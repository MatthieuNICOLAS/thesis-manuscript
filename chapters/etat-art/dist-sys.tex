\begin{itemize}
  \item Contexte des systèmes distribués à large échelle
  \item Réplique les données afin de pouvoir supporter les pannes
  \item Adopte le paradigme de la réplication optimiste \cite{2005-optimistic-replication-saito}
  \item Autorise les noeuds à consulter et à modifier la donnée sans aucune coordination entre eux
  \item Autorise alors les noeuds à diverger temporairement
  \item Permet d'être toujours disponible, de toujours répondre aux requêtes même en cas de partition réseau
  \item Permet aussi, en temps normal, de réduire le temps de réponse (privilégie la latence) \cite{pacelc2012}
  \item Comme ce modèle autorise les noeuds à modifier la donnée sans se coordonner, possible d'effectuer des modifications concurrentes
  \item Généralement, un mécanisme de résolution de conflits est nécessaire afin d'assurer la convergence des noeuds dans une telle situation
  \item Plusieurs approches ont été proposées pour implémenter un tel mécanisme
\end{itemize}
