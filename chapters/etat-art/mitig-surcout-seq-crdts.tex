L'augmentation du surcoût des \acp{CRDT} pour Séquence, qu'il soit dû à des pierres tombales ou à des identifiants de taille non-bornée, est un problème bien identifié dans la littérature \cite{ROH2011354,2009-treedoc-preguica,letia:hal-01248270,zawirski:hal-01248197,lseq2013,lseq2017}.
Plusieurs approches ont donc été proposées pour réduire sa croissance.

\subsection{Mécanisme de \ac{GC} des pierres tombales}

Pour réduire l'impact des pierres tombales sur les performances de \ac{RGA}, \textcite{ROH2011354} proposent un mécanisme de \ac{GC} des pierres tombales.
Pour rappel, ce mécanisme nécessite qu'une pierre tombale ne puisse plus être utilisée comme prédecesseur par une opération $\trm{ins}$ reçue dans le futur pour pouvoir être supprimée définitivement.
En d'autres termes, ce mécanisme repose sur la stabilité causale de l'opération de suppression pour supprimer la pierre tombale correspondante.

La stabilité causale est une contrainte forte, peu adaptée aux systèmes \ac{P2P} dynamiques à large échelle.
Notamment, la stabilité causale nécessite que chaque noeud du système fournisse régulièrement des informations sur son avancée, \ie quelles opérations il a intégré, pour progresser.
Ainsi, il suffit qu'un noeud du système se déconnecte pour bloquer la stabilité causale, ce qui apparaît extrêmement fréquent dans le cadre d'un système \ac{P2P} dynamique dans lequel nous n'avons pas de contrôle sur les noeuds.

À notre connaissance, il s'agit du seul mécanisme proposé pour l'approche à pierres tombales.

\subsection{Ré-équilibrage de l'arbre des identifiants de position}
\label{sec:etat-art-core-nebula}

Concernant l'approche à identifiants densément ordonnés, \textcite{letia:hal-01248270} puis \textcite{zawirski:hal-01248197} proposent un mécanisme de ré-équilibrage de l'arbre des identifiants de position pour Treedoc \cite{2009-treedoc-preguica}.
Pour rappel, Treedoc souffre des problèmes suivants :
\begin{enumerate}
    \item Le déséquilibrage de son arbre des identifiants de position si les insertions sont effectuées de manière séquentielle à une position.
    \item La présence de pierres tombales dans son arbre des identifiants de position lorsque des identifiants correspondants à des noeuds intermédiaires de l'arbre sont supprimés.
\end{enumerate}

Pour répondre à ces problèmes, les auteurs présentent un mécanisme de ré-équilibrage de l'arbre supprimant par la même occasion les pierres tombales existantes, \ie un mécanisme réattribuant de nouveaux identifiants de position aux éléments encore présents.
Ce mécanisme prend la forme d'une nouvelle opération, que nous notons $\trm{bal}$.

Notons que l'opération $\trm{bal}$ contrevient à une des propriétés des identifiants de position densément ordonnés : leur \emph{immutabilité} \cf{def:id-prop}.
L'opération $\trm{bal}$ est donc intrinséquement non-commutative avec les opérations $\trm{ins}$ et $\trm{rmv}$ concurrentes.
Pour assurer la convergence à terme des copies, les auteurs mettent en place un mécanisme de \emph{catch-up}.
Ce mécanisme consiste à transformer les opérations concurrentes aux opérations $\trm{bal}$ avant de les intégrer, de façon à prendre en compte les effets des ré-équilibrages.

Toutefois, l'opération $\trm{bal}$ n'est pas non plus commutative avec elle-même.
Cette approche nécessite d'empêcher la génération d'opérations $\trm{bal}$ concurrentes.
Pour cela, les auteurs proposent de reposer sur un protocole de consensus entre les noeuds pour la génération d'opérations $\trm{bal}$.

De nouveau, l'utilisation d'un protocole de consensus est une contrainte forte, peu adaptée aux systèmes \ac{P2P} dynamique à large échelle.
Pour pallier à ce point, les auteurs proposent de répartir les noeuds dans deux groupes : le \emph{core} et la \emph{nebula}.

Le \emph{core} est un ensemble, de taille réduite, de noeuds stables et hautement connectés tandis que la \emph{nebula} est un ensemble, de taille non-bornée, de noeuds.
Seuls les noeuds du \emph{core} participent à l'exécution du protocole de consensus.
Les noeuds de la \emph{nebula} contribuent toujours au document par le biais des opérations $\trm{ins}$ et $\trm{rmv}$.

Ainsi, cette solution permet d'adapter l'utilisation d'un protocole de consensus à un système \ac{P2P} dynamique.
Cependant, elle requiert de disposer de noeuds stables et bien connectés dans le système pour former le \emph{core}.
Cette condition est un obstacle pour le déploiement et la pérennité de cette solution.

\subsection{Réduction de la croissance des identifiants de position}
\label{sec:etat-art-lseq}

L'approche LSEQ \cite{lseq2013, lseq2017} est une approche visant à réduire la croissance des identifiants dans les Séquences \acp{CRDT} à identifiants densément ordonnés.
Au lieu de réduire périodiquement la taille des métadonnées liées aux identifiants à l'aide d'un mécanisme de renommage coûteux, les auteurs définissent de nouvelles stratégies d'allocation des identifiants pour réduire leur vitesse de croissance.

Dans ces travaux, les auteurs notent que les stratégies d'allocation des identifiants proposées pour Logoot dans \cite{2009-logoot-weiss} et \cite{2010-logoot-undo-weiss} ne sont adaptées qu'à un seul comportement d'édition : l'édition séquentielle.
Si les insertions sont effectuées en suivant d'autres comportements, les identifiants générés saturent rapidement l'espace des identifiants pour une taille donnée.
Les insertions suivantes déclenchent alors une augmentation de la taille des identifiants.
En conséquent, la taille des identifiants dans Logoot augmente de façon linéaire au nombre d'insertions, au lieu de suivre la progression logarithmique attendue.

LSEQ définit donc plusieurs stratégies d'allocation d'identifiants adaptées à différents comportements d'édition.
Les noeuds choisissent de manière aléatoire mais déterministe une de ces stratégies pour chaque taille d'identifiants.
De plus, LSEQ adopte une structure d'arbre exponentiel pour allouer les identifiants : l'espace des identifiants possibles double à chaque fois que la taille des identifiants augmente.
Cela permet à LSEQ de choisir avec soin la taille des identifiants et la stratégie d'allocation en fonction des besoins.
En combinant les différentes stratégies d'allocation avec la structure d'arbre exponentiel, LSEQ offre une croissance polylogarithmique de la taille des identifiants en fonction du nombre d'insertions.

Cette solution ne repose sur aucune coordination synchrone entre les noeuds.
Sa complexité ne dépend pas non plus du nombre de noeuds du système.
Elle nous apparaît donc adaptée aux systèmes \ac{P2P} dynamique à large échelle.

Nous conjecturons cependant que cette approche perd ses bienfaits lorsqu'elle est couplée avec un \ac{CRDT} pour Séquence à granularité variable.
En effet, comme évoqué précédemment, toute insertion au sein d'un bloc provoque une augmentation de la taille de l'identifiant résultant \cf{sec:growth-ids-split}.

\subsection{Synthèse}

Ainsi, plusieurs approches ont été proposées dans la littérature pour réduire le surcoût des \acp{CRDT} pour Séquence.
Cependant, aucune de ces approches ne nous apparaît adaptée pour les \acp{CRDT} pour Séquence à granularité variable dans le contexte de systèmes \ac{P2P} dynamiques :

\begin{enumerate}
    \item Les approches présentées dans \cite{ROH2011354,letia:hal-01248270,zawirski:hal-01248197} reposent chacune sur des contraintes fortes dans les systèmes \ac{P2P} dynamiques, \ie respectivement la stabilité causale des opérations et l'utilisation d'un protocole de consensus.
        Dans un système dans lequel nous n'avons aucun contrôle sur les noeuds et notamment leur disponibilité, ces contraintes nous apparaissent rédhibitoires.
    \item L'approche présentée dans \cite{lseq2013,lseq2017} est conçue pour les \acp{CRDT} pour Séquence à identifiants densément ordonnés à granularité fixe.
        L'introduction de mécanismes d'aggrégation dynamique des éléments en blocs comme ceux présentés dans \cite{2013-logootsplit,briot:hal-01343941}, avec les contraintes qu'ils introduisent, nous semble contrarier les efforts effectués pour réduire la croissance des identifiants de position.
\end{enumerate}

Nous considérons donc la problématique du surcoût des \acp{CRDT} pour Séquence à granularité variable toujours ouverte.
