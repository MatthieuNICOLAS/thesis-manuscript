Comme évoqué précédemment, d'autres sémantiques sont spécifiques au type de données concerné.
Ainsi, nous abordons à présent des sémantiques spécifiques au type de l'Ensemble.

Dans le cadre de l'Ensemble, un conflit est provoqué lorsque des modifications $\trm{add}$ et $\trm{remove}$ d'un même élément sont effectuées en concurrence.
Ainsi, deux approches peuvent être proposées pour résoudre le conflit :

\begin{enumerate}
  \item Une sémantique où la modification $\trm{add}$ d'une élément prend la précédence sur les modifications concurrentes $\trm{remove}$ du même élément, nommée \acf{AW}.
    L'élément est alors présent dans l'état obtenu à l'issue de la résolution du conflit.
  \item Une sémantique où la modification $\trm{remove}$ d'une élément prend la précédence sur les opérations concurrentes $\trm{add}$ du même élément, nommée \acf{RW}.
    L'élément est alors absent de l'état obtenu à l'issue de la résolution du conflit.
\end{enumerate}

La \autoref{fig:set-aw-rw} illustre l'application de chacune de ces sémantiques sur notre exemple.

\begin{figure}[!ht]

  \subfloat[Application de la sémantique \ac{AW}]{
      \begin{minipage}{\columnwidth}
        \resizebox{\columnwidth}{!}{
          \centering
          \begin{tikzpicture}
            \path
                node {\textbf{A}}
                ++(0:0.5) node (a) {}
                +(0:21) node (a-end) {}
                +(0:2) node[point, label=above right:{$\{a\}$}] (a-initial) {}
                +(0:7) node[point, label=above right:{$\{\}$}, label=below left:{$\trm{rmv}(a)$}] (a-removes) {}
                +(0:12) node[point, label=above right:{$\{a\}$}, label=below left:{$\trm{add}(a)$}] (a-add) {}
                +(0:19) node[point, label=above right:{$\{a\}$}] (a-conflicts) {};

            \draw[dotted] (a) -- (a-initial) (a-conflicts) -- (a-end);
            \draw[->, thick] (a-initial) --  (a-removes) -- (a-add) -- (a-conflicts);

            \path
                ++(270:3) node {\textbf{B}}
                ++(0:0.5) node (b) {}
                +(0:21) node (b-end) {}
                +(0:2) node[point, label=below right:{$\{a\}$}] (b-initial) {}
                +(0:15.5) node[point, label=below right:{$\{\}$}, label=above left:{$\trm{rmv}(a)$}] (b-removes) {}
                +(0:19) node[point, label=below right:{$\{a\}$}] (b-conflicts) {};

            \draw[dotted] (b) -- (b-initial) (b-conflicts) -- (b-end);
            \draw[->, thick] (b-initial) --  (b-removes) -- (b-conflicts);

            \draw[->, dashed, shorten >= 1] (a-add) edge node[above right, near start] {\emph{sync}} (b-conflicts);
            \draw[->, dashed, shorten >= 1] (b-removes) edge node[below right, near start, xshift=-10pt, yshift=-7pt] {\emph{sync}} (a-conflicts);
          \end{tikzpicture}
          \label{fig:set-aw}}
      \end{minipage}}
  \hfil
  \subfloat[Application de la sémantique \ac{RW}]{
      \begin{minipage}{\columnwidth}
        \centering
        \resizebox{\columnwidth}{!}{
          \begin{tikzpicture}
            \path
                node {\textbf{A}}
                ++(0:0.5) node (a) {}
                +(0:21) node (a-end) {}
                +(0:2) node[point, label=above right:{$\{a\}$}] (a-initial) {}
                +(0:7) node[point, label=above right:{$\{\}$}, label=below left:{$\trm{rmv}(a)$}] (a-removes) {}
                +(0:12) node[point, label=above right:{$\{a\}$}, label=below left:{$\trm{add}(a)$}] (a-add) {}
                +(0:19) node[point, label=above right:{$\{\}$}] (a-conflicts) {};

            \draw[dotted] (a) -- (a-initial) (a-conflicts) -- (a-end);
            \draw[->, thick] (a-initial) --  (a-removes) -- (a-add) -- (a-conflicts);

            \path
                ++(270:3) node {\textbf{B}}
                ++(0:0.5) node (b) {}
                +(0:21) node (b-end) {}
                +(0:2) node[point, label=below right:{$\{a\}$}] (b-initial) {}
                +(0:15.5) node[point, label=below right:{$\{\}$}, label=above left:{$\trm{rmv}(a)$}] (b-removes) {}
                +(0:19) node[point, label=below right:{$\{\}$}] (b-conflicts) {};

            \draw[dotted] (b) -- (b-initial) (b-conflicts) -- (b-end);
            \draw[->, thick] (b-initial) --  (b-removes) -- (b-conflicts);

            \draw[->, dashed, shorten >= 1] (a-add) edge node[above right, near start] {\emph{sync}} (b-conflicts);
            \draw[->, dashed, shorten >= 1] (b-removes) edge node[below right, near start, xshift=-10pt, yshift=-7pt] {\emph{sync}} (a-conflicts);
          \end{tikzpicture}
          \label{fig:set-rw}}
      \end{minipage}}
  \caption{Résolution du conflit en utilisant soit la sémantique \ac{AW}, soit la sémantique \ac{RW}}
  \label{fig:set-aw-rw}
\end{figure}
