Comme indiqué précédemment, la taille des identifiants provenant d'un ordre total dense est variable.
Quand les noeuds insèrent de nouveaux éléments entre deux autres ayant la même valeur de \emph{position}, LogootSplit n'a pas d'autre choix que d'augmenter la taille de l'identifiant résultant.
La \autoref{fig:example-split} illustre de tels cas.
Dans cet exemple, puisque le noeud A insère un nouvel élément entre deux identifiants contigus \id{i}{B0}{0} et \id{i}{B0}{1}, LogootSplit ne peut pas générer un identifiant adapté de la même taille.
Pour respecter l'ordre souhaité, LogootSplit génère un identifiant en ajoutant un nouveau tuple à l'identifiant du prédecesseur : \id{i}{B0}{0}\id{f}{A0}{0}.

\begin{figure}[!ht]
  \centering
  \begin{tikzpicture}
      \path
          node {\textbf{A}}
          ++(0:\widthletter) node[block, label=below:{\id{i}{B0}{0..2}}] (HLO) {HLO}
          ++(0:5 * \widthletter) node[letter, label=below:{\id{i}{B0}{0}}] (H) {H}
          ++(0:\widthletter) node[letter, fill=mydarkorange, label=above:{\color{mydarkorange}\id{i}{B0}{0}\id{f}{A0}{0}}] {E}
          ++(0:\widthletter) node[block, label=below:{\id{i}{B0}{1..2}}] {LO};

      \draw[->, thick] (HLO) -- node[below, align=center]{\emph{insert "E"}\\\emph{between}\\\emph{"H" and "L"}} (H);
  \end{tikzpicture}
  \caption{Insertion menant à une augmentation de la taille des identifiants}
  \label{fig:example-split}
\end{figure}

Par conséquent, la taille des identifiants a tendance à croître alors que le système progresse.
Cette croissance impacte négativement les performances de la structure de données sur plusieurs aspects.
Puisque les identifiants attachés aux éléments deviennent plus long, le surcoût en métadonnées augmente.
Ceci augmente aussi la consommation en bande-passante puisque les noeuds doivent diffuser les identifiants aux autres.

\mnnote{TODO: Ajouter une phrase pour expliquer que la croissance des identifiants impacte aussi le temps d'intégration des modifications}

De plus, le nombre de blocs composant la séquence répliquée augmente au fil du temps.
En effet, plusieurs contraintes sur la génération d'identifiants empêchent les noeuds d'ajouter des nouveaux éléments aux blocs existants.
Par exemple, seul le noeud qui a généré un bloc peut ajouter un élément à ce dernier.
Ces limitations provoquent la génération de nouveau blocs.
La séquence se retrouve finalement fragmentée en de nombreux blocs de seulement quelques caractères chacun.
Cependant, aucun mécanisme pour fusionner les blocs à posteriori n'est fourni.
L'efficacité de la structure décroît donc puisque chaque bloc entraîne un surcoût.

Comme illustré plus loin, nous avons mesuré au cours de nos évaluations que le contenu représente à terme moins de 1\% de taille de la structure de données.
Les 99\% restants correspondent aux métadonnées utilisées par la séquence répliquée.
Il est donc nécessaire de proposer des mécanismes et techniques afin de mitiger les problèmes soulignés précédemments.
