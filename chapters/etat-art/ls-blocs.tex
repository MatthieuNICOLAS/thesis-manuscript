\label{sec:blocs}

Afin de réduire le surcoût de la séquence, LogootSplit propose d'aggréger de façon dynamique les éléments et leur identifiants dans des blocs.
Pour cela, LogootSplit introduit la notion d'intervalle d'identifiants :
\begin{definition}[Intervalle d'identifiants]
  Un intervalle d'identifiants est un couple $\langle \trm{idBegin},\trm{offsetEnd} \rangle$ avec :
  \begin{enumerate}
    \item $\trm{idBegin}$, l'identifiant du premier élément de l'interval.
    \item $\trm{offsetEnd}$, l'offset du dernier identifiant de l'interval.
  \end{enumerate}
\end{definition}

Les intervalles d'identifiants permettent à LogootSplit d'assigner logiquement un identifiant à un ensemble d'éléments, tout en ne stockant de manière effective que l'identifiant de son premier élément et le dernier offset de son dernier élément.

LogootSplit regroupe les éléments avec des identifiants \emph{contigus} dans un intervalle.
\begin{definition}[Identifiants contigus]
  Deux identifiants sont contigus si et seulement si les deux identifiants sont identiques à l'exception de leur dernier offset et que leur derniers offsets sont consécutifs.

  De manière plus formelle, étant donné deux identifiants $\trm{id} = t_1 \oplus t_2 \oplus ... \oplus t_{n-1} \oplus \langle \trm{pos},\trm{nodeId},\trm{seq},\trm{offset} \rangle$ et $\trm{id'} = t^\prime_1 \oplus t^\prime_2 \oplus ... \oplus t^\prime_{n-1} \oplus \langle \trm{pos}',\trm{nodeId}',\trm{seq}',\trm{offset}' \rangle$, on a :
  \begin{equation*}
    \begin{split}
      \trm{contigus}(id,id') \quad \trm{iff} \quad     & (\forall i \in [1,n[ \cdot t_i = t'_i) \quad \land \\
                                            & (\trm{pos} = \trm{pos}' \land \trm{nodeId} = \trm{nodeId}' \land \trm{seq} = \trm{seq}' \land \trm{offset} + 1 = \trm{offset}') \\
    \end{split}
  \end{equation*}

\end{definition}

Nous représentons un intervalle d'identifiants à l'aide du formalisme suivant : $\betterid{position}{nodeId~nodeSeq}{begin..end}$ où $\trm{begin}$ est l'offset du premier identifiant de l'intervalle et $\trm{end}$ du dernier.

LogootSplit utilise une structure de données pour associer un intervalle d'identifiants aux éléments correspondants : les blocs.

\begin{definition}[Bloc]
  Un bloc est un triplet $\langle \trm{idInterval}, \trm{elts}, \trm{isAppendable} \rangle$ avec :
  \begin{enumerate}
    \item $\trm{idInterval}$, l'intervalle d'identifiants associés au bloc.
    \item $\trm{elts}$, les éléments contenus dans le bloc.
    \item $\trm{isAppendable}$, un booléen indiquant si l'auteur du bloc peut ajouter de nouveaux éléments en fin de bloc
    \footnote{
      De manière similaire, il est possible de permettre à l'auteur du bloc d'ajouter de nouveaux éléments en début de bloc à l'aide d'un booléen $\trm{isPrependable}$.
      Cette fonctionnalité est cependant incompatible avec le mécanisme que nous proposons dans le \autoref{chap:rls}.
      Nous faisons donc le choix de la retirer.
    }.
  \end{enumerate}
\end{definition}

Nous représentons un exemple de séquence LogootSplit dans la \autoref{fig:logootsplit-seq}.
Dans la \autoref{fig:logootsplit-seq-as-letters}, les identifiants $\betterid{i}{B1}{1}$, $\betterid{i}{B1}{2}$ et $\betterid{i}{B1}{3}$ forment une chaîne d'identifiants contigus.
LogootSplit est donc capable de regrouper ces éléments en un bloc représentant l'intervalle d'identifiants $\betterid{i}{B1}{1..3}$ pour minimiser les métadonnées stockées, comme illustré dans la \autoref{fig:logootsplit-seq-as-block}.

\begin{figure}[!ht]
  \centering
  \subfloat[Éléments avec leur identifiant correspondant]{
      \begin{minipage}{.48\linewidth}
      \centering
          \begin{tikzpicture}
              \path
                  node[letter, label=below:{\id{i}{B1}{1}}] {H}
                  ++(0:\widthletter) node[letter, label=below:{\id{i}{B1}{2}}] {L}
                  ++(0:\widthletter) node[letter, label=below:{\id{i}{B1}{3}}] {O};
          \end{tikzpicture}
          \label{fig:logootsplit-seq-as-letters}
      \end{minipage}}
  \hfil
  \subfloat[Éléments regroupés en un bloc]{
      \begin{minipage}{.48\linewidth}
      \centering
          \begin{tikzpicture}
              \path
                  node[block, label=below:{\id{i}{B1}{1..3}}] {HLO};
          \end{tikzpicture}
          \label{fig:logootsplit-seq-as-block}
      \end{minipage}}
  \caption{Représentation d'une séquence LogootSplit contenant les éléments "HLO"}
  \label{fig:logootsplit-seq}
\end{figure}

Cette fonctionnalité permet d'améliorer les performances de la structure de données sur plusieurs aspects :
\begin{enumerate}
  \item Elle réduit le nombre d'identifiants stockés au sein de la structure de données.
    En effet, les identifiants sont désormais conservés à l'échelle des blocs plutôt qu'à l'échelle de chaque élément.
    Ceci permet de réduire le surcoût en métadonnées du \ac{CRDT}.
  \item L'utilisation de blocs comme niveau de granularité, en lieu et place des éléments, permet de réduire la complexité en temps des manipulations de la structure de données.
  \item L'utilisation de blocs permet aussi d'effectuer des modifications à l'échelle de plusieurs éléments, et non plus un par un seulement.
    Ceci permet de réduire la taille des messages diffusés sur le réseau.
\end{enumerate}

Comme pour les identifiants, il est intéressant de noter que la paire $\langle \trm{nodeId},\trm{seq} \rangle$ du dernier tuple d'un identifiant permet d'identifier de manière unique la partie commune des identifiants de l'intervalle d'identifiants auquel il appartient.
Ainsi, nous pouvons identifier de manière un intervalle d'identifiants avec le quadruplet $\langle \trm{nodeId}, \trm{seq}, \trm{offsetBegin}, \trm{offsetEnd} \rangle$.
Par exemple, l'intervalle d'identifiants $\betterid{i}{B1}{2}\betterid{f}{A1}{2..4}$ peut être référencé à l'aide du quadruplet $\langle A,1,2,4 \rangle$.
