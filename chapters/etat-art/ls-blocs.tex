\label{sec:blocs}

Au lieu de stocker les identifiants de chaque élément de la séquence, LogootSplit propose d'aggréger de façon dynamique les éléments dans des blocs.
Pour cela, LogootSplit introduit la notion d'intervalle d'identifiants :

\begin{definition}[IdInterval]
  Un \emph{IdInterval} est un couple $\langle$idBegin, offsetEnd$\rangle$ où
  \begin{itemize}
    \item idBegin est l'identifiant du premier élément de l'interval.
    \item offsetEnd est l'offset du dernier identifiant de l'interval.
  \end{itemize}
\end{definition}

Les intervalles d'identifiants permettent à LogootSplit d'assigner logiquement un identifiant à un ensemble d'éléments, tout en ne stockant réellement que l'identifiant de son premier élément et le dernier offset de son dernier élément.

LogootSplit regroupe les éléments avec des identifiants \emph{contigus} dans un interval.
Nous appelons \emph{contigus} deux identifiants qui partagent une même base (\ie qui sont identiques à l'exception de leur dernier offset) et dont les \emph{offsets} sont consécutifs.
Nous représentons un intervalle d'identifiants à l'aide du formalisme suivant : \id{position}{nodeId~nodeSeq}{begin..end} où $\trm{begin}$ est l'offset du premier identifiant de l'intervalle et $\trm{end}$ du dernier.

Les blocs permettent d'associer un intervalle d'identifiants aux éléments correspondant.
Les blocs sont définis de la manière suivante :

\begin{definition}[Bloc]
  Un \emph{Bloc} est un quadruplet $\langle$idInterval, elts, isAppendable, isPrependable$\rangle$ où
  \begin{itemize}
    \item idInterval est l'intervalle d'identifiants formant le bloc
    \item elts sont les éléments contenus dans le bloc
    \item isAppendable (resp. isPrependable) est un booléen indiquant si l'auteur du bloc peut ajouter un nouvel élément en fin (resp. début) de bloc
  \end{itemize}
\end{definition}

La \autoref{fig:logootsplit-seq} présente un exemple de séquence LogootSplit : dans la \autoref{fig:logootsplit-seq-as-letters}, les identifiants \id{i}{B0}{0}, \id{i}{B0}{1}, \id{i}{B0}{2} forment une chaîne d'identifiants contigus.
LogootSplit est donc capable de regrouper ces éléments en un bloc représentant l'intervalle d'identifiants \id{i}{B0}{0..2} pour minimiser les métadonnées stockées, comme montré dans la \autoref{fig:logootsplit-seq-as-block}.

\begin{figure}[!ht]
  \centering
  \subfloat[Éléments avec leur identifiant correspondant]{
      \begin{minipage}{.48\linewidth}
      \centering
          \begin{tikzpicture}
              \path
                  node[letter, label=below:{\id{i}{B0}{0}}] {H}
                  ++(0:\widthletter) node[letter, label=below:{\id{i}{B0}{1}}] {L}
                  ++(0:\widthletter) node[letter, label=below:{\id{i}{B0}{2}}] {O};
          \end{tikzpicture}
          \label{fig:logootsplit-seq-as-letters}
      \end{minipage}}
  \hfil
  \subfloat[Éléments regroupés en un bloc]{
      \begin{minipage}{.48\linewidth}
      \centering
          \begin{tikzpicture}
              \path
                  node[block, label=below:{\id{i}{B0}{0..2}}] {HLO};
          \end{tikzpicture}
          \label{fig:logootsplit-seq-as-block}
      \end{minipage}}
  \caption{Représentation d'une séquence LogootSplit contenant les éléments "HLO"}
  \label{fig:logootsplit-seq}
\end{figure}

Cette fonctionnalité réduit le nombre d'identifiants stockés au sein de la structure de données, puisque les identifiants sont conservés à l'échelle des blocs plutôt qu'à l'échelle de chaque élément.
Ceci permet de réduire de manière significative le surcoût en métadonnées de la structure de données.
L'utilisation de blocs améliore aussi les performances de la structure de données.
En effet, l'utilisation de blocs permet de parcourir plus efficacement la structure de données.
Les blocs permettent aussi d'effectuer des modifications à l'échelle de la chaîne de caractères et non plus seulement caractère par caractère.

\mnnote{TODO: indiquer que le couple $\langle$nodeId, nodeSeq$\rangle$ permet d'identifier de manière unique la base d'un bloc ou d'un identifiant}

Notons que pour une séquence donnée, nous pouvons identifier chacun de ses identifiants par le triplet $\langle$nodeId, nodeSeq, offset$\rangle$ issue de leur dernier Tuple.
Par exemple, le triplet $\langle$B, 0, 2$\rangle$ désigne de manière unique l'identifiant \id{i}{B0}{2} dans \autoref{fig:logootsplit-seq}.
