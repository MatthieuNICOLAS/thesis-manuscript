L'approche de la synchronisation par états propose que les noeuds diffusent leurs modifications en transmettant leur état.
Les \acp{CRDT} adoptant cette approche doivent définir une fonction \texttt{merge}.
Cette fonction correspond à la fonction de fusion mentionnée précédemment \cf{def:spec-crdts} : elle prend en paramètres une paire d'états et génère en retour l'état correspondant à leur \ac{LUB}.
Cette fonction doit être associative, commutative et idempotente.

Ainsi, lorsqu'un noeud reçoit l'état d'un autre noeud, il fusionne ce dernier avec son état courant à l'aide de la fonction \texttt{merge}.
Il obtient alors un nouvel état intégrant l'ensemble des modifications ayant été effectuées sur les deux états.

La nature croissante des états des \acp{CRDT} couplée aux propriétés d'associativité, de commutativité et d'idempotence de la fonction \texttt{merge} permettent de reposer sur la couche de livraison sans lui imposer de contraintes fortes : les messages peuvent être perdus, réordonnés ou même dupliqués.
Les noeuds convergeront tant que la couche de livraison garantit que les noeuds seront capables de transmettre leur état aux autres à terme.
Il s'agit là de la principale force des \acp{CRDT} synchronisés par états.

Néanmoins, la définition de la fonction \texttt{merge} offrant ces propriétés peut s'avérer complexe et a des répercussions sur la spécification même du \ac{CRDT}.
Notamment, les états doivent conserver une trace de l'existence des éléments et de leur suppression afin d'éviter qu'une fusion d'états ne les fassent ressurgirent.
Ainsi, les \acp{CRDT} synchronisés par états utilisent régulièrement des pierres tombales.

\begin{definition}[Pierre tombale]
  Une pierre tombale est un marqueur de la présence passée d'un élément.

  Dans le contexte des \acp{CRDT}, un identifiant est généralement associé à chaque élément.
  Lors de la suppression de l'élément, l'élément peut être supprimé de manière effective, mais son identifiant est conservé dans la structure de données.
\end{definition}

En plus de l'utilisation de pierres tombales, la taille de l'état peut croître de manière non-bornée dans le cas de certains types de donnés, \eg l'Ensemble ou la Séquence.
Ainsi, ces structures peuvent atteindre à terme des tailles conséquentes.
Dans de tels cas, diffuser l'état complet à chaque modification induirait alors un coût rédhibitoire.
L'approche de la synchronisation par états s'avère donc inadaptée aux systèmes temps réel et repose généralement sur une synchronisation périodique.

Nous illustrons le fonctionnement de cette approche avec la \autoref{fig:sync-model-state}.
Dans cet exemple, après que les noeuds aient effectués leurs modifications respectives, le mécanisme de synchronisation périodique de chaque noeud se déclenche.
Le noeud A (resp. B) diffuse alors son état $\set{a,b,c,e}$ (resp. $\set{a,d,e}$) à B (resp. A).

\begin{figure}[!ht]

  \centering
  \resizebox{\columnwidth}{!}{
    \begin{tikzpicture}
      \path
          node {\textbf{A}}
          ++(0:0.5) node (a) {}
          +(0:21) node (a-end) {}
          +(0:2) node[point, label=above right:{$\set{a,e}$}] (a-initial) {}
          +(0:7) node[point, label=above right:{$\set{a,b,e}$}, label=-170:{$\trm{add}(b)$}] (a-add-b) {}
          +(0:11) node[point, label=above right:{$\set{a,b,c,e}$}, label=-170:{$\trm{add}(c)$}] (a-add-c) {}
          +(0:16) node[point, label=below left:{$\trm{sync}$}, label={[xshift=10pt]-10:{$\set{a,b,c,e}$}}] (a-sends-state) {}
          +(0:19) node[point, label=above right:{$\set{a,b,c,d,e}$}] (a-final) {};

      \draw[dotted] (a) -- (a-initial) (a-final) -- (a-end);
      \draw[->, thick] (a-initial) --  (a-add-b) -- (a-add-c) -- (a-sends-state) -- (a-final);

      \path
          ++(270:3) node {\textbf{B}}
          ++(0:0.5) node (b) {}
          +(0:21) node (b-end) {}
          +(0:2) node[point, label=below right:{$\set{a,e}$}] (b-initial) {}
          +(0:11) node[point, label=below right:{$\set{a,d,e}$}, label=170:{$\trm{add}(d)$}] (b-add-d) {}
          +(0:16) node[point, label=170:{$\trm{sync}$}, label={[xshift=15pt]10:{$\set{a,d,e}$}}] (b-sends-state) {}
          +(0:19) node[point, label=below right:{$\set{a,b,c,d,e}$}] (b-final) {};

      \draw[dotted] (b) -- (b-initial) (b-final) -- (b-end);
      \draw[->, thick] (b-initial) --  (b-add-d) -- (b-sends-state) -- (b-final);

      \draw[->, dashed, shorten >= 1] (a-sends-state) -- (b-final);
      \draw[->, dashed, shorten >= 1] (b-sends-state) -- (a-final);
    \end{tikzpicture}
  }
  \caption{Synchronisation des noeuds A et B en adoptant le modèle de synchronisation par états}
  \label{fig:sync-model-state}
\end{figure}

À la réception de l'état, chaque noeud utilise la fonction \texttt{merge} pour intégrer les modifications de l'état reçu dans son propre état.
Dans le cadre de l'Ensemble répliqué, cette fonction consiste généralement à faire l'union des états, en prenant en compte l'estampille et le statut (présent ou non) associé à chaque élément.
Ainsi la fusion de leur état respectif, $\set{a,b,c,e} \cup \set{a,d,e}$, permet aux noeuds de converger à l'état souhaité : $\set{a,b,c,d,e}$.

Avant de conclure, il est intéressant de noter que les \acp{CRDT} adoptant ce modèle de synchronisation respectent de manière intrinsèque le modèle de cohérence causale \cite{2011-consistency-availability-convergence-mahajan}.
En effet, ce modèle de synchronisation assure l'intégration soit de toutes les modifications connues d'un noeud, soit d'aucune.
Par exemple, dans la \autoref{fig:sync-model-state}, le noeud B ne peut pas recevoir et intégrer l'élément $c$ sans l'élement $b$.
Ainsi, ce modèle permet naturellement d'éviter ce qui pourrait être interprétées comme des anomalies par les utilisateur-rices.
