Pour ce faire, LogootSplit associe aux éléments des identifiants définis de la manière suivante :

\begin{definition}[Identifiant LogootSplit]
  Un identifiant LogootSplit est une liste de tuples LogootSplit.
  Les tuples LogootSplit sont définis de la manière suivante :
  \begin{subdefinition}[Tuple LogootSplit]
    Un tuple LogootSplit est un quadruplet $\langle \trm{pos},\trm{nodeId},$ \\
    $\trm{seq},\trm{offset} \rangle$ avec :
    \begin{enumerate}
      \item $\trm{pos}$, un entier représentant la position relative du tuple dans l'espace dense,
      \item $\trm{nodeId}$, l'identifiant du noeud auteur de l'élément,
      \item $\trm{seq}$, le numéro de séquence courant du noeud auteur de l'élément.
      \item $\trm{offset}$, la position de l'élément au sein d'un bloc. Nous reviendrons plus en détails sur ce composant dans la \autoref{sec:blocs}.
    \end{enumerate}
  \end{subdefinition}
\end{definition}

Dans ce manuscrit, nous représentons les tuples LogootSplit par le biais de la notation suivante : $\betterid{position}{nodeId~nodeSeq}{offset}$.
Sans perdre en généralité, nous utiliserons des lettres minuscules comme valeurs pour $\trm{pos}$, des lettres majuscules pour $\trm{nodeId}$ et des entiers naturels pour $\trm{seq}$ et $\trm{offset}$.
Par exemple, nous représentons l'identifiant $\langle \langle i,A,1,1 \rangle \langle f,B,1,1 \rangle \rangle$ par $\betterid{i}{A1}{1}\betterid{f}{B1}{1}$.

LogootSplit utilise les identifiants de position pour ordonner relativement les éléments les uns par rapport aux autres.
LogootSplit définit une relation d'ordre strict total sur les identifiants : $\lid$.
Cette relation repose sur l'ordre lexicographique.

\begin{definition}[Relation $\lid$]
  Étant donné deux identifiants $id = t_1 \oplus t_2 \oplus ... \oplus t_n$ et $id' = t'_1 \oplus t'_2 \oplus ... \oplus t'_m$, nous avons :
  \begin{equation*}
    \begin{split}
      id \lid id' \quad \trm{iff} \quad     & (n < m \land \forall i \in [1,n] \cdot t_i = t'_i) \quad \lor \\
                                            & (\exists j \leq m \cdot \forall i < j \cdot t_i = t'_i \land t_j \ltuple t'_j) \\
    \end{split}
  \end{equation*}
  avec $\ltuple$ défini de la manière suivante :
  \begin{subdefinition}[Relation $\ltuple$]
    Étant donné deux tuples $t = \langle \trm{pos},\trm{nodeId},\trm{seq},\trm{offset} \rangle$ et $t' = \langle \trm{pos'},\trm{nodeId'},\trm{seq'},\trm{offset'} \rangle$, nous avons :
    \begin{equation*}
      \begin{split}
        t \ltuple t' \quad \trm{iff} \quad  & (\trm{pos} < \trm{pos}') \quad \lor \\
                                            & (\trm{pos} = \trm{pos'} \land \trm{nodeId} < \trm{nodeId'}) \quad \lor \\
                                            & (\trm{pos} = \trm{pos'} \land \trm{nodeId} = \trm{nodeId'} \land \trm{seq} < \trm{seq'}) \\
                                            & (\trm{pos} = \trm{pos'} \land \trm{nodeId} = \trm{nodeId'} \land \trm{seq} = \trm{seq'} \land \trm{offset} = \trm{offset'}) \\
      \end{split}
    \end{equation*}
  \end{subdefinition}
\end{definition}

Par exemple, nous avons :
\begin{enumerate}
  \item  \id{i}{A1}{1} $\lid$ \id{i}{B1}{1} car le tuple composant le premier est inférieur au tuple composant le second, \ie \id{i}{A1}{0} $\ltuple$ \id{i}{B0}{0}.
  \item \id{i}{B0}{0} $\lid$ \id{i}{B0}{0}\id{f}{A0}{0} car le premier est un préfixe du second.
\end{enumerate}

Il est intéressant de noter que le triplet $\langle \trm{nodeId},\trm{seq},\trm{offset} \rangle$ du dernier tuple d'un identifiant permet de l'identifier de manière unique.
