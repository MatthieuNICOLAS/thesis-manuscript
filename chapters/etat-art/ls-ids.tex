\subsection{Identifiants}

Pour ce faire, LogootSplit assigne des identifiants composés d'une liste de tuples aux éléments.
Les tuples sont définis de la manière suivante :

\begin{definition}[Tuple]
  Un \emph{Tuple} est un quadruplet $\langle$position, nodeId, nodeSeq, offset$\rangle$ où
  \begin{itemize}
    \item position incarne la position souhaitée de l'élément.
    \item nodeId est l'identifiant unique du noeud qui a généré le tuple.
    \item nodeSeq est le numéro de séquence courant du noeud à la génération du tuple.
    \item offset indique la position de l'élément au sein d'un bloc. Nous reviendrons plus en détails sur ce composant dans la \autoref{sec:blocs}.
  \end{itemize}
\end{definition}

\mnnote{TODO: Ajouter une relation d'ordre sur les tuples}

Dans ce manuscrit, nous représentons les tuples par le biais de la notation suivante : \id{position}{nodeId~nodeSeq}{offset} où $\trm{position}$ est une lettre minuscule, $\trm{nodeId}$ une lettre majuscule et $\trm{nodeSeq}$ et $\trm{offset}$ des entiers, \eg \id{i}{B0}{0}.

À partir de là, les identifiants LogootSplit sont définis de la manière suivante :

\begin{definition}[Identifiant]
  Un \emph{Identifiant} est une liste de \emph{Tuples}.
\end{definition}

\mnnote{TODO: Définir la notion de base (et autres fonctions utiles sur les identifiants ? genre isPrefix, concat, getTail...)}

Nous représentons les identifiants en listant les tuples qui les composent.
Par exemple, l'identifiant composé des tuples $\langle\langle$i, B, 0, 0$\rangle\langle$f, A, 0, 0$\rangle\rangle$ est présenté de la manière suivante : \id{i}{B0}{0}\id{f}{A0}{0}.

Les identifiants ont pour rôle d'ordonner les éléments relativement les uns par rapport aux autres.
Pour ce faire, une relation d'ordre total aux identifiants est associée à l'ensemble des identifiants :

\begin{definition}[Relation $\lid$]
  La relation $\lid$ est un ordre strict total sur l'ensemble des identifiants.
  Elle permet aux noeuds de comparer n'importe quelle paire d'identifiants.
  Elle est définie en utilisant l'ordre lexicographique sur les composants des différents tuples des identifiants comparés.
\end{definition}

\begin{itemize}
  \item En utilisant cette relation d'ordre, les noeuds peuvent ordonner les éléments grâce à leur identifiant.
  \item Par exemple, déterminent que \id{i}{A1}{0} $\lid$ \id{i}{B0}{0} car les positions sont identiques et que le \emph{nodeId} (A) du premier est plus petit que le \emph{nodeId} (B) du second
  \item et que \id{i}{B0}{0} $\lid$ \id{i}{B0}{0}\id{f}{A0}{0} car le premier est un préfixe du second
\end{itemize}

\mnnote{TODO: Montrer que cet ensemble d'identifiants est un ensemble dense}
