Une manière simple pour résoudre un conflit consiste à trancher de manière arbitraire et de sélectionner une modification parmi l'ensemble des modifications en conflit.
Pour faire cela de manière déterministe, une approche est de reproduire et d'utiliser l'ordre total sur les modifications qui serait instauré par une horloge globale pour choisir la modification à prioritiser.

Cette approche, présentée dans \cite{johnson1975rfc0677}, correspond à la sémantique nommée \acf{LWW}.
De par son fonctionnement, cette sémantique est générique et est donc utilisée par une variété de \acp{CRDT} pour des types différents.
La \autoref{fig:set-lww} illustre son application à l'Ensemble pour résoudre le conflit de la \autoref{fig:set-conflict}.

\begin{figure}[!ht]

  \centering
  \resizebox{\columnwidth}{!}{
    \begin{tikzpicture}
      \path
          node {\textbf{A}}
          ++(0:0.5) node (a) {}
          +(0:21) node (a-end) {}
          +(0:2) node[point, label=above right:{$\{a\}$}] (a-initial) {}
          +(0:7) node[point, label=above right:{$\{\}$}, label=below left:{$\trm{rmv}(a)$}] (a-removes) {}
          +(0:12) node[point, label=above right:{$\{a\}$}, label=below left:{$\trm{add}(a)$}] (a-add) {}
          +(0:19) node[point, label=above right:{$\{\}$}] (a-conflicts) {};

      \draw[dotted] (a) -- (a-initial) (a-conflicts) -- (a-end);
      \draw[->, thick] (a-initial) --  (a-removes) -- (a-add) -- (a-conflicts);

      \path
          ++(270:3) node {\textbf{B}}
          ++(0:0.5) node (b) {}
          +(0:21) node (b-end) {}
          +(0:2) node[point, label=below right:{$\{a\}$}] (b-initial) {}
          +(0:15.5) node[point, label=below right:{$\{\}$}, label=above left:{$\trm{rmv}(a)$}] (b-removes) {}
          +(0:19) node[point, label=below right:{$\{\}$}] (b-conflicts) {};

      \draw[dotted] (b) -- (b-initial) (b-conflicts) -- (b-end);
      \draw[->, thick] (b-initial) --  (b-removes) -- (b-conflicts);

      \draw[->, dashed, shorten >= 1] (a-add) edge node[above right, near start] {\emph{sync}} (b-conflicts);
      \draw[->, dashed, shorten >= 1] (b-removes) edge node[below right, near start, xshift=-10pt, yshift=-7pt] {\emph{sync}} (a-conflicts);
    \end{tikzpicture}
  }
  \caption{Résolution du conflit en utilisant la sémantique \ac{LWW}}
  \label{fig:set-lww}
\end{figure}

Comme indiqué précédemment, le scénario illustré dans la \autoref{fig:set-lww} présente un conflit entre les modifications concurrentes $\trm{add}(a)$ et $\trm{rmv}(a)$ générées de manière concurrente respectivement par les noeuds A et B.
Pour le résoudre, la sémantique \ac{LWW} associe à chaque modification une estampille.
L'ordre créé entre les modifications par ces dernières permet de déterminer quelle modification désigner comme prioritaire.
Ici, nous considérons que $\trm{add}(a)$ a eu lieu plus tôt que $\trm{rmv}(a)$.
La sémantique \ac{LWW} désigne donc $\trm{rmv}(a)$ comme prioritaire et ignore $\trm{add}(a)$.
L'état obtenu à l'issue de cet exemple par chaque noeud est donc $\{\}$.

Il est à noter que si la modification $\trm{rmv}(a)$ du noeud B avait eu lieu plus tôt dans notre exemple, l'état final obtenu aurait été $\{a\}$.
Ainsi, des exécutions reproduisant le même ensemble de modifications produiront des résultats différents en fonction de l'ordre créé par les estampilles associées à chaque modification.
Ces estampilles étant des métadonnées du mécanisme de résolution de conflits, elles sont dissimulées aux utilisateur-rices.
Le comportement de cette sémantique peut donc être perçu comme aléatoire et s'avérer perturbant pour les utilisateur-rices.

La sémantique \ac{LWW} repose sur l'horloge de chaque noeud pour attribuer une estampille à chacune de leurs modifications.
Les horloges physiques étant sujettes à des imprécisions et notamment des décalages, utiliser les estampilles qu'elles fournissent peut provoquer des anomalies vis-à-vis de la relation \hb.
Les systèmes distribués préfèrent donc généralement utiliser des horloges logiques \cite{1978-happen-before-lamport}.
\mnnote{TODO: Ajouter refs des horloges logiques plus intelligentes (Interval Tree Clock, Hybrid Clock...)}
