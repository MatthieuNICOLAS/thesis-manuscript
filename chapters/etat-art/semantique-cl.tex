Une nouvelle sémantique pour l'Ensemble fut proposée \cite{2020-cl-set-weihai} récemment.
Cette sémantique se base sur les observations suivantes :

\begin{enumerate}
  \item $\trm{add}$ et $\trm{remove}$ d'un élément prennent place à tour de rôle, chaque modification invalidant la précédente.
  \item $\trm{add}$ (resp. $\trm{remove}$) concurrents d'un même élément représentent la même intention.
    Prendre en compte une de ces modifications concurrentes revient à prendre en compte leur ensemble.
\end{enumerate}

À partir de ces observations, \textcite{2020-cl-set-weihai} proposent de déterminer pour chaque élément la chaîne d'ajouts et retraits la plus longue.
C'est cette chaîne, et précisément son dernier maillon, qui indique si l'élément est présent ou non dans l'ensemble final.
La \autoref{fig:set-cl} illustre son fonctionnement.

\begin{figure}[!ht]

  \centering
  \resizebox{\columnwidth}{!}{
    \begin{tikzpicture}
      \path
          node {\textbf{A}}
          ++(0:0.5) node (a) {}
          +(0:21) node (a-end) {}
          +(0:2) node[point, label=above right:{$\{a\}$}] (a-initial) {}
          +(0:7) node[point, label=above right:{$\{\}$}, label=below left:{$\trm{rmv}(a)$}] (a-removes) {}
          +(0:12) node[point, label=above right:{$\{a\}$}, label=below left:{$\trm{add}(a)$}] (a-add) {}
          +(0:19) node[point, label=above right:{$\{a\}$}] (a-conflicts) {};

      \draw[dotted] (a) -- (a-initial) (a-conflicts) -- (a-end);
      \draw[->, thick] (a-initial) --  (a-removes) -- (a-add) -- (a-conflicts);

      \path
          ++(270:3) node {\textbf{B}}
          ++(0:0.5) node (b) {}
          +(0:21) node (b-end) {}
          +(0:2) node[point, label=below right:{$\{a\}$}] (b-initial) {}
          +(0:15.5) node[point, label=below right:{$\{\}$}, label=above left:{$\trm{rmv}(a)$}] (b-removes) {}
          +(0:19) node[point, label=below right:{$\{a\}$}] (b-conflicts) {};

      \draw[dotted] (b) -- (b-initial) (b-conflicts) -- (b-end);
      \draw[->, thick] (b-initial) --  (b-removes) -- (b-conflicts);

      \draw[->, dashed, shorten >= 1] (a-add) edge node[above right, near start] {\emph{sync}} (b-conflicts);
      \draw[->, dashed, shorten >= 1] (b-removes) edge node[below right, near start, xshift=-10pt, yshift=-7pt] {\emph{sync}} (a-conflicts);
    \end{tikzpicture}
  }
  \caption{Résolution du conflit en utilisant la sémantique \ac{CL}}
  \label{fig:set-cl}
\end{figure}

Dans notre exemple, la modification $\trm{rmv}(a)$ effectuée par B est en concurrence avec une modification identique effectuée par A.
La sémantique \ac{CL} définit que ces deux modifications partagent la même intention.
Ainsi, A ayant déjà appliqué sa propre modification préalablement, il ne prend pas en compte \emph{de nouveau} cette modification lorsqu'il la reçoit de B.
Son état reste donc inchangé.

À l'inverse, la modification $\trm{add(a)}$ effectuée par A fait suite à sa modification $\trm{remove}(a)$.
La sémantique \ac{CL} définit alors qu'elle fait suite à toute autre modification $\trm{remove}(a)$ concurrente.
Ainsi, B intègre cette modification lorsqu'il la reçoit de A.
Son état évolue donc pour devenir $\{a\}$.
