\label{sec:etat-art-proposition}

Dans le cadre de cette thèse, nous proposons et présentons un nouveau mécanisme de réduction du surcoût pour les \acp{CRDT} pour le type Séquence à identifiants densément ordonnés et à granularité variable.

Ce mécanisme se distingue des travaux existants, notamment de \cite{letia:hal-01248270,zawirski:hal-01248197}, par les aspects suivants :
\begin{enumerate}
    \item Il ne nécessite pas de coordination synchrone entre les noeuds.
    \item Il ré-aggrége les éléments de la séquence en de nouveaux blocs pour réduire leur nombre.
\end{enumerate}

Nous concevons ce mécanisme pour le \ac{CRDT} LogootSplit.
Toutefois, le principe de notre approche est générique.
Ainsi, ce mécanisme peut être adapté pour proposer un équivalent pour d'autres \acp{CRDT} pour le type Séquence, \eg RGASplit.

Nous présentons et détaillons ce mécanisme dans le chapitre suivant.
