Dans le modèle de réplication optimiste, les noeuds divergent momentanément lorsqu'ils effectuent des modifications locales.
Pour ensuite converger vers des états équivalents, les noeuds doivent propager et intégrer l'ensemble des modifications.
La \autoref{fig:sync-model-base} illustre ce point.

\begin{figure}[!ht]

  \centering
  \resizebox{\columnwidth}{!}{
    \begin{tikzpicture}
      \path
          node {\textbf{A}}
          ++(0:0.5) node (a) {}
          +(0:21) node (a-end) {}
          +(0:2) node[point, label=above right:{$\set{a,e}$}] (a-initial) {}
          +(0:7) node[point, label=above right:{$\set{a,b,e}$}, label=below left:{$\trm{add}(b)$}] (a-add-b) {}
          +(0:12) node[point, label=above right:{$\set{a,b,c,e}$}, label=below left:{$\trm{add}(c)$}] (a-add-c) {}
          +(0:19) node (a-final) {};

      \draw[dotted] (a) -- (a-initial) (a-final) -- (a-end);
      \draw[->, thick] (a-initial) --  (a-add-b) -- (a-add-c) -- (a-final);

      \path
          ++(270:3) node {\textbf{B}}
          ++(0:0.5) node (b) {}
          +(0:21) node (b-end) {}
          +(0:2) node[point, label=below right:{$\set{a,e}$}] (b-initial) {}
          +(0:10) node[point, label=below right:{$\set{a,d,e}$}, label=above left:{$\trm{add}(d)$}] (b-add-d) {}
          +(0:19) node (b-final) {};

      \draw[dotted] (b) -- (b-initial) (b-final) -- (b-end);
      \draw[->, thick] (b-initial) --  (b-add-d) -- (b-final);
    \end{tikzpicture}
  }
  \caption{Modifications en concurrence d'un Ensemble répliqué par les noeuds A et B}
  \label{fig:sync-model-base}
\end{figure}

Dans cet exemple, deux noeuds A et B partagent et éditent un même Ensemble à l'aide d'un \ac{CRDT}.
Les deux noeuds possèdent le même état initial : $\set{a,e}$.

Le noeud A effectue les modifications $\trm{add}(b)$ puis $\trm{add}(c)$.
Il obtient ainsi l'état $\set{a,b,c,e}$.
De son côté, le noeud B effectue la modification suivante : $\trm{add}(d)$.
Son état devient donc $\set{a,d,e}$.
Ainsi, les noeuds doivent encore s'échanger leur modifications pour converger vers l'état souhaité\footnote{Le scénario ne comportant uniquement des modifications $\trm{add}$, aucun conflit n'est produit malgré la concurrence des modifications.}, \ie $\set{a,b,c,d,e}$.

% Dans ce scénario, aucune communication n'ayant eu lieu entre A et B au cours de cette exécution.
% Leurs opérations respectives sont donc concurrentes.
% Il est à noter que ce scénario ne présente uniquement des modifications $\trm{add}$.
% Ainsi, aucun conflit n'est produit malgré la concurrence des modifications.
% La sémantique utilisée par le \ac{CRDT} dans cet exemple n'a donc pas d'importance.

Dans le cadre des \acp{CRDT}, le choix de la méthode pour synchroniser les noeuds n'est pas anodin.
En effet, ce choix impacte la spécification même du \ac{CRDT} et ses prérequis.

Initialement, deux approches ont été proposées : une méthode de synchronisation par états \cite{shapiro_2011_crdt, shapiro:inria-00555588} et une méthode de synchronisation par opérations \cite{shapiro_2011_crdt, shapiro:inria-00555588, 2014-making-op-based-crdts-op-based, baquero2017pure}.
Une troisième approche, nommée synchronisation par différence d'états \cite{almeida2015delta, Almeida_2018}, fut spécifiée par la suite.
Le but de cette dernière est d'allier le meilleur des deux approches précédentes.

Dans la suite de cette section, nous présentons ces approches ainsi que leurs caractéristiques respectives.
Pour les illustrer, nous complétons l'exemple décrit ici.
Cependant, nous nous focalisons uniquement sur les messages envoyés par les noeuds et n'évoquons seulement les métadonnées introduites par chaque modèle de synchronisation, par soucis de clarté et de simplicité.
