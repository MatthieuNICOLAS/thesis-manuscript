Pour résumer, la couche de livraison des opérations associée à LogootSplit doit respecter le modèle de livraison suivant :

\begin{definition}[Modèle de livraison LogootSplit]
  Le modèle de livraison LogootSplit définit que :
  \begin{enumerate}
    \item Une opération doit être livrée exactement une fois à chaque noeud.
    \item Les opérations $\trm{ins}$ peuvent être délivrées dans un ordre quelconque.
    \item L'opération $\trm{rmv}(\trm{idIntervals})$ ne peut délivrée qu'après la livraison des opération d'insertions des éléments formant les $\trm{idIntervals}$.
  \end{enumerate}
\end{definition}

Il est à noter que \textcite{2021-these-vic} a récemment proposé dans ses travaux de thèse Dotted LogootSplit, un nouveau \ac{CRDT} pour Séquence dont la synchronisation est basée sur les différences d'états.
Inspiré de Logoot et LogootSplit, ce nouveau \ac{CRDT} associe une séquence à identifiants densément ordonnés à un contexte causal.
Le contexte causal est une structure de données permettant à Dotted LogootSplit de représenter et de maintenir efficacement les informations des modifications déjà intégrées à l'état courant.
Cette association permet à Dotted LogootSplit de fonctionner de manière autonome, sans imposer de contraintes particulières à la couche livraison autres que la livraison à terme.
