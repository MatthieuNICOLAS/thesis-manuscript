\subsubsection{Définition du modèle de livraison}

Pour résumer, la couche de livraison des opérations associée à LogootSplit doit respecter le modèle de livraison suivant :

\begin{definition}[Exactly-once + Causal remove]
  \label{def:ls-delivery-model}
  Le modèle de livraison \emph{Exactly-once + Causal remove} définit les 3 règles suivantes sur la livraison des opérations :
  \begin{enumerate}
    \item Une opération doit être délivrée à l'ensemble des noeuds à terme,
    \item Une opération doit être délivrée qu'une seule et unique fois aux noeuds,
    \item Une opération \emph{remove} doit être délivrée à un noeud une fois que les opérations \emph{insert} des éléments concernés par la suppression ont été délivrées à ce dernier.
  \end{enumerate}
\end{definition}

Il est à noter que \textcite{2021-these-vic} a récemment proposé dans ses travaux de thèse Dotted LogootSplit, un nouveau Sequence \ac{CRDT} basée sur les différences.
Inspiré de Logoot et LogootSplit, ce nouveau \ac{CRDT} associe une séquence à identifiants densément ordonnés à un contexte causal.
Le contexte causal est une structure de données permettant à Dotted LogootSplit de représenter et de maintenir efficacement les informations des modifications déjà intégrées à l'état courant.
Cette association permet à Dotted LogootSplit de fonctionner de manière autonome, sans imposer de contraintes particulières à la couche livraison autres que la livraison à terme.
