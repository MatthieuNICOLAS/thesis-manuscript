La \emph{Séquence}, aussi appelée \emph{Liste}, est un type abstrait de données représentant une collection ordonnée et de taille dynamique d'éléments.
Dans une séquence, un même élément peut apparaître à de multiples reprises.
Chacune des occurrences de cet élément est alors considérée comme distincte.

Dans le cadre de ce manuscrit, nous travaillons sur des séquences de caractères.
Cette restriction du domaine se fait sans perte en généralité.
Nous illustrons par la \autoref{fig:seq} notre représentation des séquences.

\begin{figure}[!ht]

  \centering
  \resizebox{0.2\columnwidth}{!}{
    \begin{tikzpicture}
      \path
        node[letter, label={below:0}] {H}
        ++(0:\widthletter) node[letter, label={below:1}] {E}
        ++(0:\widthletter) node[letter, label={below:2}] {L}
        ++(0:\widthletter) node[letter, label={below:3}] {L}
        ++(0:\widthletter) node[letter, label={below:4}] {O};
    \end{tikzpicture}
  }
  \caption{Représentation de la séquence "HELLO"}
  \label{fig:seq}
\end{figure}

Dans la \autoref{fig:spec-seq}, nous présentons la spécification algébrique du type Séquence que nous utilisons.

\begin{figure}[!ht]

  \centering
  % \resizebox{0.35\columnwidth}{!}{
    \begin{tabular}{llll}
      $S \in \trm{Seq}\langle E \rangle$ & & & \\
      \\
      $\trm{emp}$ & : &                       & $\longrightarrow S$   \\
      \\
      $\trm{ins}$ & : & $S \times N \times E$ & $\longrightarrow S$   \\
      $\trm{rmv}$ & : & $S \times N$          & $\longrightarrow S$   \\
      \\
      $\trm{len}$ & : & $S$                   & $\longrightarrow N$   \\
      $\trm{rd}$  & : & $S$                   & $\longrightarrow \trm{Arr}\langle E \rangle$ \\
    \end{tabular}
  % }
  \caption{Spécification algébrique du type abstrait usuel Séquence}
  \label{fig:spec-seq}
\end{figure}

Celle-ci définit deux modifications :
\begin{enumerate}
  \item $\trm{ins}$, qui permet d'insérer un élément donné $e$ à un index donné $i$ dans une séquence $s$ de taille $m$.
    Cette modification renvoie une nouvelle séquence construite de la manière suivante :
    \begin{multline*}
      \forall s \in S, e \in E, i \in [0,m]~|~m = \trm{len}(s) - 1,
      s = \langle e_0, ..., e_{i-1}, e_{i}, ..., e_m \rangle~\cdot \\
      \trm{ins}(s,i,v) = \langle e_0, ..., e_{i-1}, e, e_{i}, ..., e_m  \rangle
    \end{multline*}
  \item $\trm{rmv}$, qui permet de retirer l'élément situé à l'index $i$ dans une séquence $s$ de taille $m$.
    Cette modification renvoie une nouvelle séquence construite de la manière suivante :
    \begin{multline*}
      \forall s \in S, e \in E, i \in [0,m]~|~m = \trm{len}(s) - 1,
      s = \langle e_0, ..., e_{i-1}, e_{i}, e{i+1}, ..., e_m \rangle~\cdot \\
      \trm{rmv}(s,i) = \langle e_0, ..., e_{i-1}, e_{i+1}, ..., e_m  \rangle
    \end{multline*}
\end{enumerate}
Les modifications définies dans \autoref{fig:spec-seq}, $\trm{ins}$ et $\trm{rmv}$, ne permettent respectivement que l'insertion ou la suppression d'un élément à la fois.
Cette simplification du type se fait cependant sans perte de généralité, la spécification pouvant être étendue pour insérer successivement plusieurs éléments à partir d'un index donné ou retirer plusieurs éléments consécutifs.

Cette spécification du type Séquence est une spécification séquentielle.
Les modifications sont définies pour être effectuées l'une après l'autre.
Si plusieurs noeuds répliquent une même séquence et la modifient en concurrence, l'intégration de leurs opérations respectives dans des ordres différents résulte en des états différents.
Nous illustrons ce point avec la \autoref{fig:seq-conflit}.

\begin{figure}[!ht]

  \centering
  \resizebox{\columnwidth}{!}{
    \begin{tikzpicture}

      \newcommand\initialstate[3]{
        \path
          #1
          ++#2
          ++(0:0.5) node[letter, label={#3:0}] {W}
          ++(0:\widthletter) node[letter, label={#3:1}] {R}
          ++(0:\widthletter) node[letter, label={#3:2}] {D}
      }

      \newcommand\inso[3]{
        \path
          #1
          ++#2
          ++(0:0.5) node[letter, label={#3:0}] {W}
          ++(0:\widthletter) node[letter, label={#3:1}] {O}
          ++(0:\widthletter) node[letter, label={#3:2}] {R}
          ++(0:\widthletter) node[letter, label={#3:3}] {D}
      }

      \newcommand\insl[3]{
        \path
          #1
          ++#2
          ++(0:0.5) node[letter, label={#3:0}] {W}
          ++(0:\widthletter) node[letter, label={#3:1}] {R}
          ++(0:\widthletter) node[letter, label={#3:2}] {L}
          ++(0:\widthletter) node[letter, label={#3:3}] {D}
      }

      \newcommand\finalstatea[3]{
        \path
          #1
          ++#2
          ++(0:0.5) node[letter, label={#3:0}] {W}
          ++(0:\widthletter) node[letter, label={#3:1}] {O}
          ++(0:\widthletter) node[letter, label={#3:2}] {L}
          ++(0:\widthletter) node[letter, label={#3:3}] {R}
          ++(0:\widthletter) node[letter, label={#3:4}] {D}
      }

      \newcommand\finalstateb[3]{
        \path
          #1
          ++#2
          ++(0:0.5) node[letter, label={#3:0}] {W}
          ++(0:\widthletter) node[letter, label={#3:1}] {O}
          ++(0:\widthletter) node[letter, label={#3:2}] {R}
          ++(0:\widthletter) node[letter, label={#3:3}] {L}
          ++(0:\widthletter) node[letter, label={#3:4}] {D}
      }

      \newcommand\offseta{ (90:0.7) }
      \newcommand\offsetb{ (270:0.7) }

      \path
          node {\textbf{A}}
          ++(0:0.5) node (a) {}
          +(0:30) node (a-end) {}
          +(0:2) node[point] (a-initial) {}
          +(0:12) node[point, label=-170:{$\trm{ins}(1, O)$}] (a-ins-o) {}
          +(0:20) node[point] (a-recv-ins-l) {}
          +(0:28) node (a-final) {};

      \initialstate{(a-initial)}{\offseta}{90};
      \inso{(a-ins-o)}{\offseta}{90};
      \finalstatea{(a-recv-ins-l)}{\offseta}{90};

      \draw[dotted] (a) -- (a-initial) (a-final) -- (a-end);
      \draw[->, thick] (a-initial) --  (a-ins-o) --  (a-recv-ins-l) -- (a-final);

      \path
          ++(270:3) node {\textbf{B}}
          ++(0:0.5) node (b) {}
          +(0:30) node (b-end) {}
          +(0:2) node[point] (b-initial) {}
          +(0:12) node[point, label=170:{$\trm{ins}(2, L)$}] (b-ins-l) {}
          +(0:20) node[point] (b-recv-ins-o) {}
          +(0:28) node (b-final) {};

      \initialstate{(b-initial)}{\offsetb}{-90};
      \insl{(b-ins-l)}{\offsetb}{-90};
      \finalstateb{(b-recv-ins-o)}{\offsetb}{-90};

      \draw[dotted] (b) -- (b-initial) (b-final) -- (b-end);
      \draw[->, thick] (b-initial) --  (b-ins-l) -- (b-recv-ins-o) -- (b-final);

      \draw[->, dashed, shorten >= 1] (a-ins-o) -- (b-recv-ins-o);
      \draw[->, dashed, shorten >= 1] (b-ins-l) -- (a-recv-ins-l);
    \end{tikzpicture}
  }
  \caption{Modifications concurrentes d'une séquence}
  \label{fig:seq-conflit}
\end{figure}

Dans cet exemple, deux noeuds A et B partagent et éditent collaborativement une même séquence.
Celle-ci correspond initialement à la chaîne de caractères "WRD".
Le noeud A insère le caractère "O" à l'index 1, obtenant ainsi la séquence "WORD".
En concurrence, le noeud B insère lui le caractère "L" à l'index 2 pour obtenir "WRLD".

Les deux noeuds diffusent ensuite leur opération respective puis intègre celle de leur pair.
Nous constatons alors une divergence.
En effet, l'intégration de la modification $\trm{ins}(2,L)$ par le noeud A ne produit pas l'effet escompté, \ie produire la chaîne "WORLD", mais la chaîne "WOLRD.

Cette divergence est dûe à la non-commutativité de la modification $\trm{ins}$ avec elle-même.
En effet, celle-ci se base sur un index pour déterminer où placer le nouvel élément.
Cependant, les index sont eux-mêmes modifiés par $\trm{ins}$.
Ainsi, l'intégration dans des ordres différents de modifications $\trm{ins}$ sur un même état initial résulte en des états différents.
Plus généralement, nous observons que chaque paire possible de modifications du type Séquence, \ie $\langle \trm{ins},\trm{ins} \rangle$, $\langle \trm{ins},\trm{del} \rangle$ et $\langle \trm{del},\trm{del} \rangle$, n'est pas commutative.

La non-commutativité des modifications du type Séquence fut l'objet de nombreux travaux de recherche dans le domaine de l'édition collaborative.
Pour résoudre ce problème, l'approche \ac{OT} \cite{1989-grove-ellis-gibbs, 1998-ot-issues-algorithms-achievements-sun} fut initialement proposée.
Cette approche propose de transformer une modification par rapport aux modifications concurrentes intégrées pour tenir compte de leur effet.
Elle se décompose en deux parties :
\begin{enumerate}
  \item Un algorithme de contrôle \cite{1996-adopted-ressel-nitsch-ruhland-gunzenhauser, 1996-reduce-sun-yang-zhang-chen, 2009-cot-sun}, qui définit par rapport à quelles modifications une nouvelle modification distante doit être transformée avant d'être intégrée à la copie.
  \item Des fonctions de transformations \mnnote{TODO: Ajouter refs}, qui définissent comment une modification doit être transformée par rapport à une autre modification pour tenir compte de son effet.
\end{enumerate}

Cependant, bien que de nombreuses fonctions de transformations pour le type Séquence ont été proposées, seule la correction des \acfp{TTF} \cite{2006-tombstone-transformation-functions-oster} a été éprouvée à notre connaissance.
\mnnote{TODO: Vérifier que c'est pas plutôt les seules fonctions de transformations qui sont correctes et compatibles avec un système P2P.}
De plus, les algorithmes de contrôle compatibles reposent sur une livraison causale des modifications, et donc l'utilisation de vecteurs d'horloges.
Cette approche est donc inadaptée aux systèmes \ac{P2P} dynamiques.

Néanmoins, une contribution importante de l'approche \ac{OT} fut la définition d'un modèle de cohérence que doivent respecter les systèmes d'édition collaboratif : le modèle \ac{CCI} \cite{1998-cci-sun}.

\begin{definition}[Modèle \ac{CCI}]
  Le modèle de cohérence \ac{CCI} définit qu'un système d'édition collaboratif doit respecter les critères suivants :
  \begin{subdefinition}[Convergence]
    Le critère de \emph{Convergence} indique que des noeuds ayant intégrés le même ensemble de modifications convergent à un état équivalent.
  \end{subdefinition}
  \begin{subdefinition}[Causality preservation]
    \label{def:causality}
    Le critère de \emph{Causality preservation} indique que si une modification $m_1$ précède une autre modification $m_2$ d'après la relation \emph{happens-before}, \ie $m_1 \rightarrow m_2$, $m_1$ doit être intégrée avant $m_2$ par les noeuds du système.
  \end{subdefinition}
  \begin{subdefinition}[Intention preservation]
    Le critère de \emph{Intention preservation} indique que l'intégration d'une modification par un noeud distant doit reproduire l'effet de la modification sur la copie du noeud d'origine, indépendamment des modifications concurrentes intégrées.
  \end{subdefinition}
\end{definition}

De manière similaire à \cite{2009-logoot-weiss}, nous ajoutons le critère de \emph{Scalability} à ces critères :

\begin{definition}[Scalability]
  \label{def:scalability}
  Le critère de \emph{Scalability} indique que le nombre de noeuds du système ne doit avoir qu'un impact sous-linéaire sur les complexités en temps, en espace et sur le nombre et la taille des messages.
\end{definition}

Nous constatons cependant que les critères \ref{def:causality} et \ref{def:scalability} peuvent être contraires.
En effet, pour respecter le modèle de cohérence causale, un système peut nécessiter une livraison causale des modifications, \eg un \ac{CRDT} synchronisé par opérations dont seules les opérations concurrentes sont commutatives.
La livraison causale implique un surcoût computationnel, en métadonnées et en taille des messages qui est fonction du nombre de participants du système \cite{1991-concerning-size-logical-clocks-charron-bost}.
Ainsi, dans le cadre de nos travaux, nous cherchons à nous affranchir du modèle de livraison causale des modifications, ce qui peut nécessiter de relaxer le modèle de cohérence causale.

C'est dans une optique similaire que fut proposé \cite{2006-woot-oster}, un modèle de séquence répliquée qui pose les fondations des \acp{CRDT}.
Depuis, plusieurs \acp{CRDT} pour Séquence furent définies \cite{ROH2011354, 2009-treedoc-preguica, 2009-logoot-weiss}.
Ces \acp{CRDT} peuvent être répartis en deux approches : l'approche à pierres tombales \cite{2006-woot-oster,ROH2011354} et l'approche à identifiants densément ordonnés \cite{2009-treedoc-preguica,2009-logoot-weiss}.
L'état d'une séquence pouvant croître de manière infinie, ces \acp{CRDT} sont synchronisés par opérations pour limiter la taille des messages diffusés.
À notre connaissance, seul \cite{2021-these-vic} propose un \ac{CRDT} pour Séquence synchronisé par différence d'états.

Dans la suite de cette section, nous présentons les différents \acp{CRDT} pour Séquence de la littérature.
