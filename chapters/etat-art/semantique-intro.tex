Plusieurs sémantiques peuvent être proposées pour résoudre les conflits.
Certaines de ces sémantiques ont comme avantage d'être générique, \ie applicable à l'ensemble des types de données.
En contrepartie, elles souffrent de cette même généricité, en ne permettant que des comportements simples en cas de conflits.

À l'inverse, la majorité des sémantiques proposées dans la littérature sont spécifiques à un type de données.
Elles visent ainsi à prendre plus finement en compte l'intention des modifications pour proposer des comportements plus précis.

Dans la suite de cette section, nous présentons ces sémantiques génériques ainsi que celles spécifiques à l'Ensemble et, à titre d'exemple, les illustrons à l'aide du scénario présenté dans la \autoref{fig:set-conflict}.
