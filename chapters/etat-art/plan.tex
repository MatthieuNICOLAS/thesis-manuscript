Dans ce chapitre, nous définissons le modèle du système que nous considérons (\autoref{sec:system-model}).
Puis nous présentons le fonctionnement de LogootSplit, le \acf{CRDT} pour le type Séquence qui sert de base pour nos travaux (\autoref{sec:logootsplit}).
Ensuite, nous présentons les approches proposées pour réduire le surcoût des \acp{CRDT} pour le type Séquence et identifions leurs limites (\autoref{sec:etat-art-core-nebula} et \autoref{sec:etat-art-lseq}).
Finalement, nous introduisons l'approche que nous proposons (\autoref{sec:etat-art-proposition}) pour répondre à notre première problématique de recherche \cf{sec:research-questions-rls}, que nous présentons en détails par la suite dans le \autoref{chap:rls}.

Néanmoins, afin d'offrir une vision plus globale de notre domaine de recherche, nous complétons notre état de l'art de plusieurs points.
Dans la \autoref{sec:etat-art-crdts-intro}, nous rappelons la notion de \acp{CRDT}, \ie de types de données répliquées sans conflits.
Ce rappel se compose d'une section présentant la notion de sémantique pour un mécanisme de résolution de conflits automatiques (\autoref{sec:etat-art-semantique}) et d'une section présentant les différents modèles de synchronisation pour \acp{CRDT} définis dans la littérature, \ie la synchronisation par états, la synchronisation par opérations et la synchronisation par différences d'états (\autoref{sec:etat-art-synchro}).
À notre connaissance, nous présentons une des études les plus complètes comparant ces modèles de synchronisation en guise de synthèse de cette même section.

De manière similaire, nous rappelons les différents \acp{CRDT} pour le type Séquence définis dans la littérature dans la \autoref{sec:seq-crdts}.
Ce rappel prend la forme d'un historique des \acp{CRDT} pour le type Séquence, catégorisés en fonction de l'approche sur laquelle se base leur mécanisme de résolution de conflits, \ie l'approche à pierres tombales ou l'approche à identifiants densément ordonnés.
De nouveau, ce rappel aboutit à notre connaissance à l'une des études les plus précises comparant ces deux approches (\autoref{sec:seq-crdts-synth}).
