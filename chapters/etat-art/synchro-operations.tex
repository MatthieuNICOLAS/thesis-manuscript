L'approche de la synchronisation par opérations propose quant à elle que les noeuds diffusent leurs modifications sous la forme d'opérations.
Pour chaque modification possible, les \acp{CRDT} synchronisés par opérations doivent définir deux fonctions : \texttt{prepare} et \texttt{effect} \cite{baquero2017pure}.

La fonction \texttt{prepare} a pour but de générer une opération correspondant à la modification effectuée, et commutative avec les potentielles opérations concurrentes.
Cette fonction prend en paramètres la modification ainsi que ses paramètres, et l'état courant du noeud.
Cette fonction n'a pas d'effet de bord, \ie ne modifie pas l'état courant, et génère en retour l'opération à diffuser à l'ensemble des noeuds.

Une opération est un message.
Son rôle est d'encoder la modification sous la forme d'un ou plusieurs éléments irréductibles du sup-demi-treillis.

\begin{definition}[Élément irréductible]
  \label{def:irreducible-element}
  Un élément irréductible d'un sup-demi-treillis est un élément atomique de ce dernier.
  Il ne peut être obtenu par la fusion d'autres états.
\end{definition}

Il est à noter que pour certains \acp{CRDT}, leurs modifications s'avèrent indépendantes de l'état courant (\eg l'Ensemble LWW).
Pour ces \acp{CRDT} synchronisés par opérations, les modifications estampillées avec leur information de causalité correspondent à des éléments irréductibles, \ie à des opérations.
La fonction \texttt{prepare} peut donc être omise pour cette sous-catégorie de \acp{CRDT} synchronisés par opérations.
Ces \acp{CRDT} sont nommés \acp{CRDT} purs synchronisés par opérations \cite{baquero2017pure}.

La fonction \texttt{effect} permet quant à elle d'intégrer les effets d'une opération générée ou reçue.
Elle prend en paramètre l'état courant et l'opération, et retourne un nouvel état.
Ce nouvel état correspond à la \ac{LUB} entre l'état courant et le ou les éléments irréductibles encodés par l'opération.

La diffusion des modifications par le biais d'opérations présentent plusieurs avantages.
Tout d'abord, la taille des opérations est généralement fixe et inférieure à la taille de l'état complet du \ac{CRDT}, puisque les opérations servent à encoder un de ses éléments irréductibles.
Ensuite, l'expressivité des opérations permet de proposer plus simplement des algorithmes efficaces pour leur intégration par rapport aux modifications équivalentes dans les \acp{CRDT} synchronisés par états.
Par exemple, la suppression d'un élément dans un Ensemble se traduit en une opération de manière presque littérale, tandis que pour les \acp{CRDT} synchronisés par états, c'est l'absence de l'élément dans l'état qui va rendre compte de la suppression effectuée.
Ces avantages rendent possible la diffusion et l'intégration une à une des modifications et rendent ainsi plus adaptés les \acp{CRDT} synchronisés par opérations pour construire des systèmes temps réels.

Il est à noter que la seule contrainte imposée aux \acp{CRDT} synchronisés par opérations est que leurs opérations concurrentes soient commutatives \cite{shapiro_2011_crdt}.
Ainsi, il n'existe aucune contrainte sur la commutativité des opérations liées causalement.
De la même manière, aucune contrainte n'est définie sur l'idempotence des opérations.
Ces libertés impliquent qu'il peut être nécessaire que les opérations soient livrées au \ac{CRDT} en respectant un ordre donné et en garantissant leur livraison en exactement une fois pour garantir la convergence \cite{2018-crdts-overview-preguica}.
Ainsi, un intergiciel chargé de la diffusion et de la livraison des opérations est usuellement associé aux \acp{CRDT} synchronisés par opérations pour respecter ces contraintes.
Il s'agit de la couche de livraison de messages que nous avons introduit dans le cadre de notre modèle du système \cf{sec:system-model}.

Généralement, les \acp{CRDT} synchronisés par opérations sont présentés dans la littérature comme nécessitant une livraison causale des opérations.

\begin{definition}[Modèle de livraison causale]
  \label{def:causal-delivery}
  Le modèle de livraison causale définit que, pour toute paire de messages $m_1$ et $m_2$ d'une exécution, si $m_1 \to m_2$, alors la couche de livraison de l'ensemble des noeuds doit livrer le message $m_1$ à l'application avant de livrer le message $m_2$.
\end{definition}

Ce modèle de livraison permet de respecter le modèle de cohérence causale.

Ce modèle de livraison introduit néanmoins plusieurs effets négatifs.
Tout d'abord, ce modèle peut provoquer un délai dans l'intégration des modifications.
En effet, la perte d'une opération par le réseau provoque la mise en attente de la livraison des opérations suivantes.
Les opérations mises en attente ne pourront en effet être livrées qu'une fois l'opération perdue re-diffusée et livrée.

De plus, il nécessite que des informations de causalité précises soient attachées à chaque opération.
Pour cela, les systèmes reposent généralement sur l'utilisation de vecteurs de versions \cite{1988-version-vector-mattern,1991-version-vector-fidge}.
Or, la taille de cette structure de données croît de manière linéaire avec le nombre de noeuds du système.
Les métadonnées de causalité peuvent ainsi représenter la majorité des données diffusées sur le réseau\footnote{
  La relation de causalité étant transitive, les opérations et leurs relations de causalité forment un DAG.
  \cite{1997-causal-barrier} propose d'ajouter en dépendances causales d'une opération seulement les opérations correspondant aux extremités du DAG au moment de sa génération.
  Ce mécanisme plus complexe permet de réduire la consommation réseau, mais induit un surcoût en calculs et en mémoire utilisée.
} \cite{enes2019}.
Cependant, nous observons que la livraison dans l'ordre causal de toutes les opérations n'est pas toujours nécessaire pour la convergence.
Par exemple, l'ordre d'intégration de deux opérations d'ajout d'éléments différents dans un Ensemble n'a aucun impact sur le résultat obtenu.
Nous pouvons alors nous affranchir du modèle de livraison causale pour accélérer la vitesse d'intégration des modifications et pour réduire les métadonnées envoyées.

% Une autre contrainte généralement associée aux \acp{CRDT} synchronisés par opérations est la nécessité d'une livraison en exactement un exemplaire de chaque opération.
% Cette contrainte dérive de :
% \begin{enumerate}
%   \item La potentielle non-idempotence des opérations.
%   \item La nécessité de la livraison de chaque opération pour la livraison causale.
% \end{enumerate}
% Toutefois, nous observons que des opérations peuvent être sémantiquement rendues obsolètes par d'autres opérations, \eg une opération d'ajout d'un élément dans un Ensemble est rendue obsolète par une opération de suppression ultérieure du même élément.
% Ainsi, l'intergiciel de livraison peut se contenter d'assurer une livraison en un exemplaire au plus des opérations non-obsolètes.
% Ce choix permet de réduire la consommation réseau en évitant la diffusion d'opérations désormais non-pertinentes.

Pour compenser la perte d'opérations par le réseau et ainsi garantir la livraison à terme des opérations, la couche de livraison des opérations doit mettre en place un mécanisme d'anti-entropie, \ie un mécanisme permettant de détecter et ré-échanger les messages perdus.
Plusieurs mécanismes de ce type ont été proposés dans la littérature \cite{1983-anti-entropy-vv, 2007-dynamo, 2015-approximate-hash-based-set-reconciliation, 2017-anti-entropy-without-merkle-trees} \mnnote{TODO: Ajouter refs Scuttlebutt si applicable à Op-based} et proposent des compromis variés entre complexité en temps, complexité spatiale et consommation réseau.

Nous illustrons le modèle de synchronisation par opérations à l'aide de la \autoref{fig:sync-model-op}.
Dans ce nouvel exemple, les noeuds diffusent les modifications qu'ils effectuent sous la forme d'opérations.
Nous considèrons que le \ac{CRDT} utilisé est un \ac{CRDT} pur synchronisé par opérations, \ie que les modifications et opérations sont confondues, et qu'il autorise une livraison dans le désordre des opérations $\trm{add}$.

\begin{figure}[!ht]

  \centering
  \resizebox{\columnwidth}{!}{
    \begin{tikzpicture}
      \path
          node {\textbf{A}}
          ++(0:0.5) node (a) {}
          +(0:26) node (a-end) {}
          +(0:2) node[point, label=above right:{$\set{a,e}$}] (a-initial) {}
          +(0:7) node[point, label=above right:{$\set{a,b,e}$}, label=-170:{$\trm{add}(b)$}] (a-add-b) {}
          +(0:11) node[point, label=above right:{$\set{a,b,c,e}$}, label=-170:{$\trm{add}(c)$}] (a-add-c) {}
          +(0:14) node[point, label=above right:{$\set{a,b,c,d,e}$}] (a-receives-add-d) {}
          +(0:21) node[point, label={[xshift=8pt]-10:{$\trm{add(b)}$}}] (a-receives-query-sync) {}
          +(0:24) node (a-final) {};

      \draw[dotted] (a) -- (a-initial) (a-final) -- (a-end);
      \draw[->, thick] (a-initial) --  (a-add-b) -- (a-add-c) -- (a-receives-add-d) -- (a-receives-query-sync) -- (a-final);

      \path
          ++(270:3) node {\textbf{B}}
          ++(0:0.5) node (b) {}
          +(0:26) node (b-end) {}
          +(0:2) node[point, label=below right:{$\set{a,e}$}] (b-initial) {}
          +(0:11) node[point, label=below right:{$\set{a,d,e}$}, label=170:{$\trm{add}(d)$}] (b-add-d) {}
          +(0:14) node[point, label=below right:{$\set{a,c,d,e}$}] (b-receives-add-c) {}
          +(0:18) node[point, label=170:{$\trm{query\ sync}$}] (b-sends-query-sync) {}
          +(0:24) node[point, label=below right:{$\set{a,b,c,d,e}$}] (b-final) {};

      \draw[dotted] (b) -- (b-initial) (b-final) -- (b-end);
      \draw[->, thick] (b-initial) --  (b-add-d) -- (b-receives-add-c) -- (b-sends-query-sync) -- (b-final);

      \draw[->, dashed, shorten >= 1] (a-add-c) -- (b-receives-add-c);
      \draw[->, dashed, shorten >= 1] (b-add-d) -- (a-receives-add-d);
      \draw[->, dashed, shorten >= 1] (b-sends-query-sync) -- (a-receives-query-sync);
      \draw[->, dashed, shorten >= 1] (a-receives-query-sync) -- (b-final);

      \path
        ++(270:1.5)
        ++(0:0.5)
        +(0:8.5) node[cross] (network-error) {};

      \draw[->, dashed, shorten >= 1] (a-add-b) -- (network-error);
    \end{tikzpicture}
  }
  \caption{Synchronisation des noeuds A et B en adoptant le modèle de synchronisation par opérations}
  \label{fig:sync-model-op}
\end{figure}

Le noeud A diffuse donc les opérations $\trm{add}(b)$ et $\trm{add}(c)$.
Il reçoit ensuite l'opération $\trm{add}(d)$ de B, qu'il intègre à sa copie.
Il obtient alors l'état $\set{a,b,c,d,e}$.

De son côté, le noeud B ne reçoit initialement pas l'opération $\trm{add}(b)$ suite à une perte de message.
Il génère et diffuse $\trm{add}(d)$ puis reçoit l'opération $\trm{add}(c)$.
Comme indiqué précédemment, nous considérons que la livraison causale des opérations $\trm{add}$ n'est pas obligatoire dans cet exemple, cette opération est alors intégrée sans attendre.
Le noeud B obtient alors l'état $\set{a,c,d,e}$.

Ensuite, le mécanisme d'anti-entropie du noeud B se déclenche.
Le noeud B envoie alors à A une demande de synchronisation contenant un résumé de son état, \eg son vecteur de versions.
À partir de cette donnée, le noeud A détermine que B n'a pas reçu l'opération $\trm{add}(a)$.
Il génère alors une réponse contenant cette opération et lui envoie.
À la réception de l'opération, le noeud B l'intègre.
Il obtient l'état $\set{a,b,c,d,e}$ et converge ainsi avec A.

Avant de conclure, nous noterons qu'il est nécessaire pour les noeuds de maintenir leur journal des opérations.
En effet, les noeuds l'utilisent pour renvoyer les opérations manquées lors de l'exécution du mécanisme d'anti-entropie évoqué ci-dessus.
Ceci se traduit par une augmentation perpétuelle des métadonnées des \acp{CRDT} synchronisés par opérations.
Pour y pallier, des travaux \cite{baquero2017pure, 2021-improving-reactivity-pure-op-based-crdts-bauwens} proposent de tronquer le journal des opérations pour en supprimer les opérations connues de tous.
Les noeuds reposent alors sur la notion de stabilité causale \cite{10.1007/978-3-662-43352-2_11} pour déterminer les opérations supprimables de manière sûre.

\begin{definition}[Stabilité causale]
  Une opération est stable causalement lorsqu'elle a été intégrée par l'ensemble des noeuds du système.
  Ainsi, toute opération future dépend causalement des opérations causalement stables, \ie les noeuds ne peuvent plus générer d'opérations concurrentes aux opérations causalement stables.
\end{definition}

Un mécanisme d'instantané \cite{1980-database-snapshots-adiba} doit néanmoins être associé au mécanisme de troncature du journal pour générer un état équivalent à la partie tronquée.
Ce mécanisme est en effet nécessaire pour permettre un nouveau noeud de rejoindre le système et d'obtenir l'état courant à partir de l'instantané et du journal tronqué.

Pour résumer, cette approche permet de mettre en place un système en composant un \ac{CRDT} synchronisé par opérations avec une couche de livraison des messages.
Mais comme illustré ci-dessus, chaque \ac{CRDT} synchronisé par opérations établit les propriétés de ses différentes opérations et délègue potentiellement des responsabilités à la couche de livraison.
Une partie de la complexité de cette approche réside ainsi dans l'ajustement du couple $\langle \trm{CRDT}, \trm{couche~livraison} \rangle$ pour régler finement et optimiser leur fonctionnement en tandem.
Des travaux \cite{baquero2017pure, 2021-improving-reactivity-pure-op-based-crdts-bauwens} ont proposé un patron de conception pour modéliser ces deux composants et leurs interactions.

Cependant, ce patron repose sur l'hypothèse d'une livraison causale des opération.
Nous avons toutefois montré que la livraison causale des opérations peut ne pas être nécessaire pour la convergence (\eg opérations d'ajout d'éléments dans un Ensemble).
Ce patron implique donc l'ajout de métadonnées de causalité aux opérations, \ie des vecteurs de version, de manière parfois superflue.
Ce patron n'est donc pas optimal.
\mnnote{TODO: Vérifier que c'est bien le cas dans \cite{2021-improving-reactivity-pure-op-based-crdts-bauwens}}
