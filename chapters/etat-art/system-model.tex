\label{sec:system-model}

Le système que nous considérons est un système \acf{P2P} à large échelle.
Il est composé d'un ensemble de noeuds dynamique.
En d'autres termes, un noeud peut rejoindre ou quitter le système à tout moment.

À un instant donné, un noeud est soit connecté, soit déconnecté.
Nous considérons possible qu'un noeud se déconnecte de manière définitive, sans indication au préalable.
Ainsi, du point de vue des autres noeuds du système, il est impossible de déterminer le statut d'un noeud déconnecté.
Ce dernier peut être déconnecté de manière temporaire ou définitive.
Toutefois, nous assimilons les noeuds déconnectés de manière définitive à des noeuds ayant quittés le système, ceux-ci ne participant plus au système.

Dans ce système, nous considérons comme confondus les noeuds et clients.
Un noeud correspond alors à un appareil d'un-e utilisateur-rice du système.
Un-e même utilisateur-rice peut prendre part au système au travers de différents appareils, nous considérons alors chaque appareil comme un noeud distinct.

Le système consiste en une application permettant de répliquer une donnée.
Chaque noeud du système possède en local une copie de la donnée.
Les noeuds peuvent consulter et éditer leur copie locale à tout moment, sans se coordonner entre eux.
Les modifications sont appliquées à la copie locale immédiatement et de manière atomique.
Les modifications sont ensuite transmises aux autres noeuds de manière asynchrone par le biais de messages, afin qu'ils puissent à leur tour intégrer les modifications à leur copie.
L'application garantit la convergence à terme des copies.

\begin{definition}[Convergence à terme]
    \label{def:eventual-consistency}
    La convergence à terme est une propriété de sûreté indiquant que l'ensemble des noeuds du système ayant intégrés le même ensemble de modifications obtiendront des états équivalents\footnotemark.
    \footnotetext{Nous considérons comme équivalents deux états pour lesquels chaque observateur du type de données renvoie un même résultat, \ie les deux états sont indifférenciables du point de vue des utilisateur-rices du système.}
\end{definition}

Les noeuds communiquent entre eux par l'intermédiaire d'un réseau non-fiable.
Les messages envoyés peuvent être perdus, ré-ordonnés et/ou dupliqués.
Le réseau est aussi sujet à des partitions, qui séparent les noeuds en des sous-groupes disjoints.
Aussi, nous considérons que les noeuds peuvent initier de leur propre chef des partitions réseau : des groupes de noeuds peuvent décider de travailler de manière isolée pendant une certaine durée, avant de se reconnecter au réseau.

Pour compenser les limitations du réseau, les noeuds reposent sur une couche de livraison de messages.
Cette couche permet de garantir un modèle de livraison donné des messages à l'application.
En fonction des garanties du modèle de livraison sélectionné, cette couche peut ré-ordonner les messages reçus avant de les livrer à l'application, dé-dupliquer les messages, et détecter et ré-échanger les messages perdus.
Nous considérons a minima que la couche de livraison garantit la livraison à terme des messages.

\begin{definition}[Livraison à terme]
    La livraison à terme est un modèle de livraison garantissant que l'ensemble des messages du système seront livrés à l'ensemble des noeuds du système à terme.
\end{definition}

Finalement, nous supposons que les noeuds du système sont honnêtes.
Les noeuds ne peuvent dévier du protocole de la couche de livraison des messages ou de l'application.
Les noeuds peuvent cependant rencontrer des défaillances.
Nous considérons que les noeuds disposent d'une mémoire durable et fiable.
Ainsi, nous considérons que les noeuds peuvent restaurer le dernier état valide, \ie pas en cours de modification, qu'il possédait juste avant la défaillance.
% Nous considérons qu'en cas de perte de données, les noeuds quittent et re-rejoignent le système sous une autre identité.
