\begin{itemize}
    \item S'agit d'une structure primordiale dans les systèmes distribués, dont pouvons difficilement nous passer.
      Utilisé notamment pour représenter le contexte causal de l'état d'un noeud, nécessaire pour :
      \begin{enumerate}
        \item Déterminer quelles opérations ont été observées (anti-entropie et couche de livraison)
        \item Déterminer quelles opérations ont observé les autres noeuds (stabilité causale)
        \item Préciser les dépendances causales d'un message
      \end{enumerate}
    \item Comme présenté précédemment, nous utilisons plusieurs vecteurs pour représenter des données dans l'application MUTE
    \item Notamment pour le vecteur de versions, utilisé pour respecter le modèle de livraison requis par le \ac{CRDT}
    \item Et pour la liste des collaborateur-rices, utilisé pour offrir des informations nécessaires à la conscience de groupe aux utilisateurs
    \item Ces vecteurs sont maintenus localement par chacun des noeuds et sont échangés de manière périodique
    \item Cependant, la taille de ces vecteurs croit de manière linéaire au nombre de noeuds impliqués dans la collaboration
    \item Les systèmes \ac{P2P} à large échelle sont sujets au \emph{churn}
    \item Dans le cadre d'un tel système, ces structures croissent de manière non-bornée
    \item Ceci pose un problème de performances, notamment d'un point de vue consommation en bande-passante
    \item Cependant, même si on observe un grand nombre de pairs différents dans le cadre d'une collaboration à large échelle
    \item Intuition est qu'une collaboration repose en fait sur un petit noyau de collaborateur-rices principaux
    \item Et que majorité des collaborateur-rices se connectent de manière éphèmère
    \item Serait intéressant de pouvoir réduire la taille des vecteurs en oubliant les collaborateur-rices éphèmères
    \item Dynamo\cite{2007-dynamo} tronque le vecteur de versions lorsqu'il dépasse une taille seuil
    \item Conduit alors à une perte d'informations
    \item Pour la liste des collaborateur-rices, approche peut être adoptée (pas forcément gênant de limiter à 100 la taille de la liste)
    \item Mais pour vecteur de versions, conduirait à une relivraison d'opérations déjà observées
    \item Approche donc pas applicable pour cette partie
    \item Autre approche possible est de réutiliser le système d'époque
    \item Idée serait de ACK un vecteur avec un changement d'époque
    \item Et de ne diffuser à partir de là que les différences
    \item Un mécanisme de transformation (une simple soustraction) permettrait d'obtenir le dot dans la nouvelle époque d'une opération concurrente au renommage
    \item Peut facilement mettre en place un mécanisme d'inversion du renommage (une simple addition) pour revenir à une époque précédente
    \item Et ainsi pouvoir circuler librement dans l'arbre des époques et gérer les opérations \emph{rename} concurrentes
    \item Serait intéressant d'étudier si on peut aller plus loin dans le cadre de cette structure de données et notamment rendre commutatives les opérations de renommage concurrentes
  \end{itemize}
