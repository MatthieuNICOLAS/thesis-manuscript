Pour privilégier leur disponibilité, latence et tolérance aux pannes, les systèmes distribués peuvent adopter le paradigme de la réplication optimiste \cite{2005-optimistic-replication-saito}.
Ce paradigme consiste à relaxer la cohérence de données entre les noeuds du système pour leur permettre de consulter et modifier leur copie locale sans se coordonner.
Leur copies peuvent alors temporairement diverger avant de converger de nouveau une fois les modifications de chacun propagées.
Cependant, cette approche nécessite l'emploi d'un mécanisme de résolution pour assurer la convergence même en cas de modifications concurrentes.
Pour cela, l'approche des \acp{CRDT} \cite{2007-crdt-shapiro,shapiro_2011_crdt} propose d'utiliser des types de données dont les modifications sont nativement commutatives.

Depuis la spécification des \acp{CRDT}, la littérature a proposé plusieurs de ces mécanismes résolution de conflits automatiques pour le type de données Séquence \cite{2006-woot-oster,ROH2011354,2009-treedoc-preguica,2009-logoot-weiss}.
Cependant, ces approches souffrent toutes d'un surcoût croissant de manière monotone.
Ce problème a été identifié par la communauté, et celle-ci a proposé pour y répondre des mécanismes permettant soit de réduire la croissance du surcoût \cite{lseq2013,lseq2017}, soit d'effectuer une \ac{GC} du surcoût \cite{ROH2011354,letia:hal-01248270,zawirski:hal-01248197}.
Nous avons cependant déterminé que ces mécanismes ne sont pas adaptés aux systèmes \ac{P2P} à large échelle souffrant de churn et utilisant des \acp{CRDT} pour Séquence à granularité variable.

Dans le cadre de cette thèse, nous avons donc souhaité proposer un nouveau mécanisme adapté à ce type de systèmes.
Pour cela, nous avons suivi l'approche proposée par \cite{letia:hal-01248270,zawirski:hal-01248197} : l'utilisation d'un mécanisme pour ré-assigner de nouveaux identifiants aux élements stockées dans la séquence.
Nous avons donc proposé un nouveau mécanisme appartenant à cette approche pour le \ac{CRDT} LogootSplit \cite{2013-logootsplit}.

Notre proposition prend la forme d'un nouvel \ac{CRDT} pour Séquence à granularité variable : RenamableLogootSplit.
Ce nouveau \ac{CRDT} associe à LogootSplit un nouveau type de modification, $\trm{ren}$, permettant de produire une nouvelle séquence équivalente à son état précédent.
Cette nouvelle modification tire profit de la granularité variable de la séquence pour produire un état de taille minimale : elle assigne à tous les éléments des identifiants de position issus d'un même intervalle.
Ceci nous permet de minimiser les métadonnées que la séquence doit stocker de manière effective.
De plus, le passage à une représentation interne minimale de la séquence nous permet d'améliorer le coût des modifications suivantes en termes de calculs.

Afin de gérer les opérations concurrentes aux opérations $\trm{ren}$, nous définissons pour ces dernières un algorithme de transformation.
Pour cela, nous définissons un mécanisme d'époques nous permettant d'identifier la concurrence entre opérations.
De plus, nous introduisons une relation d'ordre strict total, \emph{priority}, pour résoudre de manière déterministe le conflit provoqué par deux opérations $\trm{ren}$, \ie pour déterminer quelle opération $\trm{ren}$ privilégier.
Finalement, nous définissons deux algorithmes, \texttt{renameId} et \texttt{revertRenameId}, qui permettent de transformer les opérations concurrentes à une opération $\trm{ren}$ pour prendre en compte l'effet de cette dernière.
Ainsi, notre algorithme permet de détecter et de transformer les opérations concurrentes aux opérations $\trm{ren}$, sans nécessiter une coordination synchrone entre les noeuds.
Le surcoût induit par ce mécanisme, notamment en termes de calculs, est toutefois contrebalancé par l'amélioration précisée précédemment, \ie la réduction de la taille de la séquence.

Finalement, le mécanisme que nous proposons est partiellement générique : il peut être adapté à d'autres \acp{CRDT} pour Séquence à granularité variable, \eg un \ac{CRDT} pour Séquence appartenant à l'approche à pierres tombales.
Dans le cadre d'une telle démarche, nous pourrions réutiliser le système d'époques, la relation \emph{priority} et l'algorithme de contrôle qui identifie les transformations à effectuer.
Pour compléter une telle adaptation, nous devrions cependant concevoir de nouveaux algorithmes \texttt{renameId} et \texttt{revertRenameId} spécifiques et adaptés au \ac{CRDT} choisi.

Le mécanisme de renommage que nous présentons souffre néanmoins de plusieurs limites.
La première d'entre elles concerne ses performances.
En effet, notre évaluation expérimentale a mis en lumière le coût important en l'état de la modification $\trm{ren}$ par rapport aux autres modifications en termes de calculs \cf{sec:integration-time-rename}.
De plus, chaque opération $\trm{ren}$ comporte une représentation de l'ancien état qui doit être maintenue par les noeuds jusqu'à leur stabilité causale.
Le surcoût en métadonnées introduit par un ensemble d'opérations $\trm{ren}$ concurrentes peut donc s'avérer important, voire pénalisant \cf{sec:evaluation-metadata}.
Pour répondre à ces problèmes, nous identifions trois axes d'amélioration :
\begin{enumerate}
    \item La définition de stratégies de déclenchement du renommage efficaces.
        Le but de ces stratégies serait de déclencher le mécanisme de renommage de manière fréquente, de façon à garder son temps d'exécution acceptable, mais tout visant à minimiser la probabilité que les noeuds produisent des opérations $\trm{ren}$ concurrentes, de façon à minimiser le surcoût en métadonnées.
    \item La définition de relations \emph{priority} efficaces.
        Nous développons ce point dans la \autoref{sec:alternative-priority}.
    \item La proposition d'algorithmes de renommage efficaces.
        Cette amélioration peut prendre la forme de nouveaux algorithmes pour \texttt{renameId} et \texttt{revertRenameId} offrant une meilleure complexité en temps.
        Il peut aussi s'agir de la conception d'une nouvelle approche pour renommer l'état et gérer les modifications concurrentes, \eg un mécanisme de renommage basé sur le journal des opérations \cf{sec:log-based-rename-mechanism}.
\end{enumerate}

Une seconde limite de RenamableLogootSplit que nous identifions concerne son mécanisme de \ac{GC} des métadonnées introduites par le mécanisme de renommage.
En effet, pour fonctionner, ce dernier repose sur la stabilité causale des opérations $\trm{ren}$.
Pour rappel, la stabilité causale représente le contexte causal commun à l'ensemble des noeuds du système.
Pour le déterminer, chaque noeud doit récupérer le contexte causal de l'ensemble des noeuds du système.
Ainsi, l'utilisation de la stabilité causale comme pré-requis pour la \ac{GC} de métadonnées constitue une contrainte forte, voire prohibitive, dans les systèmes \ac{P2P} à large échelle sujet au churn.
En effet, un noeud du système déconnecté de manière définitive suffit pour empêcher la stabilité causale de progresser, son contexte causal étant alors indéterminé du point de vue des autres noeuds.
Il s'agit toutefois d'une limite récurrente des mécanismes de \ac{GC} distribués et asynchrones \cite{ROH2011354,baquero2017pure,2018-prunable-authenticated-log-vic}.
Nous présentons une piste de travail possible pour pallier ce problème dans la \autoref{sec:manual-merge}.
