Les applications collaboratives permettent à des utilisateur-rices de réaliser collaborativement une tâche.
Elles permettent à plusieurs utilisateur-rices de consulter la version actuelle du document, de la modifier et de partager leurs modifications avec les autres.
Ceci permet de mettre en place une réflexion de groupe, ce qui améliore la qualité du résultat produit \cite{2004-empirical-study-collaborative-writing,2005-internet-encyclopaedias-head-to-head}.

Cependant, les applications collaboratives sont historiquement des applications centralisées, \eg Google Docs \cite{gdocs}.
Ce type d'architecture induit des défauts d'un point de vue technique, \eg faible capacité de passage à l'échelle et faible tolérance aux pannes, mais aussi d'un point de vue utilisateur, \eg perte de la souveraineté des données et absence de garantie de pérennité.

Les travaux de l'équipe Coast s'inscrivent dans une mouvance souhaitant résoudre ces problèmes et qui a conduit à la définition d'un nouveau paradigme d'applications : les \acp{LFS} \cite{localfirstsoftware2019}.
Le but de ce paradigme est la conception d'applications collaboratives, \ac{P2P}, pérennes et rendant la souveraineté de leurs données aux utilisateur-rices.

Dans le cadre de cette démarche, l'équipe Coast développe depuis plusieurs années l'application \acf{MUTE}, un éditeur de texte web collaboratif \ac{P2P} temps réel chiffré de bout en bout.
Cette application sert à la fois de plateforme de démonstration et de recherche pour les travaux de l'équipe, mais aussi de \ac{PoC} pour les \acp{LFS}.

Dans le cadre de cette thèse, nous avons implémenté dans \ac{MUTE} nos travaux de recherche portant sur le nouvel \ac{CRDT} pour le type Séquence : RenamableLogootSplit.
\ac{MUTE} a aussi servi à l'équipe pour présenter ses travaux concernant l'authentification des utilisateur-rices dans un système \ac{P2P} \cite{2018-trusternity-long}.
Finalement, \ac{MUTE} nous a permis de nous d'étudier et/ou de présenter les travaux de recherche existants concernant :
\begin{enumerate}
    \item Les protocoles distribués d'appartenance au groupe \cite{swim2002}.
    \item Les mécanismes d'anti-entropie \cite{1983-anti-entropy-vv}.
    \item Les protocoles d'établissement de clés de chiffrement de groupe \cite{1995-burmester-desmedt}.
    \item Les protocoles d'établissement de topologies réseaux efficientes \cite{2018-spray-nedelec}.
    \item Les mécanismes de conscience de groupe.
\end{enumerate}

\ac{MUTE} offre donc, à notre connaissance, le tour d'horizon le plus complet des travaux de recherche permettant la conception d'applications \acp{LFS}.
Cependant, cela ne dispense pas \ac{MUTE} de souffrir de plusieurs limites.

Tout d'abord, l'environnement web implique un certain nombre de contraintes, notamment au niveau des technologies et protocoles disponibles.
Notamment, le protocole WebRTC repose sur l'utilisation de serveurs de signalisation, \ie de points de rendez-vous des pairs, et de serveurs de relai, \ie d'intermédiaires pour communiquer entre pairs lorsque les configurations de leur réseaux respectifs interdisent l'établissement d'une connection directe.
Ainsi, les applications \ac{P2P} web doivent soit déployer et maintenir leur propre infrastructure de serveurs, soit reposer sur une infrastructure existante, \eg celle proposée par OpenRelay \cite{openrelay}.
Afin de minimiser l'effort requis aux applications \ac{P2P} et la confiance exigée à leurs utilisateur-rices, nous devons supporter la mise en place d'une telle infrastructure transparente et pérenne.

Une autre limite de ce système que nous identifions concerne l'utilisabilité des systèmes \ac{P2P} de manière générale.
L'expérience vécue suivante constitue à notre avis un exemple éloquent des limites actuelles de l'application \ac{MUTE} dans ce domaine.
Après avoir rédigé une version initiale d'un document, nous avons envoyé le lien du document à notre collaborateur pour relecture et validation.
Lorsque notre collaborateur a souhaité accéder au document, celui-ci s'est retrouvé devant une page blanche : comme nous nous étions déconnecté du système entretemps, \ie plus aucun pair n'était disponible pour effectuer une synchronisation.
Notre collaborateur était donc dans l'incapacité de récupérer l'état et d'effectuer sa tâche.
Afin de pallier ce problème, une solution possible est de faire reposer \ac{MUTE} sur un réseau \ac{P2P} global, \eg le réseau de \ac{IPFS} \cite{ipfs}, et d'utiliser les pairs de ce dernier, potentiellement des pairs étrangers à l'application, comme pairs de stockage pour permettre une synchronisation future.
Cette solution limite ainsi le risque qu'un pair ne puisse récupérer l'état du document faute de pairs disponibles.
Cependant, elle nécessite de mettre en place un mécanisme de réplication de données additionnel.
Ce mécanisme de réplication sur des pairs additionnels doit cependant garantir qu'il n'introduit pas de vulnérabilités, \eg la possibilité pour les pairs de stockage selectionnés de reconstruire et consulter le document.
% \item Finalement, une dernière limite que nous identifions est la pérennité économique de ce type d'applications.
%     Selon nous, les systèmes \ac{P2P} chiffrés de bout en bout s'interdisent les modèles économiques dominants et acceptés par les organisations et utilisateur-rices, \ie la collecte et revente de données.
%     En effet,
%     , de par les propriétés qu'ils assurent, notamment la confidentialité des données .
%     \mnnote{TODO: Voir si on a des données sur les entreprises/organisations encourageant le chiffrement de bout-en-bout dans leurs outils internes, }
