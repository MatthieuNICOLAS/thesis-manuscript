\begin{itemize}
    \item La spécification récente des Delta-based CRDTs .
      Ce nouveau type de CRDTs se base sur celui des State-based CRDTs.
      Partage donc les mêmes pré-requis :
      \begin{itemize}
        \item États du type de données répliqué forment un sup-demi-treillis
        \item Modifications locales entraînent une inflation de l'état
        \item Possède une fonction de \texttt{merge}, permettant de fusionner deux états S et S', et qui
        \begin{itemize}
          \item Est associative, commutative et idempotente
          \item Retourne S", la \ac{LUB} de S et S' (\ie $\nexists S''' \cdot merge(S, S') < S''' < S''$)
        \end{itemize}
      \end{itemize}
      Et bénéficie de son principal avantage : synchronisation possible entre deux pairs en fusionnant leur états, peu importe le nombre de modifications les séparant.
    \item Spécificité des Delta-based CRDTs est de proposer une synchronisation par différence d'états.
      Plutôt que de diffuser l'entièreté de l'état pour permettre aux autres pairs de se mettre à jour, idée est de seulement transmettre la partie de l'état ayant été mise à jour.
      Correspond à un élément irréductible du sup-demi-treillis.
      Permet ainsi de mettre en place une synchronisation en temps réel de manière efficace.
      Et d'utiliser la synchronisation par fusion d'états complets pour compenser les défaillances du réseau
    \item Ainsi, ce nouveau type de CRDTs semble allier le meilleur des deux mondes :
      \begin{itemize}
        \item Absence de contrainte sur le réseau autre que la livraison à terme
        \item Propagation possible en temps réel des modifications
      \end{itemize}
      Semble donc être une solution universelle :
      \begin{itemize}
        \item Utilisable peu importe la fiabilité réseau à disposition
        \item Empreinte réseau du même ordre de grandeur qu'un Op-based CRDT
        \item Utilisable peu importe la fréquence de synchronisation désirée
      \end{itemize}
      Pose la question de l'intérêt des autres types de CRDTs.
    \item Delta-based CRDT est un State-based CRDT dont on a identifié les éléments irréductibles et qui utilise ces derniers pour la propagation des modifications plutôt que l'état complet.
      Famille des State-based CRDTs semble donc rendue obsolète par celle des Delta-based CRDTs.
      À confirmer.
    \item Les Op-based CRDTs proposent une spécification différente du type répliqué de leur équivalent Delta-based, généralement plus simple.
      À première vue, famille des Op-based CRDTs semble donc avoir la simplicité comme avantage par rapport à celle des Delta-based CRDTs.
      S'agit d'un paramètre difficilement mesurable et auquel on peut objecter si on considère qu'un Op-based CRDT s'accompagne d'une couche livraison de messages, qui cache sa part de complexité.
      Intéressant d'étudier si la spécification différente des Op-based CRDTs présente d'autres avantages par rapport aux Delta-based CRDTs : performances (temps d'intégration des modifications, délai de convergence...), fonctionnalités spécifiques (composition, undo...)
    \item But serait de fournir des guidelines sur la famille de CRDT à adopter en fonction du cas d'utilisation.
  \end{itemize}
