\begin{itemize}
    \item Absence d'un dataset réel et réutilisable sur les sessions d'édition collaborative
    \item Généralement, expériences utilisent données d'articles de Wikipédia \mnnote{TODO: Revoir références, mais me semble que c'est celui utilisé pour Logoot, LogootSplit et RGASplit entre autres}.
      Mais ces données correspondent à une exécution séquentielle, \ie aucune édition concurrente ne peut être réalisée avec le système de résolution de conflits de Wikipédia.
      \mnnote{TODO:
        Me semble que Kleppmann a aussi utilisé et mis à disposition ses traces correspondant à la rédaction d'un de ses articles.
        Mais que cet article n'était rédigé que par lui.
        Peu de chances de présence d'éditions concurrentes.
        À retrouver et vérifier.
      }
    \item Inspiré par expériences de Claudia, pourrait mener des sessions d'édition collaborative sur des outils orchestrés pour produire ce dataset
    \item Devrait rendre ce dataset agnostique de l'approche choisie pour la résolution automatique de conflits
    \item Absence de retours sur les collaborations à grande échelle
    \item Comment on collabore lorsque plusieurs centaines d'utilisateur-rices ?
  \end{itemize}
