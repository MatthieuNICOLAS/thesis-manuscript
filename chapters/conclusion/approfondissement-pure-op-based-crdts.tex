Plusieurs approches ont été proposées dans la littérature pour guider la conception de \acp{CRDT} :
\begin{enumerate}
    \item L'utilisation de la théorie des treillis pour la conception de \acp{CRDT} synchronisés par états et par différences d'états \cite{shapiro_2011_crdt,enes2019}.
    \item L'utilisation d'un framework \cite{baquero2017pure} pour la conception de \acp{CRDT} purs synchronisés par opérations.
\end{enumerate}

Malgré ses améliorations \cite{2020-flec-bauwens,2021-improving-reactivity-pure-op-based-crdts-bauwens}, le framework présenté dans \cite{baquero2017pure} souffre de plusieurs limitations, ce qui entrave la conception de nouveaux \acp{CRDT} synchronisés par opérations.

Tout d'abord, il ne permet que la conception de \acp{CRDT} purs synchronisés par opérations, \ie des \acp{CRDT} dont les modifications enrichies de leurs arguments et d'une estampille fournie par la couche de livraison des messages sont commutatives.
Cette contrainte limite à des types de données simples, \eg le Compteur ou l'Ensemble, les \acp{CRDT} pouvant être spécifiés et exclut des types de données plus complexes, \eg la Séquence ou le Graphe.

Une seconde limite de ce framework est qu'il repose sur le modèle de livraison causal des opérations.
Ce modèle induit l'ajout de données de causalité précises à chaque opération, sous la forme d'un vecteur de version \cite{1988-version-vector-mattern,1991-version-vector-fidge} ou d'une barrière causale \cite{1997-causal-barrier}.
Nous jugeons ce modèle trop coûteux pour les applications \acp{LFS} à large échelle.

Nous souhaitons donc proposer un nouveau framework pour la conception de \acp{CRDT} synchronisés par opérations répondant à ces limites.
Nos objectifs sont multiples.

Notre framework devrait permettre la conception de \acp{CRDT} non-purs.
Ce framework devrait aussi mettre en lumière la présence et le rôle de deux modèles de livraison dans les \acp{CRDT} synchronisés par opérations :
\begin{enumerate}
    \item Le modèle de livraison minimal requis par le \ac{CRDT} pour assurer la convergence forte à terme \cite{shapiro_2011_crdt}.
    \item Le modèle de livraison employé par le système qui utilise le \ac{CRDT}.
        Ce second modèle de livraison est une stratégie permettant au système de respecter un modèle de cohérence donné et régissant les règles de compaction de l'état.
        Il doit être égal ou plus contraint que modèle de livraison minimal du \ac{CRDT} et peut être amené à évoluer en fonction de l'état du système et de ses besoins.
        Par exemple, un système pourrait par défaut utiliser le modèle de livraison causale pour assurer le modèle de cohérence causal.
        Puis, lorsque le nombre de noeuds atteint un seuil donné et que le coût de la livraison causale devient trop élevé, le système pourrait passer au modèle de livraison FIFO pour assurer le modèle de cohérence PRAM afin de réduire les coûts en bande-passante.
\end{enumerate}
