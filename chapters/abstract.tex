\begin{ThesisAbstract}
    \vspace{-3cm}
    \begin{FrenchAbstract}
        Un système collaboratif permet à plusieurs utilisateur-rices de créer ensemble un contenu.
        Afin de supporter des collaborations impliquant des millions d'utilisateurs, ces systèmes adoptent une architecture décentralisée pour garantir leur haute disponibilité, tolérance aux pannes et capacité de passage à l'échelle.
        % Cependant, de part le rôle prédominant des serveurs dans leur fonctionnement, ces sytèmes échouent à garantir un autre ensemble de propriétés : confidentialité des données, souveraineté des données, pérennité et résistance à la censure.
        Cependant, ces sytèmes échouent à garantir la confidentialité des données, souveraineté des données, pérennité et résistance à la censure.
        % Pour répondre à ce problème, la littérature propose la conception d'applications \acf{LFS} : des applications collaboratives \acf{P2P} réléguant les serveurs à un simple rôle de support de la collaboration.
        Pour répondre à ce problème, la littérature propose la conception d'applications \acf{LFS} : des applications collaboratives \acf{P2P}.

        Une pierre angulaire des applications \ac{LFS} sont les \acfp{CRDT}.
        Il s'agit de nouvelles spécifications des types de données, tels que l'Ensemble ou la Séquence, permettant à un ensemble de noeuds de répliquer une donnée.
        Les \acp{CRDT} permettent aux noeuds de consulter et de modifier la donnée sans coordination préalable, et incorporent un mécanisme de résolution de conflits pour intégrer les modifications concurrentes.
        Cependant, les \acp{CRDT} pour le type Séquence souffrent d'une croissance monotone du surcoût de leur mécanisme de résolution de conflits.
        Pouvons-nous proposer un mécanisme de réduction du surcoût des \acp{CRDT} pour le type Séquence qui soit compatible avec les applications \ac{LFS} ?
        Dans cette thèse, nous proposons un nouveau \ac{CRDT} pour le type Séquence, RenamableLogootSplit.
        Ce \ac{CRDT} intègre un mécanisme de renommage qui minimise périodiquement le surcoût de son mécanisme de résolution de conflits ainsi qu'un mécanisme de résolution de conflits pour intégrer les modifications concurrentes à un renommage.
        Finalement, nous proposons un mécanisme de \acf{GC} qui supprime à terme le propre surcoût du mécanisme de renommage.

      % \KeyWords{CRDTs, édition collaborative en temps réel, cohérence à terme, optimisation mémoire, performance}
    \end{FrenchAbstract}
    \begin{EnglishAbstract}
        A collaborative system enables multiple users to work together to create content.
        To support collaborations involving millions of users, these systems adopt a decentralised architecture to ensure high availability, fault tolerance and scalability.
        However, these systems fail to guarantee the data confidentiality, data sovereignty, longevity and resistance to censorship.
        To address this problem, the literature proposes the design of \acf{LFS} applications: collaborative peer-to-peer applications.

        A cornerstone of LFS applications are \acfp{CRDT}.
        \acp{CRDT} are new specifications of data types, \eg Set or Sequence, enabling a set of nodes to replicate a data.
        \acp{CRDT} enable nodes to access and modify the data without prior coordination, and incorporate a conflict resolution mechanism to integrate concurrent modifications.
        However, Sequence \acp{CRDT} suffer from a monotonous growth in the overhead of their conflict resolution mechanism.
        Can we propose a mechanism for reducing the overhead of Sequence-type CRDTs that is compatible with LFS applications?
        In this thesis, we propose a novel \ac{CRDT} for the Sequence type, RenamableLogootSplit.
        This \ac{CRDT} embeds a renaming mechanism that periodically minimizes the overhead of its conflict resolution mechanism as well as a conflict resolution mechanism to integrate concurrent modifications to a rename.
        Finally, we propose a mechanism of \acf{GC} that eventually removes the own overhead of the renaming mechanism.
      % \KeyWords{CRDTs, real-time collaborative editing, eventual consistency, memory-wise optimisation, performance}
    \end{EnglishAbstract}
\end{ThesisAbstract}
