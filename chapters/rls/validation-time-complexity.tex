\label{sec:validation-time-complexity}

\subsubsection{Hypothèses}

Les complexités en temps des opérations de RenamableLogootSplit dépendent de plusieurs paramètres, \eg nombre d'identifiants et de blocs stockés au sein de la structure, taille des identifiants ou encore structures de données utilisées.
Afin d'établir les valeurs de complexité des différentes opérations, nous prenons les hypothèses suivantes vis-à-vis des paramètres.

Nous supposons que le nombre $n$ d'identifiants présents dans la séquence a tendance à croître, \ie que plus d'insertions sont effectuées que de suppressions.
Nous considérons que la taille des identifiants, qui elle croît avec le nombre d'insertions mais qui est réinitialisée à chaque renommage, devient négligeable par rapport au nombre d'identifiants.
Nous ne prenons donc pas en considération ce paramètre dans nos complexités et considérons que les manipulations d'identifiants (comparaison, génération) s'effectuent en temps constant.
Afin de simplifier les complexités, nous considérons que les \emph{anciens états} associés aux époques contiennent aussi $n$ identifiants.
Finalement, nous considérons que nous utilisons comme structures de données un arbre AVL pour représenter l'état interne de la séquence, des tableaux pour les \emph{anciens états} et une table de hachage pour l'\emph{arbre des époques}.

\subsubsection{Complexité en temps des opérations $\trm{insert}$ et $\trm{remove}$}

À partir de ces hypothèses, nous établissons les complexités en temps des opérations.
Nous distinguons pour chaque opération deux complexités : une complexité pour l'intégration de l'opération \emph{locale}, une pour l'intégration de l'opération \emph{distante}.

Nous désignons par intégration de l'opération \emph{locale} le processus d'application d'une modification issue du noeud lui-même sur son état courant, \eg l'insertion d'un nouvel élément à partir de son index et l'allocation de l'identifiant de position correspondant, et de génération de l'opération correspondante.
L'intégration de l'opération \emph{distante} correspond quant à elle au processus d'application d'une opération provenant d'un autre noeud, \eg l'insertion d'un nouvel élément à partir de son identifiant de position.

La complexité de l'intégration de l'opération $\trm{insert}$ locale est inchangée par rapport à celle obtenue pour LogootSplit.
Son intégration consiste toujours à déterminer entre quels identifiants se situe les nouveaux éléments insérés, à générer de nouveaux identifiants correspondants à l'ordre souhaité puis à insérer le bloc dans l'arbre AVL.
D'après \cite{2013-logootsplit}, nous obtenons donc une complexité de \bigO{\log{}b} pour cette opération locale, où $b$ représente le nombre de blocs dans la séquence.\\

La complexité de l'intégration de l'opération $\trm{insert}$ distante, elle, évolue par rapport à celle définie pour LogootSplit.
En effet, plusieurs étapes se rajoutent au processus d'intégration de l'opération notamment dans le cas où celle-ci provient d'une autre époque que l'époque courante.

Tout d'abord, il est nécessaire d'identifier l'époque \ac{LCA} entre l'époque de l'opération et l'époque courante.
L'algorithme correspondant consiste à déterminer la première intersection entre deux branches de l'\emph{arbre des époques}.
Cette étape peut être effectuée en \bigO{h}, où $h$ représente la hauteur de l'\emph{arbre des époques}.

L'obtention de l'époque \ac{LCA} entre l'époque de l'opération et l'époque courante permet de déterminer les $k$ renommages dont les effets doivent être retirés de l'opération et les $l$ renommages dont les effets doivent être intégrés à l'opération.
Le noeud intégrant l'opération procède ainsi aux $k$ inversions de renommages successives puis aux $l$ application de renommages, et ce pour tous les $s$ identifiants insérés par l'opération.

Pour retirer les effets des renommages à inverser, le noeud intégrant l'opération utilise \texttt{revertRenameId}.
Cet algorithme retourne pour un identifiant donné un nouvel identifiant correspondant à l'époque précédente.
Pour cela, \texttt{revertRenameId} utilise le prédécesseur et le successeur de l'identifiant donné dans l'\emph{ancien état} renommé.
Pour retrouver ces deux identifiants au sein de l'\emph{ancien état}, \texttt{revertRenameId} utilise l'offset du premier tuple de l'identifiant donné.
Par définition, cet élément correspond à l'index du prédecesseur de l'identifiant donné dans l'\emph{ancien état}.
Aucun parcours de l'\emph{ancien état} n'est nécessaire.
Le reste de \texttt{revertRenameId} consistant en des comparaisons et manipulations d'identifiants, nous obtenons que \texttt{revertRenameId} s'effectue en \bigO{1}.

Pour inclure les effets des renommages à appliquer, le noeud utilise ensuite \texttt{renameId}.
De manière similaire à \texttt{revertRenameId}, \texttt{renameId} génère pour un identifiant donné un nouvel identifiant équivalent à l'époque suivante en se basant sur son prédecesseur.
Cependant, il est nécessaire ici de faire une recherche pour déterminer le prédecesseur de l'identifiant donné dans l'\emph{ancien état}.
L'\emph{ancien état} étant un tableau trié d'identifiants, il est possible de procéder à une recherche dichotomique.
Cela permet de trouver le prédecesseur en \bigO{\log{}n}, où $n$ correspond ici au nombre d'identifiants composant l'\emph{ancien état}.
Comme pour \texttt{revertRenameId}, les instructions restantes consistent en des comparaisons et manipulations d'identifiants.
La complexité de \texttt{renameId} est donc de \bigO{\log{}n}.

Une fois les identifiants introduits par l'opération $\trm{insert}$ renommés pour l'époque courante, il ne reste plus qu'à les insérer dans la séquence.
Cette étape se réalise en \bigO{\log{}b} pour chaque identifiant, le temps nécessaire pour trouver son emplacement dans l'arbre AVL.\\

Ainsi, en reprenant l'ensemble des étapes composant l'intégration de l'opération $\trm{insert}$ distante, nous obtenons la complexité suivante : \bigO{h + s \cdot (k + l \cdot \log{}n + \log{}b)}.

Le procédé de l'intégration de l'opération $\trm{remove}$ étant similaire à celui de l'opération $\trm{insert}$, aussi bien en local qu'en distant, nous obtenons les mêmes complexités en temps.

\subsubsection{Complexité en temps de l'opération $\trm{rename}$}
\label{sec:validation-time-complexity-rename}

L'opération $\trm{rename}$ locale se décompose en 2 étapes :
\begin{enumerate}
  \item La génération de l'\emph{ancien état} à intégrer au message de l'opération \cf{def:rename-op}.
  \item Le remplacement de la séquence courante par une séquence équivalente, renommée.
\end{enumerate}
La première étape consiste à parcourir la séquence actuelle pour en extraire les intervalles d'identifiants.
Elle s'effectue donc en \bigO{b}.
La seconde étape consiste à instancier une nouvelle séquence vide, et à y insérer un bloc qui associe le contenu actuel de la séquence à l'intervalle d'identifiants \id{pos}{nodeId~nodeSeq}{0..n-1}, avec $\trm{pos}$ la position du premier tuple du premier identifiant de l'état, $\trm{nodeId}$ et $\trm{nodeSeq}$ l'identifiant et le numéro de séquence actuel du noeud et $\trm{n}$ la taille du contenu.
Cette seconde étape s'effectue en \bigO{1}.
L'opération $\trm{rename}$ locale a donc une complexité de \bigO{b}.\\

L'intégration de l'opération distante $\trm{rename}$ se décompose en les étapes suivantes :
\begin{enumerate}
  \item L'insertion de la nouvelle époque et de l'\emph{ancien état} associé dans l'\emph{arbre des époques}.
  \item La récupération des $n$ identifiants formant l'état courant.
  \item \label{item:distant-rename-op-3} Le calcul de l'époque \ac{LCA} entre l'époque courante et l'époque cible.
  \item L'identification des $k$ opérations $\trm{rename}$ à inverser et des $l$ opérations $\trm{rename}$ à jouer.
  \item Le renommage de chacun des identifiants à l'aide de \texttt{revertRenameId} et \texttt{renameId}.
  \item \label{item:distant-rename-op-6} L'insertion de chacun des identifiants renommés dans une nouvelle séquence.
\end{enumerate}
L'\emph{arbre des époques} étant représenté à l'aide d'une table de hachage, la première étape s'effectue en \bigO{1}.
La seconde étape nécessite elle de parcourir l'arbre AVL et de convertir chaque intervalle d'identifiants en liste d'identifiants, ce qui nécessite \bigO{n} instructions.

Les étapes \ref{item:distant-rename-op-3} à \ref{item:distant-rename-op-6} peuvent être effectuées en réutilisant pour chaque identifiant l'algorithme pour l'intégration d'opérations $\trm{insert}$ distantes analysé précédemment.
Ces étapes s'effectuent donc en \bigO{n \cdot (k + l \cdot \log{}n + \log{}b)}.

Nous obtenons donc une complexité en temps de \bigO{h + n \cdot (k + l \cdot \log{}n + \log{}b)} pour l'intégration de l'opération $\trm{rename}$ distante.\\

Nous pouvons néanmoins améliorer ce premier résultat.
Notamment, nous pouvons tirer parti des faits suivants :
\begin{enumerate}
  \item Le fonctionnement de \texttt{renameId} repose sur l'utilisation de l'identifant prédecesseur comme préfixe.
  \item Les identifiants de l'état courant et de l'\emph{ancien état} forment tous deux des listes triées.
\end{enumerate}
Ainsi, plutôt que d'effectuer une recherche dichotomique sur l'\emph{ancien état} pour trouver le prédecesseur de l'identifiant à renommer, nous pouvons parcourir les deux listes en parallèle.
Ceci nous permet de renommer l'intégralité des identifiants en un seul parcours de l'état courant et de l'\emph{ancien état}, \ie en \bigO{n} instructions.
Ensuite, plutôt que d'insérer les identifiants un à un dans la nouvelle séquence, nous pouvons recomposer au préalable les différents blocs en parcourant la liste des identifiants et en les aggrégeant au fur et à mesure.
Il ne reste plus qu'à constituer la nouvelle séquence à partir des blocs obtenus.
Ces actions s'effectuent respectivement en \bigO{n} et \bigO{b} instructions.\\

Ainsi, ces améliorations nous permettent d'obtenir une complexité en temps en \bigO{h + n \cdot (k + l) + b} pour le traitement de l'opération $\trm{rename}$ distante.

\subsubsection{Récapitulatif}

Nous récapitulons les complexités en temps présentées précédemment dans le \autoref{tab:time-complexity-operations}.

\begin{table}[!ht]
  \centering
  \caption{Complexité en temps des différentes opérations}
  \label{tab:time-complexity-operations}
  %\resizebox{0.5\columnwidth}{!}{
  \begin{tabular}{lcc}
      \toprule
      Type d'opération & \multicolumn{2}{c}{Complexité en temps} \\
      \cmidrule(lr){2-3}
        & Locale &   Distante \\
      \midrule
      $\trm{insert}$ & \bigO{\log{}b} & \bigO{h + s \cdot (k + l \cdot \log{}n + \log{}b)} \\
      $\trm{remove}$ & \bigO{\log{}b} & \bigO{h + s \cdot (k + l \cdot \log{}n + \log{}b)} \\
      \emph{naive rename} & \bigO{b} & \bigO{h + n \cdot (k + l \cdot \log{}n + \log{}b)} \\
      $\trm{rename}$ & \bigO{b} & \bigO{h + n \cdot (k + l) + b} \\
      \bottomrule
  \end{tabular}
  \caption*{$b$ : nombre de blocs, $n$ : nombre d'éléments de l'état courant et des \emph{anciens états}, $h$ : hauteur de l'\emph{arbre des époques}, $k$ : nombre de renommages à inverser, $l$ : nombre de renommages à appliquer, $s$ : nombre d'éléments insérés/supprimés par l'opération}
  %}
\end{table}

\subsubsection{Complexité en temps du mécanisme de \ac{GC} des anciens états obsolètes}

Pour compléter notre analyse théorique des performances de RenamableLogootSplit, nous proposons une analyse en complexité en temps du mécanisme présenté en \autoref{sec:gc-mechanism} qui permet de supprimer les époques devenues obsolètes et de récupérer la mémoire occupée par leur \emph{ancien état} respectif.

L'algorithme du mécanisme de récupération de la mémoire se compose des étapes suivantes.
Tout d'abord, il établit le vecteur de versions des opérations causalement stables.
Pour cela, chaque noeud doit maintenir une matrice des vecteurs de versions de tous les noeuds.
L'algorithme génère le vecteur de versions des opérations causalement stable en récupérant pour chaque noeud la valeur minimale qui y est associée dans la matrice des vecteurs de versions.
Cette étape correspond à fusionner $p$ vecteurs de versions contenant $p$ entrées, elle s'exécute donc en \bigO{p^2} instructions.

La seconde étape consiste à parcourir l'arbre des époques de manière inverse à l'ordre défini par la relation \emph{priority}.
Ce parcours s'effectue jusqu'à trouver l'époque maximale causalement stable, \ie la première époque pour laquelle l'opération $\trm{rename}$ associée est causalement stable.
Pour chaque époque parcourue, le mécanisme de récupération de mémoire calcule et stocke son chemin jusqu'à la racine.
Cette étape s'exécute donc en \bigO{e \cdot h}, avec $e$ le nombre d'époques composant l'arbre des époques et $h$ la hauteur de l'arbre.

À partir de ces chemins, le mécanisme calcule l'époque \ac{LCA}.
Pour ce faire, l'algorithme calcule de manière successive la dernière intersection entre le chemin de la racine jusqu'à l'époque \ac{LCA} courante et les chemins précédemment calculés.
L'époque \ac{LCA} est la dernière époque du chemin résultant.
Cette étape s'exécute aussi en \bigO{e \cdot h}.

L'algorithme peut alors calculer l'ensemble des \emph{époques requises}.
Pour cela, il parcourt les chemins calculés au cours de la seconde étape.
Pour chaque chemin, il ajoute les époques se trouvant après l'époque \ac{LCA} à l'ensemble des \emph{époques requises}.
De nouveau, cette étape s'exécute en \bigO{e \cdot h}.

Après avoir déterminé l'ensemble des \emph{époques requises}, le mécanisme peut supprimer les époques obsolètes.
Il parcourt l'arbre des époques et supprime toute époque qui n'appartient pas à cet ensemble.
Cette étape finale s'exécute en \bigO{e}.

Ainsi, nous obtenons que la complexité en temps du mécanisme de récupération de mémoire des époques est en \bigO{p^2 + e \cdot h}.
Nous récapitulons ce résultat dans la \autoref{tab:time-complexity-gc-mechanism-epochs}.

\begin{table}[!ht]
  \centering
  \caption{Complexité en temps du mécanisme de \ac{GC} des anciens états obsolètes}
  \label{tab:time-complexity-gc-mechanism-epochs}
  \resizebox{.7\columnwidth}{!}{
      \begin{tabular}{lc}
          \toprule
          Étape & Temps \\
          \midrule
          \emph{calculer le vecteur de versions des opérations causalement stables} & \bigO{p^2} \\
          \emph{calculer les chemins de la racine aux \emph{potentielles époques courantes}} & \bigO{e \cdot h} \\
          \emph{identifier le \ac{LCA}} & \bigO{e \cdot h} \\
          \emph{calculer l'ensembe des \emph{époques requises}} & \bigO{e \cdot h} \\
          \emph{supprimer les époques obsolètes} & \bigO{e} \\
          \midrule
          \emph{total} & \bigO{p^2 + e \cdot h} \\
          \bottomrule
      \end{tabular}
  }
  \caption*{$p$: nombre de noeuds (ou pairs) du système, $e$: nombre d'époques dans l'\emph{arbre des époques}, $h$: hauteur de l'\emph{arbre des époques}}
\end{table}

Malgré sa complexité en temps, le mécanisme de récupération de mémoire des époques devrait avoir un impact limité sur les performances de l'application.
En effet, ce mécanisme n'appartient pas au chemin critique de l'application, \ie l'intégration des modifications.
Il peut être déclenché occasionnellement, en tâche de fond.
Nous pouvons même viser des fenêtres spécifiques pour le déclencher, \eg pendant les périodes d'inactivité.
Ainsi, nous avons pas étudié plus en détails cette partie de RenamableLogootSplit dans le cadre de cette thèse.
Des améliorations de ce mécanisme doivent donc être possibles.
