\label{sec:gc-mechanism}

Les noeuds stockent les époques et les \emph{anciens états} correspondant pour transformer les identifiants d'une époque à l'autre.
Au fur et à mesure que le système progresse, certaines époques et métadonnées associées deviennent obsolètes puisque plus aucune opération ne peut être émise depuis ces époques.
Les noeuds peuvent alors supprimer ces époques.
Dans cette section, nous présentons un mécanisme permettant aux noeuds de déterminer les époques obsolètes.

Pour proposer un tel mécanisme, nous nous reposons sur la notion de \emph{stabilité causale des opérations} \cite{10.1007/978-3-662-43352-2_11}.
Une opération est causalement stable une fois qu'elle a été livrée à tous les noeuds.
Dans le contexte de l'opération $\trm{rename}$, cela implique que tous les noeuds ont progressé à l'époque introduite par cette opération ou à une époque plus prioritaire d'après la relation \emph{priority}.
À partir de ce constat, nous définissons les \emph{potentielles époques courantes} :

\begin{definition}[Potentielles époques courantes]
  L'ensemble des époques auxquelles les noeuds peuvent se trouver actuellement et à partir desquelles ils peuvent émettre des opérations, du point de vue du noeud courant.
  Il s'agit d'un sous-ensemble de l'ensemble des époques, composé de l'époque maximale introduite par une opération $\trm{rename}$ causalement stable et de toutes les époques plus prioritaire que cette dernière d'après la relation \emph{priority}.
\end{definition}

Les prochaines opérations livrées seront donc générées à partir de ces époques seulement\footnote{
    Nous considérons ici que l'ordre de livraison des opérations satisfait le modèle de livraison FIFO, \ie que les messages d'un noeud sont livrés aux autres noeuds dans le même ordre qu'ils ont été envoyés, et le modèle de livraison RenamableLogootSplit \cf{def:rls-delivery-model}.
    Ainsi, une opération de renommage ne peut être causalement stable pour un noeud que si ce dernier a reçu toutes les opérations ayant eu lieu avant d'après la relation \hb et toutes les opérations concurrentes.
}.
Pour traiter ces prochaines opérations, les noeuds doivent maintenir les chemins entre toutes les époques de l'ensemble des \emph{potentielles époques courantes}.
Nous appelons \emph{époques requises} l'ensemble des époques correspondant.

\begin{definition}[Époques requises]
  L'ensemble des époques qu'un noeud doit conserver pour traiter les potentielles prochaines opérations.
  Il s'agit de l'ensemble des époques qui forment les chemins entre chaque époque appartenant à l'ensemble des \emph{potentielles époques courantes} et leur \acf{LCA}.
\end{definition}

Il s'ensuit que toute époque qui n'appartient pas à l'ensemble des \emph{époques requises} peut être retirée par les noeuds.
La \autoref{fig:GC-epochs} illustre un cas d'utilisation du mécanisme de récupération de mémoire proposé.

\begin{figure}[!ht]
  \subfloat[Exécution d'opérations $\trm{rename}$]{
      \begin{minipage}{\linewidth}
          \centering
          \resizebox{0.7\columnwidth}{!}{
          \begin{tikzpicture}
            \path
                node {\textbf{A}}
                ++(0:0.5) node (a) {}
                +(0:14) node (a-end) {}
                +(0:1) node (a-initial) {}
                +(0:3) node[point, label=above:{$\trm{ren}(\betterepoch{0}, A,2)$}] (a-ren-A2) {}
                +(0:7) node[point, label=above:{$\trm{ren}(\betterepoch{A2}, A,8)$}] (a-ren-a8) {}
                +(0:9) node[point] (a-recv-ren-B3) {}
                +(0:11) node[point, label=above:{$\trm{ren}(\betterepoch{B3}, A,9)$}] (a-ren-a9) {}
                +(0:13) node (a-final) {};

            \draw[dotted] (a) -- (a-initial) (a-final) -- (a-end);
            \draw[->, thick] (a-initial) --  (a-ren-A2) --  (a-ren-a8) -- (a-recv-ren-B3) --  (a-ren-a9) -- (a-final);

            \path
                ++(270:2) node {\textbf{B}}
                ++(0:0.5) node (b) {}
                +(0:14) node (b-end) {}
                +(0:1) node (b-initial) {}
                +(0:3) node[point, label=below:{$\trm{ren}(\betterepoch{0},B,3)$}] (b-ren-B3) {}
                +(0:5) node[point] (b-recv-ren-A2) {}
                +(0:11) node[point, label=below:{$\trm{ren}(\betterepoch{B3},B,7)$}] (b-ren-b7) {}
                +(0:13) node (b-final) {};

            \draw[dotted] (b) -- (b-initial) (b-final) -- (b-end);
            \draw[->, thick] (b-initial) -- (b-ren-B3) -- (b-recv-ren-A2) -- (b-ren-b7) -- (b-final);

            \draw[->, dashed, shorten >= 1] (a-ren-A2) -- (b-recv-ren-A2);
            \draw[->, dashed, shorten >= 1] (b-ren-B3) -- (a-recv-ren-B3);
          \end{tikzpicture}}
          \label{fig:GC-execution}
      \end{minipage}}
  \hfil
  \subfloat[\emph{Arbres des époques} respectifs avec les ensembles \emph{potentielles époques courantes} et \emph{époques requises} illustrés]{
      \begin{minipage}{\linewidth}
          \centering
          \begin{tikzpicture}[scale=0.8,every node/.style={scale=0.8}]
              \path
                  node {\textbf{A}}
                  ++(270:1) node[causalop] (Ae0) {\epoch{0}}
                  ++(225:1.5) node[op] (AeA2) {\epoch{A2}}
                  ++(270:1.5) node[op] (AeA8) {\epoch{A8}};
              \path
                  ++(270:1)
                  ++(315:1.5) node[causalop] (AeB3) {\epoch{B3}}
                      node[outer sep=12pt] (A-tl-potential-epochs) {}
                      node[outer sep=15pt] (A-tl-required-epochs) {}
                  ++(270:1.5) node[op, red] (AeA9) {\epoch{A9}}
                      node[outer sep=12pt] (A-br-potential-epochs) {}
                      node[outer sep=15pt] (A-br-required-epochs) {};

              \draw[dashed] (A-tl-potential-epochs.north west) rectangle (A-br-potential-epochs.south east);
              \draw[dotted, thick, darkgreen] (A-tl-required-epochs.north west) rectangle (A-br-required-epochs.south east);

              \path
                  ++(0:4) node {\textbf{B}}
                  ++(270:1) node[causalop] (Be0) {\epoch{0}}
                  ++(315:1.5) node[op] (BeB3) {\epoch{B3}}
                  ++(270:1.5) node[op, red] (BeB7) {\epoch{B7}}
                      node[outer sep=12pt] (B-br-potential-epochs) {}
                      node[outer sep=15pt] (B-br-required-epochs) {};
              \path
                  ++(0:4)
                  ++(270:1)
                  ++(225:1.5) node[causalop] (BeA2) {\epoch{A2}}
                      node[outer sep=12pt] (B-tl-potential-epochs) {};

              \draw[dashed] (B-tl-potential-epochs.north west) rectangle (B-br-potential-epochs.south east);
              \draw[dotted, thick, darkgreen] (B-tl-potential-epochs.north west)[xshift=-3pt, yshift=33pt] rectangle (B-br-required-epochs.south east);

              \draw[thick, ->] (Ae0) edge (AeA2) (AeA2) edge (AeA8) (Ae0) edge (AeB3) (AeB3) edge (AeA9);
              \draw[thick, ->] (Be0) edge (BeB3) (BeB3) edge (BeB7) (Be0) edge (BeA2);
              \draw[->, dashed, thick, red] (AeA9.135) -- (AeB3.225) (AeB3.180) -- (AeA8.45) -- (AeA2.315) (AeA2.0) -- (Ae0.270);
              \draw[->, dashed, thick, red] (BeB7.135) -- (BeB3.225) (BeB3.180) -- (BeA2.0) -- (Be0.270);

          \end{tikzpicture}
          \label{fig:GC-epoch-trees}
      \end{minipage}}
  \caption{Suppression des époques obsolètes et récupération de la mémoire des \emph{anciens états} associés}
  \label{fig:GC-epochs}
\end{figure}

Dans la \autoref{fig:GC-execution}, nous représentons une exécution au cours de laquelle deux noeuds A et B génère respectivement plusieurs opérations $\trm{rename}$.
Dans la \autoref{fig:GC-epoch-trees}, nous représentons les \emph{arbre des époques} respectifs de chaque noeud.
Les époques introduites par des opérations $\trm{rename}$ causalement stables sont representées en utilisant des doubles cercles.
L'ensemble des \emph{potentielles époques courantes} est montré sous la forme d'un rectangle noir tireté, tandis que l'ensemble des \emph{époques requises} est représenté par un rectangle vert pointillé.

Le noeud A génère tout d'abord une opération $\trm{rename}$ vers \epoch{A2} et ensuite une opération $\trm{rename}$ vers \epoch{A8}.
Il reçoit ensuite une opération $\trm{rename}$ du noeud B qui introduit \epoch{B3}.
Puisque \epoch{B3} est plus grand que son époque courante actuelle (\epoch{e0}\epoch{A2}\epoch{A8} < \epoch{e0}\epoch{B3}), le noeud A la sélectionne comme sa nouvelle époque cible et procède au renommage de son état en conséquence.
Finalement, le noeud A génère une troisième opération $\trm{rename}$ vers \epoch{A9}.

De manière concurrente, le noeud B génère l'opération $\trm{rename}$ vers \epoch{B3}.
Il reçoit ensuite l'opération $\trm{rename}$ vers \epoch{A2} du noeud A.
Cependant, le noeud B conserve \epoch{B3} comme époque courante (puisque \epoch{e0}\epoch{A2} < \epoch{e0}\epoch{B3}).
Après, le noeud B génère une autre opération $\trm{rename}$ vers \epoch{B7}.

À la livraison de l'opération $\trm{rename}$ introduisant l'époque \epoch{B3} au noeud A, cette opération devient causalement stable.
À partir de ce point, le noeud A sait que tous les noeuds ont progressé jusqu'à cette époque ou une plus grande d'après la relation \emph{priority}.
Les époques \epoch{B3} et \epoch{A9} forment donc l'ensemble des \emph{potentielles époques courantes} et les noeuds peuvent seulement émettre des opérations depuis ces époques ou une de leur descendante encore inconnue.
Le noeud A procède ensuite au calcul de l'ensemble des \emph{époques requises}.
Pour ce faire, il détermine le \ac{LCA} des \emph{potentielles époques courantes} : \epoch{B3}.
Il génère ensuite l'ensemble des \emph{époques requises} en ajoutant toutes les époques formant les chemins entre \epoch{B3} et les \emph{potentielles époques courantes}.
Les époques \epoch{B3} et \epoch{A9} forment donc l'ensemble des \emph{époques requises}.
Le noeud A déduit que les époques \epoch{0}, \epoch{A2} et \epoch{A8} peuvent être supprimées de manière sûre.

À l'inverse, la livraison de l'opération $\trm{rename}$ vers \epoch{A2} au noeud B ne lui permet pas de supprimer la moindre métadonnée.
À partir de ses connaissances, le noeud B calcule que \epoch{A2}, \epoch{B3} et \epoch{B7} forment l'ensemble des \emph{potentielles époques courantes}.
De cette information, le noeud B détermine que ces époques et leur \ac{LCA} forment l'ensemble des \emph{époques requises}.
Toute époque connue appartient donc à l'ensemble des \emph{époques requises}, empêchant leur suppression.\\

À terme, une fois que le système devient inactif, les noeuds atteignent la même époque et l'opération $\trm{rename}$ correspondante devient causalement stable.
Les noeuds peuvent alors supprimer toutes les autres époques et métadonnées associées, supprimant ainsi le surcoût mémoire introduit par le mécanisme de renommage.\\

Notons que le mécanisme de récupération de mémoire peut être simplifié dans les systèmes empêchant les opérations $\trm{rename}$ concurrentes.
Puisque les époques forment une chaîne dans de tels systèmes, la dernière époque introduite par une opération $\trm{rename}$ causalement stable devient le \ac{LCA} des \emph{potentielles époques courantes}.
Il s'ensuit que cette époque et ses descendantes forment l'ensemble des \emph{époques requises}.
Les noeuds n'ont donc besoin que de suivre les opérations $\trm{rename}$ causalement stables pour déterminer quelles époques peuvent être supprimées dans les systèmes sans opérations $\trm{rename}$ concurrentes.\\

Pour déterminer qu'une opération $\trm{rename}$ donnée est causalement stable, les noeuds doivent être conscients des autres et de leur avancement.
Un protocole de gestion de groupe tel que \cite{swim2002,lifeguard2018} est donc requis.

La stabilité causale peut prendre un certain temps à être atteinte.
En attendant, les noeuds peuvent néanmoins décharger les anciens états sur le disque dur puisqu'ils ne sont seulement nécessaires que pour traiter les opérations concurrentes aux opérations $\trm{rename}$.
Nous approfondissons ce sujet dans la \autoref{sec:offloading-former-states}.
