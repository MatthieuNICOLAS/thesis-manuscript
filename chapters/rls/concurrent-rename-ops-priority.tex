\label{sec:priority}

Pour que chaque noeud sélectionne la même époque cible de manière non-coordonnée, nous définissons la relation de priorité sur les époques \lepoch.

\begin{definition}[Relation \emph{priority} \lepoch]
  \label{def:priority-relation}
  La relation \emph{priority} \lepoch est un ordre strict total sur l'ensemble des époques.
  Elle permet aux noeuds de comparer n'importe quelle paire d'époques.
\end{definition}

En utilisant la relation \emph{priority}, nous définissons l'époque cible de la manière suivante :

\begin{definition}[Époque cible]
  L'époque cible est l'époque de l'ensemble des époques vers laquelle les noeuds doivent progresser.
  Les noeuds sélectionnent comme époque cible l'époque maximale d'après l'ordre établi par \emph{priority}.
\end{definition}

Pour définir la relation \emph{priority}, nous pouvons choisir entre plusieurs stratégies.
Dans le cadre de ce travail, nous utilisons l'ordre lexicographique sur le chemin des époques dans l'\emph{arbre des époques}.
La \autoref{fig:priority-example} fournit un exemple.

\begin{figure}[!ht]
  \subfloat[Exécution d'opérations \emph{rename} concurrentes]{
      \begin{minipage}{\linewidth}
          \centering
          \begin{tikzpicture}[scale=0.8,every node/.style={scale=0.8}]
              \path
                  node {\textbf{A}}
                  ++(0:1) node (a0) {}
                  ++(0:1) node[point, label=above:{rename to \epoch{A1}}] (a1) {}
                  ++(0:5) node (a-end) {};

              \draw[->, thick] (a0) --  (a1) -- (a-end);

              \path
                  ++(270:1.5) node {\textbf{B}}
                  ++(0:1) node (b0) {}
                  ++(0:1) node[point, label=below:{rename to \epoch{B2}}] (b2) {}
                  ++(0:4) node[point, label=above:{rename to \epoch{B7}}] (b7) {}
                  ++(0:1) node (b-end) {};

              \draw[->, thick] (b0) -- (b2) -- (b7) -- (b-end);

              \path
                  ++(270:3) node {\textbf{C}}
                  ++(0:1) node (c0) {}
                  ++(0:3) node (c-receives-a1) {}
                  ++(0:1) node[point, label=below:{rename to \epoch{C6}}] (c6) {}
                  ++(0:2) node (c-end) {};

              \draw[->, thick] (c0) -- (c6) -- (c-end);

              \draw[->, dashed, shorten >= 1] (a1) -- (c-receives-a1);
          \end{tikzpicture}
          \label{fig:priority-execution}
      \end{minipage}}
  \hfil
  \subfloat[\emph{Arbre des époques} final correspondant avec la relation \emph{priority} illustrée]{
      \begin{minipage}{\linewidth}
          \centering
          \begin{tikzpicture}[scale=0.8,every node/.style={scale=0.8}]
              \path
                  node[op] (e0) {\epoch{0}}
                  ++(225:1.5) node[op] (eA1) {\epoch{A1}}
                  ++(270:1.5) node[op] (eC6) {\epoch{C6}};
              \path
                  ++(315:1.5) node[op] (eB2) {\epoch{B2}}
                  ++(270:1.5) node[op, red] (eB7) {\epoch{B7}};

              \draw[->, thick] (e0) edge (eA1) (eA1) edge (eC6) (e0) edge (eB2) (eB2) edge (eB7);
              \draw[->, dashed, thick, red] (eB7.135) -- (eB2.225) (eB2.180) -- (eC6.45) -- (eA1.315) (eA1.0) -- (e0.270);
          \end{tikzpicture}
          \label{fig:priority-epoch-tree}
      \end{minipage}}
  \caption{Sélectionner l'époque cible d'une exécution d'opérations \emph{rename} concurrentes}
  \label{fig:priority-example}
\end{figure}

La \autoref{fig:priority-execution} décrit une exécution dans laquelle trois noeuds A, B et C générent plusieurs opérations avant de se synchroniser à terme.
Comme seules les opérations \emph{rename} sont pertinentes pour le problème qui nous occupe, nous représentons seulement ces opérations dans cette figure.
Initialement, le noeud A génère une opération \emph{rename} qui introduit l'époque \epoch{A1}.
Cette opération est livrée au noeud C, qui génère ensuite sa propre opération \emph{rename} qui introduit l'époque \epoch{C6}.
De manière concurrente à ces opérations, le noeud B génère deux opérations \emph{rename}, introduisant \epoch{B2} et \epoch{B7}.

Une fois que les noeuds se sont synchronisés, ils obtiennent l'\emph{arbre des époques} représenté dans la \autoref{fig:priority-epoch-tree}.
Dans cette figure, la flèche tireté rouge représente l'ordre entre les époques d'après la relation \emph{priority} tandis que l'époque cible choisie est représentée sous la forme d'un noeud rouge.

Pour déterminer l'époque cible, les noeuds reposent sur la relation \emph{priority}.
D'après l'ordre lexicographique sur le chemin des époques dans l'\emph{arbre des époques}, nous avons \epoch{0}~<~\epoch{0}\epoch{A1}~<~\epoch{0}\epoch{A1}\epoch{C6}~<~\epoch{0}\epoch{B2}~<~\epoch{0}\epoch{B2}\epoch{B7}.
Chaque noeud sélectionne donc \epoch{B7} comme époque cible de manière non-coordonnée.

D'autres stratégies pourraient être proposées pour définir la relation \emph{priority}.
Par exemple, l'ordre proposé \emph{priority} pourrait se baser sur une représentation du travail effectué sur le document, à l'aide de métriques additionnelles intégrées au sein des opérations \emph{rename}.
Cela permettrait de favoriser la branche de l'\emph{arbre des époques} avec le plus d'activité, que nous conjecturons corrélé avec le nombre de noeuds actifs, pour minimiser la quantité globale de calculs effectués par les noeuds du système.
Nous approfondissons ce sujet dans la \autoref{sec:alt-priority-relation}.
