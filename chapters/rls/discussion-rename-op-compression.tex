
Pour limiter la consommation en bande passante des opérations $\trm{rename}$, nous proposons la technique de compression suivante.
Au lieu de diffuser les identifiants complets formant l'\emph{ancien état}, les noeuds peuvent diffuser seulement les éléments nécessaires pour identifier de manière unique les intervalles d'identifiants.
En effet, un identifiant peut être caractérisé de manière unique par le triplet composé de l'\emph{identifiant de noeud}, du \emph{numéro de séquence} et de l'\emph{offset} de son dernier tuple.
Par conséquent, un intervalle d'identifiants peut être identifié de manière unique à partir du triplet signature de son identifiant de début et de sa longueur, \ie du quadruplet $\langle\trm{nodeId}, \trm{nodeSeq}, \trm{offsetBegin}, \trm{offsetEnd}\rangle$.
Cette méthode nous permet de réduire les données à diffuser dans le cadre de l'opération $\trm{rename}$ à un montant fixe par intervalle.

Pour décompresser l'opération reçue, les noeuds doivent reformer les intervalles d'identifiants correspondant aux quadruplets reçus.
Pour cela, ils parcourent leur état.
Lorsqu'ils rencontrent un identifiant partageant le même couple $\langle\trm{nodeId}, \trm{nodeSeq}\rangle$ qu'un des intervalles de l'opération $\trm{rename}$, les noeuds disposent de l'ensemble des informations requises pour le reconstruire.
Cependant, certains couples $\langle\trm{nodeId}, \trm{nodeSeq}\rangle$ peuvent avoir été supprimés en concurrence et ne plus être présents dans la séquence.
Dans ce cas, il est nécessaire de parcourir le journal des opérations $\trm{remove}$ concurrentes pour retrouver les identifiants correspondants et reconstruire l'opération $\trm{rename}$ originale.

Grâce à cette méthode de compression, nous pouvons instaurer une taille maximale à l'opération $\trm{rename}$.
En effet, les noeuds peuvent émettre une opération $\trm{rename}$ dès que leur état courant atteint un nombre donné d'intervalles d'identifiants, bornant ainsi la taille du message à diffuser.
