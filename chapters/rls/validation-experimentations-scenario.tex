Le scénario reproduit la rédaction d'un article par plusieurs pairs de manière collaborative, en temps réel.
La collaboration ainsi décrite se décompose en 2 phases.

Dans un premier temps, les pairs spécifient principalement le contenu de l'article.
Quelques opérations $\trm{remove}$ sont tout même générées pour simuler des fautes de frappes.
Une fois que le document atteint une taille critique (définie de manière arbitraire), les pairs passent à la seconde phase de la collaboration.
Lors de cette seconde phase, les pairs arrêtent d'ajouter du nouveau contenu mais se concentre à la place sur la reformulation et l'amélioration du contenu existant.
Ceci est simulé en équilibrant le ratio entre les opérations $\trm{insert}$ et $\trm{remove}$.

Chaque pair doit émettre un nombre donné d'opérations $\trm{insert}$ et $\trm{remove}$.
La simulation prend fin une fois que tous les pairs ont reçu toutes les opérations.
Pour suivre l'évolution de l'état des pairs, nous prenons des instantanés de leur état à plusieurs points donnés de la simulation.
