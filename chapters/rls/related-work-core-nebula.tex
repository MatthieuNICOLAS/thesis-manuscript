L'approche \emph{core-nebula}\cite{letia:hal-01248270,zawirski:hal-01248197} a été proposée pour réduire la taille des identifiants dans Treedoc \cite{2009-treedoc-preguica}.
Dans ces travaux, les auteurs définissent l'opération \emph{rebalance} qui permet aux noeuds de réassigner des identifiants plus courts aux éléments du document.
Cependant, cette opération \emph{rebalance} n'est ni commutative avec les opérations $\trm{insert}$ et $\trm{remove}$, ni avec elle-même.
Pour assurer la convergence à terme \cite{10.1145/224057.224070}, l'approche \emph{core-nebula} empêche la génération d'opérations \emph{rebalance} concurrentes.
Pour ce faire, l'approche requiert un consensus entre les noeuds pour générer les opérations \emph{rebalance}.
Des opérations $\trm{insert}$ et $\trm{remove}$ sont elles toujours générées sans coordination entre les noeuds et peuvent donc être concurrentes aux opérations \emph{rebalance}.
Pour gérer les opérations concurrentes aux opérations \emph{rebalance}, les auteurs proposent de transformer les opérations concernées par rapport aux effets des opérations \emph{rebalance}, à l'aide un mécanisme de \emph{catch-up}, avant de les appliquer.

Cependant, les protocoles de consensus ne passent pas à l'échelle et ne sont pas adaptés aux systèmes distribués à large échelle.
Pour pallier ce problème, l'approche \emph{core-nebula} propose de répartir les noeuds dans deux groupes : le \emph{core} et la \emph{nebula}.
Le \emph{core} est un ensemble, de taille réduite, de noeuds stables et hautement connectés tandis que la \emph{nebula} est un ensemble, de taille non-bornée, de noeuds.
Seuls les noeuds du \emph{core} participent à l'exécution du protocole de consensus.
Les noeuds de la \emph{nebula} contribuent toujours au document par le biais des opérations $\trm{insert}$ et $\trm{remove}$.

Notre travail peut être vu comme une extension de celui présenté dans \emph{core-nebula}.
Avec RenamableLogootSplit, nous adaptons l'opération \emph{rebalance} et le mécanisme de \emph{catch-up} à LogootSplit pour tirer partie de la fonctionnalité offerte par les blocs.
De plus, nous proposons un mécanisme pour supporter les opérations $\trm{rename}$ concurrentes, ce qui supprime la nécessité de l'utilisation d'un protocole de consensus.
Notre contribution est donc une approche plus générique puisque RenamableLogootSplit est utilisable dans des systèmes composés d'un \emph{core} et d'une \emph{nebula}, ainsi que dans les systèmes ne disposant pas de noeuds stables pour former un \emph{core}.

Dans les systèmes disposant d'un \emph{core}, nous pouvons donc combiner RenamableLogootSplit avec un protocole de consensus pour éviter la génération d'opérations $\trm{rename}$ concurrentes.
Cette approche offre plusieurs avantages.
Elle permet de se passer de tout ce qui à attrait au support d'opérations $\trm{rename}$ concurrentes, \ie la définition d'une relation \emph{priority} et l'implémentation de \textsc{revertRenameId}.
Elle permet aussi de simplifier l'implémentation du mécanisme de récupération de mémoire des époques et \emph{anciens états} pour reposer seulement sur la stabilité causale des opérations.
Concernant ses performances, cette approche se comporte de manière similaire à RenamableLogootSplit avec un seul \emph{renaming bot} (cf. \autoref{sec:evaluation-results}), mais avec un surcoût correspondant au coût du protocole de consensus sélectionné.
