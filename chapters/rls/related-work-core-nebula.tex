\cite{letia:hal-01248270} puis \cite{zawirski:hal-01248197} définissent un mécanisme de ré-équilibrage de l'arbre des identifiants de position pour Treedoc \cite{2009-treedoc-preguica}.
Ce mécanisme, qui prend la forme d'une nouvelle opération $\trm{rebalance}$, permet aux noeuds de réassigner des identifiants plus courts aux éléments du document.
Cependant, cette nouvelle opération n'est ni commutative avec les opérations $\trm{insert}$ et $\trm{remove}$, ni avec elle-même.
Pour assurer la convergence à terme \cite{10.1145/224057.224070} du système, \cite{zawirski:hal-01248197} fait le choix d'empêcher la génération d'opérations $\trm{rebalance}$ concurrentes.
Pour cela, cette approche requiert un consensus entre les noeuds pour générer les opérations $\trm{rebalance}$.
Les opérations $\trm{insert}$ et $\trm{remove}$ sont elles toujours générées sans coordination entre les noeuds et peuvent donc être concurrentes aux opérations $\trm{rebalance}$.
Pour gérer les opérations concurrentes aux opérations $\trm{rebalance}$, les auteurs proposent de transformer les opérations concernées par rapport aux effets des opérations$\trm{rebalance}$, à l'aide un mécanisme de \emph{catch-up}, avant de les appliquer.

Cependant, les protocoles de consensus ne passent pas à l'échelle et ne sont pas adaptés aux systèmes distribués à large échelle.
Pour pallier ce problème, les auteurs proposent de répartir les noeuds dans deux groupes : le \emph{core} et la \emph{nebula}.
Le \emph{core} est un ensemble, de taille réduite, de noeuds stables et hautement connectés tandis que la \emph{nebula} est un ensemble, de taille non-bornée, de noeuds.
Seuls les noeuds du \emph{core} participent à l'exécution du protocole de consensus.
Les noeuds de la \emph{nebula} contribuent toujours au document par le biais des opérations $\trm{insert}$ et $\trm{remove}$.\\

Notre travail peut être vu comme une extension des travaux.
Avec RenamableLogootSplit, nous adaptons l'opération $\trm{rebalance}$ et le mécanisme de \emph{catch-up} à LogootSplit pour tirer partie de la fonctionnalité offerte par les blocs.
De plus, nous proposons un mécanisme pour supporter les opérations $\trm{rename}$ concurrentes, ce qui supprime la nécessité de l'utilisation d'un protocole de consensus.
Notre contribution est donc une approche plus générique puisque RenamableLogootSplit est utilisable dans des systèmes composés d'un \emph{core} et d'une \emph{nebula}, ainsi que dans les systèmes ne disposant pas de noeuds stables pour former un \emph{core}.

Dans les systèmes disposant d'un \emph{core}, nous pouvons donc combiner RenamableLogootSplit avec un protocole de consensus pour éviter la génération d'opérations $\trm{rename}$ concurrentes.
Cette approche offre plusieurs avantages.
Elle permet de se passer de tout ce qui à attrait au support d'opérations $\trm{rename}$ concurrentes, \ie la définition d'une relation \emph{priority} et l'implémentation de \texttt{revertRenameId}.
Elle permet aussi de simplifier l'implémentation du mécanisme de récupération de mémoire des époques et \emph{anciens états} pour reposer seulement sur la stabilité causale des opérations.
Concernant ses performances, cette approche se comporte de manière similaire à RenamableLogootSplit avec un seul \emph{renaming bot} \cf{sec:evaluation-results}, mais avec un surcoût correspondant au coût du protocole de consensus sélectionné.
