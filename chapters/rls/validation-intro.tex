Dans cette section, nous présentons notre validation de notre contribution.

Cette validation prend deux formes.
Dans un premier temps, nous effectuons une évaluation théorique de ses performances.
Ainsi, nous présentons dans la \autoref{sec:validation-time-complexity} une évaluation de la complexité en temps des opérations de RenamableLogootSplit, ainsi que de son mécanisme de \ac{GC} des métadonnées des anciens états obsolètes.

Puis nous effectuons dans un second temps une évaluation empirique pour confirmer ces résultats.
Dans la \autoref{sec:validation-experimentations}, nous présentons le protocole expérimental que nous avons mis en place pour cette évaluation empirique.
Puis nous présentons et analysons les résultats obtenus dans la \autoref{sec:evaluation-results}.
