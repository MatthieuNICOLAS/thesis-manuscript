\label{sec:ops-concurrent-to-rename}

Après avoir appliqué des opérations $\trm{rename}$ sur leur état local, les noeuds peuvent recevoir des opérations concurrentes.
La \autoref{fig:concurrent-insert-rename-inconsistent} illustre de tels cas.

\begin{figure}[!ht]
  \centering
  \resizebox{\columnwidth}{!}{
    \begin{tikzpicture}
        \newcommand\nodeh[1]{
            node[letter, label=#1:{$\betterid{i}{B1}{0}$}] {H}
        }
        \newcommand\nodee[1]{
            node[letter, fill=\colorblockone, label=#1:{$\coloridone\betterid{i}{B1}{0}\betterid{f}{A1}{0}$}] {E}
        }
        \newcommand\nodelo[1]{
            node[block, label=#1:{$\betterid{i}{B1}{1..2}$}] {LO}
        }
        \newcommand\renhelo[1]{
            node[block, fill=\colorblocktwo, label=#1:{$\coloridtwo\betterid{i}{A2}{0..3}$}] {HELO}
        }
        \newcommand\nodel[1]{
            node[letter, fill=\colorblockthree, label=#1:{$\coloridthree\betterid{i}{B1}{0}\betterid{m}{B2}{0}$}] {L}
        }
        \newcommand\crossl[1]{
            node[letter, cross, fill=\colorblockthree, label=#1:{$\coloridthree\betterid{i}{B1}{0}\betterid{m}{B2}{0}$}] {L}
        }

        \newcommand\initialstate[3]{
            \path
                #1
                ++#2
                ++(0:0.5)
                ++(#3:0.5) \nodeh{-#3}
                ++(0:\widthletter) \nodee{#3}
                ++(0:\widthletter) \nodelo{-#3};
        }

        \newcommand\ren[3]{
            \path
                #1
                ++#2
                ++(0:0.5)
                ++(#3:0.5) \renhelo{-#3};
        }

        \newcommand\insl[3]{
            \path
            #1
            ++#2
            ++(0:0.5)
            ++(#3:0.5) \nodeh{-#3}
            ++(0:\widthletter) \nodee{#3}
            ++(0:\widthletter) \nodel{-#3}
            ++(0:\widthletter) \nodelo{#3};
        }

        \newcommand\finalstate[3]{
            \path
                #1
                ++#2
                ++(0:0.5)
                ++(#3:0.5) \renhelo{-#3}
                ++(0:1.18 * \widthblock) \crossl{#3};
        }

        \newcommand\offseta{ (90:0.7) }
        \newcommand\offsetb{ (-90:0.7) }

        \path
            node {\textbf{A}}
            ++(0:0.5) node (a) {}
            +(0:20) node (a-end) {}
            +(0:2) node[point] (a-initial) {}
            +(0:8) node[point, label=-170:{$\trm{ren}()$}, label={[xshift=0pt]-10:{$\trm{ren}(A,2)$}}] (a-ren-a1) {}
            +(0:14) node[point] (a-recv-ins-l) {}
            +(0:18) node (a-final) {};

        \initialstate{(a-initial)}{\offseta}{90};
        \ren{(a-ren-a1)}{\offseta}{90};
        \finalstate{(a-recv-ins-l)}{\offseta}{90};

        \draw[dotted] (a) -- (a-initial) (a-final) -- (a-end);
        \draw[->, thick] (a-initial) --  (a-ren-a1) -- (a-recv-ins-l) -- (a-final);

        \path
            ++(270:3) node {\textbf{B}}
            ++(0:0.5) node (b) {}
            +(0:20) node (b-end) {}
            +(0:2) node[point] (b-initial) {}
            +(0:8) node[point, label=170:{$\trm{ins}(E \prec L \prec O)$}, label={[xshift=45pt]10:{$\trm{ins}({\coloridthree\betterid{i}{B1}{0}\betterid{m}{B2}{0}},L)$}}] (b-ins-l) {}
            +(0:18) node (b-final) {};


        \initialstate{(b-initial)}{\offsetb}{-90};
        \insl{(b-ins-l)}{\offsetb}{-90};

        \draw[dotted] (b) -- (b-initial) (b-final) -- (b-end);
        \draw[->, thick] (b-initial) --  (b-ins-l) -- (b-final);

        \draw[->, dashed, shorten >= 1] (b-ins-l) -- (a-recv-ins-l);
    \end{tikzpicture}
  }
  \caption{Modifications concurrentes menant à une anomalie}
  \label{fig:concurrent-insert-rename-inconsistent}
\end{figure}

Dans cet exemple, le noeud B insère un nouvel élément "L", lui assigne l'identifiant \id{i}{B1}{0}\id{m}{B2}{0} et diffuse cette modification, de manière concurrente à l'opération $\trm{rename}$ décrite dans la \autoref{fig:renaming}.
À la réception de l'opération $\trm{insert}$, le noeud A ajoute l'élément inséré au sein de sa séquence, en utilisant l'identifiant de l'élément pour déterminer sa position.
Cependant, puisque les identifiants ont été modifiés par l'opération $\trm{rename}$ concurrente, le noeud A insère le nouvel élément à la fin de sa séquence (puisque \id{i}{A2}{3} $\lid$ \id{i}{B1}{0}\id{m}{B2}{0}) au lieu de l'insérer à la position souhaitée.
Comme illustré par cet exemple, appliquer naivement les modifications concurrentes provoquerait des anomalies.
Il est donc nécessaire de traiter les opérations concurrentes aux opérations $\trm{rename}$ de manière particulière.\\

Tout d'abord, les noeuds doivent détecter les opérations concurrentes aux opérations $\trm{rename}$.
Pour cela, nous utilisons un système basé sur des \emph{époques}.
Initialement, la séquence répliquée débute à l'époque \emph{origine} notée \epoch{0}.
Chaque opération $\trm{rename}$ introduit une nouvelle époque et permet aux noeuds d'y avancer depuis l'époque précédente.
Par exemple, l'opération $\trm{rename}$ décrite dans la \autoref{fig:concurrent-insert-rename-inconsistent} permet aux noeuds de faire progresser leur état de \epoch{0} à \epoch{A2}.
Nous définissons les époques de la manière suivante :

\begin{definition}[Époque]
  Une époque est un couple $\langle \trm{nodeId}, \trm{nodeSeq} \rangle$ où
  \begin{enumerate}
    \item $\trm{nodeId}$ est l'identifiant du noeud qui a généré cette époque.
    \item $\trm{nodeSeq}$ est le numéro de séquence du noeud au moment de la génération de cette époque.
  \end{enumerate}
\end{definition}

Notons que l'époque générée est caractérisée et identifiée de manière unique par son couple $\langle \trm{nodeId}, \trm{nodeSeq} \rangle$.

Au fur et à mesure que les noeuds reçoivent des opérations $\trm{rename}$, ils construisent et maintiennent localement la \emph{chaîne des époques}.
Cette structure de données ordonne les époques en fonction de leur relation \emph{parent-enfant} et associe à chaque époque l'\emph{ancien état} correspondant (\ie l'\emph{ancien état} inclus dans l'opération $\trm{rename}$ qui a généré cette époque).
De plus, les noeuds marquent chaque opération avec leur époque courante au moment de génération de l'opération.
À la réception d'une opération, les noeuds comparent l'époque de l'opération à l'époque courante de leur séquence.

Si les époques diffèrent, les noeuds doivent transformer l'opération avant de pouvoir l'intégrer.
Les noeuds déterminent par rapport à quelles opérations $\trm{rename}$ doit être transformée l'opération reçue en calculant le chemin entre l'époque de l'opération et leur époque courante en utilisant la \emph{chaîne des époques}.\\

Les noeuds utilisent la fonction \texttt{renameId}, décrite dans l'\autoref{alg:renameId}\footnotemark, pour transformer les opérations $\trm{insert}$ et $\trm{remove}$ par rapport aux opérations $\trm{rename}$.
\footnotetext{
    Pour expliciter le formalisme de notre pseudo-code : nous utilisons l'\emph{assignation par décomposition}.
    Par exemple, les valeurs du tableau $\text{array} = [42, 7, 53]$ peuvent être affectées respectivement aux variables $\text{val}_1$, $\text{val}_2$ et $\text{val}_3$ via l'instruction $[\text{val}_1,\text{val}_2,\text{val}_3] \leftarrow \text{array}$.
    De plus, nous désignons par $\_$ une ou un ensembe de valeurs non-nécessaires par notre algorithme.
    Ainsi, pour récupérer uniquement le premier tuple d'un identifiant $\text{id}$, nous utilisons la notation suivante : $\text{firstTuple} \oplus \_ \leftarrow id$.
}
Cet algorithme associe les identifiants d'une époque \emph{parente} aux identifiants correspondant dans l'époque \emph{enfant}.
Pour cela, la fonction \texttt{renameId} distingue deux cas :

\begin{enumerate}
    \item L'identifiant à renommer $\trm{id}$ n'était pas présent lors de la génération de l'opération $\trm{rename}$, \ie $\trm{id}$ a été inséré en concurrence à cette dernière\footnotemark.
        \footnotetext{
            Dans le cas où nous autorisons une livraison non-causale de l'opération $\trm{rename}$ \cf{sec:renamablelogootsplit-delivery-model}, $\trm{id}$ peut aussi avoir été supprimé au préalable et l'opération $\trm{remove}$ correspondante pas encore livrée au noeud courant
        }
        Ce cas est géré par l'ensemble de l'algorithme, à l'exception des lignes \ref{alg:renameId-id-in-renamedids-begin} à \ref{alg:renameId-id-in-renamedids-end}.
        \label{item:renameId-id-not-in-former-state}
    \item L'identifiant à renommer $\trm{id}$ fait partie des identifiants présents à la génération de l'opération $\trm{rename}$, \ie $\trm{id}$ appartient à l'ancien état (nommé $\trm{renamedIds}$ dans \autoref{alg:renameId}).
        Ce cas est géré par les lignes \ref{alg:renameId-id-in-renamedids-begin} à \ref{alg:renameId-id-in-renamedids-end}.
        \label{item:renameId-id-in-former-state}
\end{enumerate}

\begin{algorithm}[!ht]
    \footnotesize
    \begin{algorithmic}[1]
        \Function {renameId}{id $\in \mathbb{I}$, renamedIds $\in \trm{Array}\langle \mathbb{I}\rangle$, nodeId $\in \mathbb{N}$, nodeSeq $\in \mathbb{N}$} {: $\mathbb{I}$}

            \Statex
            \State $[\text{id}_0, \text{id}_1, \cdots, \text{id}_{n-2}, \text{id}_{n-1}] \gets \text{renamedIds}$
            \Comment Retrieve the identifiers of the \emph{former state}
            % \State \Comment{$id$ is the id to rename}
            % \State \Comment{$renamedIds$ is the list of ids of the \emph{former state}}
            % \State \Comment{$nId$ is $node~id$ of the node that issued the $\trm{rename}$ op}
            % \State \Comment{$nSeq$ is $node~seq$ of the node that issued the $\trm{rename}$ op}
            \State $\text{firstId} \gets \text{id}_0$
            \State $\text{lastId} \gets \text{id}_{n - 1}$
            \State $\langle \text{pos}, \_, \_, \_ \rangle \oplus \_ \gets \text{firstId}$
              \label{alg:renameId-get-pos}
            \Comment Retrieve the value \emph{pos} of the first tuple of \emph{id}
            \Statex
            \If{$\text{id} \lid \text{firstId}$}
                \State $\text{newFirstId} \gets \newFirstId$
                \State \Return renIdLessThanFirstId(id, newFirstId)
            \ElsIf{$\text{id} \in \text{renamedIds}$}
                \label{alg:renameId-id-in-renamedids-begin}
                \State Retrieve the index \emph{i} of \emph{id} in \emph{renamedIds}
                \State \Return $\langle \text{pos}, \text{nodeId}, \text{nodeSeq}, i \rangle$ \label{alg:renameId-id-in-renamedids-end}
                \Comment Use \emph{i} as the \emph{offset} of the generated id
            \ElsIf{$\text{lastId} \lid \text{id}$}
                \State $\text{newLastId} \gets \newLastId$
                \State \Return renIdGreaterThanLastId(id, newLastId)
            \Else
                  \label{alg:renameId-main-case}
                \Statex \Comment $\text{firstId} \lid \text{id} \lid \text{lastId} \land \text{id} \notin \text{renamedIds}$
                \State \Return renIdFromPredId(id, renamedIds, pos, nodeId, nodeSeq)
            \EndIf
        \EndFunction
        \Statex
        \Function {renIdFromPredId}{id $\in \mathbb{I}$, renamedIds $\in \trm{Array} \langle \mathbb{I} \rangle$, pos $\in \mathbb{N}$, nodeId $\in \mathbb{N}$, nodeSeq $\in \mathbb{N}$} {: $\mathbb{I}$}
            \Statex
            \State $[\text{id}_0, \text{id}_1, \cdots, \text{id}_{n-2}, \text{id}_{n-1}] \gets \text{renamedIds}$
            \State Search $\text{id}_i \in \text{renamedIds such that id}_i \lid \text{id} \lid \text{id}_{i+1}$ \label{alg:renameId-find-predecessor}
            \State $\text{newPredId} \gets \langle \text{pos}, \text{nodeId}, \text{nodeSeq}, i \rangle$ \label{alg:renameId-rename-predecessor}
            \State \Return $\text{newPredId} \oplus \text{id}$ \label{alg:renameId-concat-predecessor}
        \EndFunction
    \end{algorithmic}
    \caption{Fonctions principales pour renommer un identifiant}
    \label{alg:renameId}
  \end{algorithm}

\subsubsection{Renommage d'un identifiant inséré en concurrence de l'opération $\trm{rename}$}
\label{sec:renameId-id-not-in-former-state}

Pour renommer les identifiants inconnus au moment de la génération de l'opération $\trm{rename}$, \texttt{renameId} utilise plusieurs stratégies.
La stratégie principale repose sur l'utilisation du prédecesseur dans l'ancien état de l'identifiant à renommer.
Un exemple est présenté dans la \autoref{fig:concurrent-insert-rename-fixed}.
Cette figure décrit le même scénario que la \autoref{fig:concurrent-insert-rename-inconsistent}, à l'exception que le noeud A utilise \texttt{renameId} pour renommer l'identifiant généré de façon concurrente, $\betterid{i}{B1}{0}\betterid{m}{B2}{0}$, avant de l'insérer dans son état.
L'algorithme procède de la manière suivante.

\begin{figure}[!ht]
  \centering
  \resizebox{\columnwidth}{!}{
    \begin{tikzpicture}
        \newcommand\nodeh[1]{
            node[letter, label=#1:{$\betterid{i}{B1}{0}$}] {H}
        }
        \newcommand\nodee[1]{
            node[letter, fill=\colorblockone, label=#1:{\coloridone$\betterid{i}{B1}{0}\betterid{f}{A1}{0}$}] {E}
        }
        \newcommand\nodelo[1]{
            node[block, label=#1:{$\betterid{i}{B1}{1..2}$}] {LO}
        }
        \newcommand\renhelo[1]{
            node[block, fill=\colorblocktwo, label=#1:{\coloridtwo$\betterid{i}{A2}{0..3}$}] {HELO}
        }
        \newcommand\nodel[1]{
            node[letter, fill=\colorblockthree,label=#1:{\coloridthree$\betterid{i}{B1}{0}\betterid{m}{B2}{0}$}] {L}
        }
        \newcommand\renhe[1]{
            node[block, fill=\colorblocktwo, label=#1:{\coloridtwo$\betterid{i}{A2}{0..1}$}] {HE}
        }
        \newcommand\renl[1]{
            node[letter, fill=\colorblockfour, label=#1:{\coloridfour$\betterid{i}{A2}{1}\betterid{i}{B1}{0}\betterid{m}{B2}{0}$}] {L}
        }
        \newcommand\renlo[1]{
            node[block, fill=\colorblocktwo, label=#1:{\coloridtwo$\betterid{i}{A2}{2..3}$}] {LO}
        }

        \newcommand\initialstate[3]{
            \path
                #1
                ++#2
                ++(0:0.5)
                ++(#3:0.5) node[epoch] {\epoch{0}}
                ++(0:1.05 * \widthoriginepoch) \nodeh{-#3}
                ++(0:\widthletter) \nodee{#3}
                ++(0:\widthletter) \nodelo{-#3};
        }

        \newcommand\ren[3]{
            \path
                #1
                ++#2
                ++(0:0.5)
                ++(#3:0.5) node[epoch] {\epoch{A2}}
                ++(0:1.3 * \widthepoch) \renhelo{-#3};
        }

        \newcommand\insl[3]{
            \path
            #1
            ++#2
            ++(0:0.5)
            ++(#3:0.5) node[epoch] {\epoch{0}}
            ++(0:1.05 * \widthoriginepoch) \nodeh{-#3}
            ++(0:\widthletter) \nodee{#3}
            ++(0:\widthletter) \nodel{-#3}
            ++(0:\widthletter) \nodelo{#3};
        }

        \newcommand\finalstate[3]{
            \path
                #1
                ++#2
                ++(0:0.5)
                ++(#3:0.5) node[epoch] {\epoch{A2}}
                ++(0:1.3 * \widthepoch) \renhe{-#3}
                ++(0:\widthblock) \renl{#3}
                ++(0:\widthletter) \renlo{-#3};
        }

        \newcommand\offseta{ (90:0.7) }
        \newcommand\offsetb{ (-90:0.7) }

        \path
            node {\textbf{A}}
            ++(0:0.5) node (a) {}
            +(0:20) node (a-end) {}
            +(0:2) node[point] (a-initial) {}
            +(0:8) node[point, label=-170:{$\trm{ren}()$}, label={[xshift=0pt]-10:{$\trm{ren}(\betterepoch{0}, A,2)$}}] (a-ren-a1) {}
            +(0:14) node[point] (a-recv-ins-l) {}
            +(0:18) node (a-final) {};

        \initialstate{(a-initial)}{\offseta}{90};
        \ren{(a-ren-a1)}{\offseta}{90};
        \finalstate{(a-recv-ins-l)}{\offseta}{90};

        \draw[dotted] (a) -- (a-initial) (a-final) -- (a-end);
        \draw[->, thick] (a-initial) --  (a-ren-a1) -- (a-recv-ins-l) -- (a-final);

        \path
            ++(270:3) node {\textbf{B}}
            ++(0:0.5) node (b) {}
            +(0:20) node (b-end) {}
            +(0:2) node[point] (b-initial) {}
            +(0:8) node[point, label=170:{$\trm{ins}(E \prec L \prec O)$}, label={[xshift=45pt]10:{$\trm{ins}(\betterepoch{0}, {\coloridthree\betterid{i}{B1}{0}\betterid{m}{B2}{0}},L)$}}] (b-ins-l) {}
            +(0:18) node (b-final) {};


        \initialstate{(b-initial)}{\offsetb}{-90};
        \insl{(b-ins-l)}{\offsetb}{-90};

        \draw[dotted] (b) -- (b-initial) (b-final) -- (b-end);
        \draw[->, thick] (b-initial) --  (b-ins-l) -- (b-final);

        \draw[->, dashed, shorten >= 1] (b-ins-l) -- (a-recv-ins-l);
    \end{tikzpicture}
  }
  \caption{Renommage de la modification concurrente avant son intégration en utilisant \texttt{renameId} afin de maintenir l'ordre souhaité}
  \label{fig:concurrent-insert-rename-fixed}
\end{figure}

Tout d'abord, il détermine quelle stratégie utiliser pour renommer l'identifiant ($\betterid{i}{B1}{0}\betterid{m}{B2}{0}$) en le comparant aux identifiants présents dans l'ancien état, dont notamment deux identifiants clés : $\trm{firstId}$ et $\trm{lastId}$.
Ces identifiants correspondent respectivement au premier ($\betterid{i}{B1}{0}$) et au dernier identifiants ($\betterid{i}{B1}{2}$) présents dans l'ancien état ($[\betterid{i}{B1}{0},\betterid{i}{B1}{0}\betterid{f}{A1}{0},\betterid{i}{B1}{1},\betterid{i}{B1}{2}]$).
Ici, nous avons :
\[\trm{firstId} <_{id} \trm{id} <_{id} \trm{lastId}\]
Nous sommes donc dans le cas correspondant à la ligne \ref{alg:renameId-main-case} et l'algorithme utilise \texttt{renIdFromPredId} pour renommer l'identifiant.

Cet algorithme recherche le prédecesseur de l'identifiant donné (\id{i}{B1}{0}\id{m}{B2}{0}) dans l'ancien état, \ie l'identifiant maximal présent dans l'ancien état\footnote{Notons que l'ancien état est un tableau trié d'identifiants, il est donc possible d'effectuer une recherche dichotomique pour trouver efficacement l'index d'un identifiant.} et qui est inférieur à l'identifiant donné (ligne \ref{alg:renameId-find-predecessor}).
Dans le cas présent, cette recherche établit que $\betterid{i}{B1}{0}\betterid{f}{A1}{0}$ est le prédécesseur de $\betterid{i}{B1}{0}\betterid{m}{B2}{0}$.

Ensuite, l'algorithme renomme le prédecesseur obtenu pour déterminer l'identifiant correspondant à l'époque \emph{enfant} (ligne \ref{alg:renameId-rename-predecessor}).
Puisque le prédecesseur appartient à l'ancien état, ce renommage correspond au cas \ref{item:renameId-id-in-former-state} expliqué ci-après.
Dans notre exemple, $\betterid{i}{B1}{0}\betterid{f}{A1}{0}$ est renommé en $\betterid{i}{A2}{1}$.

Finalement, le noeud A concatène (noté $\oplus$) le prédecesseur renommé et l'identifiant donné pour générer l'identifiant correspondant à l'époque \emph{enfant} : \id{i}{A2}{1}\id{i}{B1}{0}\id{m}{B2}{0} (ligne \ref{alg:renameId-concat-predecessor}).
En réassignant cet identifiant à l'élément inséré de manière concurrente, le noeud A peut l'insérer à son état tout en préservant l'ordre souhaité entre les éléments.

\subsubsection{Renommage d'un identifiant appartenant à l'ancien état}
\label{sec:renameId-id-in-former-state}

\texttt{renameId} permet aussi aux noeuds de gérer le cas contraire : intégrer des opérations $\trm{rename}$ distantes sur leur copie locale alors qu'ils ont précédemment intégré des modifications concurrentes.
Ce cas correspond à celui du noeud B dans la \autoref{fig:concurrent-insert-rename-fixed}.
Dans ce cas, \texttt{renameId} reproduit l'algorithme que nous avons décrit de manière informelle lors de la présentation de l'opération \emph{rename} \cf{sec:rename-op-proposition}.
Cet algorithme associe à chaque identifiant appartenant à l'ancien état un nouvel identifiant composé d'un unique tuple.

Pour chaque identifiant renommé, le nouvel identifiant obtenu est de la forme $\langle \trm{pos},\trm{nodeId},\trm{nodeSeq},\trm{offset} \rangle$ où :
\begin{enumerate}
    \item $\trm{pos}$ est la valeur du champ $\trm{pos}$ du premier identifiant appartenant à l'ancien état.
        Dans notre exemple, l'ancien état est le suivant :
        \begin{equation*}
            [\betterid{i}{B1}{0},\betterid{i}{B1}{0}\betterid{f}{A1}{0},\betterid{i}{B1}{1},\betterid{i}{B1}{2}]
        \end{equation*}
        La valeur de $\trm{pos}$ est donc \emph{i} (ligne \ref{alg:renameId-get-pos}).
    \item $\trm{nodeId}$ est l'identifiant du noeud qui a généré l'opération \emph{rename} (\textbf{A}, obtenu par paramètre).
    \item $\trm{nodeSeq}$ est le numéro de séquence du noeud qui a généré l'opération \emph{rename} (2, obtenu par paramètre).
    \item $\trm{offset}$ est l'index de l'identifiant renommé dans l'ancien état ($\betterid{i}{B1}{0} \rightarrow 0,\betterid{i}{B1}{0}\betterid{f}{A1}{0} \rightarrow 1, \dots$).
    Ce dernier peut être récupéré via une recherche de l'identifiant donné dans l'ancien état (ligne \ref{alg:renameId-id-in-renamedids-begin}).
\end{enumerate}

Ainsi, à la réception de l'opération $\trm{rename}$ du noeud A, le noeud B utilise \texttt{renameId} sur chacun des identifiants de son état pour le renommer et atteindre un état équivalent à celui du noeud A.\\

\subsubsection{Notes additionnelles}

L'\autoref{alg:renameId} présente seulement le cas principal de \texttt{renameId}, \ie le cas où l'identifiant à renommer appartient à l'intervalle des identifiants formant l'ancien état ($\trm{firstId} \leq_{id} \trm{id} \leq_{id} \trm{lastId}$).
Les fonctions pour gérer les autres cas, \ie les cas où l'identifiant à renommer n'appartient pas à cet intervalle ($\trm{id} <_{id} \trm{firstId}$ ou $\trm{lastId} <_{id} \trm{id}$), sont présentées dans l'\autoref{app:rename-id}.\\

Nous notons que l'algorithme que nous présentons ici permet aux noeuds de renommer leur état identifiant par identifiant.
Une extension possible est de concevoir \texttt{renameBlock}, une version améliorée qui renomme l'état bloc par bloc.
\texttt{renameBlock} réduirait le temps d'intégration des opérations $\trm{rename}$, puisque sa complexité en temps ne dépendrait plus du nombre d'identifiants (\ie du nombre d'éléments) mais du nombre de blocs.
De plus, son exécution réduirait le temps d'intégration des prochaines opérations $\trm{rename}$ puisque le mécanisme de renommage réduit le nombre de blocs.
