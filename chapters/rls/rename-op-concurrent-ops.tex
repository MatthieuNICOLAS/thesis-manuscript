\label{sec:ops-concurrent-to-rename}

Après avoir appliqué des opérations \emph{rename} sur leur état local, les noeuds peuvent recevoir des opérations concurrentes.
La \autoref{fig:concurrent-insert-rename-inconsistent} illustre de tels cas.

\begin{figure}[!ht]
  \centering
  \resizebox{\columnwidth}{!}{
    \begin{tikzpicture}
        \path
            node {\textbf{A}}
            ++(0:\widthletter) node[letter, label=below:{\id{i}{B0}{0}}] {H}
            ++(0:\widthletter) node[letter, fill=mydarkorange, label=above:{\color{mydarkorange}\id{i}{B0}{0}\id{f}{\,A0}{0}}] {E}
            ++(0:\widthletter) node[block, label=below:{\id{i}{B0}{1..2}}] (S0A-right) {LO}
            ++(0:5 * \widthletter) node[block, fill=mydarkblue,
                    label={below:{\color{mydarkblueid}\id{i}{A1}{0..3}} }
                        ] (S1A) {HELO}
            ++(0:8 * \widthletter) node[block, fill=mydarkblue,
                    label={below:{\color{mydarkblueid}\id{i}{A1}{0..3}} }
                        ] (S2A-left) {HELO}
            ++(0:1.18 * \widthblock) node[letter, fill=mylightorange, cross,
                    label={above:{\color{mylightorange}\id{i}{B0}{0}\id{m}{B1}{0}} }
                        ] {L};

        \path
            ++(270:2) node {\textbf{B}}
            ++(0:\widthletter) node[letter, label=below:{\id{i}{B0}{0}}] {H}
            ++(0:\widthletter) node[letter, fill=mydarkorange, label=above:{\color{mydarkorange}\id{i}{B0}{0}\id{f}{\,A0}{0}}] {E}
            ++(0:\widthletter) node[block, label=below:{\id{i}{B0}{1..2}}] (S0B-right) {LO}
            ++(0:5 * \widthletter) node[letter, label=below:{\id{i}{B0}{0}}] (S1B-left) {H}
            ++(0:\widthletter) node[letter, fill=mydarkorange, label=above:{\color{mydarkorange}\id{i}{B0}{0}\id{f}{\,A0}{0}}] {E}
            ++(0:\widthletter) node[letter, fill=mylightorange, label=below:{\color{mylightorange}\id{i}{B0}{0}\id{m}{B1}{0}}] {L}
            ++(0:\widthletter) node[block, label=above:{\id{i}{B0}{1..2}}] (S1B-right) {LO};


        \draw[->, thick] (S0A-right) -- node[above, align=center]{\emph{rename}} (S1A);
        \draw[dotted] (S1A) -- (S2A-left);
        \draw[->, thick] (S0B-right) -- node[below, align=center]{\emph{insert "L"}\\\emph{between}\\\emph{"E" and "L"}} (S1B-left);
        \draw[dashed, ->, thick, shorten >= 3] (S1B-right.east) -- node[below right, align=center]{\emph{insert "L" at} {\color{mylightorange}\id{i}{B0}{0}\id{m}{B1}{0}}} (S2A-left.west);

    \end{tikzpicture}
  }
  \caption{Modifications concurrentes menant à une anomalie}
  \label{fig:concurrent-insert-rename-inconsistent}
\end{figure}

Dans cet exemple, le noeud B insère un nouvel élément "L", lui assigne l'identifiant \id{i}{B0}{0}\id{m}{B1}{0} et diffuse cette modification, de manière concurrente à l'opération \emph{rename} décrite dans la \autoref{fig:concurrent-insert-rename-inconsistent}.
À la réception de l'opération \emph{insert}, le noeud A ajoute l'élément inséré au sein de sa séquence, en utilisant l'identifiant de l'élément pour déterminer sa position.
Cependant, puisque les identifiants ont été modifiés par l'opération \emph{rename} concurrente, le noeud A insère le nouvel élément à la fin de sa séquence (puisque \id{i}{A1}{3} $\lid$ \id{i}{B0}{0}\id{m}{B1}{0}) au lieu de l'insérer à la position souhaitée.
Comme illustré par cet exemple, appliquer naivement les modifications concurrentes provoquerait des anomalies.
Il est donc nécessaire de traiter les opérations concurrentes aux opérations \emph{rename} de manière particulière.

Tout d'abord, les noeuds doivent détecter les opérations concurrentes aux opérations \emph{rename}.
Pour cela, nous utilisons un système basé sur des \emph{époques}.
Initialement, la séquence répliquée débute à l'époque \emph{origine} notée \epoch{0}.
Chaque opération \emph{rename} introduit une nouvelle époque et permet aux noeuds d'y avancer depuis l'époque précédente.
Par exemple, l'opération \emph{rename} décrite dans la \autoref{fig:concurrent-insert-rename-inconsistent} permet aux noeuds de faire progresser leur état de \epoch{0} à \epoch{A1}.
Nous définissons les époques de la manière suivante :

\begin{definition}[Époque]
  Une époque est un couple $\langle \trm{nodeId}, \trm{nodeSeq} \rangle$ où
  \begin{itemize}
    \item $\trm{nodeId}$ est l'identifiant du noeud qui a généré cette époque.
    \item $\trm{nodeSeq}$ est le numéro de séquence du noeud au moment de la génération de cette époque.
  \end{itemize}
\end{definition}

Notons que l'époque générée est caractérisée et identifiée de manière unique par son couple $\langle \trm{nodeId}, \trm{nodeSeq} \rangle$.

Au fur et à mesure que les noeuds reçoivent des opérations \emph{rename}, ils construisent et maintiennent localement la \emph{chaîne des époques}.
Cette structure de données ordonne les époques en fonction de leur relation \emph{parent-enfant} et associe à chaque époque l'\emph{ancien état} correspondant (\ie l'\emph{ancien état} inclus dans l'opération \emph{rename} qui a généré cette époque).
De plus, les noeuds marquent chaque opération avec leur époque courante au moment de génération de l'opération.
À la réception d'une opération, les noeuds comparent l'époque de l'opération à l'époque courante de leur séquence.

Si les époques diffèrent, les noeuds doivent transformer l'opération avant de pouvoir l'intégrer.
Les noeuds déterminent par rapport à quelles opérations \emph{rename} doit être transformée l'opération reçue en calculant le chemin entre l'époque de l'opération et leur époque courante en utilisant la \emph{chaîne des époques}.

Les noeuds utilisent la fonction \textsc{renameId}, décrite dans l'\autoref{alg:renameId}, pour transformer les opérations \emph{insert} et \emph{remove} par rapport aux opérations \emph{rename}.
Cet algorithme associe les identifiants d'une époque \emph{parente} aux identifiants correspondant dans l'époque \emph{enfant}.
L'idée principale de cet algorithme est de renommer les identifiants inconnus au moment de la génération de l'opération \emph{rename} en utilisant leur prédecesseur.
Un exemple est présenté dans la \autoref{fig:concurrent-insert-rename-fixed}.
Cette figure décrit le même scénario que la \autoref{fig:concurrent-insert-rename-inconsistent}, à l'exception que le noeud A utilise \textsc{renameId} pour renommer l'identifiants généré de façon concurrente avant de l'insérer dans son état.

\begin{algorithm}[!ht]
  \footnotesize
  \begin{algorithmic}[1]
      \Function {renameId}{id $\in \mathbb{I}$, renamedIds $\in \trm{Arr}\langle \mathbb{I}\rangle$, nodeId $\in \mathbb{N}$, nodeSeq $\in \mathbb{N}$} {: $\mathbb{I}$}
          \Statex \Comment $\text{renamedIds} = [\text{id}_0, \text{id}_1, \cdots, \text{id}_{n-2}, \text{id}_{n-1}]$
          % \State \Comment{$id$ is the id to rename}
          % \State \Comment{$renamedIds$ is the list of ids of the \emph{former state}}
          % \State \Comment{$nId$ is $node~id$ of the node that issued the \emph{rename} op}
          % \State \Comment{$nSeq$ is $node~seq$ of the node that issued the \emph{rename} op}
          \State $\text{firstId} \gets \text{id}_0$
          \State $\text{lastId} \gets \text{id}_{n - 1}$
          \Statex \Comment $\text{firstId} = \langle \text{pos}, \_, \_, \_ \rangle \oplus \text{suffix}$
          \\
          \If{$\text{id} \lid \text{firstId}$}
              \State $\text{newFirstId} \gets \newFirstId$
              \State \Return renIdLessThanFirstId(id, newFirstId)
          \ElsIf{$\text{id} \in \text{renamedIds}$}
              \Statex \Comment $\text{id} = \text{id}_i$
              \State \Return $\langle \text{pos}, \text{nodeId}, \text{nodeSeq}, i \rangle$
          \ElsIf{$\text{lastId} \lid \text{id}$}
              \State $\text{newLastId} \gets \langle \text{pos}, \text{nodeId}, \text{nodeSeq}, n-1 \rangle$
              \State \Return renIdGreaterThanLastId(id, newLastId)
          \Else
              \State \Return renIdFromPredId(id, renamedIds, pos, nodeId, nodeSeq)
          \EndIf
      \EndFunction
      \\
      \Function {renIdFromPredId}{id $\in \mathbb{I}$, renamedIds $\in \trm{Arr} \langle \mathbb{I} \rangle$, pos $\in \mathbb{N}$, nodeId $\in \mathbb{N}$, nodeSeq $\in \mathbb{N}$} {: $\mathbb{I}$}
          \Statex \Comment $\text{renamedIds} = [\text{id}_0, \text{id}_1, \cdots, \text{id}_{n-2}, \text{id}_{n-1}]$
          \State Find $\text{id}_i \in \text{renamedIds such that id}_i \lid \text{id} \lid \text{id}_{i+1}$
          \State $\text{newPredId} \gets \langle \text{pos}, \text{nodeId}, \text{nodeSeq}, i \rangle$
          \State \Return $\text{newPredId} \oplus \text{id}$
      \EndFunction
  \end{algorithmic}
  \caption{Fonctions principales pour renommer un identifiant}
  \label{alg:renameId}
\end{algorithm}

\begin{figure}[!ht]
  \centering
  \resizebox{\columnwidth}{!}{
    \begin{tikzpicture}
        \path
            node {\textbf{A}}
            ++(0:0.5 * \widthletter) node[epoch] {\epoch{0}}
            ++(0:1.05 * \widthoriginepoch) node[letter, label=below:{\id{i}{B0}{0}}] {H}
            ++(0:\widthletter) node[letter, fill=mydarkorange, label=above:{\color{mydarkorange}\id{i}{B0}{0}\id{f}{\,A0}{0}}] {E}
            ++(0:\widthletter) node[block, label=below:{\id{i}{B0}{1..2}}] (S0A-right) {LO}
            ++(0:5 * \widthletter) node[epoch] (S1A-left) {\epoch{A1}}
            ++(0:1.3 * \widthepoch) node[block, fill=mydarkblue,
                    label={below:{\color{mydarkblueid}\id{i}{A1}{0..3}} }
                        ] (S1A-right) {HELO}
            ++(0:8 * \widthletter) node[epoch] (S2A-left) {\epoch{A1}}
            ++(0:1.3 * \widthepoch) node[block, fill=mydarkblue,
                    label={below:{\color{mydarkblueid}\id{i}{A1}{0..1}} }
                        ] {HE}
            ++(0:\widthblock) node[letter, fill=mylightblue,
                    label={above:{\color{mylightblue!20!mydarkblueid}\id{i}{A1}{1}\id{i}{B0}{0}\id{m}{B1}{0}} }
                        ] {L}
            ++(0:\widthletter) node[block, fill=mydarkblue,
                    label={below:{\color{mydarkblueid}\id{i}{A1}{2..3}} }
                        ] {LO};

        \path
            ++(270:2) node {\textbf{B}}
            ++(0:0.5 * \widthletter) node[epoch] {\epoch{0}}
            ++(0:1.05 * \widthoriginepoch) node[letter, label=below:{\id{i}{B0}{0}}] {H}
            ++(0:\widthletter) node[letter, fill=mydarkorange, label=above:{\color{mydarkorange}\id{i}{B0}{0}\id{f}{\,A0}{0}}] {E}
            ++(0:\widthletter) node[block, label=below:{\id{i}{B0}{1..2}}] (S0B-right) {LO}
            ++(0:5 * \widthletter) node[epoch] (S1B-left) {\epoch{0}}
            ++(0:1.05 * \widthoriginepoch) node[letter, label=below:{\id{i}{B0}{0}}] {H}
            ++(0:\widthletter) node[letter, fill=mydarkorange, label=above:{\color{mydarkorange}\id{i}{B0}{0}\id{f}{\,A0}{0}}] {E}
            ++(0:\widthletter) node[letter, fill=mylightorange, label=below:{\color{mylightorange}\id{i}{B0}{0}\id{m}{B1}{0}}] {L}
            ++(0:\widthletter) node[block, label=above:{\id{i}{B0}{1..2}}] (S1B-right) {LO};


        \draw[->, thick] (S0A-right) -- node[above, align=center]{\emph{rename to \epoch{A1}}} (S1A-left);
        \draw[dotted] (S1A-right) -- (S2A-left);
        \draw[->, thick] (S0B-right) -- node[below, align=center]{\emph{insert "L"}\\\emph{between}\\\emph{"E" and "L"}} (S1B-left);
        \draw[dashed, ->, thick, shorten >= 3] (S1B-right.east) -- node[below right, align=center]{\emph{insert "L" at} {\color{mylightorange}\id{i}{B0}{0}\id{m}{B1}{0}}} (S2A-left.west);

    \end{tikzpicture}
  }
  \caption{Renommage de la modification concurrente avant son intégration en utilisant \textsc{renameId} afin de maintenir l'ordre souhaité}
  \label{fig:concurrent-insert-rename-fixed}
\end{figure}

L'algorithme procède de la manière suivante.
Tout d'abord, le noeud récupère le prédecesseur de l'identifiant donné \id{i}{B0}{0}\id{m}{B1}{0} dans l'ancien état : \id{i}{B0}{0}\id{f}{A0}{0}.
Ensuite, il calcule l'équivalent de \id{i}{B0}{0}\id{f}{A0}{0} dans l'état renommé : \id{i}{A1}{1}.
Finalement, le noeud A concatène cet identifiant et l'identifiant donné pour générer l'identifiant correspondant dans l'époque \emph{enfant} : \id{i}{A1}{1}\id{i}{B0}{0}\id{m}{B1}{0}.
En réassignant cet identifiant à l'élément inséré de manière concurrente, le noeud A peut l'insérer à son état tout en préservant l'ordre souhaité.

\textsc{renameId} permet aussi aux noeuds de gérer le cas contraire : intégrer des opérations \emph{rename} distantes sur leur copie locale alors qu'ils ont précédemment intégré des modifications concurrentes.
Ce cas correspond à celui du noeud B dans la \autoref{fig:concurrent-insert-rename-fixed}.
À la réception de l'opération \emph{rename} du noeud A, le noeud B utilise \textsc{renameId} sur chacun des identifiants de son état pour le renommer et atteindre un état équivalent à celui du noeud A.

L'\autoref{alg:renameId} présente seulement le cas principal de \textsc{renameId}, \ie le cas où l'identifiant à renommer appartient à l'intervalle des identifiants formant l'ancien état ($\trm{firstId} \leq_{id} \trm{id} \leq_{id} \trm{lastId}$).
Les fonctions pour gérer les autres cas, \ie les cas où l'identifiant à renommer n'appartient pas à cet intervalle ($\trm{id} <_{id} \trm{firstId}$ ou $\trm{lastId} <_{id} \trm{id}$), sont présentées dans l'\autoref{app:rename-id}.

Nous notons que l'algorithme que nous présentons ici permet aux noeuds de renommer leur état identifiant par identifiant.
Une extension possible est de concevoir \textsc{renameBlock}, une version améliorée qui renomme l'état bloc par bloc.
\textsc{renameBlock} réduirait le temps d'intégration des opérations \emph{rename}, puisque sa complexité en temps ne dépendrait plus du nombre d'identifiants (\ie du nombre d'éléments) mais du nombre de blocs.
De plus, son exécution réduirait le temps d'intégration des prochaines opérations \emph{rename} puisque le mécanisme de renommage regroupe les éléments en moins de blocs.
