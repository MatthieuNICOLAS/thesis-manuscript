Comme indiqué dans la \autoref{sec:def-rename-op}, les opérations \emph{rename} sont des opérations systèmes.
C'est donc aux concepteurs de systèmes qu'incombe la responsabilité de déterminer quand les noeuds devraient générer des opérations \emph{rename} et de définir une stratégie correspondante.
Il n'existe cependant pas de solution universelle, chaque système ayant ses particuliarités et contraintes.

Plusieurs aspects doivent être pris en compte lors de la définition de la stratégie de génération des opérations \emph{rename}.
Le premier porte sur la taille de la structure de données.
Comme illustré dans la \autoref{fig:evolution-document-size}, les métadonnées augmentent de manière progressive jusqu'à représenter 99\% de la structure de données.
En utilisant les opérations \emph{rename}, les noeuds peuvent supprimer les métadonnées et ainsi réduire la taille de la structure à un niveau acceptable.
Pour déterminer quand générer des opérations \emph{rename}, les noeuds peuvent donc monitorer le nombre d'opérations effectuées depuis la dernière opération \emph{rename}, le nombre de blocs qui composent la séquence ou encore la taille des identifiants.

Un second aspect à prendre en compte est le temps d'intégration des opérations \emph{rename}.
Comme indiqué dans le \autoref{tab:evolution-integration-time-rename}, l'intégration des opérations \emph{rename} distantes peut nécessiter des secondes si elles sont retardées trop longtemps.
Bien que les opérations \emph{rename} travaillent en coulisses, elles peuvent néanmoins impacter négativement l'expérience utilisateur.
Notamment, les noeuds ne peuvent pas intégrer d'autres opérations \emph{distantes} tant qu'ils sont en train de traiter des opérations \emph{rename}.
Du point de vue des utilisateurs, les opérations \emph{rename} peuvent alors être perçues comme des pics de latence.
Dans le domaine de l'édition collaborative temps réel, \textcite{2014-effect-delay-collaborative-editing-ignat,2015-cope-delay-collaborative-note-taking-ignat} ont montré que le délai dégradait la qualité des collaborations.
Il est donc important de générer fréquemment des opérations \emph{rename} pour conserver leur temps d'intégration sous une limite perceptible.

Finalement, le dernier aspect à considérer est le nombre d'opérations \emph{rename} concurrentes.
La \autoref{fig:evolution-document-size} montre que les opérations \emph{rename} concurrentes accroissent la taille de la structure de données tandis que la \autoref{fig:time-to-replay-log} illustre qu'elles augmentent le temps nécessaire pour rejouer le log d'opérations.
La stratégie proposée doit donc viser à minimiser le nombre d'opérations \emph{rename} concurrentes générées.
Cependant, elle doit éviter d'utiliser des coordinations synchrones entre les noeuds pour cela (\eg algorithmes de consensus), pour des raisons de performances.
Pour réduire la probabilité de générer des opérations \emph{rename} concurrentes, plusieurs méthodes peuvent être proposées.
Par exemple, les noeuds peuvent monitorer à quels autres noeuds ils sont connectés actuellement et déléguer au noeud ayant le plus grand \emph{identifiant de noeud} la responsabilité de générer les opérations \emph{rename}.

Pour récapituler, nous pouvons proposer plusieurs stragégies de génération des opérations \emph{rename}, pour minimiser de manière individuelle chacun des paramètres présentés.
Mais bien que certaines de ces stratégies convergent (minimiser la taille de la structure de données et minimiser le temps d'intégration des opérations \emph{rename}), d'autres entrent en conflit (générer une opération \emph{rename} dès qu'un seuil est atteint vs. minimiser le nombre d'opérations \emph{rename} concurrentes générées).
Les concepteurs de systèmes doivent proposer un compromis entre les différents paramètres en fonction des contraintes du système concerné (application temps réel ou asynchrone, limitations matérielles des noeuds...).
Il est donc nécessaire d'analyser le système pour évaluer ses performances sur chaque aspect, ses usages et trouver le bon compromis entre tous les paramètres de la stratégie de renommage.
Par exemple, dans le contexte des systèmes d'édition collaborative temps réel, \cite{2014-effect-delay-collaborative-editing-ignat} a montré que le délai diminue la qualité de la collaboration.
Dans de tels systèmes, nous viserions donc à conserver le temps d'intégration des opérations (en incluant les opérations \emph{rename}) en dessous du temps limite correspondant à leur perception par les utilisateurs.
