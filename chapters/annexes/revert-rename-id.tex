\Annex{Algorithmes \textsc{revertRenameId}}
\label{app:revert-rename-id}

Nous présentons dans cette annexe le reste des fonctions utilisées par la fonction \texttt{revertRenameId} pour annuler l'effet sur un identifiant d'une opération de renommage appliquée précédemment \cf{alg:revertRenameId}.

Pour rappel, \texttt{revertRenameId} prend comme paramètres initiaux :
\begin{enumerate}
    \item $\trm{id}$, l'identifiant à renommer.
    \item $\trm{renamedIds}$, une collection ordonnée des identifiants composant l'\emph{ancien état}, \ie l'état du noeud auteur de l'opération de renommage à sa génération \cf{def:former-state}.
    \item $\trm{nodeId}$, l'identifiant du noeud auteur de l'opération de renommage.
    \item $\trm{nodeSeq}$, le numéro de séquence du noeud auteur de l'opération de renommage à sa génération.
\end{enumerate}
et repose sur des valeurs particulières :
\begin{enumerate}
    \item $\trm{firstId}$, le premier identifiant de $\trm{renamedIds}$.
    \item $\trm{newFirstId}$, l'identifiant correspondant à $\trm{firstId}$ dans l'époque enfant.
    \item $\trm{lastId}$, le dernier identifiant de $\trm{renamedIds}$.
    \item $\trm{newLastId}$, l'identifiant correspondant à $\trm{lastId}$ dans l'époque enfant.
    \item $\bott$ (resp. $\topt$), le tuple LogootSplit minimal (resp. maximal). Son utilisation est réservé au mécanisme de renommage.
\end{enumerate}

La fonction \texttt{revRenIdLessThanNewFirstId} (resp. \texttt{revRendIdGreaterThanNewLastId}) permet de renommer un identifiant $\trm{id}$ tel que $\trm{id} \lid \trm{newFirstId}$ (resp. $\trm{newLastId} < \trm{id}$).

\begin{algorithm}[!ht]
  \footnotesize
  \begin{algorithmic}[1]
      \Function{revRenIdLessThanNewFirstId}{id $\in \mathbb{I}$, firstId $\in \mathbb{I}$, newFirstId $\in \mathbb{I}$}{: $\mathbb{I}$}
        \Statex \Comment $\text{newFirstId} = \newFirstId$
        \Statex \Comment $\text{predNewFirstId} = \predNewFirstId$

          \If{$\text{id} = \text{predNewId} \oplus \text{tail}$}
              \If{$\text{tail} \lid \text{firstId}$}
                  \State \Return tail \label{appendix-revert-rename-id-concurrent-1}
              \Else
                  \Statex \Comment id has been inserted causally after the \emph{rename} op
                  \Statex \Comment $\text{firstId} = \text{prefix} \oplus \logootsplituple{n}$

                  \State $\text{predFirstId} \gets \text{prefix} \oplus \langle \text{pos}_n,\text{nodeId}_n,\text{nodeSeq}_n,\text{offset}_n - 1 \rangle$
                  \State \Return $\text{predFirstId} \oplus \topt \oplus tail$ \Comment \commenttopt \label{appendix-revert-rename-id-causally-after-1}
              \EndIf
          \Else
              \State \Return id \label{appendix-revert-rename-id-simple-1}
          \EndIf
      \EndFunction
      \\
      \Function{revRenIdGreaterThanNewLastId}{id $\in \mathbb{I}$, lastId $\in \mathbb{I}$}{: $\mathbb{I}$}
          \If{$\text{id} \lid \text{lastId}$}
              \Statex \Comment id has been inserted causally after the \emph{rename} op
              \State \Return $\text{lastId} \oplus \bott \oplus \text{id}$ \Comment \commentbott \label{appendix-revert-rename-id-causally-after-2}
          \ElsIf{$\text{id} = \text{newLastId} \oplus \text{tail}$}
              \If{$\text{tail} \lid \text{lastId}$}
                  \Statex \Comment id has been inserted causally after the \emph{rename} op
                  \State \Return $\text{lastId} \oplus \bott \oplus \text{tail}$ \Comment \commentbott \label{appendix-revert-rename-id-causally-after-3}
              \ElsIf{$\text{tail} \lid \text{newLastId}$}
                  \State \Return tail \label{appendix-revert-rename-id-concurrent-2}
              \Else
                  \Statex \Comment id has been inserted causally after the \emph{rename} op
                  \State \Return id \label{appendix-revert-rename-id-simple-2}
              \EndIf
          \Else
              \State \Return id \label{appendix-revert-rename-id-simple-3}
          \EndIf
      \EndFunction
  \end{algorithmic}
  \caption{Remaining functions to revert an identifier renaming}
  \label{alg:appendix-revert-rename-id}
\end{algorithm}

Comme les fonctions présentées dans l'\autoref{app:rename-id}, ces fonctions ne modifient pas les identifiants n'étant pas compris dans l'intervalle affecté par le renommage (lignes \ref{appendix-revert-rename-id-simple-1}, \ref{appendix-revert-rename-id-simple-2} et \ref{appendix-revert-rename-id-simple-3}).

Pour traiter les cas restants, ces fonctions utilisent des stratégies différentes en fonction du motif formant $\trm{id}$.
Si \texttt{revRenIdLessThanNewFirstId} (resp. \texttt{revRenIdGreaterThanNewLastId}) infère de la valeur de l'identifiant que ce dernier a potentiellement été obtenu en renommant un identifiant généré en concurrence de l'opération de renommage à l'aide de \texttt{renIdLessThanFirstId} (resp. \texttt{renGreaterThanLastId}), elle retire simplement le préfixe ajouté (ligne \ref{appendix-revert-rename-id-concurrent-1} (resp. ligne \ref{appendix-revert-rename-id-concurrent-2})).


Si à la place la valeur de l'identifiant indique qu'il a été inséré après l'opération de renommage\footnote{D'après la relation \hb.}, \texttt{revRenIdLessThanNewFirstId} (resp. \texttt{revRenIdGreaterThanNewLastId}) génère l'identifiant correspondant à l'époque parent qui préserve l'ordre souhaité en utilisant $\trm{predFirstId}$ et le tuple exclusif au mécanisme de renommage $\topt$ (resp. $\trm{lastId}$ et le tuple exclusif au mécanisme de renommage $\bott$) (ligne \ref{appendix-revert-rename-id-causally-after-1} (resp. ligne \ref{appendix-revert-rename-id-causally-after-2} et \ref{appendix-revert-rename-id-causally-after-3})).

