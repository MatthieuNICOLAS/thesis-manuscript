\Annex{Algorithmes \textsc{revertRenameId}}
\label{app:revert-rename-id}

\autoref{alg:appendix-revert-rename-id} features the remaining functions composing \textsc{revertRenameId}.
\textsc{revRenIdLessThanNewFirstId} (resp. \textsc{revRenIdGreaterThanNewLastId}) enables to revert the effects of a previously applied \emph{rename} operation on identifiers with $\trm{id}$ such as $\trm{id} < \trm{newFirstId}$ (resp. $\trm{newLastId}$ < $\trm{id}$).

\begin{algorithm}[!ht]
  \footnotesize
  \begin{algorithmic}[1]
      \Function{revRenIdLessThanNewFirstId}{id, firstId, newFirstId}
          \State predNewFirstId $\gets$ createIdFromBase(newFirstId, -1)
          \If{isPrefix(predNewFirstId, id)}
              \State tail $\gets$ getTail(id, 1)
              \If{tail < firstId}
                  \State \Return tail
              \Else
                  \State \Comment{$id$ has been inserted causally after the \emph{rename} op}
                  \State offset $\gets$ getLastOffset(firstId)
                  \State predFirstId $\gets$ createIdFromBase(firstId, offset)
                  \State \Return concat(predFirstId, MAX\_TUPLE, tail)
              \EndIf
          \Else
              \State \Return id
          \EndIf
      \EndFunction
      \\
      \Function{revRenIdGreaterThanNewLastId}{id, lastId}
          \If{id < lastId}
              \State \Comment{$id$ has been inserted causally after the \emph{rename} op}
              \State \Return concat(lastId, MIN\_TUPLE, id)
          \ElsIf{isPrefix(newLastId, id)}
              \State tail $\gets$ getTail(id, 1)
              \If{tail < lastId}
                  \State \Comment{$id$ has been inserted causally after the \emph{rename} op}
                  \State \Return concat(lastId, MIN\_TUPLE, tail)
              \ElsIf{tail < newLastId}
                  \State \Return tail
              \Else
                  \State \Comment{$id$ has been inserted causally after the \emph{rename} op}
                  \State \Return id
              \EndIf
          \Else
              \State \Return id
          \EndIf
      \EndFunction
  \end{algorithmic}
  \caption{Remaining functions to revert an identifier renaming}
  \label{alg:appendix-revert-rename-id}
\end{algorithm}

Like \textsc{renIdLessThanFirstId} and \textsc{renIdGreaterThanFirstId}, these functions do not modify identifiers out of the range of renamed identifiers and of resulting identifiers.
%
To process other cases, the functions study the pattern of the given identifier.
If \textsc{revRenIdLessThanNewFirstId} (resp. \textsc{revRenIdGreaterThanNewLastId}) infers from the value of the identifier that it may come from a concurrent \emph{insert} operation to the \emph{rename} one, \textsc{revRenIdLessThanNewFirstId} (resp. \textsc{revRenIdGreaterThanNewLastId}) simply removes the prefix added by \textsc{renIdLessThanFirstId} (resp. \textsc{renIdGreaterThanLastId}).
%
If the value of an identifier instead indicates that it was inserted causally after the \emph{rename} operation, \textsc{revRenIdLessThanNewFirstId} (resp. \textsc{revRenIdGreaterThanNewLastId}) generates a corresponding identifier at the parent epoch preserving the intended order using $\trm{predFirstId}$ and $\trm{MAX\_TUPLE}$ (resp. $\trm{lastId}$ and $\trm{MIN\_TUPLE}$).
