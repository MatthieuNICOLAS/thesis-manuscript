\Annex{Algorithmes \textsc{renameId}}
\label{app:rename-id}

Nous présentons dans cette annexe le reste des fonctions utilisées par la fonction \texttt{renameId} pour renommer un identifiant d'une époque parente à son époque enfant \cf{alg:renameId}.

Pour rappel, \texttt{renameId} prend comme paramètres initiaux :
\begin{enumerate}
    \item $\trm{id}$, l'identifiant à renommer.
    \item $\trm{renamedIds}$, une collection ordonnée des identifiants composant l'\emph{ancien état}, \ie l'état du noeud auteur de l'opération de renommage à sa génération \cf{def:former-state}.
    \item $\trm{nodeId}$, l'identifiant du noeud auteur de l'opération de renommage.
    \item $\trm{nodeSeq}$, le numéro de séquence du noeud auteur de l'opération de renommage à sa génération.
\end{enumerate}
et repose sur des valeurs particulières :
\begin{enumerate}
    \item $\trm{firstId}$, le premier identifiant de $\trm{renamedIds}$.
    \item $\trm{newFirstId}$, l'identifiant correspondant à $\trm{firstId}$ dans l'époque enfant.
    \item $\trm{lastId}$, le dernier identifiant de $\trm{renamedIds}$.
    \item $\trm{newLastId}$, l'identifiant correspondant à $\trm{lastId}$ dans l'époque enfant.
\end{enumerate}

La fonction \texttt{renIdLessThanFirstId} (resp. \texttt{renIdGreaterThanLastId}) permet de renommer un identifiant $\trm{id}$ tel que $\trm{id} \lid \trm{firstId}$ (resp. $\trm{lastId} \lid \trm{id}$).

\begin{algorithm}[!ht]
  \footnotesize
  \begin{algorithmic}[1]
      \Function{renIdLessThanFirstId}{id $\in \mathbb{I}$, newFirstId $\in \mathbb{I}$}{: $\mathbb{I}$}
      \Statex \Comment $\text{newFirstId} = \newFirstId$
      \Statex \Comment $\text{predNewFirstId} = \predNewFirstId$

      \If{$\text{id} \lid \text{newFirstId}$}
          \State \Return id
      \Else
          \State \Return $\text{predNewFirstId} \oplus \text{id}$ \label{alg:appendix-rename-id-min}
        \EndIf
      \EndFunction
      \\
      \Function{renIdGreaterThanLastId}{id $\in \mathbb{I}$, newLastId $\in \mathbb{I}$}{: $\mathbb{I}$}
          \If{$\text{id} \lid \text{newLastId}$}
              \State \Return $\text{newLastId} \oplus \text{id}$ \label{alg:appendix-rename-id-max}
          \Else
              \State \Return id
          \EndIf
      \EndFunction
  \end{algorithmic}
  \caption{Remaining functions to rename an identifier}
  \label{alg:appendix-rename-id}
\end{algorithm}

L'idée principale derrière ces fonctions est la suivante : \emph{il n'est pas nécessaire de modifier les identifiants qui ne sont pas compris dans l'intervalle affecté par le renommage}.
L'intervalle affecté par le renommage est le suivant :
\begin{equation*}
    ]\trm{min}(\trm{firstId},\trm{predNewFirstId}\footnote{La borne inférieure des identifiants générés par \texttt{renameId} (cf. \autoref{alg:appendix-rename-id}, ligne \ref{alg:appendix-rename-id-min}).}),\trm{max}(\trm{lastId},\trm{succNewLastId}\footnote{La borne supérieure des identifiants générés par \texttt{renameId} (cf. \autoref{alg:appendix-rename-id}, ligne \ref{alg:appendix-rename-id-max}).})[
\end{equation*}

Dans le cas où l'identifiant à renommer appartient à cet intervalle, ces fonctions utilisent le même procédé que \texttt{renIdFromPredId} pour le renommer : elles ajoutent à ce dernier un préfixe adapté pour préserver l'ordre souhaité :
\begin{enumerate}
    \item \texttt{renIdLessThanFirstId} utilise $\trm{predNewFirstId}$, le prédecesseur de $\trm{newFirstId}$.
    \item \texttt{renIdGreaterThanLastId} utilise $\trm{newLastId}$.
\end{enumerate}
