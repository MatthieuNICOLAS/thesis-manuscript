\Annex{Entrelacement d'insertions concurrentes dans Treedoc}
\label{app:treedoc-interleaving}

Dans cette annexe, nous présentons un contre-exemple à la conjecture présentée dans \cite{2019-interleaving-anomalies-collaborative-editors-kleppmann}.
Cette conjecture suppose que Treedoc ne souffre pas d'anomalies d'entrelacement, \ie que des éléments insérés en concurrence ne se retrouvent pas entrelacés après l'intégration de l'ensemble des modifications.
La \autoref{fig:treedoc-interleaving} illustre notre contre-exemple.

\begin{figure}[!ht]

  \centering
  \resizebox{\columnwidth}{!}{
    \begin{tikzpicture}[level distance=80, level/.style={sibling distance=80/#1}]
      \newcommand\treedocnode[2]{
        node[rounded corners, matrix, draw] {
          \node {$d_#1$};           \\
          \node[circle, draw] {#2}; \\
        }
      }

      \newcommand\nodeh{ \treedocnode{1}{H} }
      \newcommand\nodee{ \treedocnode{2}{E} }
      \newcommand\nodeo{ \treedocnode{3}{O} }
      \newcommand\nodel{ \treedocnode{4}{L} }

      \newcommand\treedocsupernode[4]{
        node[rounded corners, matrix, draw, ampersand replacement=\&, column sep=3] {
          \node {$d_#1$};            \& \node {$d_#3$};           \\
          \node[circle, draw] {#2}; \& \node[circle, draw] {#4}; \\
        }
      }

      \newcommand\nodeeo{ \treedocsupernode{2}{E}{3}{O} }

      \newcommand\initialstate[2]{
        \path
          #1
          ++#2
          \nodeh
      }

      \newcommand\he[2]{
        \path
          #1
          ++#2
          \nodeh
            child[missing]
            child {
              \nodee edge from parent node[right] {1}
            };
      }

      \newcommand\hel[2]{
        \path
          #1
          ++#2
          \nodeh
            child[missing]
            child {
              \nodee
                child[missing]
                child {
                  \nodel edge from parent node[right] {1}
                }
              edge from parent node[right] {1}
            };
      }

      \newcommand\ho[2]{
        \path
          #1
          ++#2
          \nodeh
            child[missing]
            child {
              \nodeo edge from parent node[right] {1}
            };
      }

      \newcommand\finalstate[2]{
        \path
          #1
          ++#2
          \nodeh
            child[missing]
            child {
              \nodeeo
                child[missing]
                child {\nodel edge from parent node[right] {1}}
                edge from parent node[right] {1}
            };
      }

      \newcommand\offseta{ (90:7) }
      \newcommand\offsetb{ (270:1.5) }

      \path
          node {\textbf{A}}
          ++(0:0.5) node (a) {}
          +(0:28) node (a-end) {}
          +(0:2) node[point] (a-initial) {}
          +(0:6) node[point, label=-170:{$\trm{ins}(H \prec E)$}, label={[xshift=1em]-10:{$\trm{ins}(\langle 1,d_2 \rangle,e)$}}] (a-ins-e) {}
          +(0:14) node[point, label=-170:{$\trm{ins}(E \prec L)$}, label={[xshift=1em]-10:{$\trm{ins}(1 \oplus \langle 1,d_4 \rangle,L)$}}] (a-ins-l) {}
          +(0:20) node[point, label=-170:{$\trm{sync}$}] (a-send-sync) {}
          +(0:24) node[point] (a-recv-sync) {}
          +(0:26) node (a-final) {};


      \initialstate{(a-initial)}{\offseta};
      \he{(a-ins-e)}{\offseta};
      \hel{(a-ins-l)}{\offseta};
      \finalstate{(a-recv-sync)}{\offseta};

      \draw[dotted] (a) -- (a-initial) (a-final) -- (a-end);
      \draw[->, thick] (a-initial) --  (a-ins-e) --  (a-ins-l) -- (a-send-sync) -- (a-recv-sync) -- (a-final);

      \path
          ++(270:3) node {\textbf{B}}
          ++(0:0.5) node (b) {}
          +(0:28) node (b-end) {}
          +(0:2) node[point] (b-initial) {}
          +(0:6) node[point, label=170:{$\trm{ins}(H \prec O)$}, label={[xshift=1em]10:{$\trm{ins}(\langle 1, d_3 \rangle,O)$}}] (b-ins-o) {}
          +(0:20) node[point, label=170:{$\trm{sync}$}] (b-send-sync) {}
          +(0:24) node[point] (b-recv-sync) {}
          +(0:26) node (b-final) {};

      \initialstate{(b-initial)}{\offsetb};
      \ho{(b-ins-o)}{\offsetb};
      \finalstate{(b-recv-sync)}{\offsetb};

      \draw[dotted] (b) -- (b-initial) (b-final) -- (b-end);
      \draw[->, thick] (b-initial) --  (b-ins-o) -- (b-send-sync) -- (b-recv-sync) -- (b-final);

      \draw[->, dashed, shorten >= 1] (a-send-sync) -- (b-recv-sync);
      \draw[->, dashed, shorten >= 1] (b-send-sync) -- (a-recv-sync);
    \end{tikzpicture}
  }
  \caption{Entrelacement d'insertions concurrentes dans une séquence Treedoc}
  \label{fig:treedoc-interleaving}
\end{figure}

Dans cet exemple, deux noeuds A et B partagent et éditent collaborativement une séquence répliquée Treedoc.
Initialement, ils possèdent le même état : la séquence contient l'élément "H".

Le noeud A insère l'élément "E" en fin de séquence, \ie $\trm{insert}(H \prec E)$.
Treedoc génère l'opération correspondante, $\trm{insert}(\langle 1,d_2 \rangle, E)$, et l'intègre à sa copie locale.
Puis A insère l'élément "L", toujours en fin de séquence.
La modification $\trm{insert}(E \prec L)$ est convertie en opération $\trm{insert}(1 \oplus \langle 1,d_4 \rangle, L)$ et intégrée.

En concurrence, le noeud B insère l'élément "O" en fin de séquence.
Cette modification résulte en l'opération $\trm{insert}(\langle 1,d_3 \rangle, O)$, qui est intégrée.

Les deux noeuds procèdent ensuite à une synchronisation, échangeant leurs opérations respectives.
Lorsque A (resp. B) intègre $\trm{insert}(\langle 1,d_3 \rangle, O)$ (resp. $\trm{insert}(\langle 1,d_2 \rangle, E)$), il ajoute cet élément avec son désambiguateur dans son noeud de chemin $1$, après (resp. avant) l'élément existant (on considère que $d_2 < d_3$).

Ainsi, l'état obtenu après l'intégration de l'ensemble des modifications par les noeuds A et B est "HEOL".
L'élément "O", inséré par le noeud B en concurrence de l'insertion des éléments "E" et "L" par le noeud A, se retrouve inséré entre ces derniers.

Nous présentons donc dans cet exemple un entrelacement des insertions concurrentes, contredisant la conjecture de \cite{2019-interleaving-anomalies-collaborative-editors-kleppmann}.
