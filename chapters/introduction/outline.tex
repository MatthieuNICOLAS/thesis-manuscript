
Ce manuscrit de thèse est organisé de la manière suivante :

Dans le \autoref{chap:state-of-art}, nous introduisons le modèle du système que nous considérons, \ie les systèmes \ac{P2P} à large échelle sujets au churn et sans autorités centrales.
Puis nous présentons dans ce chapitre l'état de l'art des mécanismes de résolution de conflits automatiques utilisés dans les systèmes adoptant le paradigme de la réplication optimiste.
À partir de cet état de l'art, nous identifions et motivons notre problématique de recherche, \ie l'absence de mécanisme adapté aux systèmes \ac{P2P} à large échelle sujets au churn permettant de réduire le surcoût induit par les mécanismes de résolution de conflits automatiques pour le type Séquence.

Dans le \autoref{chap:renamablelogootsplit}, nous présentons notre approche pour présenter un tel mécanisme, \ie un mécanisme de résolution de conflits automatiques pour le type Séquence auquel nous associons un mécanisme de \ac{GC} de son surcoût ne nécessitant pas de coordination synchrone entre les noeuds du système.
Nous détaillons le fonctionnement de notre approche, sa validation par le biais d'une évaluation empirique puis comparons notre approche par rapport aux approches existantes
Finalement, nous concluons la présentation de notre approche en identifiant et en détaillant plusieurs de ses limites.

Dans le \autoref{chap:mute}, nous présentons \ac{MUTE}, l'éditeur de texte collaboratif temps réel \ac{P2P} chiffré de bout en bout que notre équipe de recherche développe dans le cadre de ses travaux de recherche.
Nous présentons les différentes couches logicielles formant un pair et les services tiers avec lesquels les pairs interagissent, et détaillons nos travaux dans le cadre de ce projet, \ie l'intégration de notre mécanisme de résolution de conflits automatiques pour le type Séquence et le développement de la couche de livraison des messages associée.
Pour chaque couche logicielle, nous identifions ses limites et présentons de potentielles pistes d'améliorations.

Finalement, nous récapitulons dans le \autoref{chap:conclusions-perspectives} les contributions réalisées dans le cadre de cette thèse.
Puis nous clotûrons ce manuscrit en introduisant plusieurs des pistes de recherches que nous souhaiterons explorer dans le cadre de nos travaux futurs.
