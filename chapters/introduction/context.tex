\begin{itemize}
    \item Systèmes collaboratifs (wikis, plateformes de contenu, réseaux sociaux) et leurs bienfaits (qualité de l'info, vitesse de l'info (exemple de crise ?), diffusion de la parole).
      Démocratisation (sic) de ces systèmes au cours de la dernière décennie.
    \item En raison du volume de données et de requêtes, adoptent architecture décentralisée.
      Permet ainsi de garantir disponibilité, tolérance aux pannes et capacité de passage à l'échelle.
    \item Mais échoue à adresser problèmes non-techniques : confidentialité, souveraineté, protection contre censure, dépendance et nécessité de confiance envers autorité centrale.
    \item À l'heure où les entreprises derrière ces systèmes font preuve d'ingérence et d'intérêts contraires à ceux de leurs utilisateur-rices (Cambridge Analytica, Prism, non-modération/mise en avant de contenus racistes \footnote{\emph{Algorithms of Oppression}, Safiya Umoja Noble}\footnote{\url{https://www.researchgate.net/publication/342113147_The_YouTube_Algorithm_and_the_Alt-Right_Filter_Bubble}}\footnote{\url{https://www.wsj.com/articles/the-facebook-files-11631713039}}, invisibilisation de contenus féministes, dissolution du comité d'éthique de Google\footnote{\url{https://www.bbc.com/news/technology-56135817}}, inégalité d'accès à la méta-machine affectante\footnote{\emph{Je suis une fille sans histoire}, Alice Zeniter, p. 75}\footnote{Qui cite \emph{Les affects de la politique}, Frédéric Lordon}\footnote{\url{https://www.bbc.com/news/technology-59011271}}), parait fondamental de proposer les moyens technologiques accessibles pour concevoir et déployer des alternatives.
    \item \mnnote{TODO: Voir si angle écologique/réduction consommation d'énergie peut être pertinent.}
    \item Systèmes pair-à-pair sont une direction intéressante pour répondre à ces problématiques, de part leur absence d'autorité centrale, la distribution des tâches et leur conception mettant le pair au centre.
      Mais posent de nouvelles problématiques de recherche.
    \item Ces systèmes ne disposent d'aucun contrôle sur les noeuds qui les composent.
      Le nombre de noeuds peut donc croître de manière non-bornée et atteindre des centaines de milliers de noeuds.
      La complexité des algorithmes de ces systèmes ne doit donc pas dépendre de ce paramètre, ou alors de manière logarithmique.
    \item De plus, ces noeuds n'offrent aucune garantie sur leur stabilité.
      Ils peuvent donc rejoindre et participer au système de manière éphèmère.
      S'agit du phénomène connu sous le nom de churn.
      Les algorithmes de ces systèmes ne peuvent donc pas reposer sur des mécanismes nécessitant une coordination synchrone d'une proportion des noeuds.
    \item Finalement, ces noeuds n'offrent aucune garanties sur leur fiabilité et intentions.
      Les noeuds peuvent se comporter de manière byzantine.
      Pour assurer la confidentialité, l'absence de confiance requise et le bon fonctionnement du système, ce dernier doit être conçu pour résister aux comportements byzantins de ses acteurs.
    \item Ainsi, il est nécessaire de faire progresser les technologies existantes pour les rendre compatible avec ce nouveau modèle de système.
      Dans le cadre de cette thèse, nous nous intéressons aux mécanismes de réplication de données dans les systèmes collaboratifs pair-à-pair temps réel.
\end{itemize}
