\subsection{Ré-identification sans coordination synchrone pour les \acp{CRDT} pour le type Séquence}
\label{sec:research-questions-rls}

Les \acfp{CRDT} \cite{2007-crdt-shapiro,shapiro_2011_crdt} sont des types de données répliqués.
Ils sont conçus pour permettre à un ensemble de noeuds d'un système de répliquer une donnée et pour leur permettre de la consulter, de la modifier sans aucune coordination préalable et d'assurer à terme la convergence des copies.
Dans ce but, les \acp{CRDT} incorporent des mécanismes de résolution de conflits automatiques directement au sein leur spécification.

Cependant, ces mécanimes induisent un surcoût, aussi bien en termes de métadonnées et de calculs que de bande-passante.
Ces surcoûts sont néanmoins jugés acceptables par la communauté pour une variété de types de données, \eg le Registre ou l'Ensemble.
Cependant, le surcoût des \acp{CRDT} pour le type Séquence constitue toujours une problématique de recherche.

En effet, la particuliarité des \acp{CRDT} pour le type Séquence est que leur surcoût croît de manière monotone au cours de la durée de vie de la donnée, \ie au fur et à mesure des modifications effectuées.
Le surcoût introduit par les \acp{CRDT} pour ce type de données se révèle donc handicapant dans le contexte de collaborations sur de longues durées ou à large échelle.

De manière plus précise, le surcoût des \acp{CRDT} pour le type Séquence provient de la croissance des métadonnées utilisées par leur mécanisme de résolution de conflits automatique.
Ces métadonnées correspondent à des identifiants qui sont associés aux éléments de la Séquence.
Ces identifiants permettent de résoudre les conflits, \eg en précisant quel est l'élement à supprimer ou en spécifiant la position d'un nouvel élément à insérer par rapport aux autres.

Plusieurs approches ont été proposées pour réduire le coût induit par ces identifiants.
Notamment, \cite{letia:hal-01248270,zawirski:hal-01248197} proposent un mécanisme de ré-assignation des identifiants pour réduire leur coût a posteriori.
Ce mécanisme génère toutefois des conflits en cas de modifications concurrentes de la séquence, \ie l'insertion ou la suppression d'un élément.
Les auteurs résolvent ce problème en proposant un mécanisme de transformation des modifications concurrentes par rapport à l'effet du mécanisme de ré-assignation des identifiants.

Cependant, l'exécution en concurrence du mécanisme de ré-assignation des identifiants par plusieurs noeuds provoque elle-même un conflit.
Pour éviter ce dernier type de conflit, les auteurs choisissent de subordonner à un algorithme de consensus l'exécution du mécanisme de ré-assignation des identifiants.
Ainsi, le mécanisme de ré-assignation des identifiants ne peut être déclenché en concurrence par plusieurs noeuds du systèmes.

Comme nous l'avons évoqué précédemment, reposer sur un algorithme de consensus qui requiert une coordination synchrone entre une proportion de noeuds du système est une contrainte incompatible avec les systèmes \ac{P2P} à large échelle sujets au churn.\\

Notre problématique de recherche est donc la suivante : \emph{pouvons-nous proposer un mécanisme sans coordination synchrone de réduction du surcoût des \acp{CRDT} pour Séquence, \ie adapté aux applications \acp{LFS} ?}\\

Pour répondre à cette problématiquee, nous proposons RenamableLogootSplit, un nouveau \ac{CRDT} pour le type Séquence.
Ce \ac{CRDT} intègre un mécanisme de ré-assignation des identifiants, dit de renommage, directement au sein de sa spécification.
Nous associons au mécanisme de renommage un mécanisme de résolution de conflits automatique additionnel pour gérer ses exécutions concurrentes.
Ainsi, nous proposons un \ac{CRDT} pour le type Séquence dont le surcoût est périodiquement réduit par le biais d'un mécanisme n'introduisant aucune contrainte de coordination synchrone entre les noeuds du système.

\subsection{Éditeur de texte collaboratif \ac{P2P} temps réel chiffré de bout en bout}
\label{sec:research-questions-mute}

Comme évoqué précédemment, la conception d'applications \acp{LFS} à large échelle présente un ensemble de problématiques issues de domaines variés, \eg
\begin{enumerate}
    \item Comment permettre aux utilisateur-rices de collaborer en l'absence d'autorités centrales pour résoudre les conflits de modifications ?
    \item Comment authentifier les utilisateur-rices en l'absence d'autorités centrales ?
    \item Comment structurer le réseau de manière efficace, \ie en limitant le nombre de connexions par pair ?
\end{enumerate}

Cet ensemble de questions peut être résumé en la problématique suivante : \emph{pouvons-nous concevoir une application \ac{LFS} à large échelle, sûre et sans autorités centrales ?}\\

Pour étudier cette problématique, l'équipe Coast développe l'application \acf{MUTE}\footnote{Disponible à l'adresse : \url{https://mutehost.loria.fr}} \cite{MUTE2017}.
Il s'agit d'un \ac{PoC} d'éditeur de texte web collaboratif \ac{P2P} temps réel chiffré de bout en bout.

Ce projet permet à l'équipe de présenter ses travaux de recherche portant sur les mécanismes de résolutions de conflits automatiques pour le type Séquence \cite{2013-logootsplit,2021-these-vic,2022-rls-tpds-nicolas} et les mécanismes d'authentification des pairs dans les systèmes sans autorités centrales \cite{2018-trusternity-short,2018-trusternity-long}.

De plus, en inscrivant ses travaux dans le cadre d'un système complet, ce projet permet à l'équipe d'identifier de nouvelles problématiques en relation avec les nombreux domaines de recherche nécessaires à la conception d'un tel système, \ie le domaine des protocoles d'appartenance aux groupes \cite{swim2002, lifeguard2018}, des topologies réseaux \ac{P2P} \cite{2018-spray-nedelec} ou encore des protocoles d'établissement de clés de chiffrement de groupe \cite{1995-burmester-desmedt}.\\

Dans le cadre de notre thèse, nous avons contribué au développement de ce projet.
Nous avons notamment implémenté plusieurs \acp{CRDT} pour le type Séquence \cite{2013-logootsplit,2022-rls-tpds-nicolas} et le protocole d'appartenance au réseau SWIM \cite{swim2002}.
