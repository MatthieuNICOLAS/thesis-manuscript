Pour se connecter, les noeuds doivent s'échanger plusieurs informations logicielles et matérielles, notamment leur adresse IP publique respective.
Cependant, un noeud n'a pas accès à cette donnée lorsque son routeur utilise le protocole NAT.
Le noeud doit alors la récupérer.

Pour permettre aux noeuds de découvrir leur adresse IP publique, \ac{WebRTC} repose sur le protocole STUN.
Ce protocole consiste simplement à contacter un serveur tiers dédié à cet effet.
Ce serveur retourne en réponse au noeud qui le contacte son adresse IP publique.
