Il est possible que des noeuds provenant de réseaux différents ne puissent établir une connexion \ac{P2P} directe entre eux, par exemple à cause de restrictions imposées par leur pare-feux respectifs.
Pour contourner ce cas de figure, \ac{WebRTC} utilise le protocole TURN.

Ce protocole consiste à utiliser un serveur tiers comme relais entre les noeuds.
Ainsi, les noeuds peuvent communiquer par son intermédiaire tout au long de la collaboration.
Les échanges sont chiffrés, afin que le serveur TURN ne représente pas une faille de sécurité.
