\label{sec:swim-failure-detection}

Le mécanisme de détection des défaillances des pairs est executé de manière périodique, toutes les $T$ unités de temps, par chacun des noeuds du système de manière non-coordonnée.
Son fonctionnement est illustré par la \autoref{fig:swim-failure-detection}.

\begin{figure}[!ht]
  \centering
  \resizebox{\columnwidth}{!}{
    \begin{tikzpicture}
      \path
        node {\textbf{A}}
        +(0:26) node (a-end) {}
        ++(0:0.5) node (a0) {}
        ++(0:15) node[point] (a-receives-c-pingreq) {}
        ++(0:6) node [point] (a-receives-b-ack) {};

      \draw[thick] (a0) --  (a-receives-c-pingreq) -- (a-receives-b-ack) -- (a-end);

      \path
          ++(270:2) node {\textbf{B}}
          +(0:26) node (b-end) {}
          ++(0:0.5) node (b0) {}
          ++(0:6) node[point] (b-receives-c-ping) {}
          ++(0:12) node[point] (b-receives-a-ping) {};

      \draw[thick] (b0) -- (b-receives-c-ping) -- (b-receives-a-ping) -- (b-end);

      \path
          ++(270:4) node {\textbf{C}}
          +(0:26) node (c-end) {}
          ++(0:0.5) node (c0) {}
          ++(0:3) node[point, label=below:{\emph{choose B to ping}}] (c-ping) {}
          ++(0:6) node (c-receives) {}
          +(90:0.5) node[cross] (fail-ack) {}
          ++(0:3) node[point, label=below:{\emph{choose A to ping-req B}}] (c-pingreq) {}
          ++(0:12) node[point] (c-receives-a-ack) {};

      \draw[thick] (c0) -- (c-ping) -- (c-pingreq) --  (c-receives-a-ack) -- (c-end);

      \draw[->, dashed, shorten >= 1]
        (a-receives-c-pingreq.east) edge node[below left] {\emph{ping}} (b-receives-a-ping.west)
        (a-receives-b-ack.east) edge node[near end, below left] {\emph{ack B}} (c-receives-a-ack)
        (b-receives-c-ping.east) edge node[below left] {\emph{ack}} (fail-ack.west)
        (b-receives-a-ping.east) edge node[above left] {\emph{ack}} (a-receives-b-ack.west)
        (c-ping.east) edge node[above left] {\emph{ping}} (b-receives-c-ping.west)
        (c-pingreq.east) edge node[near end, above left] {\emph{ping-req B}} (a-receives-c-pingreq.west);

    \end{tikzpicture}
  }
  \caption{Exécution du mécanisme de détection des défaillances par le noeud C pour tester le noeud B}
  \label{fig:swim-failure-detection}
\end{figure}

Dans cet exemple, le réseau est composé des trois noeuds A, B et C.
Le noeud C démarre l'exécution du mécanisme de détection des défaillances.

Tout d'abord, le noeud C sélectionne un noeud cible de manière aléatoire, ici B, et lui envoie un message \emph{ping}.
À la réception de ce message, le noeud B lui signifie qu'il est toujours opérationnel en lui répondant avec un message \emph{ack}.
À la réception de ce message par C, cette exécution du mécanisme de détection des défaillances prendrait fin.
Mais dans l'exemple présenté ici, ce message est perdu par le réseau.

En l'absence de réponse de la part de B au bout d'un temps spécifié au préalable, le noeud C passe à l'étape suivante du mécanisme.
Le noeud C sélectionne un autre noeud, ici A, et lui demande de vérifier via le message \emph{ping-req B} si B a eu une défaillance.
À la réception de la requête de ping, le noeud A envoie un message \emph{ping} à B.
Comme précédemment, B répond au \emph{ping} par le biais d'un \emph{ack} à A.
A informe alors C du bon fonctionnement du B via le message \emph{ack B}.
Le mécanisme prend alors fin, jusqu'à sa prochaine exécution.

Si C n'avait pas reçu de réponse suite à sa \emph{ping-req B} envoyée à A, C aurait supposé que B a eu une défaillance.
Afin de réduire le taux de faux positifs, SWIM ne considère pas directement les noeuds n'ayant pas répondu comme en panne : ils sont tout d'abord \emph{suspectés} d'être en panne.
Après un certain temps sans signe de vie d'un noeud suspecté d'être en panne, le noeud est \emph{confirmé} comme défaillant.

L'information qu'un noeud est suspecté d'être en panne est propagé dans le réseau via le mécanisme de dissémination des mises à jour du groupe décrit ci-dessous.
Si un noeud apprend qu'il est suspecté d'une panne, il dissémine à son tour l'information qu'il est toujours opérationnel pour éviter d'être confirmé comme défaillant.

Pour éviter qu'un message antérieur n'invalide une suspicion d'une défaillance et retarde ainsi sa détection, SWIM introduit un numéro d'\emph{incarnation}.
Chaque noeud maintient un numéro d'incarnation.
Lorsqu'un noeud apprend qu'il est suspecté d'une panne, il incrémente son numéro d'incarnation avant de propager l'information contradictoire.

Ainsi, afin de représenter la liste des collaborateur-rices, le protocole SWIM utilise la structure de données présentée par la \autoref{def:collaborator-list} :

\begin{definition}[Liste des collaborateur-rices]
  \label{def:collaborator-list}
  La \emph{liste des collaborateur-rices} est un ensemble de triplets $\langle \trm{nodeId},\trm{nodeStatus},\trm{nodeIncarn} \rangle$ où
  \begin{itemize}
    \item $\trm{nodeId}$ correspond à l'identifiant du noeud correspondant à ce tuple.
    \item $\trm{nodeStatus}$ correspond au statut courant du noeud correspondant à ce tuple, \ie $\trm{Alive}$ s'il est considéré comme opérationnel, $\trm{Suspect}$ s'il est suspecté d'une défaillance, $\trm{Confirm}$ s'il est considéré comme défaillant.
    \item $\trm{nodeIncarn}$ correspond au numéro d'incarnation maximal, \ie le plus récent, connu pour le noeud correspondant à ce tuple.
  \end{itemize}
\end{definition}

Chaque noeud réplique cette liste et la fait évoluer au cours de l'exécution du mécanisme présenté jusqu'ici.
Lorsqu'une mise à jour est effectuée, celle-ci est diffusée de la manière présentée ci-dessous.
