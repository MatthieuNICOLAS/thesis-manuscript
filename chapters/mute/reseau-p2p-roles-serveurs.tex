Ainsi, \ac{WebRTC} implique l'utilisation de plusieurs serveurs.

Les serveurs de signalisation et STUN sont nécessaires pour permettre à de nouveaux noeuds de rejoindre la collaboration.
Autrement dit, leur rôle est ponctuel : une fois le réseau \ac{P2P} établit, les noeuds n'ont plus besoin d'eux.
Ces serveurs peuvent alors être coupés sans impacter la collaboration.

À l'inverse, les serveurs TURN jouent un rôle plus prédominant dans la collaboration.
Ils sont nécessaires dès lors que des noeuds proviennent de réseaux différents et sont alors requis tout au long de la collaboration.
Une panne de ces derniers entraverait la collaboration puisqu'elle résulterait en une partition des noeuds.
Il est donc primordial de s'assurer de la disponibilité et fiabilité de ces serveurs.
