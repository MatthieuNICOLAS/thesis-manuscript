Nous avons ensuite apporté plusieurs modifications à la version du protocole SWIM présentée dans \cite{swim2002}.
Notre première modification porte sur l'ordre de priorité entre les états d'un pair.\\

\textbf{Modification de l'ordre de précédence.}
Dans la version originale, un pair désigné comme défaillant l'est de manière irrévocable.
Ce comportement est dû au fait que la relation d'ordre $\ltuple$ utilise d'abord les valeurs de $\trm{nodeStatus}$ pour ordonner deux états pour un noeud donné.
Ce n'est seulement qu'en cas d'égalité que $\ltuple$ considère les valeurs de $\trm{nodeIncarn}$.
Ainsi, un noeud déclaré comme défaillant par un autre noeud doit changer d'identité pour rejoindre de nouveau le groupe.

Ce choix n'est cependant pas anodin : il implique que la taille de la liste des collaborateur-rices croît de manière linéaire avec le nombre de connexions.
S'agissant du paramètre avec le plus grand ordre de grandeur de l'application, nous avons cherché à le diminuer.

Nous avons donc modifié la relation d'ordre $\ltuple$ de la manière suivante :

\begin{definition}[Relation $\ltupleupdated$]
    Étant donné deux tuples $t = \langle \trm{nodeStatus},\trm{nodeIncarn} \rangle$ et $t' = \langle \trm{nodeStatus'},\trm{nodeIncarn'} \rangle$, nous avons :
    \begin{equation*}
      \begin{split}
        t \ltupleupdated t' \quad \trm{iff} \quad  & (\trm{nodeIncarn} < \trm{nodeIncarn'}) \quad \lor \\
                                            & (\trm{nodeIncarn} = \trm{nodeIncarn'} \land \trm{nodeStatus'} \lstatus \trm{nodeStatus'}) \\
      \end{split}
    \end{equation*}
  \end{definition}

Ces modifications permettent de donner la précédence au numéro d'incarnation, et d'utiliser le statut du collaborateur pour trancher seulement en cas d'égalité par rapport au numéro d'incarnation actuel.
Ceci permet à un noeud auparavant déclaré comme défaillant de revenir dans le groupe en incrémentant son numéro d'incarnation.
La taille de la liste des collaborateur-rices devient dès lors linéaire par rapport au nombre de noeuds.

Ces modifications n'ont pas d'impact sur la convergence des listes des collaborateur-rices des différents noeuds.
Une étude approfondie reste néanmoins à effectuer pour déterminer si ces modifications ont un impact sur la vitesse à laquelle un noeud défaillant est déterminé comme tel par l'ensemble des noeuds.\\

\textbf{Ajout d'un mécanisme de synchronisation.}
La seconde modification que nous avons effectué concerne l'ajout d'un mécanisme de synchronisation entre pairs.
En effet, le papier ne précise pas de procédure particulière lorsqu'un nouveau pair rejoint le réseau.
Pour obtenir la liste des collaborateur-rices, ce dernier doit donc la demander à un autre pair.

Nous avons donc implémenté pour la liste des collaborateur-rices un mécanisme d'anti-entropie : à sa connexion, puis de manière périodique, un noeud envoie une requête de synchronisation à un noeud cible choisi de manière aléatoire.
Ce message sert aussi à transmettre l'état courant du noeud source au noeud cible.
En réponse, le noeud cible lui envoie l'état courant de sa liste.
À la réception de cette dernière, le noeud source fusionne la liste reçue avec sa propre liste.
Cette fusion conserve l'entrée la plus récente pour chaque noeud.\\

Pour récapituler, les mises à jour du groupe sont diffusées de manière atomique de façon épidémique, en utilisant les messages du mécanisme de détection des défaillances des noeuds.
De manière additionnelle, un mécanisme d'anti-entropie permet à deux noeuds de synchroniser leur état.
Ce mécanisme nous permet de pallier les défaillances éventuelles du réseau.
Ainsi, dans les faits, nous avons mis en place un \ac{CRDT} synchronisé par différences d'états \cf{sec:delta-crdts} pour la liste des collaborateur-rices.
