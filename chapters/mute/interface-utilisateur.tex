Comme illustré par la \autoref{fig:interface-mute-editor}, l'interface de la page d'un document se compose principalement d'un éditeur de texte.
Ce dernier supporte le langage de balisage Markdown \cite{2004-markdown}.
Ainsi, l'éditeur permet d'inclure plusieurs éléments légers de style.
Les balises du langage Markdown étant du texte, elles sont répliquées nativement par le \ac{CRDT} utilisé en interne par \ac{MUTE} pour représenter la séquence.\\

L'interface de la page de l'éditeur de document est agrémentée de plusieurs mécanismes permettant d'établir une conscience de groupe entre les collaborateur-rices.
L'indicateur en haut à droite de la page représente le statut de connexion de l'utilisateur-rice.
Celui-ci permet d'indiquer à l'utilisateur-rice s'iel est actuellement connecté-e au réseau \ac{P2P}, en cours de connexion, ou si un incident réseau a lieu.

De plus, \ac{MUTE} affiche sur la droite de l'éditeur la liste des collaborateur-rices actuellement connecté-es.
Un curseur ou une sélection distante est associée pour chaque membre de la liste.
Ces informations permettent d'indiquer à l'utilisateur-rice dans quelles sections du document ses collaborateur-rices sont en train de travailler.
Ainsi, iels peuvent se répartir la rédaction du document de manière implicite ou suivre facilement les modifications d'un-e collaborateur-rice.\\

Bien que fonctionnelle, cette interface souffre néanmoins de plusieurs limites.
Notamment, nous n'avons pas encore pu étudier la littérature concernant les mécanismes de conscience pour supporter la collaboration, au-delà du système de curseurs distants.

Nous identifions ainsi plusieurs axes de travail pour ces mécanismes.
Tout d'abord, l'axe des \emph{mécanismes de conscience des changements}.
Le but serait de proposer des mécanismes pour :
\begin{enumerate}
    \item Mettre en lumière de manière intelligible les modifications effectuées par les collabateur-rices dans le cadre de collaborations temps réel à large échelle.
        Un tel mécanisme représente un défi de part le débit important de changements, potentiellement à plusieurs endroits du document de manière quasi-simultanée, à présenter à l'utilisateur-rice.
    \item Mettre en lumière de manière intelligible les modifications effectuées par les collaborateur-rices dans le cadre de collaborations asynchrones.
        De nouveau, ce mécanisme représente un défi de part la quantité massive de changements, une fois encore potentiellement à plusieurs endroits du document, à présenter à l'utilisateur-rice.
\end{enumerate}
Une piste de travail potentiellement liée serait l'ajout d'une fonctionnalité d'historique du document, permettant aux utilisateur-rices de parcourir ses différentes versions obtenues au fur et à mesure des modifications.
L'intégration d'une telle fonctionnalité dans un éditeur \ac{P2P} pose cependant plusieurs questions : quel historique présenter aux utilisateur-rices, sachant que chacun-e a potentiellement observé un ordre différent des modifications ?
Doit-on convenir d'une seule version de l'historique ?
Dans ce cas, comment choisir et construire cet historique ?\\

Le second axe de travail sur les mécanismes de conscience concerne les \emph{mécanismes de conscience de groupe}.
Actuellement, nous affichons l'ensemble des collaborateur-rices actuellement connecté-es.
Cette approche s'avère lourde voire entravante dans le cadre de collaborations à large échelle où le nombre de collaborateur-rices dépasse plusieurs centaines.
Il convient donc de déterminer quelles informations présenter à l'utilisateur-rice dans cette situation, \eg une liste compacte de pairs et leur curseur respectif, ainsi que le nombre de pairs total.\\

Ainsi, pour chacun de ses axes d'amélioration, il convient d'étudier leur littérature respective et de déterminer les solutions proposées qui sont adaptées à \ac{MUTE}.
