Toujours dans le but d'offrir des fonctionnalités de conscience de groupe aux utilisateurs pour leur permettre de se coordonner aisément, nous avons implémenté dans MUTE l'affichage des curseurs distants.

Pour représenter fidèlement la position des curseurs des collaborateur-rices distants, nous nous reposons sur les identifiants du \ac{CRDT} choisi pour représenter la séquence.
Le fonctionnement est similaire à la gestion des modifications du document : lorsque l'éditeur indique que l'utilisateur a déplacé son curseur, nous récupérons son nouvel index.
Nous recherchons ensuite l'identifiant correspondant à cet index dans la séquence répliquée et le diffusons aux collaborateur-rices.

À la réception de la position d'un curseur distant, nous récupérons l'index correspondant à cet identifiant dans la séquence répliquée et représentons un curseur à cet index.
Il est intéressant de noter que si l'identifiant a été supprimé en concurrence, nous pouvons à la place récupérer l'index de l'élément précédent et ainsi indiquer à l'utilisateur où son collaborateur est actuellement en train de travailler.

De façon similaire, nous gérons les sélections de texte à l'aide de deux curseurs : un curseur de début et un curseur de fin de sélection.
