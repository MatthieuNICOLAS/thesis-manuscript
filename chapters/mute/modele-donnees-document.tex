\label{sec:mute-crdts}

MUTE propose plusieurs alternatives pour représenter le document texte.
MUTE permet de soit utiliser une implémentation de LogootSplit\footnote{Les deux implémentations proviennent de la librairie \texttt{mute-structs} : \url{https://github.com/coast-team/mute-structs}} \cf{sec:logootsplit}, soit de RenamableLogootSplit\footnotemark[\value{footnote}] \cf{chap:renamablelogootsplit} ou soit de Dotted LogootSplit \footnote{Implémentation fournie par la librairie suivante : \url{https://github.com/coast-team/dotted-logootsplit}} \cite{2021-these-vic}.
Ce choix est effectué via une valeur de configuration de l'application choisie au moment de son déploiement.\\

Le modèle de données utilisé interagit avec l'éditeur de texte par l'intermédiaire d'opérations texte, \ie de messages au format $\langle \trm{insert}, \trm{index},\trm{elts} \rangle$ ou $\langle \trm{remove}, \trm{index}, \trm{length} \rangle$.
Lorsque l'utilisateur effectue des modifications locales, celles-ci sont détectées par l'éditeur et mises sous la forme d'opérations texte.
Elles sont transmises au modèle de données, qui les intègre alors à la structure de données répliquées.
Le \ac{CRDT} retourne en résultat l'opération distante à propager aux autres noeuds.

De manière complémentaire, lorsqu'une opération distante est livrée au modèle de données, elle est intégrée par le \ac{CRDT} pour actualiser son état.
Le \ac{CRDT} génère les opérations texte correspondantes et les transmet à l'éditeur de texte pour mettre à jour la vue.\\

En plus du texte, MUTE maintient un ensemble de métadonnées par document.
Par exemple, les utilisateurs peuvent donner un titre au document.
Pour représenter cette donnée additionnelle, nous associons un Last-Writer-Wins Register \ac{CRDT} synchronisé par états \cite{shapiro_2011_crdt} au document.
De façon similaire, nous utilisons un First-Writer-Wins Register \ac{CRDT} synchronisé par états pour représenter la date de création du document.\\

L'utilisation de ces structures de données nous permet donc de représenter le document texte ainsi que ses métadonnées.
Nous identifions cependant plusieurs axes d'évolution pour cette couche.
La première d'entre elles concerne l'ajout de styles au document.
Comme indiqué dans la \autoref{sec:mute-interface-utilisateur}, nous utilisons le langage de balisage Markdown pour inclure plusieurs éléments de style.
Cette solution a pour principal intérêt de reposer sur du texte qui s'intègre directement dans le contenu du document.
Ainsi, les balises de style sont répliquées nativement avec le contenu du document par le \ac{CRDT} représentant ce dernier.
Cette solution montre cependant ses limites lorsque plusieurs éléments styles sont ajoutés sur des zones de texte se superposant, notamment en concurrence.
Les balises Markdown produites sont alors entrelacées.
La figure \autoref{fig:mute-interleaving-markdown} illustre un tel exemple.

\begin{figure}
    \centering
    \includegraphics[width=0.5\columnwidth]{img/screenshot-mute-interleaving-markdown}
    \caption{Entrelacement de balises Markdown produisant une anomalie de style}
    \label{fig:mute-interleaving-markdown}
\end{figure}

Dans cet exemple, le texte "diam venenatis" a été mis en gras tandis que le texte "venenatis vitae" a été mis en italique.
Le résultat attendu est donc "\textbf{diam \emph{venenatis}} \emph{vitae}".
Il ne s'agit toutefois pas du résultat affiché par notre éditeur de texte, la syntaxe du langage Markdown n'étant pas respectée.
Les utilisateur-rices doivent donc manuellement corriger l'anomalie de style engendrée.

Afin de prévenir ce type d'anomalie, il conviendrait d'intégrer un \ac{CRDT} pour représenter le style du document, tel que Peritext \cite{2022-peritext-litt}.
Un type de donnée dédié nous permettrait de plus d'inclure des effets de style non-disponibles dans le langage Markdown, \eg la mise en page ou encore l'utilisation de couleurs.\\

Une second piste d'évolution consiste à rendre possible l'intégration et l'édition collaborative d'autres types de contenu que du texte au sein d'un document \ac{MUTE}, \eg des listes de tâches, des feuilles de calcul ou encore des schémas, comme le propose par exemple \cite{dropbox-paper}.
L'évolution d'éditeur collaboratif de documents texte à éditeur collaboratif de documents multimédia élargirait ainsi les contextes d'utilisation de \ac{MUTE}.

Cette piste de recherche nécessite de concevoir et d'intégrer des \acp{CRDT} pour chaque type de document supplémentaire.
Elle nécessite aussi la conception d'un mécanisme assurant la composition de ces \acp{CRDT}, \ie la conception d'un méta-\ac{CRDT}.
Ce méta-\ac{CRDT} devrait assurer plusieurs responsabilités.
Tout d'abord, il devrait permettre l'ajout et la suppression de \acp{CRDT} au sein du document multimédia, ainsi que leur ré-ordonnement.
Ensuite, le méta-\ac{CRDT} devrait définir les sémantiques de résolution de conflits employées en cas de modifications concurrentes sur le document, \eg la modification du contenu d'un des \acp{CRDT} du document en concurrence de la suppression dudit \ac{CRDT}.
Finalement, le méta-\ac{CRDT} devrait proposer différents modèles de cohérence à l'application.
Il devrait par exemple proposer de garantir le modèle de cohérence causal, \ie d'ordonner les modifications sur le méta-\ac{CRDT} et les \acp{CRDT} qui composent le document en utilisant la relation \hb \cf{def:happens-before}.
