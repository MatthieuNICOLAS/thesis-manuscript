\label{sec:mute-membership-protocol}

Pour assurer la qualité de la collaboration même à distance, il est important d'offrir des fonctionnalités de conscience de groupe aux utilisateurs.
Une de ces fonctionnalités est de fournir la liste des collaborateur-rices actuellement connectés.
Les protocoles d'appartenance au réseau sont une catégorie de protocoles spécifiquement dédiée à cet effet.
Ainsi, nous devions en implémenter un dans MUTE.

MUTE présente cependant plusieurs contraintes liées à notre modèle du système que le protocole sélectionné doit respecter.
Tout d'abord, le protocole doit être compatible avec un environnement \ac{P2P}, où les noeuds partagent les mêmes droits et responsabilités.
De plus, le protocole doit présenter une capacité de passage à l'échelle pour être adapté aux collaborations à large échelle.

En raison de ces contraintes, notre choix s'est porté sur le protocole SWIM \cite{swim2002}.
Proposé par \citeauthor{swim2002}, ce protocole d'appartenance au réseau offre les propriétés intéressantes suivantes.
Tout d'abord, le nombre de messages diffusés sur le réseau est proportionnel de façon linéaire au nombre de pairs.
Pour être plus précis, le nombre de messages envoyés par un pair par période du protocole est constant.
De plus, il fournit à chaque noeud une vue de la liste des collaborateur-rices cohérente à terme, même en cas de réception désordonnée des messages du protocoles.
Finalement, il intègre un mécanisme permettant de réduire le taux de faux positifs, \ie le taux de pairs déclarés injustement comme défaillants.

Pour cela, SWIM découple les deux composants d'un protocole d'appartenance au réseau : le mécanisme de \emph{détection des défaillances des pairs} et le mécanisme de \emph{dissémination des mises à jour du groupe}.
