\label{sec:swim-update-dissemination}

Quand l'exécution du mécanisme de détection des défaillances par un noeud met en lumière une évolution de la liste des collaborateur-rices, cette mise à jour doit être propagée au reste des noeuds.

Or, diffuser cette mise à jour à l'ensemble du réseau serait coûteux pour un seul noeud.
Afin de propager cette information de manière efficace, SWIM propose d'utiliser un protocole de diffusion épidémique : le noeud transmet la mise à jour qu'à un nombre réduit $\lambda$\footnote{
  \cite{swim2002} montre que choisir une valeur constante faible comme $\lambda$ suffit néanmoins à garantir la dissémination des mises à jour à l'ensemble du réseau.
} de pairs, qui se chargeront de la transmettre à leur tour.
Le mécanisme de dissémination des mises à jour de SWIM fonctionne donc de la manière suivante.

Chaque mise à jour du groupe est stockée dans une liste et se voit attribuer un compteur entier, initialisé avec $\lambda \log{} n$ où $n$ est le nombre de noeuds.
À chaque génération d'un message pour le mécanisme de détection des défaillances, un nombre arbitraire de mises à jour sont sélectionnées dans la liste et attachées au message.
Leur compteurs respectifs sont décrémentés.
Une fois que le compteur d'une mise à jour atteint 0, celle-ci est retirée de la liste.

À la réception d'un message, le noeud le traite comme définit précédemment en \autoref{sec:swim-failure-detection}.
De manière additionnelle, il intègre dans sa liste des collaborateur-rices les mises à jour attachées au message en utilisant les règles suivantes :

\begin{definition}[Relation $\lstatus$]
  La relation $\lstatus$ est la relation d'ordre total strict sur les valeurs de $\trm{nodeStatus}$ suivante :
  \begin{equation*}
    \trm{Alive} \lstatus \trm{Suspect} \lstatus \trm{Confirm}
  \end{equation*}
\end{definition}

\begin{definition}[Relation $\ltuple$]
  Étant donné deux tuples $t = \langle \trm{nodeStatus},\trm{nodeIncarn} \rangle$ et $t' = \langle \trm{nodeStatus'},\trm{nodeIncarn'} \rangle$, nous avons :
  \begin{equation*}
    \begin{split}
      t \ltuple t' \quad \trm{iff} \quad  & (\trm{nodeStatus} \lstatus \trm{nodeStatus'}) \quad \lor \\
                                          & (\trm{nodeStatus} = \trm{nodeStatus'} \land \trm{nodeIncarn} < \trm{nodeIncarn'}) \\
    \end{split}
  \end{equation*}
\end{definition}

Ainsi, le mécanisme de dissémination des mises à jour du groupe réutilise les messages du mécanisme de détection des défaillances pour diffuser les modifications.
Cela permet de propager les évolutions de la liste des collaborateur-rices sans ajouter de message supplémentaire.
De plus, les règles de précédence sur l'état d'un collaborateur permettent aux noeuds de converger même si les mises à jour sont reçues dans un ordre distinct.
