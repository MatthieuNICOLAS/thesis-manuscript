Dans ce chapitre, nous avons présenté \acf{MUTE}, l'éditeur collaboratif temps réel \ac{P2P} chiffré de bout en bout développé par notre équipe de recherche.\\

MUTE permet d'éditer de manière collaborative des documents texte.
Pour représenter les documents, MUTE propose plusieurs \acp{CRDT} pour le type Séquence \cite{2013-logootsplit,2021-these-vic,2022-rls-tpds-nicolas} issus des travaux de l'équipe.
Ces \acp{CRDT} offrent de nouvelles méthodes de collaborer, notamment en permettant de collaborer de manière synchrone ou asynchrone de manière transparente.\\

Pour permettre aux noeuds de communiquer, MUTE repose sur la technologie WebRTC.
Cette technologie permet de construire un réseau \ac{P2P} directement entre plusieurs navigateurs.
Plusieurs serveurs sont néanmoins requis, notamment pour la découverte des pairs et pour la communication entre des noeuds lorsque leur pare-feux respectifs empêchent l'établissement d'une connexion directe.\\

Finalement, MUTE implémente un mécanisme de chiffrement de bout en bout garantissant l'authenticité et la confidentialité des échanges entre les noeuds.
Ce mécanisme repose sur une clé de groupe de chiffrement qui est établie à l'aide du protocole \cite{1995-burmester-desmedt}.

Ce protocole nécessite que chaque noeud possède une paire de clés de chiffrement et qu'ils partagent leur clé publique.
\ac{MUTE} repose sur des \acp{PKI} pour cela.
Avant de détecter tout éventuel comportement malicieux de la part de ces derniers, \ac{MUTE} intègre un mécanisme d'audit \cite{2018-trusternity-short,2018-trusternity-long}.\\
