Dans ce chapitre, nous avons présenté \acf{MUTE}, l'éditeur collaboratif temps réel \ac{P2P} chiffré de bout en bout développé par notre équipe de recherche.\\

MUTE permet d'éditer de manière collaborative des documents texte.
Pour représenter les documents, MUTE propose plusieurs \acp{CRDT} pour le type Séquence \cite{2013-logootsplit,2021-these-vic,2022-rls-tpds-nicolas} issus des travaux de l'équipe.
Ces \acp{CRDT} offrent de nouvelles méthodes de collaborer, notamment en permettant de collaborer de manière synchrone ou asynchrone de manière transparente.\\

Pour permettre aux noeuds de communiquer, MUTE repose sur la technologie \acf{WebRTC}.
Cette technologie permet de construire un réseau \ac{P2P} directement entre plusieurs navigateurs.
Plusieurs serveurs sont néanmoins requis, notamment pour la découverte des pairs et pour la communication entre des noeuds lorsque leur pare-feux respectifs empêchent l'établissement d'une connexion directe.\\

Finalement, MUTE implémente un mécanisme de chiffrement de bout en bout garantissant l'authenticité et la confidentialité des échanges entre les noeuds.
Ce mécanisme repose sur une clé de groupe de chiffrement qui est établie à l'aide du protocole \cite{1995-burmester-desmedt}.

Ce protocole nécessite que chaque noeud possède une paire de clés de chiffrement et qu'ils partagent leur clé publique.
Pour partager leur clé publique, les noeuds utilisent des \acfp{PKI}.
Cependant, afin de détecter un éventuel comportement malicieux de la part de ces derniers, \ac{MUTE} intègre un mécanisme d'audit \cite{2018-trusternity-short,2018-trusternity-long}.\\

Plusieurs tâches restent néanmoins à réaliser afin de répondre à notre objectif initial, \ie la conception d'un éditeur collaboratif \ac{P2P} temps réel, à large échelle, sûr et sans autorités centrales.
Une première d'entre elles concerne les droits d'accès aux documents.
Actuellement, tout-e utilisateur-rice possédant l'URL d'un document peut découvrir les noeuds travaillant sur ce document, établir une connexion avec eux, participer à la génération d'une nouvelle clé de chiffrement de groupe puis obtenir l'état du document en se synchronisant avec un noeuds.
Pour empêcher un tier ou un-e collaborateur-rice expulsé-e précédemment d'accéder au document, il est nécessaire d'intégrer un mécanisme de gestion de droits d'accès adapté aux systèmes \ac{P2P} à large échelle \cite{2021-access-control-crdts,2022-dist-access-control-pa}.\\

Une seconde piste de travail concerne l'amélioration de la couche réseau de \ac{MUTE}.
Comme mentionné précédemment \cf{sec:mute-topologie-protocole-diffusion}, notre couche réseau actuelle souffre de plusieurs limites.
Tout d'abord, les noeuds qui travaillent sur un même document établissent un réseau \ac{P2P} entièrement maillé.
Cette topologie réseau requiert un nombre de connexions par noeud qui dépend linéairement du nombre total de noeuds.
Cette topologie s'avère donc coûteuse et inadaptée aux collaborations à large échelle.

Ensuite, le protocole de diffusion des messages que nous employons s'avère lui aussi inadapté aux collaborations à large échelle.
En effet, c'est le noeud auteur d'un message qui a pour responsabilité d'en envoyer une copie à chacun des noeuds de la collaboration.
Ainsi, ce protocole concentre toute la charge de travail pour la diffusion d'un message sur un seul noeud.
Cette tâche s'avère excessive pour un unique noeud quand la collaboration est à large échelle.

Ces limites entravent donc la capacité à utiliser \ac{MUTE} dans le cadre de collaborations à large échelle.
Pour y répondre, nous identifions deux pistes d'amélioration :
\begin{enumerate}
    \item L'utilisation d'un protocole d'échantillonnage aléatoire de pairs adaptatif, tel que Spray \cite{2018-spray-nedelec}, qui limite la taille du voisinage d'un noeud à un facteur logarithmique du nombre total de noeuds.
    \item L'emploi d'un protocole de diffusion épidémique des messages \cite{1999-epidemic-dissemination-birman}, qui répartit sur l'ensemble des noeuds la charge de travail pour diffuser d'un message à tout le réseau.
\end{enumerate}
