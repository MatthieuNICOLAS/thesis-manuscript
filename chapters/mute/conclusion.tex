Dans ce chapitre, nous avons présenté \acf{MUTE}, notre éditeur collaboratif temps réel \ac{P2P} chiffré de bout en bout.

MUTE permet d'éditer de manière collaborative des documents texte.
Pour représenter les documents, MUTE implémente les structures de données répliquées décrites dans la \autoref{sec:logootsplit} et le \autoref{chap:renamablelogootsplit}.
Ces \acp{CRDT} offrent de nouvelles méthodes de collaborer, notamment en permettant de collaborer de manière synchrone ou asynchrone de manière transparente.

Pour permettre aux noeuds de communiquer, MUTE utilise WebRTC.
Cette technologie permet de construire un réseau \ac{P2P} entre navigateurs.
Plusieurs serveurs sont néanmoins requis, notamment pour la découverte des pairs et pour la communication entre des noeuds dont les pare-feux respectifs empêche l'établissement d'une connexion directe.

Finalement, MUTE implémente un mécanisme de chiffrement de bout en bout garantissant l'authenticité et la confidentialité des échanges entre les noeuds.
Ce mécanisme reposant sur d'autres serveurs, les PKIs, MUTE intègre un mécanisme d'audit permettant de détecter leurs éventuels comportements malicieux.
