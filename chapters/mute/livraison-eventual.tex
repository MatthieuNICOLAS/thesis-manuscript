\label{sec:mute-anti-entropy}

La contrainte restante du modèle de livraison précise que toutes les opérations doivent être livrées à l'ensemble des noeuds à terme.
Cependant, le réseau étant non-fiable, des messages peuvent être perdus au cours de l'exécution.
Il est donc nécessaire que les noeuds rediffusent les messages perdus pour assurer leur livraison à terme.

Pour cela, nous implémentons un mécanisme d'anti-entropie basé sur \cite{1983-anti-entropy-vv}.
Ce mécanisme permet à un noeud source de se synchroniser avec un autre noeud cible.
Il est executé par l'ensemble des noeuds de manière indépendante.
Nous décrivons ci-dessous son fonctionnement.\\

De manière périodique, le noeud choisit un autre noeud cible de manière aléatoire.
Le noeud source lui envoie alors une représentation de son état courant, \ie son vecteur de version.

À la réception de ce message, le noeud cible compare le vecteur de version reçu par rapport à son propre vecteur de version.
À partir de ces données, il identifie les dots des opérations de sa connaissance qui sont inconnues au noeud source.
Grâce à leur dot, le noeud cible retrouve ces opérations depuis son log des opérations.
Il envoie alors une réponse composée de ces opérations au noeud source.

À la réception de la réponse, le noeud source intègre normalement les opérations reçues.
La \autoref{fig:anti-entropy} illustre ce mécanisme.\\

\begin{figure}[!ht]
  \centering
  \resizebox{\columnwidth}{!}{
    \begin{tikzpicture}
      \path
        node {\textbf{A}}
        +(0:37) node (a-end) {}
        ++(0:0.5) node (a0) {}
        ++(0:3) node[op, label=above:{$\langle A:8,B:3,C:1 \rangle$}] (a8) {a8}
        ++(0:6) node [op, label=above:{$\langle A:9,B:3,C:1 \rangle$}] (a9) {a9}
        ++(0:15) node[point, label=above:{$\langle A:9,B:4,C:1 \rangle$}, label=below:{\emph{deliver}}] (a-receives-b4) {};


      \draw[thick] (a0) --  (a8) -- (a9) -- (a-receives-b4) -- (a-end);

      \path
          ++(270:2) node {\textbf{B}}
          +(0:37) node (b-end) {}
          ++(0:0.5) node (b0) {}
          ++(0:9) node[point, label=below:{$\langle A:8,B:3,C:1 \rangle$}, label=above:{\emph{deliver}}] (b-receives-a8) {}
          ++(0:6) node[point, label=below:{$\langle A:9,B:3,C:1 \rangle$}, label=above:{\emph{deliver}}](b-receives-a9) {}
          ++(0:4) node[op, label=above:{$\langle A:9,B:4,C:1 \rangle$}] (b4) {b4}
          ++(0:12) node[point, label=above:{\emph{reply to sync query}}] (b-receives-c-sync-request) {};

      \draw[thick] (b0) -- (b-receives-a8) -- (b-receives-a9) -- (b4) -- (b-receives-c-sync-request) -- (b-end);

      \path
          ++(270:4) node {\textbf{C}}
          +(0:37) node (c-end) {}
          ++(0:0.5) node (c0) {}
          ++(0:8) node (c-receives-a8) {}
          +(90:0.5) node[cross] (fail-a8) {}
          ++(0:6) node (c-receives-a9) {}
          +(90:0.5) node[cross] (fail-a9) {}
          ++(0:10) node[point, label=below:{$\langle A:7,B:4,C:1 \rangle$}, label=above:{\emph{deliver}}] (c-receives-b4) {}
          ++(0:4) node[op, label=below:{$\langle A:7,B:4,C:1 \rangle$}] (c-query-sync) {sync}
          ++(0:6) node[point, label=below:{$\langle A:9,B:4,C:1 \rangle$}, label=above:{\emph{deliver}}] (c-receives-b-sync-reply) {};


      \draw[thick] (c0) -- (c-receives-b4) -- (c-query-sync) --  (c-receives-b-sync-reply) -- (c-end);

      \draw[->, dashed, shorten >= 1]
        (a8.east) edge (b-receives-a8.west)
        (a8.east) edge (fail-a8)
        (a9.east) edge (b-receives-a9.west)
        (a9.east) edge (fail-a9)
        (b4.east) edge (a-receives-b4.west)
        (b4.east) edge (c-receives-b4.west)
        (b-receives-c-sync-request.east) edge (c-receives-b-sync-reply.west)
        (c-query-sync.east) edge (b-receives-c-sync-request.west);

    \end{tikzpicture}
  }
  \caption{Utilisation du mécanisme d'anti-entropie par le noeud C pour se synchroniser avec le noeud B}
  \label{fig:anti-entropy}
\end{figure}

Dans cette figure, nous représentons une exécution à laquelle participent trois noeuds : A, B et C.
Initialement, les trois noeuds sont synchronisés.
Leur vecteurs de version sont identiques et ont pour valeur $\langle A:7,B:3,C:1 \rangle$.

Le noeud A effectue les opérations \emph{a8} puis \emph{a9} et les diffusent sur le réseau.
Le noeud B reçoit ces opérations et les livre à son \ac{CRDT}.
Il effectue ensuite et propage l'opération \emph{b4}, qui est reçue et livrée par A.
Ils atteignent tous deux la version représenté par le vecteur $\langle A:9,B:4,C:1 \rangle$

De son côté, le noeud C ne reçoit pas les opérations \emph{a8} et \emph{a9} à cause d'une défaillance réseau.
Néanmoins, cela ne l'empêche pas de livrer l'opération \emph{b4} à sa réception et d'obtenir la version $\langle A:7,B:4,C:1 \rangle$.

Le noeud C déclenche ensuite son mécanisme d'anti-entropie.
Il choisit aléatoirement le noeud B comme noeud cible.
Il lui envoie un message de synchronisation avec pour contenu le vecteur de version $\langle A:7,B:8,C:1 \rangle$.

À la réception de ce message, le noeud B compare ce vecteur avec le sien.
Il détermine que le noeud C n'a pas reçu les opérations \emph{a8} et \emph{a9}.
B les récupère depuis son log des opérations et les envoie à C par le biais d'un nouveau message.

À la réception de la réponse de B, le noeud C livre les opérations \emph{a8} et \emph{a9}.
Il atteint alors le même état que A et B, représenté par le vecteur de version $\langle A:9,B:4,C:1 \rangle$.\\

Ce mécanisme d'anti-entropie nous permet ainsi de garantir la livraison à terme de toutes les opérations et de compenser les défaillances du réseau.
Il nous sert aussi de mécanisme de synchronisation : à la connexion d'un pair, celui-ci utilise ce mécanisme pour récupérer les opérations effectuées depuis sa dernière connexion.
Dans le cas où il s'agit de la première connexion du pair, il lui suffit d'envoyer un vecteur de version vide pour récupérer l'intégralité des opérations.

Ce mécanisme propose plusieurs avantages.
Son exécution n'implique que le noeud source et le noeud cible, ce qui limite les coûts de coordination.
De plus, si une défaillance a lieu lors de l'exécution du mécanisme (perte d'un des messages, panne du noeud cible...), cette défaillance n'est pas critique : le noeud source se synchronisera à la prochaine exécution du mécanisme.
Ensuite, ce mécanisme réutilise le vecteur de version déjà nécessaire pour la livraison en exactement un exemplaire, comme présenté en \autoref{sec:mute-exactly-once-delivery}.
Il ne nécessite donc pas de stocker une nouvelle structure de données pour détecter les différences entre noeuds.

En contrepartie, la principale limite de ce mécanisme d'anti-entropie est qu'il nécessite de maintenir et de parcourir périodiquement le log des opérations pour répondre aux requêtes de synchronisation.
La complexité spatiale et en temps du mécanisme dépend donc linéairement du nombre d'opérations.
Qui plus est, nous sommes dans l'incapacité de tronquer le log des opérations en se basant sur la stabilité causale des opérations puisque nous utilisons ce mécanisme pour mettre à niveau les nouveaux pairs.
À moins de mettre en place un mécanisme de compression du log comme évoqué en \autoref{sec:rename-as-compression-mechanism}, ce log des opérations croit de manière monotone.
Néanmoins, une alternative possible est de mettre en place un système de chargement différé des opérations pour ne pas surcharger la mémoire.
