\label{sec:mute-delivery-model}

Comme indiqué précédemment, la couche livraison est formée d'un unique composant, que nous nommons module de livraison.
Ce module est associé aux \acp{CRDT} synchronisés par opérations représentant le document texte, \ie LogootSplit ou RenamableLogootSplit.

Le rôle de ce module est de garantir que le modèle de livraison des opérations requis par le \ac{CRDT} pour assurer la cohérence à terme \cf{def:eventual-consistency} soit satisfait, \ie que l'ensemble des opérations soient livrées dans un ordre correct à l'ensemble des noeuds.

Pour cela, le module de livraison doit implémenter les contraintes imposées par ces \acp{CRDT} sur l'ordre de livraison des opérations (cf. \autoref{def:ls-delivery-model}, page \pageref{def:ls-delivery-model} et \autoref{def:rls-delivery-model}, page \pageref{def:ls-delivery-model}).
Pour rappel, le modèle de livraison de RenamableLogootSplit est le suivant :
\begin{enumerate}
    \item Une opération doit être livrée à l'ensemble des noeuds à terme.
    \item Une opération doit être livrée qu'une seule et unique fois aux noeuds.
    \item Une opération $\trm{remove}$ doit être livrée à un noeud une fois que les opérations $\trm{insert}$ des éléments concernés par la suppression ont été livrées à ce dernier.
    \item Une opération peut être délivrée à un noeud qu'à partir du moment où l'opération $\trm{rename}$ qui a introduit son époque de génération a été délivrée à ce même noeud.
\end{enumerate}

Nous décrivons ci-dessous comment nous assurons chacune de ces contraintes.
