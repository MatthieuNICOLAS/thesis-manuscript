\label{sec:mute-exactly-once-delivery}

Afin de respecter la contrainte de \emph{exactly-once delivery}, il est nécessaire d'identifier de manière unique chaque opération.
Pour cela, le module \texttt{Sync} ajoute un \emph{Dot} \cite{2014-scalable-accurate-causality-tracking} à chaque opération :

\begin{definition}[Dot]
  Un \emph{Dot} est une paire $\langle$nodeId, nodeSyncSeq$\rangle$ où
  \begin{itemize}
    \item nodeId est l'identifiant unique du noeud qui a généré l'opération.
    \item nodeSyncSeq est le numéro de séquence courant du noeud à la génération de l'opération.
  \end{itemize}
\end{definition}

Il est à noter que \emph{nodeSyncSeq} est différent du \emph{nodeSeq} utilisé dans LogootSplit et RenamableLogootSplit \cf{def:logootsplit}.
En effet, \emph{nodeSyncSeq} se doit d'augmenter à chaque opération tandis que \emph{nodeSeq} n'augmente qu'à la création d'un nouveau bloc.
Les contraintes étant différentes, il est nécessaire de distinguer ces deux données.

Chaque noeud maintient une structure de données représentant l'ensemble des opérations reçues par le pair.
Elle permet de vérifier à la réception d'une opération si le dot de cette dernière est déjà connu.
S'il s'agit d'un nouveau dot, le module \texttt{Sync} peut livrer l'opération au \ac{CRDT} et ajouter son dot à la structure.
Le cas échéant, cela indique que l'opération a déjà été livrée précédemment et doit être ignorée cette fois-ci.

Plusieurs structures de données sont adaptées pour maintenir l'ensemble des opérations reçues.
Dans le cadre de MUTE, nous avons choisi d'utiliser un vecteur de versions.
Cette structure nous permet de réduire à un dot par noeud le surcoût en métadonnées du module \texttt{Sync}, puisqu'il ne nécessite que de stocker le dot le plus récent par noeud.
Cette structure permet aussi de vérifier en temps constant si une opération est déjà connue.
La \autoref{fig:exactly-once-delivery} illustre son fonctionnement.

\begin{figure}[!ht]
  \subfloat[Exécution avec livraison multiple d'une opération \emph{insert}]{
      \begin{minipage}{\linewidth}
          \centering
          \resizebox{\columnwidth}{!}{
            \begin{tikzpicture}
                \path
                    node {\textbf{A}}
                    ++(0:0.5 * \widthletter) node[epoch] {\epoch{0}}
                    ++(0:1.05 * \widthoriginepoch) node[block, label=below:{\id{p}{A0}{0..4}}] (S0A) {OGNON}
                    ++(0:5 * \widthletter) node[epoch] (S1A-left) {\epoch{0}}
                    ++(0:1.05* \widthoriginepoch) node[letter, label=below:{\id{p}{A0}{0}}]  {O}
                    ++(0:\widthletter) node[letter, fill=mydarkorange, label=above:{\id{p}{A0}{0}\id{m}{A1}{0}}] {I}
                    ++(0:\widthletter) node[block, label=below:{\id{p}{A0}{1..4}}] (S1A-right) {GNON}
                    ++(0:21 * \widthletter) node[epoch] (S2A-left) {\epoch{0}}
                    ++(0:1.05 * \widthoriginepoch) node[letter, label=below:{\id{p}{A0}{0}}] {O}
                    ++(0:\widthletter) node[block, label=below:{\id{p}{A0}{1..4}}] {GNON};


                \path
                    ++(270:4) node {\textbf{B}}
                    ++(0:0.5 * \widthletter) node[epoch] {\epoch{0}}
                    ++(0:1.05 * \widthoriginepoch) node[block, label=below:{\id{p}{A0}{0..4}}] (S0B) {OGNON}
                    ++(0:12 * \widthletter) node[epoch] (S1B-left) {\epoch{0}}
                    ++(0:1.05 * \widthoriginepoch) node[letter, label=below:{\id{p}{A0}{0}}] {O}
                    ++(0:\widthletter) node[letter, fill=mydarkorange, label=above:{\id{p}{A0}{0}\id{m}{A1}{0}}] {I}
                    ++(0:\widthletter) node[block, label=below:{\id{p}{A0}{1..4}}] (S1B-right) {GNON}
                    ++(0:5 * \widthletter) node[epoch] (S2B-left) {\epoch{0}}
                    ++(0:1.05 * \widthoriginepoch) node[letter, label=below:{\id{p}{A0}{0}}] {O}
                    ++(0:\widthletter) node[block, label=below:{\id{p}{A0}{1..4}}] (S2B-right) {GNON}
                    ++(0:8 * \widthletter) node[epoch] (S3B-left) {\epoch{0}}
                    ++(0:1.05 * \widthoriginepoch) node[letter, label=below:{\id{p}{A0}{0}}] {O}
                    ++(0:\widthletter) node[block, label=below:{\id{p}{A0}{1..4}}] {GNON};

                \draw[->, thick]
                  (S0A) edge node[above, align=center]{\emph{insert "I"}\\\emph{between}\\\emph{"O" and "G"}} (S1A-left)
                  (S1B-right) edge node[above, align=center]{\emph{remove "I"}} (S2B-left);

                \draw[dotted]
                  (S1A-right) -- (S2A-left)
                  (S0B) -- (S1B-left)
                  (S2B-right) -- (S3B-left);

                \draw[dashed, ->, thick, shorten >= 3]
                  (S1A-right.east) edge node[right, align=center]{\emph{insert "I" at} {\color{mydarkorange}\id{p}{A0}{0}\id{m}{A1}{0}}}  (S1B-left.west)
                  (S1A-right.east) edge node[right, align=center]{\emph{insert "I" at} {\color{mydarkorange}\id{p}{A0}{0}\id{m}{A1}{0}}}  (S3B-left.west)
                  (S2B-right.east) edge node[below right, align=center]{\emph{remove} {\color{mydarkorange}\id{i}{B0}{1..1}}} (S2A-left.west);
            \end{tikzpicture}
            \label{fig:exactly-once-delivery-rls}
          }
          \end{minipage}
        }
        \hfil
  \subfloat[État et comportement de la couche \texttt{Sync} au cours de l'exécution décrite en \autoref{fig:exactly-once-delivery-rls}]{
    \begin{minipage}{\linewidth}
      \centering
      \resizebox{\columnwidth}{!}{
        \begin{tikzpicture}
            \path
                node {\textbf{A}}
                ++(0:1) node[point, label=above:{$\langle A:5 \rangle$}] (a-start) {}
                ++(0:2) node[draw, circle, label=above:{$\langle A:6 \rangle$}] (a-a6) {a6}
                ++(0:8) node[point, label=above:{$\langle A:6,B:1 \rangle$}, label=below:{\emph{deliver}}] (a-b1) {}
                ++(0:3) node (a-end) {};


            \path
                ++(270:3) node {\textbf{B}}
                ++(0:1) node[point, label=below:{$\langle A:5 \rangle$}] (b-start) {}
                ++(0:5) node[point, label=below:{$\langle A:6 \rangle$}, label=above:{\emph{deliver}}] (b-a6-1) {}
                ++(0:2) node[draw, circle, label=below:{$\langle A:6,B:1 \rangle$}] (b-b1) {b1}
                ++(0:5) node[point, label=below:{$\langle A:6,B:1 \rangle$}, label=above:{\emph{discard}}] (b-a6-2) {}
                ++(0:1) node (b-end) {};

            \draw[->, thick]
              (a-start) edge (a-a6)
              (b-a6-1) edge (b-b1);

            \draw[dotted]
              (a-a6) -- (a-b1) -- (a-end)
              (b-start) -- (b-a6-1)
              (b-b1) -- (b-end);

            \draw[dashed, ->, thick, shorten >= 3]
              (a-a6.east) edge (b-a6-1.west)
              (a-a6.east) edge (b-a6-2.west)
              (b-b1.east) edge (a-b1.west);
        \end{tikzpicture}
        \label{fig:exactly-once-delivery-sync}
      }
      \end{minipage}
    }
  \caption{Gestion de la livraison \emph{exactly-once} des opérations}
  \label{fig:exactly-once-delivery}
\end{figure}

Dans cet exemple, qui reprend celui de la \autoref{fig:why-exactly-once-delivery}, deux noeuds A et B répliquent une séquence.
Initialement, celle-ci contient les éléments "OGNON".
Ces éléments ont été insérés un par un par le noeud A, donc par le biais des opérations \emph{a1} à \emph{a5}.
Le module \texttt{Sync} de chaque noeud maintient donc initialement le vecteur de version $\langle A:5 \rangle$.

Le noeud A insère l'élément "I" entre les éléments "O" et "G".
Cette modification est alors labellisée \emph{a6} par son module \texttt{Sync} et est envoyée au noeud B.
À la réception de cette opération, le module \texttt{Sync} de B compare son dot avec son vecteur de version local.
L'opération \emph{a6} étant la prochaine opération attendue de A, celle-ci est acceptée : elle est alors livrée au \ac{CRDT} et le vecteur de version est mis à jour.

Le noeud B supprime ensuite l'élément nouvellement inséré.
S'agissant de la première modification de B, cette modification \emph{b1} ajoute l'entrée correspondante dans le vecteur de version $\langle A:6, B:1 \rangle$.
L'opération est envoyée au noeud A.
Cette opération étant la prochaine opération attendue de B, elle est acceptée et livrée.

Finalement, le noeud B reçoit de nouveau l'opération \emph{a6}.
Son module \texttt{Sync} détermine alors qu'il s'agit d'un doublon : l'opération apparaît déjà dans le vecteur de version $\langle A:6, B:1 \rangle$.
L'opération est donc ignorée, et la résurgence de l'élément "I" illustrée dans la \autoref{fig:why-exactly-once-delivery} est évitée.

Il est à noter que dans le cas où un noeud reçoit une opération avec un dot plus élevé que celui attendu (\eg le noeud A reçoit une opération \emph{b3} à la fin de l'exemple), cette opération est mise en attente.
En effet, livrer cette opération nécessiterait de mettre à jour le vecteur de version à $\langle A:6,B:3 \rangle$ et masquerait le fait que l'opération \emph{b2} n'a jamais été reçue.
L'opération \emph{b3} serait donc mise en attente jusqu'à la livraison de l'opération \emph{b2}.

Ainsi, l'implémentation de livraison \emph{exactly-once} avec un vecteur de version comme structure de données force une livraison \ac{FIFO} des opérations par noeuds.
Il s'agit d'une contrainte non-nécessaire et qui peut introduire des délais dans la collaboration, notamment si une opération d'un noeud est perdue par le réseau.
Nous jugeons cependant acceptable ce compromis entre le surcoût du mécanisme de livraison \emph{exactly-once} et son impact sur l'expérience utilisateur.

Pour retirer cette contrainte superflue, il est possible de remplacer cette structure de données par un \emph{Interval Version Vector} \cite{2014-optimized-or-sets}.
Au lieu d'enregistrer seulement le dernier dot intégré par noeud, cette structure de données enregistre les intervalles de dots intégrés.
Ceci permet une livraison \emph{out of order} des opérations tout en garantissant une livraison \emph{exactly-once} et en compactant efficacement les données stockées par le module \texttt{Sync} à terme.
