Pour générer et maintenir la liste des collaborateur-rices, nous avons implémenté le protocole distribué d'appartenance au réseau SWIM \cite{swim2002}.
Par rapport à la version originale, nous avons procédé à plusieurs modifications, notamment pour gérer plus efficacement les reconnexions successives d'un même noeud.\\

Ainsi, nous avons implémenté un mécanisme dont la complexité spatiale dépend linéairement du nombre de noeuds, et non plus du nombre de connexions \cf{sec:swim-modif-precedence-order}.
La complexité en temps et la complexité en communication restent quant à elles inchangées par rapport à la version présentée dans \cite{swim2002}, \ie elles dépendent toujours de paramètres dont nous choisissons les valeurs : la fréquence de déclenchement du mécanisme de détection de défaillance et le nombre de mises à jour du groupe propagées par message.\\

Des améliorations au protocole SWIM ont été proposées dans \cite{lifeguard2018}.
Ces modifications visent notamment à réduire le délai de détection d'un noeud défaillant, ainsi que réduire le taux de faux positifs.
Ainsi, une perspective est d'implémenter ces améliorations dans MUTE.
