La troisième contrainte spécifiée par le modèle de livraison est qu'une opération doit être livrée après l'opération \emph{rename} qui a introduit son époque de génération.

Pour cela, le module de livraison doit donc récupérer l'époque courante de la séquence répliquée, récupérer le dot de l'opération \emph{rename} l'ayant introduite et l'ajouter en tant que dépendance de chaque opération.
Cependant, dans notre implémentation, le module de livraison et le module représentant la séquence répliquée sont découplés et ne peuvent interagir directement l'un avec l'autre.\\

Pour remédier à ce problème, le module de livraison maintient une structure supplémentaire : un vecteur des dots des opérations \emph{rename} connues.
À la réception d'une opération \emph{rename} distante, l'entrée correspondante de son auteur est mise à jour avec le dot de la nouvelle époque introduite.
À la génération d'une opération locale, l'opération est examinée pour récupérer son époque de génération.
Le module conserve alors seulement l'entrée correspondante dans le vecteur des dots des opérations \emph{rename}.
À ce stade, le contenu du vecteur est ajouté en tant que dépendance de l'opération.
Ensuite, si l'opération locale s'avère être une opération \emph{rename}, le vecteur est modifié pour ne conserver que le dot de l'époque introduite par l'opération.
La \autoref{fig:epoch-based-delivery} illustre ce fonctionnement.\\

\begin{figure}[!ht]
  \subfloat[Exécution]{
      \begin{minipage}{\linewidth}
          \centering
          \resizebox{\columnwidth}{!}{
            \begin{tikzpicture}
              \path
                node {\textbf{A}}
                +(0:20) node (a-end) {}
                ++(0:0.5) node (a0) {}
                ++(0:3) node[point, label=above:{rename to \epoch{A1}}] (a1) {}
                ++(0:3) node [point, label=above:{rename to \epoch{A7}}] (a7) {}
                ++(0:10) node (a-receives-c1) {}
                ++(0:2) node (a-receives-b3) {};


              \draw[thick] (a0) --  (a1) -- (a7) edge (a-receives-c1) edge (a-receives-b3) edge (a-end);

              \path
                  ++(270:2) node {\textbf{B}}
                  +(0:20) node (b-end) {}
                  ++(0:0.5) node (b0) {}
                  ++(0:3) node[point, label=above:{rename to \epoch{B3}}] (b3) {};

              \draw[thick] (b0) -- (b3) -- (b-end);

              \path
                  ++(270:4) node {\textbf{C}}
                  +(0:20) node (c-end) {}
                  ++(0:0.5) node (c0) {}
                  ++(0:5.5) node (c-receives-b3) {}
                  ++(0:2.5) node (c-receives-a1) {}
                  ++(0:3) node (c-receives-a7) {}
                  ++(0:3) node[point, label=below:{insert}] (c1) {};

              \draw[thick] (c0) edge (c-receives-b3) edge (c-receives-a1) edge (c-receives-a7) -- (c1) -- (c-end);

              \draw[->, dashed, shorten >= 1]
                (a1) edge (c-receives-a1)
                (a7) edge (c-receives-a7)
                (b3) edge (a-receives-b3)
                (b3) edge (c-receives-b3)
                (c1) edge (a-receives-c1);
            \end{tikzpicture}
            \label{fig:epoch-based-delivery-execution}
          }
          \end{minipage}
        }
        \hfil
  \subfloat[État et comportement du module de livraison au cours de l'exécution décrite en \autoref{fig:epoch-based-delivery-execution}]{
    \begin{minipage}{\linewidth}
      \centering
      \resizebox{\columnwidth}{!}{
        \begin{tikzpicture}
          \path
            node {\textbf{A}}
            +(0:25) node (a-end) {}
            ++(0:0.5) node (a0) {}
            ++(0:3) node[op, label=above:{$\langle A:1 \rangle$}] (a1) {a1}
            ++(0:3) node [op, label=above:{$\langle A:7 \rangle$}] (a7) {a7}
            ++(0:10) node[point, label=above:{$\langle A:7 \rangle$}, label=below:{\emph{postpone}}] (a-receives-c1-postpone) {}
            ++(0:3) node[point, label=above:{$\langle A:7,B:3 \rangle$}, label=below:{\emph{deliver}}] (a-receives-b3) {}
            ++(0:3) node[point, label=above:{$\langle A:7,B:3 \rangle$}, label=below:{\emph{deliver}}] (a-receives-c1-deliver) {};


          \draw[thick] (a0) --  (a1) -- (a7) -- (a-receives-c1-postpone) -- (a-receives-b3) -- (a-receives-c1-deliver) -- (a-end);

          \path
              ++(270:2) node {\textbf{B}}
              +(0:25) node (b-end) {}
              ++(0:0.5) node (b0) {}
              ++(0:3) node[op, label=above:{$\langle B:3 \rangle$}] (b3) {b3};

          \draw[thick] (b0) -- (b3) -- (b-end);

          \path
              ++(270:4) node {\textbf{C}}
              +(0:25) node (c-end) {}
              ++(0:0.5) node (c0) {}
              ++(0:5.5) node[point, label=below:{$\langle B:3 \rangle$}, label=above:{\emph{deliver}}] (c-receives-b3) {}
              ++(0:2.5) node[point, label=below:{$\langle A:1,B:3 \rangle$}, label=above:{\emph{deliver}}] (c-receives-a1) {}
              ++(0:3) node[point, label=below:{$\langle A:7,B:3 \rangle$}, label=above:{\emph{deliver}}] (c-receives-a7) {}
              ++(0:3) node[op, label=below:{$\langle B:3 \rangle$}] (c1) {c1};

          \draw[thick] (c0) -- (c-receives-b3) -- (c-receives-a1) -- (c-receives-a7) -- (c1) -- (c-end);

          \draw[->, dashed, shorten >= 1]
            (a1.east) edge (c-receives-a1.west)
            (a7.east) edge node[near end, above right] {$\{\trm{deps}: \{\langle A:1 \rangle\}\}$} (c-receives-a7.west)
            (b3.east) edge (a-receives-b3.west)
            (b3.east) edge (c-receives-b3.west)
            (c1.east) edge node[near start, right] {$\{\trm{deps}: \{\langle B:3 \rangle\}\}$} (a-receives-c1-postpone.west);

          \draw[->, dashed, thick, shorten >= 3] (a-receives-c1-postpone.east) edge[bend right] (a-receives-c1-deliver.west);
        \end{tikzpicture}
        \label{fig:epoch-based-delivery-sync}
      }
      \end{minipage}
    }
  \caption{Gestion de la livraison des opérations après l'opération \emph{rename} qui introduit leur époque}
  \label{fig:epoch-based-delivery}
\end{figure}

Dans la \autoref{fig:epoch-based-delivery-execution}, nous décrivons une exécution suivante en ne faisant apparaître que les opérations importantes : les opérations \emph{rename} et une opération \emph{insert} finale.
Dans cette exécution, trois noeuds A, B et C répliquent et éditent collaborativement une séquence.
Initialement, aucune opération \emph{rename} n'a encore eu lieu.
Le noeud A effectue une première opération \emph{rename} (\emph{a1}) puis une seconde opération \emph{rename} (\emph{a7}), et les diffuse.
En concurrence, le noeud B génère et propage sa propre opération \emph{rename} (\emph{b3}).
De son côté, le noeud C reçoit les opérations \emph{b3}, puis \emph{a1} et \emph{a7}.
Il émet ensuite une opération \emph{insert} (\emph{c1}).
Le noeud A reçoit cette opération avant de finalement recevoir l'opération \emph{b3}.

Dans la \autoref{fig:epoch-based-delivery-sync}, nous faisons apparaître l'état du module de livraison et les décisions prises par ce dernier au cours de l'exécution.
Initialement, le vecteur des dots des opérations \emph{rename} connues est vide.
Ainsi, lorsque A génère l'opération \emph{a1}, celle-ci ne se voit ajouter aucune dépendance (nous ne représentons pas les dépendances des opérations qui correspondent à l'ensemble vide).
A met ensuite à jour son vecteur des dots des opérations \emph{rename} avec le dot $\langle A:1 \rangle$.
B procède de manière similaire avec l'opération \emph{b3}.

Quand A génère l'opération \emph{a7}, le dot $\langle A:1 \rangle$ est ajouté en tant que dépendance.
Le dot $\langle A:7 \rangle$ remplace ensuite ce dernier dans le vecteur des dots des opérations \emph{rename}.

À la réception de l'opération \emph{b3}, le module de livraison de C peut la livrer au \ac{CRDT}, l'ensemble de ses dépendances étant vérifié.
Le noeud C ajoute alors à son vecteur des dots des opérations \emph{rename} le dot $\langle B:3 \rangle$.
Il procède de même pour l'opération \emph{a1} : il la livre et ajoute le dot $\langle A:1 \rangle$.
Le module de livraison ne connaissant pas l'époque courante de la séquence répliquée, il maintient les deux dots localement.

Lorsque le noeud C reçoit l'opération \emph{a7}, l'ensemble de ses contraintes est vérifié : l'opération \emph{a1} a été livrée précédemment.
L'opération est donc livrée et le vecteur de dots des opérations \emph{rename} mis à jour avec $\langle A:7 \rangle$.

Quand le noeud C effectue l'opération locale \emph{c1}, le module de livraison obtient l'information de l'époque courante de la séquence : \epoch{b3}.
C met à jour son vecteur de dots des opérations \emph{rename} pour ne conserver que l'entrée du noeud B : $\langle B:3 \rangle$.
Ce dot est ajouté en tant que dépendance de l'opération \emph{c1} avant sa diffusion.

À la réception de l'opération \emph{c1} par le noeud A, cette opération est mise en attente par le module de livraison, l'opération \emph{b3} n'ayant pas encore été livrée.
Le noeud reçoit ensuite l'opération \emph{b3}.
Son vecteur des dots des opérations \emph{rename} est mis à jour et l'opération livrée.
Les conditions pour l'opération \emph{c1} étant désormais remplies, l'opération est alors livrée.\\

Cette implémentation de la contrainte de la livraison \emph{epoch-based} dispose de plusieurs avantages : sa complexité spatiale dépend linéairement du nombre de noeuds et les opérations de mise à jour du vecteur des dots des opérations \emph{rename} s'effectuent en temps constant.
De plus, seul un dot est ajouté en tant que dépendance des opérations, la taille du vecteur des dots étant ramené à 1 au préalable.
Finalement, cette implémentation ne contraint pas une livraison causale des opérations \emph{rename} et permet donc de les appliquer dès que possible.
