\documentclass[12pt]{thesul}
%----------------------------------------------------------------------
%                               Packages
%----------------------------------------------------------------------
\usepackage{acronym} % \ac[p], \acl[p], \acs[p], \acf[p]
\usepackage{biblatex}
\bibliography{biblio.bib}
\usepackage{csquotes}
\usepackage[inline]{enumitem}
\usepackage[french]{minitoc}

\usepackage{color}
\AtBeginDocument{
\definecolor{pdfurlcolor}{rgb}{0,0,0}
\definecolor{pdfcitecolor}{rgb}{0,0,0}
\definecolor{pdflinkcolor}{rgb}{0,0,0}
\definecolor{light}{gray}{.85}
\definecolor{vlight}{gray}{.95}
\definecolor{darkgreen}{RGB}{77,172,38}
\definecolor{darkblue}{RGB}{5,113,176}
\definecolor{mydarkblue}{RGB}{116,173,209}
\definecolor{mydarkblueid}{RGB}{83,154,198}
\definecolor{mylightblue}{RGB}{171,217,233}
\definecolor{mydarkorange}{RGB}{244,109,67}
\definecolor{mylightorange}{RGB}{252,153,54}
\definecolor{mydarkred}{RGB}{215,48,39}
\definecolor{mydarkpurple}{RGB}{140,107,177}
\definecolor{mydarkpurpleid}{RGB}{136,86,167}
}

\usepackage{hyperref}
\hypersetup{hidelinks}

\usepackage[draft,inline,nomargin,index]{fixme}
\fxsetup{theme=color,mode=multiuser,inlineface=\itshape,envface=\itshape}
\FXRegisterAuthor{go}{ago}{Gerald}
\FXRegisterAuthor{mn}{amn}{Matthieu}

\usepackage{tikz} % \begin{tikzpicture} \end{tikzpicture}
\usetikzlibrary{calc}
\usetikzlibrary{shapes.misc}

\usepackage[caption=false,font=footnotesize,labelfont=sf,textfont=sf]{subfig}

% Commands
%---------
\newcommand{\eg}{e.g. }
\newcommand{\ie}{i.e. }

\newcommand{\headerparagraph}[1]{\textbf{\emph{#1}}\quad}

\newcommand{\inbb}[1]{\in \mathbb{#1}}
\newcommand{\mathlist}[2]{\set{#1_i \in #2}_{i \inbb{N}}}
\newcommand{\new}{\textbf{new}}
\newcommand{\trm}[1]{\mathit{#1}}
\newcommand{\set}[1]{\left\{#1\right\}} % set brace notation

\newcommand{\id}[3]{$\trm{#1}^{\trm{#2}}_{\trm{#3}}$}
\newcommand{\epoch}[1]{$\varepsilon_{#1}$}

\newcommand{\widthletter}{2em}
\newcommand{\widthblock}{3em}
\newcommand{\widthoriginepoch}{1.65em}
\newcommand{\widthepoch}{1.8em}

% Tikz styles
\tikzset{
    common/.style={anchor=west, draw, rectangle, minimum height=6mm},
    letter/.style={common, minimum width=\widthletter},
    block/.style={common, minimum width=\widthblock},
    epoch/.style={letter, rounded rectangle, rounded rectangle east arc=0pt, minimum width=\widthepoch},
    op/.style={draw, circle, minimum size=2.7em},
    causalop/.style={op, double=white, inner sep=2pt},
    gc-rule-1/.style={dashed, thick, darkblue},
    gc-rule-2/.style={densely dotted, thick, darkgreen},
    cross/.style={
        path picture={
            \draw[mydarkred, very thick]
                (path picture bounding box.south east)--(path picture bounding box.north west)
                (path picture bounding box.south west)--(path picture bounding box.north east);
        }
    }
}


%-------------------------------------------------------------------
%                             Marges
%-------------------------------------------------------------------

% pour positionner les vraies marges:
%\SetRealMargins{1mm}{1mm}

%-------------------------------------------------------------------
%                             En-têtes
%-------------------------------------------------------------------

% Les en-têtes: quelques exemples
%\UppercaseHeadings
%\UnderlineHeadings
%\newcommand\bfheadings[1]{{\bf #1}}
%\FormatHeadingsWith{\bfheadings}
%\FormatHeadingsWith{\uppercase}
%\FormatHeadingsWith{\underline}
\newcommand\upun[1]{\uppercase{\underline{\underline{#1}}}}
\FormatHeadingsWith\upun

\newcommand\itheadings[1]{\textit{#1}}
\FormatHeadingsWith{\itheadings}

% pour avoir un trait sous l'en-tete:
\setlength{\HeadRuleWidth}{0.4pt}

%-------------------------------------------------------------------
%                         Les références
%-------------------------------------------------------------------

\NoChapterNumberInRef
\NoChapterPrefix

%-------------------------------------------------------------------
%                           Brouillons
%-------------------------------------------------------------------

% ceci ajoute une marque « brouillon » et la date
\ThesisDraft

%-------------------------------------------------------------------
%                   Pour collecter un glossaire et un index
%-------------------------------------------------------------------

\makeglossary
\makeindex

%-------------------------------------------------------------------
%                           Acronymes
%-------------------------------------------------------------------

% Acronyms
% --------
% \input{assets/acronyms.tex}
\acrodef{ADT}[ADT]{Abstract Data Type}
\acrodefplural{ADT}[ADTs]{Abstract Data Types}
\acrodef{CRDT}[CRDT]{Conflict-free Replicated Data Type}
\acrodefplural{CRDT}[CRDTs]{Conflict-free Replicated Data Types}
\acrodef{JIT}[JIT]{Just-In-Time}
\acrodef{OT}[OT]{Operational Transformation}
\acrodefplural{OT}[OT]{Operational Transformations}
\acrodef{P2P}[P2P]{Pair-à-Pair}
\acrodef{SEC}[SEC]{Strong Eventual Consistency}

%-------------------------------------------------------------------
%                           Couleurs
%-------------------------------------------------------------------

% \input{assets/colours.tex}

%-------------------------------------------------------------------
%                     Global custom tikz commands
%-------------------------------------------------------------------

% \input{assets/tikz_presets.tex}

\begin{document}


      \OddHead={{\leftmark\rightmark}{\hfil\slshape\rightmark}}
      \EvenHead={{\leftmark}{{\slshape\leftmark}\hfil}}
      \OddFoot={\hfil\thepage}
      \EvenFoot={\thepage\hfil}
      \pagestyle{ThesisHeadingsII}


%-------------------------------------------------------------------
%                          Encadrements
%-------------------------------------------------------------------

% encadre les chapitres dans la table des matières:
% (ces commandes doivent figurer apres \begin{document}

\FrameChaptersInToc
%\FramePartsInToc


%-------------------------------------------------------------------
%            Réinitialisation de la numérotation des chapitres
%-------------------------------------------------------------------

% Si la commande suivante est présente,
% elle doit figurer APRÈS \begin{document}
% et avant la première commande \part
\ResetChaptersAtParts

%-------------------------------------------------------------------
%               mini-tables des matières par chapitre
%-------------------------------------------------------------------

% préparer les mini-tables des matières par chapitre.
% (commande de minitoc.sty)
\dominitoc

%-------------------------------------------------------------------
%                         Page de titre:
%-------------------------------------------------------------------

\ThesisTitle{Ré-identification efficace dans les types de données répliquées sans conflit (CRDTs)}
\ThesisDate{TODO: Définir une date}
\ThesisAuthor{Matthieu Nicolas}

% Type de la these
\ThesisUL
% Jury:

% (ne pas mettre de \\ apres la dernière entree)

% Exemple de création d'une nouvelle catégorie dans le jury:

\NewJuryCategory{family}{\it Membre de la famille :}
                        {\it Membres de la famille :}

\family={Mon frère\\Ma sœur}

\def\blanc{\hspace*{1cm}}

\President    = {Stephan Merz}
\Rapporteurs  = {Le rapporteur 1&de Paris\\
                 Le rapporteur 2\\
                 \blanc suite&taratata\\
                 Le rapporteur 3}
\Examinateurs = {L'examinateur 1&d'ici\\
                 L'examinateur 2}
%\Invites=       {}

% Création de la page de titre:
\MakeThesisTitlePage

%-------------------------------------------------------------------


%-------------------------------------------------------------------
%                          remerciements
%-------------------------------------------------------------------

%\DontFrameThisInToc
\begin{ThesisAcknowledgments}
Les remerciements.
\end{ThesisAcknowledgments}

%-------------------------------------------------------------------
%                            dédicace
%-------------------------------------------------------------------

\begin{ThesisDedication}
Je dédie cette thèse\\
à ma machine.\\
Oui, à Pandore,\\
qui fut la première de toutes.
\end{ThesisDedication}


%-------------------------------------------------------------------
%                  écriture de `Chapitre' et `Partie'
%                      dans la table des matières
%-------------------------------------------------------------------

\WritePartLabelInToc
\WriteChapterLabelInToc

%-------------------------------------------------------------------
%                        table des matières
%-------------------------------------------------------------------

\tableofcontents

%-------------------------------------------------------------------
%              Exemple d'utilisation de \SpecialSection
%-------------------------------------------------------------------
%\SpecialSection{Introduction générale}

\DontWriteThisInToc
\listoffigures

\mainmatter
\NumberThisInToc
\chapter*{Introduction}
\minitoc
\section{Contexte}
\section{Questions de recherche}
\section{Contributions}
\section{Plan du manuscrit}
% Les systèmes collaboratifs temps réels permettent à plusieurs utilisateur-rices de réaliser une tâche de manière coopérative.
Ils permettent aux utilisateur-rices de consulter le contenu actuel, de le modifier et d'observer en direct les modifications effectuées par les autres collaborateur-rices.
L'observation en temps réel des modifications des autres favorise une réflexion de groupe et permet une répartition efficace des tâches.
L'utilisation des systèmes collaboratifs se traduit alors par une augmentation de la qualité du résultat produit \cite{2004-empirical-study-collaborative-writing, 2005-internet-encyclopaedias-head-to-head}.

Plusieurs outils d'édition collaborative centralisés basés sur l'approche \acf{OT} \cite{1989-grove-ellis-gibbs} ont permis de populariser l'édition collaborative temps réel de texte \cite{gdocs, etherpad}.
Ces approches souffrent néanmoins de leur architecture centralisée.
Notamment, ces solutions rencontrent des difficultés à passer à l'échelle \cite{2015-cope-delay-collaborative-note-taking-ignat, 2016-performance-collaborative-editors-dang-ignat} et posent des problèmes de confidentialité \cite{prism-washington-post, prism-guardian}.

L'approche \ac{CRDT} offre une meilleure capacité de passage à l'échelle et est compatible avec une architecture \ac{P2P} \cite{2011-evaluation-crdts-ahmed-nacer}.
Ainsi, de nombreux travaux \cite{Nedelec2016CRATE, peerpad, serenity-notes} ont été entrepris pour proposer une alternative distribuée répondant aux limites des éditeurs collaboratifs centralisés.
De manière plus globale, ces travaux s'inscrivent dans le nouveau paradigme d'application des \acfp{LFS} \cite{localfirstsoftware2019, pushpin2020}.
Ce paradigme vise le développement d'applications collaboratives, \ac{P2P}, pérennes et rendant la souveraineté de leurs données aux utilisateurs.\\

\mnnote{TODO: Serait intéressant d'ajouter une catégorisation des éditeurs collaboratifs en fonction de leurs caractéristiques (décentralisé vs. p2p, pas de chiffrement vs. chiffrement serveur vs. chiffrement de bout en bout, OT vs CRDT vs mécanisme de résolution de conflits custom...) pour mettre en avant le caractère unique de MUTE}

De manière semblable, l'équipe Coast conçoit depuis plusieurs années des applications avec ces mêmes objectifs et étudient les problématiques de recherche liées.
Elle développe \acf{MUTE} \cite{MUTE2017}\footnote{Disponible à l'adresse : \url{https://mutehost.loria.fr}}\footnote{Code source disponible à l'adresse suivante : \url{https://github.com/coast-team/mute}}, un éditeur collaboratif \ac{P2P} temps réel chiffré de bout en bout.
\ac{MUTE} sert de plateforme d'expérimentation et de démonstration pour les travaux de l'équipe.

Ainsi, nous avons contribué à son développement dans le cadre de cette thèse.
Notamment, nous avons participé à :
\begin{enumerate}
  \item L'implémentation des \acp{CRDT} LogootSplit \cite{2013-logootsplit} et RenamableLogootSplit \cite{2022-rls-tpds-nicolas} pour représenter le document texte.
  \item L'implémentation de leur modèle de livraison de livraison respectifs.
  \item L'implémentation d'un protocole d'appartenance au réseau, SWIM \cite{swim2002}.
\end{enumerate}

Dans ce chapitre, nous commençons par présenter le projet \ac{MUTE} : ses objectifs, ses fonctionnalités et son architecture système et logicielle.
Puis nous détaillons ses différentes couches logicielles : leur rôle, l'approche choisie pour leur implémentation et finalement leurs limites actuelles.
Au cours de cette description, nous mettons l'emphase sur les composants auxquelles nous avons contribué, \ie les sections \ref{sec:mute-replication}, et \ref{sec:mute-livraison}.


% \NumberThisInToc
% \chapter*{Problématique}
% \minitoc
% \input{assets/chapter_problematic}

\NumberThisInToc
\chapter{État de l'art}
\minitoc
\section{Transformées opérationnelles}
\section{Séquences répliquées sans conflits}
\subsection{Types de données répliquées sans conflits}
\subsubsection{Principes}

\begin{itemize}
  \item Nouvelles spécifications des types de données existants
  \item Structures conçues pour être répliquées au sein d'un système
  \item Et être modifiées sans coordination par ses différents noeuds
  \item Doivent donc supporter de nouveaux scénarios uniquement possible dans des exécutions parallèles
  \item Et définir une sémantique pour ces scénarios inédits
  \begin{itemize}
    \item Exemple du Registre avec LWW-Register et MV-Register ?
  \end{itemize}
  \item Pour gérer ces scénarios, intègrent un mécanisme de résolution de conflits directement au sein de leur spécification
  \item Garantissent la cohérence forte à terme
\end{itemize}

\subsubsection{Familles de types de données répliquées sans conflits}

\begin{itemize}
  \item Une catégorisation des CRDTs a été proposée
  \item Propose de répartir les CRDTs en différentes familles en fonction de la méthode de synchronisation utilisée
  \item Chacune de ces méthodes de synchronisation implique des contraintes sur la couche réseau du système et entraîne des répercussions sur la structure de données elle-même
  \item Types de données répliquées sans conflits à base d'états \cite{shapiro_2011_crdt, shapiro:inria-00555588}
  \begin{itemize}
    \item Les noeuds partagent leur état de manière périodique
    \item Une fonction \emph{merge} permet aux noeuds de fusionner leur état courant avec un autre état reçu
    \item Aucune hypothèse sur la partie réseau autre que les noeuds arrivent à communiquer à terme
    \item Pas un problème si états perdus, les prochains intégreront les informations de ces derniers
    \item Pas un problème si états reçus dans le désordre, la fonction \emph{merge} est commutative
    \item Pas un problème si états reçus plusieurs fois, \emph{merge} est idempotent
    \item Mais nécessite de conserver au sein de la structure de données assez d'informations pour proposer une telle fonction de \emph{merge}
    \item Par exemple, besoin de conserver une trace des éléments supprimés pour empêcher leur réapparition suite à une fusion d'états
    \item \mnnote{TODO: Ajouter forces, faiblesses et cas d'utilisation de cette approche}
  \end{itemize}
  \item Types de données répliquées sans conflits à base d'opérations \cite{shapiro_2011_crdt, shapiro:inria-00555588, 10.1145/2596631.2596632, baquero2017pure}
  \begin{itemize}
    \item Les noeuds partagent uniquement des opérations représentant leurs modifications
    \item Une modification peut se formaliser en deux étapes
    \item \emph{prepare}, qui permet de générer une opération correspondant à une modification
    \item \emph{effect}, qui permet d'appliquer l'effet de la modification à un état
    \item Les opérations concurrentes doivent être commutatives pour assurer la convergence
    \item Mais pas de contraintes sur les opérations causalement liées
    \item Pas de contraintes non plus sur l'idempotence des opérations
    \item Nécessite donc généralement d'ajouter une couche \emph{livraison} pour faire le lien entre le réseau et le CRDT
    \item Permet d'attacher des informations de causalité aux opérations locales avant de les envoyer
    \item Permet de ré-ordonner et filtrer les opérations distantes reçues avant de les fournir au CRDT
    \item Besoin d'un mécanisme d'anti-entropie \cite{10.1109/TSE.1983.236733} pour assurer que l'ensemble des noeuds observent l'ensemble des opérations et ainsi garantir la convergence
    \item Permet de lisser la consommation réseau
    \item Offre des temps d'intégration et de propagation des modifications rapides
    \item Mais accumule des métadonnées puisque les noeuds doivent conserver les opérations passées pour permettre à un nouveau noeud de rejoindre la collaboration et de se synchroniser
    \item Possible de tronquer le log des opérations en se basant sur la stabilité causale \cite{10.1007/978-3-662-43352-2_11} afin de limiter cette accumulation de métadonnées
  \end{itemize}
  \item Types de données répliquées sans conflits à base de différences \cite{almeida2015delta, Almeida_2018}
\end{itemize}

\subsubsection{Adoption dans la littérature et l'industrie}

\begin{itemize}
  \item Conception et développement de librairies mettant à disposition des développeurs d'applications des types de données composés \cite{Nicolaescu2015Yjs, Nicolaescu2016YATA, yjsimplem, jsoncrdt2017, automerge}
  \item Conception de langages de programmation intégrant des CRDTs comme types primitifs, destinés au développement d'applications distribuées \cite{Meiklejohn2015Lasp2, DePorre2020cscript}
  \item Conception et implémentation de bases de données distribuées, relationnelles ou non, privilégiant la disponibilité et la minimisation de la latence à l'aide des CRDTs \cite{RiakKV, AntidoteDB, Anna2021, Concordant, yu:hal-02983557}
  \item Conception d'un nouveau paradigme d'applications, Local-First Software, dont une des fondations est les CRDTs \cite{localfirstsoftware2019, pushpin2020}
  \item Éditeurs collaboratifs temps réel à large échelle et offrant de nouveaux scénarios de collaboration grâce aux CRDTs \cite{Nedelec2016CRATE, MUTE2017}
\end{itemize}

\subsection{Approches pour les séquences répliquées sans conflits}
\subsubsection{Approche à pierres tombales}

\begin{itemize}
  \item WOOT \cite{oster:inria-00108523, Weiss_2007, ahmednacer:inria-00629503}
  \item RGA \cite{ROH2011354}
  \item RGASplit \cite{briot:hal-01343941}
\end{itemize}

\subsubsection{Approche à identifiants densément ordonnés}

\begin{itemize}
  \item Treedoc \cite{5158449}
  \item Logoot \cite{WeissICDCS09, weiss:hal-00450416}
\end{itemize}

\section{LogootSplit}
\subsection{Identifiants}
\subsubsection{Composition}
\subsubsection{Stratégie d'allocation}
\subsection{Aggrégation dynamique d'élements en blocs}
\subsection{Limites}
\section{Mitigation du surcoût des séquences répliquées sans conflits}
\subsection{Core-Nebula}
\subsection{LSEQ}

\mnnote{Serait intéressant d'avoir une implémentation combinant LogootSplit et LSEQ pour vérifier si les contraintes sur la création de blocs dans LogootSplit ne "sabotent" pas la croissance polylogarithmique des identifiants de LSEQ}

\section{Synthèse}

% \include{assets/chapter_soa_ergm}

\NumberThisInToc
\chapter{Présentation de l'approche}
\minitoc

Nous proposons un nouveau \ac{CRDT} pour la \emph{Sequence} appartenant à l'approche des identifiants densément ordonnées : RenamableLogootSplit \cite{nicolas:hal-01932552,nicolas:hal-02526724}.
Cette structure de données permet aux pairs d'insérer et de supprimer des éléments au sein d'une séquence répliquée.
Nous introduisons une opération de renommage qui permet de
\begin{enumerate*}
  \item réassigner des identifiants plus courts aux différents éléments de la séquence
  \item fusionner les blocs composant la séquence.
\end{enumerate*}
Ces deux actions permettent à l'opération de renommage de produire un nouvel état minimisant son surcoût en métadonnées.

\section{Modèle du système}

Le système est composé d'un ensemble dynamique de noeuds, les noeuds pouvant rejoindre puis quitter la collaboration tout au long de sa durée.
Les noeuds collaborent afin de construire et maintenir une séquence à l'aide de RenamableLogootSplit.
Chaque noeud possède une copie de la séquence et peut l'éditer sans se coordonner avec les autres.
Les modifications des noeuds prennent la forme d'opérations qui sont appliquées immédiatement à leur copie locale.
Les opérations sont ensuite transmises de manière asynchrone aux autres noeudds pour qu'ils puissent à leur tour appliquer les modifications à leur copie.

Les noeuds communiquement par l'intermédiaire d'un réseau \ac{P2P}.
Ce réseau est non-fiable : les messages peuvent être perdus, ré-ordonnés ou même livrés à plusieurs reprises.
Le réseau peut aussi être sujet à des partitions, qui séparent alors les noeuds en des sous-groupes disjoints.
Afin de compenser les limitations du réseau, les noeuds reposent sur une couche de livraison de messages.

Puisque RenamableLogootSplit est une extension de LogootSplit, il partage les mêmes contraintes sur la livraison de messages.
La couche de livraison de messages sert donc à livrer les messages à l'application exactement une fois.
La couche de livraison de messages a aussi pour tâche de garantir la livraison des opérations de suppression après les opérations d'insertion correspondantes.
Finalement, la couche de livraison intègre aussi un mécanisme d'anti-entropie \cite{10.1109/TSE.1983.236733}.
Ce mécanisme permet aux noeuds de se synchroniser par paires, en détectant et ré-échangeant les messages perdus.

\section{Définition de l'opération de renommage}
\subsection{Objectifs}
\subsection{Propriétés}
\subsection{Contraintes} % Non-bloquante, performante (pour un utilisateur)

\NumberThisInToc
\chapter{Renommage dans un système centralisé}
\minitoc
\section{RenamableLogootSplit}
\subsection{Opération de renommage proposée}

Notre opération de renommage permet à RenamableLogootSplit de réduire le surcoût en métadonnées des séquences répliquées.
Pour ce faire, elle réassigne des identifiants arbitraires aux éléments de la séquence.

\begin{figure}[t!]
  \centering
  \subfloat[Selecting the new identifier of the first element]{
      \begin{minipage}{\linewidth}
          \centering
          \begin{tikzpicture}
              \path
                  node {\textbf{A}}
                  to ++(0:\widthletter) node[letter, label=below:{\id{i}{B0}{0}}] {H}
                  to ++(0:\widthletter) node[letter, fill=mydarkorange, label=above:{\color{mydarkorange}\id{i}{B0}{0}\id{f}{A0}{0}}] {E}
                  to ++(0:\widthletter) node[block, label=below:{\id{i}{B0}{1..2}}] (LO) {LO}
                  to ++(0:4 * \widthletter) node[letter, fill=mydarkblue, label=below:{\color{mydarkblueid}\id{i}{A1}{0}}] (H) {H};

              \draw[->, thick] (LO) -- node[below, align=center]{\emph{rename}} (H);
          \end{tikzpicture}
          \label{fig:renaming-first-id}
      \end{minipage}}
  \hfil
  \subfloat[Selecting the new identifiers of the remaining ones]{
      \begin{minipage}{\linewidth}
          \centering
          \begin{tikzpicture}
              \path
                  node {\textbf{A}}
                  to ++(0:\widthletter) node[letter, label=below:{\id{i}{B0}{0}}] {H}
                  to ++(0:\widthletter) node[letter, fill=mydarkorange, label=above:{\color{mydarkorange}\id{i}{B0}{0}\id{f}{A0}{0}}] {E}
                  to ++(0:\widthletter) node[block, label=below:{\id{i}{B0}{1..2}}] (LO) {LO}
                  to ++(0:4 * \widthletter) node[letter, fill=mydarkblue, label=below:{\color{mydarkblueid}\id{i}{A1}{0}}] (H) {H}
                  to ++(0:\widthletter) node[letter, fill=mydarkblue, label=below:{\color{mydarkblueid}\id{i}{A1}{1}}] {E}
                  to ++(0:\widthletter) node[letter, fill=mydarkblue, label=below:{\color{mydarkblueid}\id{i}{A1}{2}}] {L}
                  to ++(0:\widthletter) node[letter, fill=mydarkblue, label=below:{\color{mydarkblueid}\id{i}{A1}{3}}] {O};

                  \draw[->, thick] (LO) -- node[below, align=center]{\emph{rename}} (H);
              \end{tikzpicture}
          \label{fig:renaming-second-id}
      \end{minipage}}
  \hfil
  \subfloat[Final state obtained]{
      \begin{minipage}{\linewidth}
          \centering
          \begin{tikzpicture}
              \path
                  node {\textbf{A}}
                  to ++(0:\widthletter) node[letter, label=below:{\id{i}{B0}{0}}] {H}
                  to ++(0:\widthletter) node[letter, fill=mydarkorange, label=above:{\color{mydarkorange}\id{i}{B0}{0}\id{f}{A0}{0}}] {E}
                  to ++(0:\widthletter) node[block, label=below:{\id{i}{B0}{1..2}}] (LO) {LO}
                  to ++(0:4 * \widthletter) node[block, fill=mydarkblue, label=below:{\color{mydarkblueid}\id{i}{A1}{0..3}}] (HELO) {HELO};

              \draw[->, thick] (LO) -- node[below, align=center]{\emph{rename}} (HELO);
          \end{tikzpicture}
          \label{fig:renaming-final-state}
      \end{minipage}}
  \caption{Renaming the sequence on node \emph{A}}
  \label{fig:renaming}
\end{figure}

Son comportement est illustré dans la \autoref{fig:renaming}.
Dans cet exemple, le noeud A initie une opération \emph{rename} sur son état local.
Tout d'abord, le noeud A réutilise l'identifiant du premier élément de la séquence (\id{i}{B0}{0}) mais en le modifiant avec son propre identifiant de noeud (\textbf{A}) et numéro de séquence actuel (\emph{1}).
De plus, son offset est mis à 0.
Le noeud A réassigne l'identifiant résultant (\id{i}{A1}{0}) au premier élément de la séquence, comme décrit dans \autoref{fig:renaming-first-id}.
Ensuite, le noeud A dérive des identifiants contigus pour tous les éléments restants en incrémentant de manière successive l'offset (\id{i}{A1}{1}, \id{i}{A1}{2}, \id{i}{A1}{3}), comme présenté dans \autoref{fig:renaming-second-id}.
Comme nous assignons des identifiants consécutifs à tous les éléments de la séquence, nous pouvons au final aggréger ces éléments en un seul bloc, comme illustré en \autoref{fig:renaming-final-state}.
Ceci permet aux noeuds de bénéficier au mieux de la fonctionnalité des blocs et de minimiser le surcoût en métadonnés de l'état résultat.

\subsection{Gestion des opérations concurrentes au renommage}
\subsection{Processus d'intégration d'une opération}
\subsection{Évolution du modèle de cohérence}

\mnnote{TODO: Revoir si une livraison causale de l'opération \emph{rename} (et non epoch-based) peut avoir un intérêt}

\subsection{Récupération de la mémoire des états précédents}
\section{Validation}
\subsection{Preuve de correction}
\subsection{Complexité temporelle}
\section{Discussion}
\subsection{Stockage des états précédents sur disque}
\subsection{Utilisation de l'opération de renommage comme snapshot}

\begin{itemize}
  \item L'opération de renommage embarque la somme de toutes les opérations passées sous la forme du \emph{former state}
  \item On peut à tout moment re-calculer l'état courant du document à partir de l'opération de renommage primaire, des opérations concurrentes à cette dernière et des opérations générées depuis
  \item Peut utiliser ce principe pour le mécanisme de synchronisation
  \item Lorsqu'un nouveau pair rejoint la collaboration, le noeud avec lequel il se synchronise peut lui fournir uniquement ce sous-ensemble des opérations
  \item De la même manière, on pourrait généraliser l'utilisation de cette méthode de synchronisation
  \item À la réception d'une demande de synchronisation d'un pair présentant un important retard, le noeud peut choisir d'employer cette méthode
  \begin{itemize}
    \item Plutôt que de lui envoyer et de lui faire rejouer l'ensemble des opérations
  \end{itemize}
  \item La question étant de comment procéder pour quantifier ce retard et pour définir le seuil à partir duquel ce retard est considéré comme important
  \item Cette méthode de synchronisation pose néanmoins le problème suivant
  \item Le pair synchronisé de cette manière ne possède qu'une partie du log des opérations
  \item S'il reçoit ensuite une demande de synchronisation d'un autre pair, il est possible qu'il ne puisse y répondre
  \begin{itemize}
    \item Cas où il manque à l'autre pair juste une opération d'avant le renommage (possible si les dépendances causales ne sont pas requises pour intégrer l'opération de renommage)
  \end{itemize}
  \item Dans ce cas, ne peut pas fournir la seule opération manquante au pair qui la demande
  \begin{itemize}
    \item \mnnote{NOTE: Mais dans ce cas, le pair peut tout à fait générer un état courant à jour à partir des infos qu'il possède puisque l'opération qui lui manque est intégrée dans l'opé de renommage}
  \end{itemize}
  \item \mnnote{TODO: Étudier si y a un intérêt à privilégier la synchronisation basée sur l'intégration successive de toutes les opérations quand on a cette méthode de synchronisation par snapshot/checkpoint de possible}
\end{itemize}

\subsection{Compression de l'opération de renommage}
\subsection{Limitation de la taille de l'opération de renommage}
\section{Conclusion}
% \include{assets/chapter_baye_approach}

\NumberThisInToc
\chapter{Renommage dans un système distribué}
\minitoc
\section{RenamableLogootSplit v2}
\subsection{Conflits en cas de renommages concurrents}
\subsection{Méthode de résolution de conflits proposée}
\subsection{Relation de priorité entre renommages}
\subsection{Algorithme d'annulation de l'opération de renommage}
\subsection{Mise à jour du processus d'intégration d'une opération}
\subsection{Mise à jour des règles de récupération de la mémoire des états précédents}
\section{Validation}
\subsection{Complexité temporelle}
\subsection{Expérimentations}
\subsubsection{Scénario d'expérimentation}
\subsubsection{Implémentation des simulations}
\subsection{Résultats}
\subsubsection{Convergence}
\subsubsection{Consommation mémoire}
\subsubsection{Temps d'intégration des opérations "simples"}
\subsubsection{Temps d'intégration de l'opération de renommage}
\section{Discussion}
\subsection{Implémentation alternative à base d'operation-log}
\subsection{Définition de relations de priorité plus optimales}
\subsection{Report de la transition vers la nouvelle epoch principale}
\section{Conclusion}
% \include{assets/chapter_application_HAL}

\NumberThisInToc
\chapter{Stratégies de déclenchement du renommage}
\minitoc
\section{Motivation}
\section{Stratégies proposées}
\subsection{Propriétés}
\subsection{Stratégie 1 : ???}
\subsection{Stratégie 2 : ???}

NOTE: Peut considérer une stratégie où on prend en compte la hauteur de l'epoch tree pour déclencher un renommage : peut attendre qu'on ait plus qu'une epoch pour autoriser un nouveau renommage d'avoir lieu.
Permettrait d'empêcher le cas où un noeud revient après 6 mois d'absence et doit intégrer 100 renommages avant de pouvoir collaborer.
Mais dans le cas où un noeud ne rejoint plus la collaboration, bloque le mécanisme de renommage pour les autres

\section{Évaluation}
\section{Conclusion}
% \include{assets/chapter_extension}

\NumberThisInToc
\chapter{Conclusions et perspectives}
\minitoc
\section{Résumé des contributions}
\section{Perspectives}
\subsection{Définition de relations de priorité plus optimales}
\subsection{Redéfinition de la sémantique du renommage en déplacement d'éléments}
\subsection{Définition de types de données répliquées sans conflits plus complexes}
% Dans ce chapitre, nous avons présenté \acf{MUTE}, l'éditeur collaboratif temps réel \ac{P2P} chiffré de bout en bout développé par notre équipe de recherche.\\

MUTE permet d'éditer de manière collaborative des documents texte.
Pour représenter les documents, MUTE propose plusieurs \acp{CRDT} pour le type Séquence \cite{2013-logootsplit,2021-these-vic,2022-rls-tpds-nicolas} issus des travaux de l'équipe.
Ces \acp{CRDT} offrent de nouvelles méthodes de collaborer, notamment en permettant de collaborer de manière synchrone ou asynchrone de manière transparente.\\

Pour permettre aux noeuds de communiquer, MUTE repose sur la technologie WebRTC.
Cette technologie permet de construire un réseau \ac{P2P} directement entre plusieurs navigateurs.
Plusieurs serveurs sont néanmoins requis, notamment pour la découverte des pairs et pour la communication entre des noeuds lorsque leur pare-feux respectifs empêchent l'établissement d'une connexion directe.\\

Finalement, MUTE implémente un mécanisme de chiffrement de bout en bout garantissant l'authenticité et la confidentialité des échanges entre les noeuds.
Ce mécanisme repose sur une clé de groupe de chiffrement qui est établie à l'aide du protocole \cite{1995-burmester-desmedt}.

Ce protocole nécessite que chaque noeud possède une paire de clés de chiffrement et qu'ils partagent leur clé publique.
\ac{MUTE} repose sur des \acp{PKI} pour cela.
Avant de détecter tout éventuel comportement malicieux de la part de ces derniers, \ac{MUTE} intègre un mécanisme d'audit \cite{2018-trusternity-short,2018-trusternity-long}.\\


\Annex{Algorithmes}
% \include{assets/annex_extension}

%
%%-------------------------------------------------------------------
%%                         Le glossaire
%%-------------------------------------------------------------------
%\BeginGloWith{Voici un glossaire tout-à-fait fictif,
%              introduit par un texte sur toute la largeur
%              des deux colonnes.}
%\twocolumn
%\PrintGlossary

%-------------------------------------------------------------------
%              L'index (toujours sur deux colonnes)
%-------------------------------------------------------------------
\BeginIndWith{Voici un index}
\PrintIndex

\onecolumn

%-------------------------------------------------------------------
%                       La bibliographie
%-------------------------------------------------------------------

% La bibliographie (comme d'habitude)

%\nocite{*}
%\bibliographystyle{named}

\printbibliography

%-------------------------------------------------------------------
%                          Les résumés
%-------------------------------------------------------------------
% (si le résumé apparaît sur une colonne étroite, avec la
% bibliographie à gauche, c'est sans doute parce que vous avez
% oublié de générer les fichiers d'index et de glossaire...)

\NumberAbstractPages
\begin{ThesisAbstract}
  \begin{FrenchAbstract}
    Afin d'assurer leur haute disponibilité, les systèmes distribués à large échelle se doivent de répliquer leurs données tout en minimisant les coordinations nécessaires entre noeuds.
    Pour concevoir de tels systèmes, la littérature et l'industrie adoptent de plus en plus l'utilisation de types de données répliquées sans conflits (CRDTs).
    Les CRDTs sont des types de données qui offrent des comportements similaires aux types existants, tel l'Ensemble ou la Séquence.
    Ils se distinguent cependant des types traditionnels par leur spécification, qui supporte nativement les modifications concurrentes.
    À cette fin, les CRDTs incorporent un mécanisme de résolution de conflits au sein de leur spécification.

    Afin de résoudre les conflits de manière déterministe, les CRDTs associent généralement des identifiants aux éléments stockés au sein de la structure de données.
    Les identifiants doivent respecter un ensemble de contraintes en fonction du CRDT, telles que l'unicité ou l'appartenance à un ordre dense.
    Ces contraintes empêchent de borner la taille des identifiants.
    La taille des identifiants utilisés croît alors continuellement avec le nombre de modifications effectuées, aggravant le surcoût lié à l'utilisation des CRDTs par rapport aux structures de données traditionnelles.
    Le but de cette thèse est de proposer des solutions pour pallier ce problème.

    Nous présentons dans cette thèse deux contributions visant à répondre à ce problème :
    \begin{enumerate*}[label=(\roman*)]
      \item Un nouveau CRDT pour Séquence, RenamableLogootSplit, qui intègre un mécanisme de renommage à sa spécification.
      Ce mécanisme de renommage permet aux noeuds du système de réattribuer des identifiants de taille minimale aux éléments de la séquence.
      Cependant, cette première version requiert une coordination entre les noeuds pour effectuer un renommage.
      L'évaluation expérimentale montre que le mécanisme de renommage permet de réinitialiser à chaque renommage le surcoût lié à l'utilisation du CRDT.
      \item Une seconde version de RenamableLogootSplit conçue pour une utilisation dans un système distribué.
      Cette nouvelle version permet aux noeuds de déclencher un renommage sans coordination préalable.
      L'évaluation expérimentale montre que cette nouvelle version présente un surcoût temporaire en cas de renommages concurrents, mais que ce surcoût est à terme.
    \end{enumerate*}
    \KeyWords{CRDTs, édition collaborative en temps réel, cohérence à terme, optimisation mémoire, performance}
  \end{FrenchAbstract}
  \begin{EnglishAbstract}
    \KeyWords{CRDTs, real-time collaborative editing, eventual consistency, memory-wise optimisation, performance}
  \end{EnglishAbstract}
\end{ThesisAbstract}


\end{document}



