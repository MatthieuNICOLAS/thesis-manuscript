\documentclass[12pt]{thesul}
%----------------------------------------------------------------------
%                               Packages
%----------------------------------------------------------------------
\usepackage{acronym} % \ac[p], \acl[p], \acs[p], \acf[p]
\usepackage{biblatex}
\usepackage[inline]{enumitem}
\usepackage[french]{minitoc}

%-------------------------------------------------------------------
%                             Marges
%-------------------------------------------------------------------

% pour positionner les vraies marges:
%\SetRealMargins{1mm}{1mm}

%-------------------------------------------------------------------
%                             En-têtes
%-------------------------------------------------------------------

% Les en-têtes: quelques exemples
%\UppercaseHeadings
%\UnderlineHeadings
%\newcommand\bfheadings[1]{{\bf #1}}
%\FormatHeadingsWith{\bfheadings}
%\FormatHeadingsWith{\uppercase}
%\FormatHeadingsWith{\underline}
\newcommand\upun[1]{\uppercase{\underline{\underline{#1}}}}
\FormatHeadingsWith\upun

\newcommand\itheadings[1]{\textit{#1}}
\FormatHeadingsWith{\itheadings}

% pour avoir un trait sous l'en-tete:
\setlength{\HeadRuleWidth}{0.4pt}

%-------------------------------------------------------------------
%                         Les références
%-------------------------------------------------------------------

\NoChapterNumberInRef
\NoChapterPrefix

%-------------------------------------------------------------------
%                           Brouillons
%-------------------------------------------------------------------

% ceci ajoute une marque « brouillon » et la date
\ThesisDraft

%-------------------------------------------------------------------
%                   Pour collecter un glossaire et un index
%-------------------------------------------------------------------

\makeglossary
\makeindex

%-------------------------------------------------------------------
%                           Acronymes
%-------------------------------------------------------------------

% Acronyms
% --------
% \input{assets/acronyms.tex}
\acrodef{ADT}[ADT]{Abstract Data Type}
\acrodefplural{ADT}[ADTs]{Abstract Data Types}
\acrodef{CRDT}[CRDT]{Conflict-free Replicated Data Type}
\acrodefplural{CRDT}[CRDTs]{Conflict-free Replicated Data Types}
\acrodef{JIT}[JIT]{Just-In-Time}
\acrodef{OT}[OT]{Operational Transformation}
\acrodefplural{OT}[OT]{Operational Transformations}
\acrodef{P2P}[P2P]{Peer-to-Peer}
\acrodef{SEC}[SEC]{Strong Eventual Consistency}

%-------------------------------------------------------------------
%                           Couleurs
%-------------------------------------------------------------------

% \input{assets/colours.tex}

%-------------------------------------------------------------------
%                     Global custom tikz commands
%-------------------------------------------------------------------

% \input{assets/tikz_presets.tex}

\begin{document}


      \OddHead={{\leftmark\rightmark}{\hfil\slshape\rightmark}}
      \EvenHead={{\leftmark}{{\slshape\leftmark}\hfil}}
      \OddFoot={\hfil\thepage}
      \EvenFoot={\thepage\hfil}
      \pagestyle{ThesisHeadingsII}


%-------------------------------------------------------------------
%                          Encadrements
%-------------------------------------------------------------------

% encadre les chapitres dans la table des matières:
% (ces commandes doivent figurer apres \begin{document}

\FrameChaptersInToc
%\FramePartsInToc


%-------------------------------------------------------------------
%            Réinitialisation de la numérotation des chapitres
%-------------------------------------------------------------------

% Si la commande suivante est présente,
% elle doit figurer APRÈS \begin{document}
% et avant la première commande \part
\ResetChaptersAtParts

%-------------------------------------------------------------------
%               mini-tables des matières par chapitre
%-------------------------------------------------------------------

% préparer les mini-tables des matières par chapitre.
% (commande de minitoc.sty)
\dominitoc

%-------------------------------------------------------------------
%                         Page de titre:
%-------------------------------------------------------------------

\ThesisTitle{Ré-identification efficace dans les types de données répliquées sans conflit (CRDTs)}
\ThesisDate{28 janvier 1986}
\ThesisAuthor{Matthieu Nicolas}

% Type de la these
\ThesisUL
% Jury:

% (ne pas mettre de \\ apres la dernière entree)

% Exemple de création d'une nouvelle catégorie dans le jury:

\NewJuryCategory{family}{\it Membre de la famille :}
                        {\it Membres de la famille :}

\family={Mon frère\\Ma sœur}

\def\blanc{\hspace*{1cm}}

\President    = {Stephan Merz}
\Rapporteurs  = {Le rapporteur 1&de Paris\\
                 Le rapporteur 2\\
                 \blanc suite&taratata\\
                 Le rapporteur 3}
\Examinateurs = {L'examinateur 1&d'ici\\
                 L'examinateur 2}
%\Invites=       {}

% Création de la page de titre:
\MakeThesisTitlePage

%-------------------------------------------------------------------


%-------------------------------------------------------------------
%                          remerciements
%-------------------------------------------------------------------

%\DontFrameThisInToc
\begin{ThesisAcknowledgments}
Les remerciements.
\end{ThesisAcknowledgments}

%-------------------------------------------------------------------
%                            dédicace
%-------------------------------------------------------------------

\begin{ThesisDedication}
Je dédie cette thèse\\
à ma machine.\\
Oui, à Pandore,\\
qui fut la première de toutes.
\end{ThesisDedication}


%-------------------------------------------------------------------
%                  écriture de `Chapitre' et `Partie'
%                      dans la table des matières
%-------------------------------------------------------------------

\WritePartLabelInToc
\WriteChapterLabelInToc

%-------------------------------------------------------------------
%                        table des matières
%-------------------------------------------------------------------

\tableofcontents

%-------------------------------------------------------------------
%              Exemple d'utilisation de \SpecialSection
%-------------------------------------------------------------------
%\SpecialSection{Introduction générale}

\DontWriteThisInToc
\listoffigures

\mainmatter
\NumberThisInToc
\chapter*{Introduction}
\minitoc
\section{Contexte}
\section{Questions de recherche}
\section{Contributions}
\section{Plan du manuscrit}
% Les systèmes collaboratifs temps réels permettent à plusieurs utilisateur-rices de réaliser une tâche de manière coopérative.
Ils permettent aux utilisateur-rices de consulter le contenu actuel, de le modifier et d'observer en direct les modifications effectuées par les autres collaborateur-rices.
L'observation en temps réel des modifications des autres favorise une réflexion de groupe et permet une répartition efficace des tâches.
L'utilisation des systèmes collaboratifs se traduit alors par une augmentation de la qualité du résultat produit \cite{2004-empirical-study-collaborative-writing, 2005-internet-encyclopaedias-head-to-head}.

Plusieurs outils d'édition collaborative centralisés basés sur l'approche \acf{OT} \cite{1989-grove-ellis-gibbs} ont permis de populariser l'édition collaborative temps réel de texte \cite{gdocs, etherpad}.
Ces approches souffrent néanmoins de leur architecture centralisée.
Notamment, ces solutions rencontrent des difficultés à passer à l'échelle \cite{2015-cope-delay-collaborative-note-taking-ignat, 2016-performance-collaborative-editors-dang-ignat} et posent des problèmes de confidentialité \cite{prism-washington-post, prism-guardian}.

L'approche \ac{CRDT} offre une meilleure capacité de passage à l'échelle et est compatible avec une architecture \ac{P2P} \cite{2011-evaluation-crdts-ahmed-nacer}.
Ainsi, de nombreux travaux \cite{Nedelec2016CRATE, peerpad, serenity-notes} ont été entrepris pour proposer une alternative distribuée répondant aux limites des éditeurs collaboratifs centralisés.
De manière plus globale, ces travaux s'inscrivent dans le nouveau paradigme d'application des \acfp{LFS} \cite{localfirstsoftware2019, pushpin2020}.
Ce paradigme vise le développement d'applications collaboratives, \ac{P2P}, pérennes et rendant la souveraineté de leurs données aux utilisateurs.\\

\mnnote{TODO: Serait intéressant d'ajouter une catégorisation des éditeurs collaboratifs en fonction de leurs caractéristiques (décentralisé vs. p2p, pas de chiffrement vs. chiffrement serveur vs. chiffrement de bout en bout, OT vs CRDT vs mécanisme de résolution de conflits custom...) pour mettre en avant le caractère unique de MUTE}

De manière semblable, l'équipe Coast conçoit depuis plusieurs années des applications avec ces mêmes objectifs et étudient les problématiques de recherche liées.
Elle développe \acf{MUTE} \cite{MUTE2017}\footnote{Disponible à l'adresse : \url{https://mutehost.loria.fr}}\footnote{Code source disponible à l'adresse suivante : \url{https://github.com/coast-team/mute}}, un éditeur collaboratif \ac{P2P} temps réel chiffré de bout en bout.
\ac{MUTE} sert de plateforme d'expérimentation et de démonstration pour les travaux de l'équipe.

Ainsi, nous avons contribué à son développement dans le cadre de cette thèse.
Notamment, nous avons participé à :
\begin{enumerate}
  \item L'implémentation des \acp{CRDT} LogootSplit \cite{2013-logootsplit} et RenamableLogootSplit \cite{2022-rls-tpds-nicolas} pour représenter le document texte.
  \item L'implémentation de leur modèle de livraison de livraison respectifs.
  \item L'implémentation d'un protocole d'appartenance au réseau, SWIM \cite{swim2002}.
\end{enumerate}

Dans ce chapitre, nous commençons par présenter le projet \ac{MUTE} : ses objectifs, ses fonctionnalités et son architecture système et logicielle.
Puis nous détaillons ses différentes couches logicielles : leur rôle, l'approche choisie pour leur implémentation et finalement leurs limites actuelles.
Au cours de cette description, nous mettons l'emphase sur les composants auxquelles nous avons contribué, \ie les sections \ref{sec:mute-replication}, et \ref{sec:mute-livraison}.


\NumberThisInToc
\chapter*{Problématique}
\minitoc
% \input{assets/chapter_problematic}

\NumberThisInToc
\chapter{État de l'art}
\minitoc
\section{Transformées opérationnelles}
\section{Séquences répliquées sans conflits}
\subsection{Type de données répliquées sans conflits}
\subsection{Approches pour les séquences répliquées sans conflits}
\section{LogootSplit}
% \include{assets/chapter_soa_ergm}

\NumberThisInToc
\chapter{Renommage dans un système centralisé}
\minitoc
\section{Approche}
\section{RenamableLogootSplit}
\subsection{Opération de renommage}
\subsubsection{Propriétés}
\subsubsection{Proposition}
\subsection{Gestion des opérations concurrentes au renommage}
\subsection{Récupération de la mémoire des états précédents}
\subsection{Modèle de cohérence}
\section{Évaluation}
\subsection{Expérimentations}
\subsubsection{Scénario d'expérimentation}
\subsubsection{Implémentation des simulations}
\subsection{Résultats}
\subsubsection{Convergence}
\subsubsection{Consommation mémoire}
\subsubsection{Temps d'intégration des opérations "simples"}
\subsubsection{Temps d'intégration de l'opération de renommage}
\section{Discussion}
\subsection{Stockage d'anciens états sur disque}
\subsection{Compression de l'opération de renommage}
\subsection{Limitation de la taille de l'opération de renommage}
% \include{assets/chapter_baye_approach}

\NumberThisInToc
\chapter{Renommage dans un système distribué}
\minitoc
\section{Approche}
\section{RenamableLogootSplit v2}
\subsection{Opération de renommage et inversion du renommage}
\subsubsection{Propriétés}
\subsubsection{Proposition}
\subsection{Relation de priorité entre renommages concurrents}
\subsection{Identification des renommages à inverser et à appliquer}
\subsection{Récupération de la mémoire des états précédents}
\section{Évaluation}
\subsection{Expérimentations}
\subsubsection{Scénario d'expérimentation}
\subsubsection{Implémentation des simulations}
\subsection{Résultats}
\subsubsection{Convergence}
\subsubsection{Consommation mémoire}
\subsubsection{Temps d'intégration des opérations "simples"}
\subsubsection{Temps d'intégration de l'opération de renommage}
\section{Discussion}
\subsection{Définition de relations de priorité plus optimales}
\subsection{Report de la transition vers la nouvelle epoch principale}
% \include{assets/chapter_application_HAL}

\NumberThisInToc
\chapter{Stratégies de déclenchement du renommage}
\section{Motivation}
\section{Stratégies proposées}
\subsection{Propriétés}
\subsection{Stratégie 1 : ???}
\subsection{Stratégie 2 : ???}
\section{Évaluation}
\minitoc
% \include{assets/chapter_extension}

\NumberThisInToc
\chapter{Conclusions et perspectives}
\minitoc
\section{Résumé des contributions}
\section{Perspectives}
\subsection{Définition de relations de priorité plus optimales}
\subsection{Redéfinition de la sémantique du renommage en déplacement d'éléments}
\subsection{Définition de types de données répliquées sans conflits plus complexes}
% Dans ce chapitre, nous avons présenté \acf{MUTE}, l'éditeur collaboratif temps réel \ac{P2P} chiffré de bout en bout développé par notre équipe de recherche.\\

MUTE permet d'éditer de manière collaborative des documents texte.
Pour représenter les documents, MUTE propose plusieurs \acp{CRDT} pour le type Séquence \cite{2013-logootsplit,2021-these-vic,2022-rls-tpds-nicolas} issus des travaux de l'équipe.
Ces \acp{CRDT} offrent de nouvelles méthodes de collaborer, notamment en permettant de collaborer de manière synchrone ou asynchrone de manière transparente.\\

Pour permettre aux noeuds de communiquer, MUTE repose sur la technologie WebRTC.
Cette technologie permet de construire un réseau \ac{P2P} directement entre plusieurs navigateurs.
Plusieurs serveurs sont néanmoins requis, notamment pour la découverte des pairs et pour la communication entre des noeuds lorsque leur pare-feux respectifs empêchent l'établissement d'une connexion directe.\\

Finalement, MUTE implémente un mécanisme de chiffrement de bout en bout garantissant l'authenticité et la confidentialité des échanges entre les noeuds.
Ce mécanisme repose sur une clé de groupe de chiffrement qui est établie à l'aide du protocole \cite{1995-burmester-desmedt}.

Ce protocole nécessite que chaque noeud possède une paire de clés de chiffrement et qu'ils partagent leur clé publique.
\ac{MUTE} repose sur des \acp{PKI} pour cela.
Avant de détecter tout éventuel comportement malicieux de la part de ces derniers, \ac{MUTE} intègre un mécanisme d'audit \cite{2018-trusternity-short,2018-trusternity-long}.\\


\Annex{Algorithmes}
% \include{assets/annex_extension}

%
%%-------------------------------------------------------------------
%%                         Le glossaire
%%-------------------------------------------------------------------
%\BeginGloWith{Voici un glossaire tout-à-fait fictif,
%              introduit par un texte sur toute la largeur
%              des deux colonnes.}
%\twocolumn
%\PrintGlossary

%-------------------------------------------------------------------
%              L'index (toujours sur deux colonnes)
%-------------------------------------------------------------------
\BeginIndWith{Voici un index}
\PrintIndex

\onecolumn

%-------------------------------------------------------------------
%                       La bibliographie
%-------------------------------------------------------------------

% La bibliographie (comme d'habitude)

%\nocite{*}
%\bibliographystyle{named}

\printbibliography

%-------------------------------------------------------------------
%                          Les résumés
%-------------------------------------------------------------------
% (si le résumé apparaît sur une colonne étroite, avec la
% bibliographie à gauche, c'est sans doute parce que vous avez
% oublié de générer les fichiers d'index et de glossaire...)

\NumberAbstractPages
\begin{ThesisAbstract}
  \begin{FrenchAbstract}
    Afin d'assurer leur haute disponibilité, les systèmes distribués à large échelle se doivent de répliquer leurs données tout en minimisant les coordinations nécessaires entre noeuds.
    Pour concevoir de tels systèmes, la littérature et l'industrie adoptent de plus en plus l'utilisation de types de données répliquées sans conflits (CRDTs).
    Les \acp{CRDT} sont des types de données qui offrent des comportements similaires aux types existants, tel l'Ensemble ou la Séquence.
    Ils se distinguent cependant des types traditionnels par leur spécification, qui supporte nativement les modifications concurrentes.
    À cette fin, les \acp{CRDT} incorporent un mécanisme de résolution de conflits au sein de leur spécification.

    Afin de résoudre les conflits de manière déterministe, les \acp{CRDT} associent généralement des identifiants aux éléments stockés au sein de la structure de données.
    Les identifiants doivent respecter un ensemble de contraintes en fonction du \ac{CRDT}, telles que l'unicité ou l'appartenance à un ordre dense.
    Ces contraintes empêchent de borner la taille des identifiants.
    La taille des identifiants utilisés croît alors continuellement avec le nombre de modifications effectuées, aggravant le surcoût lié à l'utilisation des \acp{CRDT} par rapport aux structures de données traditionnelles.
    Le but de cette thèse est de proposer des solutions pour pallier ce problème.

    Nous présentons dans cette thèse deux contributions visant à répondre à ce problème :
    \begin{enumerate*}[label=(\roman*)]
      \item Un nouveau \ac{CRDT} pour Séquence, RenamableLogootSplit, qui intègre un mécanisme de renommage à sa spécification.
      Ce mécanisme de renommage permet aux noeuds du système de réattribuer des identifiants de taille minimale aux éléments de la séquence.
      Cependant, cette première version requiert une coordination entre les noeuds pour effectuer un renommage.
      L'évaluation expérimentale montre que le mécanisme de renommage permet de réinitialiser à chaque renommage le surcoût lié à l'utilisation du \ac{CRDT}.
      \item Une seconde version de RenamableLogootSplit conçue pour une utilisation dans un système distribué.
      Cette nouvelle version permet aux noeuds de déclencher un renommage sans coordination préalable.
      L'évaluation expérimentale montre que cette nouvelle version présente un surcoût temporaire en cas de renommages concurrents, mais que ce surcoût est à terme.
    \end{enumerate*}
    \KeyWords{CRDTs, édition collaborative en temps réel, cohérence à terme, optimisation mémoire, performance}
  \end{FrenchAbstract}
  \begin{EnglishAbstract}
    \KeyWords{CRDTs, real-time collaborative editing, eventual consistency, memory-wise optimisation, performance}
  \end{EnglishAbstract}
\end{ThesisAbstract}


\end{document}



