\documentclass[12pt]{thesul}
%----------------------------------------------------------------------
%                               Packages
%----------------------------------------------------------------------
\usepackage[french]{babel}
\usepackage{acronym} % \ac[p], \acl[p], \acs[p], \acf[p]
\usepackage{biblatex}
\bibliography{biblio.bib}
\usepackage{booktabs} % \toprule, \midrule, \cmidrule, \bottomrule
\usepackage{caption}
\usepackage{csquotes}
\usepackage{hyperref}
\hypersetup{hidelinks}
\usepackage[inline]{enumitem}
\usepackage[french]{minitoc}

\usepackage{xcolor}
\AtBeginDocument{
\definecolor{pdfurlcolor}{rgb}{0,0,0}
\definecolor{pdfcitecolor}{rgb}{0,0,0}
\definecolor{pdflinkcolor}{rgb}{0,0,0}
\definecolor{light}{gray}{.85}
\definecolor{vlight}{gray}{.95}
\definecolor{darkgreen}{RGB}{77,172,38}
\definecolor{darkblue}{RGB}{5,113,176}
\definecolor{mydarkblue}{RGB}{116,173,209}
\definecolor{mydarkblueid}{RGB}{83,154,198}
\definecolor{mylightblue}{RGB}{171,217,233}
\definecolor{mydarkorange}{RGB}{244,109,67}
\definecolor{mylightorange}{RGB}{252,153,54}
\definecolor{mydarkred}{RGB}{215,48,39}
\definecolor{mydarkpurple}{RGB}{140,107,177}
\definecolor{mydarkpurpleid}{RGB}{136,86,167}
}

\usepackage{amssymb}
\usepackage{amsmath}
\interdisplaylinepenalty=2500
\newtheorem{definition}{Définition}
\newtheorem{myrule}{Règle}
\newtheorem{property}{Propriété}
\newtheorem{subproperty}{Propriété}[property]

\usepackage{algpseudocode}

\usepackage[draft,inline,nomargin,index]{fixme}
\fxsetup{theme=color,mode=multiuser,inlineface=\itshape,envface=\itshape}
\FXRegisterAuthor{go}{ago}{Gerald}
\FXRegisterAuthor{mn}{amn}{Matthieu}

\usepackage{tikz} % \begin{tikzpicture} \end{tikzpicture}
\usetikzlibrary{calc}
\usetikzlibrary{graphs}
\usetikzlibrary{quotes}
\usetikzlibrary{shapes.misc}

\usepackage[caption=false,font=footnotesize,labelfont=sf,textfont=sf]{subfig}

% Commands
%---------
\newcommand{\eg}{e.g. }
\newcommand{\ie}{c.-à-d. }

\newcommand{\headerparagraph}[1]{\textbf{\emph{#1}}\quad}

\newcommand{\inbb}[1]{\in \mathbb{#1}}
\newcommand{\mathlist}[2]{\set{#1_i \in #2}_{i \inbb{N}}}
\newcommand{\new}{\textbf{new}}
\newcommand{\trm}[1]{\mathit{#1}}
\newcommand{\set}[1]{\left\{#1\right\}} % set brace notation

\newcommand{\id}[3]{$\trm{#1}^{\trm{#2}}_{\trm{#3}}$}
\newcommand{\epoch}[1]{$\varepsilon_{#1}$}
\newcommand{\lid}{$<_{id}$~}
\newcommand{\leqid}{$\leq_{id}$~}
\newcommand{\lepoch}{$<_{\varepsilon}$~}
\newcommand{\leqepoch}{$\leq_{\varepsilon}$~}

\newcommand{\widthletter}{2em}
\newcommand{\widthblock}{3em}
\newcommand{\widthoriginepoch}{1.65em}
\newcommand{\widthepoch}{1.8em}

% Tikz styles
\tikzset{
    common/.style={anchor=west, draw, rectangle, minimum height=6mm},
    letter/.style={common, minimum width=\widthletter},
    block/.style={common, minimum width=\widthblock},
    epoch/.style={letter, rounded rectangle, rounded rectangle east arc=0pt, minimum width=\widthepoch},
    point/.style={insert path={ node[scale=5*sqrt(\pgflinewidth)]{.} }},
    op/.style={draw, circle, minimum size=2.7em},
    causalop/.style={op, double=white, inner sep=2pt},
    gc-rule-1/.style={dashed, thick, darkblue},
    gc-rule-2/.style={densely dotted, thick, darkgreen},
    cross/.style={
        path picture={
            \draw[mydarkred, very thick]
                (path picture bounding box.south east)--(path picture bounding box.north west)
                (path picture bounding box.south west)--(path picture bounding box.north east);
        }
    }
}


%-------------------------------------------------------------------
%                             Marges
%-------------------------------------------------------------------

% pour positionner les vraies marges:
%\SetRealMargins{1mm}{1mm}

%-------------------------------------------------------------------
%                             En-têtes
%-------------------------------------------------------------------

% Les en-têtes: quelques exemples
%\UppercaseHeadings
%\UnderlineHeadings
%\newcommand\bfheadings[1]{{\bf #1}}
%\FormatHeadingsWith{\bfheadings}
%\FormatHeadingsWith{\uppercase}
%\FormatHeadingsWith{\underline}
\newcommand\upun[1]{\uppercase{\underline{\underline{#1}}}}
\FormatHeadingsWith\upun

\newcommand\itheadings[1]{\textit{#1}}
\FormatHeadingsWith{\itheadings}

% pour avoir un trait sous l'en-tete:
\setlength{\HeadRuleWidth}{0.4pt}

%-------------------------------------------------------------------
%                         Les références
%-------------------------------------------------------------------

\NoChapterNumberInRef
\NoChapterPrefix

%-------------------------------------------------------------------
%                           Brouillons
%-------------------------------------------------------------------

% ceci ajoute une marque « brouillon » et la date
\ThesisDraft

%-------------------------------------------------------------------
%                   Pour collecter un glossaire et un index
%-------------------------------------------------------------------

\makeglossary
\makeindex

%-------------------------------------------------------------------
%                           Acronymes
%-------------------------------------------------------------------

% Acronyms
% --------
% \input{assets/acronyms.tex}
\acrodef{ADT}[ADT]{Abstract Data Type}
\acrodefplural{ADT}[ADTs]{Abstract Data Types}
\acrodef{CRDT}[CRDT]{Conflict-free Replicated Data Type}
\acrodefplural{CRDT}[CRDTs]{Conflict-free Replicated Data Types}
\acrodef{JIT}[JIT]{Just-In-Time}
\acrodef{LCA}[PPAC]{Plus Petit Ancêtre Commun}
\acrodef{OT}[OT]{Operational Transformation}
\acrodefplural{OT}[OT]{Operational Transformations}
\acrodef{P2P}[P2P]{Pair-à-Pair}
\acrodef{SEC}[SEC]{Cohérence forte à terme}

%-------------------------------------------------------------------
%                           Couleurs
%-------------------------------------------------------------------

% \input{assets/colours.tex}

%-------------------------------------------------------------------
%                     Global custom tikz commands
%-------------------------------------------------------------------

% \input{assets/tikz_presets.tex}

\begin{document}


      \OddHead={{\leftmark\rightmark}{\hfil\slshape\rightmark}}
      \EvenHead={{\leftmark}{{\slshape\leftmark}\hfil}}
      \OddFoot={\hfil\thepage}
      \EvenFoot={\thepage\hfil}
      \pagestyle{ThesisHeadingsII}


%-------------------------------------------------------------------
%                          Encadrements
%-------------------------------------------------------------------

% encadre les chapitres dans la table des matières:
% (ces commandes doivent figurer apres \begin{document}

\FrameChaptersInToc
%\FramePartsInToc


%-------------------------------------------------------------------
%            Réinitialisation de la numérotation des chapitres
%-------------------------------------------------------------------

% Si la commande suivante est présente,
% elle doit figurer APRÈS \begin{document}
% et avant la première commande \part
\ResetChaptersAtParts

%-------------------------------------------------------------------
%               mini-tables des matières par chapitre
%-------------------------------------------------------------------

% préparer les mini-tables des matières par chapitre.
% (commande de minitoc.sty)
\dominitoc

%-------------------------------------------------------------------
%                         Page de titre:
%-------------------------------------------------------------------

\ThesisTitle{Ré-identification efficace dans les types de données répliquées sans conflit (CRDTs)}
\ThesisDate{TODO: Définir une date}
\ThesisAuthor{Matthieu Nicolas}

% Type de la these
\ThesisUL
% Jury:

% (ne pas mettre de \\ apres la dernière entree)

% Exemple de création d'une nouvelle catégorie dans le jury:

\NewJuryCategory{family}{\it Membre de la famille :}
                        {\it Membres de la famille :}

\family={Mon frère\\Ma sœur}

\def\blanc{\hspace*{1cm}}

\President    = {Stephan Merz}
\Rapporteurs  = {Le rapporteur 1&de Paris\\
                 Le rapporteur 2\\
                 \blanc suite&taratata\\
                 Le rapporteur 3}
\Examinateurs = {L'examinateur 1&d'ici\\
                 L'examinateur 2}
%\Invites=       {}

% Création de la page de titre:
\MakeThesisTitlePage

%-------------------------------------------------------------------


%-------------------------------------------------------------------
%                          remerciements
%-------------------------------------------------------------------

%\DontFrameThisInToc
\begin{ThesisAcknowledgments}
Les remerciements.
\end{ThesisAcknowledgments}

%-------------------------------------------------------------------
%                            dédicace
%-------------------------------------------------------------------

\begin{ThesisDedication}
Je dédie cette thèse\\
à ma machine.\\
Oui, à Pandore,\\
qui fut la première de toutes.
\end{ThesisDedication}


%-------------------------------------------------------------------
%                  écriture de `Chapitre' et `Partie'
%                      dans la table des matières
%-------------------------------------------------------------------

\WritePartLabelInToc
\WriteChapterLabelInToc

%-------------------------------------------------------------------
%                        table des matières
%-------------------------------------------------------------------

\tableofcontents

%-------------------------------------------------------------------
%              Exemple d'utilisation de \SpecialSection
%-------------------------------------------------------------------
%\SpecialSection{Introduction générale}

\DontWriteThisInToc
\listoffigures

\mainmatter
\NumberThisInToc
\chapter*{Introduction}
\minitoc
\section{Contexte}
\section{Questions de recherche}
\section{Contributions}
\section{Plan du manuscrit}
% Les systèmes collaboratifs temps réels permettent à plusieurs utilisateur-rices de réaliser une tâche de manière coopérative.
Ils permettent aux utilisateur-rices de consulter le contenu actuel, de le modifier et d'observer en direct les modifications effectuées par les autres collaborateur-rices.
L'observation en temps réel des modifications des autres favorise une réflexion de groupe et permet une répartition efficace des tâches.
L'utilisation des systèmes collaboratifs se traduit alors par une augmentation de la qualité du résultat produit \cite{2004-empirical-study-collaborative-writing, 2005-internet-encyclopaedias-head-to-head}.

Plusieurs outils d'édition collaborative centralisés basés sur l'approche \acf{OT} \cite{1989-grove-ellis-gibbs} ont permis de populariser l'édition collaborative temps réel de texte \cite{gdocs, etherpad}.
Ces approches souffrent néanmoins de leur architecture centralisée.
Notamment, ces solutions rencontrent des difficultés à passer à l'échelle \cite{2015-cope-delay-collaborative-note-taking-ignat, 2016-performance-collaborative-editors-dang-ignat} et posent des problèmes de confidentialité \cite{prism-washington-post, prism-guardian}.

L'approche \ac{CRDT} offre une meilleure capacité de passage à l'échelle et est compatible avec une architecture \ac{P2P} \cite{2011-evaluation-crdts-ahmed-nacer}.
Ainsi, de nombreux travaux \cite{Nedelec2016CRATE, peerpad, serenity-notes} ont été entrepris pour proposer une alternative distribuée répondant aux limites des éditeurs collaboratifs centralisés.
De manière plus globale, ces travaux s'inscrivent dans le nouveau paradigme d'application des \acfp{LFS} \cite{localfirstsoftware2019, pushpin2020}.
Ce paradigme vise le développement d'applications collaboratives, \ac{P2P}, pérennes et rendant la souveraineté de leurs données aux utilisateurs.\\

\mnnote{TODO: Serait intéressant d'ajouter une catégorisation des éditeurs collaboratifs en fonction de leurs caractéristiques (décentralisé vs. p2p, pas de chiffrement vs. chiffrement serveur vs. chiffrement de bout en bout, OT vs CRDT vs mécanisme de résolution de conflits custom...) pour mettre en avant le caractère unique de MUTE}

De manière semblable, l'équipe Coast conçoit depuis plusieurs années des applications avec ces mêmes objectifs et étudient les problématiques de recherche liées.
Elle développe \acf{MUTE} \cite{MUTE2017}\footnote{Disponible à l'adresse : \url{https://mutehost.loria.fr}}\footnote{Code source disponible à l'adresse suivante : \url{https://github.com/coast-team/mute}}, un éditeur collaboratif \ac{P2P} temps réel chiffré de bout en bout.
\ac{MUTE} sert de plateforme d'expérimentation et de démonstration pour les travaux de l'équipe.

Ainsi, nous avons contribué à son développement dans le cadre de cette thèse.
Notamment, nous avons participé à :
\begin{enumerate}
  \item L'implémentation des \acp{CRDT} LogootSplit \cite{2013-logootsplit} et RenamableLogootSplit \cite{2022-rls-tpds-nicolas} pour représenter le document texte.
  \item L'implémentation de leur modèle de livraison de livraison respectifs.
  \item L'implémentation d'un protocole d'appartenance au réseau, SWIM \cite{swim2002}.
\end{enumerate}

Dans ce chapitre, nous commençons par présenter le projet \ac{MUTE} : ses objectifs, ses fonctionnalités et son architecture système et logicielle.
Puis nous détaillons ses différentes couches logicielles : leur rôle, l'approche choisie pour leur implémentation et finalement leurs limites actuelles.
Au cours de cette description, nous mettons l'emphase sur les composants auxquelles nous avons contribué, \ie les sections \ref{sec:mute-replication}, et \ref{sec:mute-livraison}.


% \NumberThisInToc
% \chapter*{Problématique}
% \minitoc
% \input{assets/chapter_problematic}

\NumberThisInToc
\chapter{État de l'art}
\minitoc

\begin{itemize}
  \item Contexte des systèmes distribués à large échelle
  \item Réplique les données afin de pouvoir supporter les pannes
  \item Adopte le paradigme de la réplication optimiste \cite{10.1145/1057977.1057980}
  \item Autorise les noeuds à consulter et à modifier la donnée sans aucune coordination entre eux
  \item Autorise alors les noeuds à diverger temporairement
  \item Permet d'être toujours disponible, de toujours répondre aux requêtes même en cas de partition réseau
  \item Permet aussi, en temps normal, de réduire le temps de réponse (privilégie la latence) \cite{pacelc2012}
  \item Comme ce modèle autorise les noeuds à modifier la donnée sans se coordonner, possible d'effectuer des modifications concurrentes
  \item Généralement, un mécanisme de résolution de conflits est nécessaire afin d'assurer la convergence des noeuds dans une telle situation
  \item Plusieurs approches ont été proposées pour implémenter un tel mécanisme
\end{itemize}

\section{Transformées opérationnelles}

\begin{itemize}
  \item Approche permettant de gérer des modifications concurrentes sur un type de données
  \item Consiste à transformer les opérations par rapport aux effets des opérations concurrentes pour rendre les rendre commutatives.
    Permet de rendre l'ordre d'intégration des opérations sans importance par rapport à l'état final obtenu
  \item Se décompose en 2 parties : algorithmes (génériques) et fonctions de transformations (spécifiques au type de données)
  \item Plusieurs algorithmes OT adoptent une architecture centralisée (trouver citations)
  \item Cette architecture pose des problèmes de performances (bottleneck), sécurité (SPOF), coût, d'utilisabilité (mode offline), pérennité (disparition du service), vie privée et de résistance à la censure.
  \item Pour ces raisons, des algorithmes reposant sur une architecture décentralisée ont été proposés
  \item Mais ne règlent qu'en partie ces limites
  \item Notamment, ne sont pas adaptés à des systèmes P2P dynamiques
  \item Besoin de vector clocks sur chaque opération pour détecter la concurrence.
    Vector clocks adaptés dans systèmes à nombre de pairs fixe, mais pas aux systèmes dynamiques (revoir causal barrier pour p-e nuancer ce propos).
  \item Néanmoins, cette approche a permis de démocratiser les systèmes collaboratifs via son adoption par des services tels que Google Docs, Overleaf, Framapad
  \item De plus, dans le cadre de ces travaux, ont été définies les propriétés CCI \cite{10.1145/274444.274447}.
  \item Remettre en question la propriété Causalité des CCI.
    Généralement, confond causalité et happen-before et exprime en finalité une contrainte trop forte.
    Cette contrainte peut réduire la réactivité du système (exemple avec 2 insertions sans liens mais qui force d'attendre la 1ère pour intégrer la 2nde).
    Causalité pose aussi des problèmes de passage à l'échelle car repose sur vector clocks.
    IMO, doit relaxer cette propriété pour pouvoir construire systèmes à large échelle.
\end{itemize}

\mnnote{TODO: Mentionner TP1 et TP2}

\mnnote{TODO: Spécification faible et forte des séquences répliquées}

\section{Séquences répliquées sans conflits}
\subsection{Types de données répliquées sans conflits}
\subsubsection{Principes}

\begin{itemize}
  \item Nouvelles spécifications des types de données existants
  \item Structures conçues pour être répliquées au sein d'un système
  \item Et être modifiées sans coordination par ses différents noeuds
  \item Doivent donc supporter de nouveaux scénarios uniquement possible dans des exécutions parallèles
  \item Et définir une sémantique pour ces scénarios inédits
  \begin{itemize}
    \item Exemple du Registre avec LWW-Register et MV-Register ?
  \end{itemize}
  \item Pour gérer ces scénarios, intègrent un mécanisme de résolution de conflits directement au sein de leur spécification
  \item Garantissent la cohérence forte à terme
\end{itemize}

\mnnote{Faire le lien avec les travaux de Burckhardt \cite{10.1145/2535838.2535848} et les MRDTs \cite{10.1145/3360580}}

\subsubsection{Familles de types de données répliquées sans conflits}

\begin{itemize}
  \item Une catégorisation des CRDTs a été proposée
  \item Propose de répartir les CRDTs en différentes familles en fonction de la méthode de synchronisation utilisée
  \item Chacune de ces méthodes de synchronisation implique des contraintes sur la couche réseau du système et entraîne des répercussions sur la structure de données elle-même
  \item Types de données répliquées sans conflits à base d'états \cite{shapiro_2011_crdt, shapiro:inria-00555588}
  \begin{itemize}
    \item Les noeuds partagent leur état de manière périodique
    \item Une fonction \emph{merge} permet aux noeuds de fusionner leur état courant avec un autre état reçu
    \item Aucune hypothèse sur la partie réseau autre que les noeuds arrivent à communiquer à terme
    \item Pas un problème si états perdus, les prochains intégreront les informations de ces derniers
    \item Pas un problème si états reçus dans le désordre, la fonction \emph{merge} est commutative
    \item Pas un problème si états reçus plusieurs fois, \emph{merge} est idempotent
    \item Mais nécessite de conserver au sein de la structure de données assez d'informations pour proposer une telle fonction de \emph{merge}
    \item Par exemple, besoin de conserver une trace des éléments supprimés pour empêcher leur réapparition suite à une fusion d'états
    \item \mnnote{TODO: Ajouter forces, faiblesses et cas d'utilisation de cette approche}
  \end{itemize}
  \item Types de données répliquées sans conflits à base d'opérations \cite{shapiro_2011_crdt, shapiro:inria-00555588, 10.1145/2596631.2596632, baquero2017pure}
  \begin{itemize}
    \item Les noeuds partagent uniquement des opérations représentant leurs modifications
    \item Une modification peut se formaliser en deux étapes
    \item \emph{prepare}, qui permet de générer une opération correspondant à une modification
    \item \emph{effect}, qui permet d'appliquer l'effet de la modification à un état
    \item Les opérations concurrentes doivent être commutatives pour assurer la convergence
    \item Mais pas de contraintes sur les opérations causalement liées
    \item Pas de contraintes non plus sur l'idempotence des opérations
    \item Nécessite donc généralement d'ajouter une couche \emph{livraison} pour faire le lien entre le réseau et le CRDT
    \item Permet d'attacher des informations de causalité aux opérations locales avant de les envoyer
    \item Permet de ré-ordonner et filtrer les opérations distantes reçues avant de les fournir au CRDT
    \item Besoin d'un mécanisme d'anti-entropie \cite{1983-anti-entropy-vv} pour assurer que l'ensemble des noeuds observent l'ensemble des opérations et ainsi garantir la convergence
      \mnnote{TODO: Ajouter référence mécanisme d'anti-entropie basé sur Merkle Tree}
    \item Permet de lisser la consommation réseau
    \item Offre des temps d'intégration et de propagation des modifications rapides
    \item Mais accumule des métadonnées puisque les noeuds doivent conserver les opérations passées pour permettre à un nouveau noeud de rejoindre la collaboration et de se synchroniser
    \item Possible de tronquer le log des opérations en se basant sur la stabilité causale \cite{10.1007/978-3-662-43352-2_11} afin de limiter cette accumulation de métadonnées
  \end{itemize}
  \item Types de données répliquées sans conflits à base de différences \cite{almeida2015delta, Almeida_2018}
\end{itemize}

\subsubsection{Adoption dans la littérature et l'industrie}

\begin{itemize}
  \item Conception et développement de librairies mettant à disposition des développeurs d'applications des types de données composés \cite{Nicolaescu2015Yjs, Nicolaescu2016YATA, yjsimplem, jsoncrdt2017, automerge}
  \item Conception de langages de programmation intégrant des CRDTs comme types primitifs, destinés au développement d'applications distribuées \cite{Meiklejohn2015Lasp2, DePorre2020cscript}
  \item Conception et implémentation de bases de données distribuées, relationnelles ou non, privilégiant la disponibilité et la minimisation de la latence à l'aide des CRDTs \cite{RiakKV, AntidoteDB, Anna2021, Concordant, yu:hal-02983557}
  \item Conception d'un nouveau paradigme d'applications, Local-First Software, dont une des fondations est les CRDTs \cite{localfirstsoftware2019, pushpin2020}
  \item Éditeurs collaboratifs temps réel à large échelle et offrant de nouveaux scénarios de collaboration grâce aux CRDTs \cite{Nedelec2016CRATE, MUTE2017}
\end{itemize}

\subsection{Approches pour les séquences répliquées sans conflits}
\subsubsection{Approche à pierres tombales}

\begin{itemize}
  \item WOOT \cite{oster:inria-00108523, Weiss_2007, ahmednacer:inria-00629503}
  \item RGA \cite{ROH2011354}
  \item RGASplit \cite{briot:hal-01343941}
\end{itemize}

\subsubsection{Approche à identifiants densément ordonnés}

\begin{itemize}
  \item Treedoc \cite{5158449}
  \item Logoot \cite{WeissICDCS09, weiss:hal-00450416}
\end{itemize}

\mnnote{NOTE: Ajouter LogootSplit de manière sommaire aussi à cet endroit?}

\mnnote{TODO: Autres Sequence CRDTs à considérer : String-wise CRDT \cite{2012-string-wise}, Chronofold \cite{2020-chronofold}}

\section{LogootSplit}

LogootSplit \cite{2013-logootsplit} est l'état de l'art des séquences répliquées à identifiants densément ordonnés.
Comme expliqué précédemment, LogootSplit utilise des identifiants provenant d'un ordre total dense pour positionner les éléments dans la séquence répliquée.

\subsection{Identifiants}

Pour ce faire, LogootSplit assigne des identifiants composés d'une liste de tuples aux éléments.
Les tuples sont définis de la manière suivante :

\begin{definition}[Tuple]
  Un \emph{Tuple} est un quadruplet $\langle$position, nodeId, nodeSeq, offset$\rangle$ où
  \begin{itemize}
    \item position incarne la position souhaitée de l'élément.
    \item nodeId est l'identifiant unique du noeud qui a généré le tuple.
    \item nodeSeq est le numéro de séquence courant du noeud à la génération du tuple.
    \item offset indique la position de l'élément au sein d'un bloc. Nous reviendrons plus en détails sur ce composant dans la \autoref{sec:blocs}.
  \end{itemize}
\end{definition}

\mnnote{TODO: Ajouter une relation d'ordre sur les tuples}

Dans ce manuscrit, nous représentons les tuples par le biais de la notation suivante : \id{position}{nodeId~nodeSeq}{offset} où $\trm{position}$ est une lettre minuscule, $\trm{nodeId}$ une lettre majuscule et $\trm{nodeSeq}$ et $\trm{offset}$ des entiers, \eg \id{i}{B0}{0}.

À partir de là, les identifiants LogootSplit sont définis de la manière suivante :

\begin{definition}[Identifiant]
  Un \emph{Identifiant} est une liste de \emph{Tuples}.
\end{definition}

\mnnote{TODO: Définir la notion de base (et autres fonctions utiles sur les identifiants ? genre isPrefix, concat, getTail...)}

Nous représentons les identifiants en listant les tuples qui les composent.
Par exemple, l'identifiant composé des tuples $\langle\langle$i, B, 0, 0$\rangle\langle$f, A, 0, 0$\rangle\rangle$ est présenté de la manière suivante : \id{i}{B0}{0}\id{f}{A0}{0}.

Les identifiants ont pour rôle d'ordonner les éléments relativement les uns par rapport aux autres.
Pour ce faire, une relation d'ordre totale aux identifiants est associée à l'ensemble des identifiants :

\begin{definition}[Relation \lid]
  La relation \lid est un ordre total strict sur l'ensemble des identifiants.
  Elle permet aux noeuds de comparer n'importe quelle paire d'identifiants.
  Elle est définie en utilisant l'ordre lexicographique sur les composants des différents tuples des identifiants comparés.
\end{definition}

\begin{itemize}
  \item En utilisant cette relation d'ordre, les noeuds peuvent ordonner les éléments grâce à leur identifiant.
  \item Par exemple, déterminent que \id{i}{A1}{0} \lid \id{i}{B0}{0} car les positions sont identiques et que le \emph{nodeId} (A) du premier est plus petit que le \emph{nodeId} (B) du second
  \item et que \id{i}{B0}{0} \lid \id{i}{B0}{0}\id{f}{A0}{0} car le premier est un préfixe du second
\end{itemize}

\mnnote{TODO: Montrer que cet ensemble d'identifiants est un ensemble dense}

\subsection{Aggrégation dynamique d'élements en blocs}

\label{sec:blocs}

Au lieu de stocker les identifiants de chaque élément de la séquence, LogootSplit propose d'aggréger de façon dynamique les éléments dans des blocs.
Pour cela, LogootSplit introduit la notion d'interval d'identifiants :

\begin{definition}[IdInterval]
  Un \emph{IdInterval} est un couple $\langle$idBegin, offsetEnd$\rangle$ où
  \begin{itemize}
    \item idBegin est l'identifiant du premier élément de l'interval.
    \item offsetEnd est l'offset du dernier identifiant de l'interval.
  \end{itemize}
\end{definition}

Les intervals d'identifiants permettent à LogootSplit d'assigner logiquement un identifiant à un ensemble d'éléments, tout en ne stockant réellement que l'identifiant de son premier élément et le dernier offset de son dernier élément.

LogootSplit regroupe les éléments avec des identifiants \emph{contigus} dans un interval.
Nous appelons \emph{contigus} deux identifiants qui partagent une même base (\ie qui sont identiques à l'exception de leur dernier offset) et dont les \emph{offsets} sont consécutifs.
Nous représentons un interval d'identifiants à l'aide du formalisme suivant : \id{position}{nodeId~nodeSeq}{begin..end} où $\trm{begin}$ est l'offset du premier identifiant de l'interval et $\trm{end}$ du dernier.

Les blocs permettent d'associer un interval d'identifiants aux éléments correspondant.
Les blocs sont définis de la manière suivante :

\begin{definition}[Bloc]
  Un \emph{Bloc} est un quadruplet $\langle$idInterval, elts, isAppendable, isPrependable$\rangle$ où
  \begin{itemize}
    \item idInterval est l'interval d'identifiants formant le bloc
    \item elts sont les éléments contenus dans le bloc
    \item isAppendable (resp. isPrependable) est un booléen indiquant si l'auteur du bloc peut ajouter un nouvel élément en fin (resp. début) de bloc
  \end{itemize}
\end{definition}

La \autoref{fig:logootsplit-seq} présente un exemple de séquence LogootSplit : dans la \autoref{fig:logootsplit-seq-as-letters}, les identifiants \id{i}{B0}{0}, \id{i}{B0}{1}, \id{i}{B0}{2} forment une chaîne d'identifiants contigus.
LogootSplit est donc capable de regrouper ces éléments en un bloc représentant l'interval d'identifiants \id{i}{B0}{0..2} pour minimiser les métadonnées stockées, comme montré dans la \autoref{fig:logootsplit-seq-as-block}.

\begin{figure}[!ht]
  \centering
  \subfloat[Elements with their corresponding identifiers]{
      \begin{minipage}{.48\linewidth}
      \centering
          \begin{tikzpicture}
              \path
                  node[letter, label=below:{\id{i}{B0}{0}}] {H}
                  to ++(0:\widthletter) node[letter, label=below:{\id{i}{B0}{1}}] {L}
                  to ++(0:\widthletter) node[letter, label=below:{\id{i}{B0}{2}}] {O};
          \end{tikzpicture}
          \label{fig:logootsplit-seq-as-letters}
      \end{minipage}}
  \hfil
  \subfloat[Elements grouped into a block]{
      \begin{minipage}{.48\linewidth}
      \centering
          \begin{tikzpicture}
              \path
                  node[block, label=below:{\id{i}{B0}{0..2}}] {HLO};
          \end{tikzpicture}
          \label{fig:logootsplit-seq-as-block}
      \end{minipage}}
  \caption{Representation of a LogootSplit sequence containing the elements "HLO"}
  \label{fig:logootsplit-seq}
\end{figure}

Cette fonctionnalité réduit le nombre d'identifiants stockés au sein de la structure de données, puisque les identifiants sont conservés à l'échelle des blocs plutôt qu'à l'échelle de chaque élément.
Ceci permet de réduire de manière significative le surcoût en métadonnées de la structure de données.
L'utilisation de blocs améliore aussi les performances de la structure de données.
En effet, l'utilisation de blocs permet de parcourir plus efficacement la structure de données.
Les blocs permettent aussi d'effectuer des modifications à l'échelle de la chaîne de caractères et non plus seulement caractère par caractère.

\mnnote{TODO: indiquer que le couple $\langle$nodeId, nodeSeq$\rangle$ permet d'identifier de manière unique la base d'un bloc ou d'un identifiant}

Notons que pour une séquence donnée, nous pouvons identifier chacun de ses identifiants par le triplet $\langle$nodeId, nodeSeq, offset$\rangle$ issue de leur dernier Tuple.
Par exemple, le triplet $\langle$B, 0, 2$\rangle$ désigne de manière unique l'identifiant \id{i}{B0}{2} dans \autoref{fig:logootsplit-seq}.

\subsection{Modèle de données}

\textcite{2013-logootsplit} définissent une séquence LogootSplit de la manière suivante :

\begin{definition}[Séquence LogootSplit]
  Une séquence \emph{Séquence LogootSplit} est un triplet $\langle$nodeId, nodeSeq, blocks$\rangle$ où
  \begin{itemize}
    \item nodeId est l'identifiant du noeud.
    \item nodeSeq est le numéro de séquence courant du noeud.
    \item blocks est une liste de Blocs correspondant à l'état actuel de la séquence répliquée.
  \end{itemize}
\end{definition}

Plusieurs fonctions sont définies sur cette structure de données et permettent de l'interroger et de la modifier :

\begin{itemize}
  \item ins(S, index, elts) permet d'insérer les éléments elts à la position index dans la séquence S.
    Cette fonction génère et associe un interval d'identifiants valide aux éléments insérés
    Elle retourne une opération \emph{insert} permettant aux autres noeuds d'intégrer la modification à leur état.
\end{itemize}

\begin{definition}[insert]
  Une opération \emph{insert} est un couple $\langle$id, elts$\rangle$ où
  \begin{itemize}
    \item id est l'identifiant du premier élément inséré par cette opération.
    \item elts est la liste des éléments insérés par cette opération.
  \end{itemize}
\end{definition}

\begin{itemize}
  \item rem(S, index, length) permet de supprimer length éléments à partir la position index dans la séquence S.
  Cette fonction répertorie les éléments supprimés sous la forme d'intervals d'identifiants.
  Elle retourne une opération \emph{remove} permettant aux autres noeuds d'intégrer la modification à leur état.
\end{itemize}

\begin{definition}[remove]
  Une opération \emph{remove} est une liste de IdInterval où chaque idInterval désigne un ensemble d'éléments à supprimer.
\end{definition}

Nous présentons dans la \autoref{fig:logootsplit-example} un exemple d'utilisation de cette séquence répliquée.

\begin{figure}[!ht]
  \centering
  \resizebox{\columnwidth}{!}{
    \begin{tikzpicture}
        \path
            node {\textbf{A}}
            to ++(0:\widthletter) node[block, label=below:{\id{i}{B0}{0..3}}] (S0A) {HRLO}
            to ++(0:5 * \widthletter) node[letter, label=below:{\id{i}{B0}{0}}] (S1A-left) {H}
            to ++(0:\widthletter) node[block, label=below:{\id{i}{B0}{2..3}}] (S1A-right) {LO}
            to ++(0:6 * \widthletter) node[letter, label=below:{\id{i}{B0}{0}}] (S2A-left) {H}
            to ++(0:\widthletter) node[letter, fill=mydarkorange, label=above:{\color{mydarkorange}\id{i}{B0}{0}\id{f}{A0}{0}}] {E}
            to ++(0:\widthletter) node[block, label=below:{\id{i}{B0}{2..3}}] (S2A-right) {LO}
            to ++(0:6 * \widthletter) node[letter, label=below:{\id{i}{B0}{0}}] (S3A-left) {H}
            to ++(0:\widthletter) node[letter, fill=mydarkorange, label=above:{\color{mydarkorange}\id{i}{B0}{0}\id{f}{A0}{0}}] {E}
            to ++(0:\widthletter) node[block, label=below:{\id{i}{B0}{2..4}}] {LO!};


        \path
            to ++(270:4) node {\textbf{B}}
            to ++(0:\widthletter) node[block, label=below:{\id{i}{B0}{0..3}}] (S0B) {HRLO}
            to ++(0:5 * \widthletter) node[block, label=below:{\id{i}{B0}{0..4}}] (S1B) {HRLO!}
            to ++(0:7 * \widthletter) node[letter, label=below:{\id{i}{B0}{0}}] (S2B-left) {H}
            to ++(0:\widthletter) node[block, label=below:{\id{i}{B0}{2..4}}] (S2B-right) {LO!}
            to ++(0:7 * \widthletter) node[letter, label=above:{\id{i}{B0}{0}}] (S3B-left) {H}
            to ++(0:\widthletter) node[letter, fill=mydarkorange, label=below:{\color{mydarkorange}\id{i}{B0}{0}\id{f}{A0}{0}}] {E}
            to ++(0:\widthletter) node[block, label=above:{\id{i}{B0}{2..4}}] {LO!};

        \draw[->, thick]
          (S0A) edge node[above, align=center]{\emph{remove "r"}} (S1A-left)
          (S1A-right) edge node[above, align=center]{\emph{insert "e"}\\\emph{between}\\\emph{"h" and "l"}} (S2A-left)
          (S0B) edge node[below, align=center]{\emph{insert "!"}\\\emph{at the end}} (S1B);

        \draw[dotted]
          (S2A-right) -- (S3A-left)
          (S1B) -- (S2B-left)
          (S2B-right) -- (S3B-left);

        \draw[dashed, ->, thick, shorten >= 3]
          (S1A-right.east) edge node[right, align=center]{\emph{remove} \id{i}{B0}{1..1}}  (S2B-left.west)
          (S2A-right.east) edge node[below right, align=center]{\emph{insert "e" at} {\color{mydarkorange}\id{i}{B0}{0}\id{f}{A0}{0}}} (S3B-left.west)
          (S1B.east) edge node[below right, near end, align=center]{\emph{insert "!" at} \id{i}{B0}{4}} (S3A-left.west);


        % \draw[->, thick] (S0A-right) -- node[above, align=center]{\emph{rename}} (S1A);
        % \draw[dotted] (S1A) -- (S2A-left);
        % \draw[->, thick] (S0B-right) -- node[below, align=center]{\emph{insert "l"}\\\emph{between}\\\emph{"e" and "l"}} (S1B-left);
        % \draw[dashed, ->, thick, shorten >= 3] (S1B-right.east) -- node[below right, align=center]{\emph{insert "l" at} {\color{mylightorange}\id{i}{B0}{0}\id{m}{B1}{0}}} (S2A-left.west);

    \end{tikzpicture}
  }
  \caption{TODO}
  \label{fig:logootsplit-example}
\end{figure}

Dans cet exemple, deux noeuds A et B répliquent et éditent collaborativement un document texte en utilisant LogootSplit.
Ils partagent initialement le même état : une séquence composée d'un seul bloc associant les identifiants \id{i}{B0}{0..3} aux éléments "HRLO".
Les noeuds se mettent ensuite à éditer le document.

Le noeud A commence par supprimer l'élément "r" de la séquence.
LogootSplit génère l'opération \emph{remove} correspondante en utilisant l'identifiant de l'élément supprimé (\id{i}{B0}{1}).
Cette opération est envoyée au noeud B pour qu'il intègre cette modification.

Le noeud A insère ensuite un élément "e" dans la séquence, entre le "h" et le "l".
LogootSplit doit alors générer un identifiant $id$ à associer à ce nouvel élément.
Ce nouvel identifiant $id$ doit respecter la contrainte suivante : \id{i}{B0}{0} \lid $id$ \lid \id{i}{B0}{2}.
Cependant, LogootSplit ne peut pas générer un identifiant composé d'un seul tuple respectant cet ordre.
LogootSplit génère alors $id$ en recopiant le premier tuple (\id{i}{B0}{0}) et en y ajoutant un nouveau tuple (\id{f}{A0}{0}).
LogootSplit génère l'opération \emph{insert} correspondante, indiquant l'élément à insérer et sa position grâce à son identifiant.
Cette opération est ensuite diffusée sur le réseau.

En parallèle, le noeud B insère un élément "!" à la fin de la séquence.
Comme le noeud B est l'auteur du bloc \id{i}{B0}{0..3}, il peut y ajouter de nouveaux éléments.
LogootSplit associe donc l'identifiant \id{i}{B0}{4} à l'élément "!" et l'ajoute au bloc existant.

Les noeuds se synchronisent ensuite.
Le noeud A reçoit l'opération \emph{insert} de l'élément "!" à la position \id{i}{B0}{4}.
Le noeud A détermine que cet élément doit être inséré à la fin de la séquence (puisque \id{i}{B0}{3} \lid \id{i}{B0}{4}) et qu'il peut être ajouté au bloc \id{i}{B0}{2..3} (puisque \id{i}{B0}{3} et \id{i}{B0}{4} sont contigus).

De son côté, le noeud B reçoit tout d'abord l'opération \emph{remove} des éléments identifiés par l'interval \id{i}{B0}{1..1}, \ie l'élément attaché à l'identifiant \id{i}{B0}{1}.
Le noeud B supprime donc l'élément "r" de son état.

Il reçoit ensuite l'opération \emph{insert} de l'élément "e" à la position \id{i}{B0}{0}\id{f}{A0}{0}.
Le noeud B insère cet élément entre les éléments "h" et "l" (puisque \id{i}{B0}{0} \lid \id{i}{B0}{0}\id{f}{A0}{0} \lid \id{i}{B0}{2}), respectant ainsi l'intention du noeud A.

\mnnote{NOTE: Pourrait définir dans cette sous-section la notion de séquence bien-formée}

\subsection{Modèle de livraison}

\label{sec:logootsplit-delivery-model}

Afin de garantir son bon fonctionnement, LogootSplit doit être associé à une couche de livraison de messages garantissant plusieurs propriétés.

\subsubsection{Livraison des opérations en exactement un exemplaire}

Tout d'abord, la couche de livraison de messages doit assurer que toutes les opérations soient délivrées aux noeuds, mais qu'une seule et unique fois.
La \autoref{fig:why-exactly-once-delivery} représente un exemple illustrant la nécessité de cette contrainte.

\begin{figure}[!ht]
  \centering
  \resizebox{\columnwidth}{!}{
    \begin{tikzpicture}
        \path
            node {\textbf{A}}
            to ++(0:\widthletter) node[block, label=below:{\id{p}{A0}{0..4}}] (S0A) {OGNON}
            to ++(0:5 * \widthletter) node[letter, label=below:{\id{p}{A0}{0}}] (S1A-left) {O}
            to ++(0:\widthletter) node[letter, fill=mydarkorange, label=above:{\id{p}{A0}{0}\id{m}{A1}{0}}] {I}
            to ++(0:\widthletter) node[block, label=below:{\id{p}{A0}{1..4}}] (S1A-right) {GNON}
            to ++(0:21 * \widthletter) node[letter, label=below:{\id{p}{A0}{0}}] (S2A-left) {O}
            to ++(0:\widthletter) node[block, label=below:{\id{p}{A0}{1..4}}] {GNON};


        \path
            to ++(270:4) node {\textbf{B}}
            to ++(0:\widthletter) node[block, label=below:{\id{p}{A0}{0..4}}] (S0B) {OGNON}
            to ++(0:12 * \widthletter) node[letter, label=below:{\id{p}{A0}{0}}] (S1B-left) {O}
            to ++(0:\widthletter) node[letter, fill=mydarkorange, label=above:{\id{p}{A0}{0}\id{m}{A1}{0}}] {I}
            to ++(0:\widthletter) node[block, label=below:{\id{p}{A0}{1..4}}] (S1B-right) {GNON}
            to ++(0:5 * \widthletter) node[letter, label=below:{\id{p}{A0}{0}}] (S2B-left) {O}
            to ++(0:\widthletter) node[block, label=below:{\id{p}{A0}{1..4}}] (S2B-right) {GNON}
            to ++(0:8 * \widthletter) node[letter, label=below:{\id{p}{A0}{0}}] (S3B-left) {O}
            to ++(0:\widthletter) node[letter, fill=mydarkorange, label=above:{\id{p}{A0}{0}\id{m}{A1}{0}}] {I}
            to ++(0:\widthletter) node[block, label=below:{\id{p}{A0}{1..4}}] {GNON};

        \draw[->, thick]
          (S0A) edge node[above, align=center]{\emph{insert "i"}\\\emph{between}\\\emph{"o" and "g"}} (S1A-left)
          (S1B-right) edge node[above, align=center]{\emph{remove "i"}} (S2B-left);

        \draw[dotted]
          (S1A-right) -- (S2A-left)
          (S0B) -- (S1B-left)
          (S2B-right) -- (S3B-left);

        \draw[dashed, ->, thick, shorten >= 3]
          (S1A-right.east) edge node[right, align=center]{\emph{insert "i" at} {\color{mydarkorange}\id{p}{A0}{0}\id{m}{A1}{0}}}  (S1B-left.west)
          (S1A-right.east) edge node[right, align=center]{\emph{insert "i" at} {\color{mydarkorange}\id{p}{A0}{0}\id{m}{A1}{0}}}  (S3B-left.west)
          (S2B-right.east) edge node[below right, align=center]{\emph{remove} {\color{mydarkorange}\id{i}{B0}{1..1}}} (S2A-left.west);
    \end{tikzpicture}
  }
  \caption{TODO}
  \label{fig:why-exactly-once-delivery}
\end{figure}

Dans cet exemple, deux noeuds A et B répliquent et éditent collaborativement une séquence.
La séquence répliquée contient initialement les éléments "ognon", qui sont associés à l'interval d'identifiants \id{p}{A0}{0..4}.

Le noeud A commence par insérer un nouvel élément, "i", dans la séquence entre les éléments "o" et "g".
L'opération \emph{insert} résultante, insérant l'élément "i" à la position \id{p}{A0}{0}\id{m}{A1}{0}, est diffusée au noeud B.

À la réception de l'opération \emph{insert}, le noeud B l'intègre à son état.
Puis il supprime dans la foulée ce nouvel élément.
L'opération \emph{remove} générée est envoyée au noeud A.

Le noeud A intègre l'opération \emph{remove}, ce qui a pour effet de supprimer l'élément "i" associé à l'identifiant \id{p}{A0}{0}\id{m}{A1}{0}.
Il obtient alors un état équivalent à celui du noeud B.

Cependant, l'opération \emph{insert} insérant l'élément "i" à la position \id{p}{A0}{0}\id{m}{A1}{0} est de nouveau délivrée au noeud B.
De multiples raisons peuvent être à l'origine de cette nouvelle livraison : perte du message d'\emph{acknowledgment}, utilisation d'un protocole de diffusion épidémique des messages, déclenchement du mécanisme d'anti-entropie en concurrence...
Le noeud B ré-intègre alors l'opération \emph{insert}, ce qui fait revenir l'élément "i" et l'identifiant associé.
L'état du noeud B diverge désormais de celui-ci du noeud A.

Pour se prémunir de ce type de scénarios, LogootSplit requiert que la couche de livraison des messages assure une livraison en exactement un exemplaire des opérations.
Cette contrainte permet d'éviter que d'anciens éléments et identifiants ressurgissent après leur suppression chez certains noeuds uniquement à cause d'une livraison multiple de l'opération \emph{insert} correspondante.

\mnnote{QUESTION: Ajouter quelques lignes ici sur comment faire ça en pratique (Ajout d'un dot aux opérations, maintien d'un dot store au niveau de la couche livraison, vérification que dot pas encore présent dans dot store avant de passer opération à la structure de données) ? Ou je garde ça pour le chapitre sur MUTE ?}

\subsubsection{Livraison de l'opération \emph{remove} après l'opération \emph{insert}}

Une autre propriété que doit assurer la couche de livraison de messages est que les opérations \emph{remove} doivent être livrées au \ac{CRDT} après les opérations \emph{insert} correspondantes.
La \autoref{fig:why-causal-remove} présente un exemple justifiant cette contrainte.

\begin{figure}[!ht]
  \centering
  \resizebox{\columnwidth}{!}{
    \begin{tikzpicture}
        \path
            node {\textbf{A}}
            to ++(0:\widthletter) node[block, label=below:{\id{p}{A0}{0..4}}] (S0A) {OGNON}
            to ++(0:5 * \widthletter) node[letter, label=below:{\id{p}{A0}{0}}] (S1A-left) {O}
            to ++(0:\widthletter) node[letter, fill=mydarkorange, label=above:{\id{p}{A0}{0}\id{m}{A1}{0}}] {I}
            to ++(0:\widthletter) node[block, label=below:{\id{p}{A0}{1..4}}] (S1A-right) {GNON}
            to ++(0:25 * \widthletter) node[letter, label=below:{\id{p}{A0}{0}}] (S2A-left) {O}
            to ++(0:\widthletter) node[block, label=below:{\id{p}{A0}{1..4}}] {GNON};


        \path
            to ++(270:4) node {\textbf{B}}
            to ++(0:\widthletter) node[block, label=below:{\id{p}{A0}{0..4}}] (S0B) {OGNON}
            to ++(0:12 * \widthletter) node[letter, label=below:{\id{p}{A0}{0}}] (S1B-left) {O}
            to ++(0:\widthletter) node[letter, fill=mydarkorange, label=above:{\id{p}{A0}{0}\id{m}{A1}{0}}] {I}
            to ++(0:\widthletter) node[block, label=below:{\id{p}{A0}{1..4}}] (S1B-right) {GNON}
            to ++(0:5 * \widthletter) node[letter, label=below:{\id{p}{A0}{0}}] (S2B-left) {O}
            to ++(0:\widthletter) node[block, label=below:{\id{p}{A0}{1..4}}] (S2B-right) {GNON};

        \path
            to ++(270:8) node {\textbf{C}}
            to ++(0:\widthletter) node[block, label=below:{\id{p}{A0}{0..4}}] (S0C) {OGNON}
            to ++(0:30 * \widthletter) node[letter, label=below:{\id{p}{A0}{0}}] (S1C-left) {O}
            to ++(0:\widthletter) node[block, label=below:{\id{p}{A0}{1..4}}] (S1C-right) {GNON}
            to ++(0:8 * \widthletter) node[letter, label=below:{\id{p}{A0}{0}}] (S2C-left) {O}
            to ++(0:\widthletter) node[letter, fill=mydarkorange, label=above:{\id{p}{A0}{0}\id{m}{A1}{0}}] {I}
            to ++(0:\widthletter) node[block, label=below:{\id{p}{A0}{1..4}}] {GNON};

        \draw[->, thick]
          (S0A) edge node[above, align=center]{\emph{insert "i"}\\\emph{between}\\\emph{"o" and "g"}} (S1A-left)
          (S1B-right) edge node[above, align=center]{\emph{remove "i"}} (S2B-left);

        \draw[dotted]
          (S1A-right) -- (S2A-left)
          (S0B) -- (S1B-left)
          (S0C) -- (S1C-left)
          (S1C-right) -- (S2C-left);

        \draw[dashed, ->, thick, shorten >= 3]
          (S1A-right.east) edge node[right, align=center]{\emph{insert "i" at} {\color{mydarkorange}\id{p}{A0}{0}\id{m}{A1}{0}}}  (S1B-left.west)
          (S1A-right.east) edge node[pos=0.85, right, align=center]{\emph{insert "i" at} {\color{mydarkorange}\id{p}{A0}{0}\id{m}{A1}{0}}}  (S2C-left.west)
          (S2B-right.east) edge node[below right, align=center]{\emph{remove} {\color{mydarkorange}\id{i}{B0}{1..1}}} (S2A-left.west)
          (S2B-right.east) edge node[pos=0.80, right, align=center]{\emph{remove} {\color{mydarkorange}\id{i}{B0}{1..1}}} (S1C-left.west);
    \end{tikzpicture}
  }
  \caption{TODO}
  \label{fig:why-causal-remove}
\end{figure}

Dans cet exemple, trois noeuds A, B et C répliquent et éditent collaborativement une séquence.
Le noeud A commence par insérer un nouvel élément, "i", dans la séquence entre les éléments "o" et "g".
L'opération \emph{insert} résultante, insérant l'élément "i" à la position \id{p}{A0}{0}\id{m}{A1}{0}, est diffusée aux autres noeuds.

À la réception de l'opération \emph{insert}, le noeud B l'intègre à son état.
Cependant, le noeud B supprime dans la foulée l'élément nouvellement ajouté.
Il diffuse ensuite l'opération \emph{remove} générée.

Toutefois, suite à un aléa du réseau, l'opération \emph{remove} supprimant l'élément "i" est livrée au noeud C avant l'opération \emph{insert} l'ajoutant à son état.
Lorsque le noeud C reçoit l'opération \emph{remove}, il parcourt son état à la recherche de l'élément "i" pour le supprimer.
Cependant, celui-ci n'est pas présent dans son état courant.
L'intégration de l'opération s'achève donc sans effectuer de modification.

Le noeud C reçoit ensuite l'opération \emph{insert}.
Le noeud C intègre ce nouvel élément dans la séquence en utilisant son identifiant (\id{p}{A0}{0} \lid \id{p}{A0}{0}\id{m}{A1}{0} \lid \id{p}{A0}{1}).

Ainsi, l'état du noeud C diverge de celui-ci des autres noeuds à terme, et cela malgré que les noeuds A, B et C aient intégré le même ensemble d'opérations.
Ce résultat transgresse la propriété de \ac{SEC} que doivent assurer les \acp{CRDT}.
Afin d'empêcher ce scénario de se produire, LogootSplit impose donc la livraison causale des opérations \emph{remove} par rapport aux opérations \emph{insert} correspondantes.

\mnnote{QUESTION: Même que pour la exactly-once delivery, est-ce que j'explique ici comment assurer cette contrainte plus en détails (Ajout des dots des opérations \emph{insert} en dépendances de l'opération \emph{remove}, vérification que dots présents dans dot store avant de passer l'opération \emph{remove} à la structure de données) ou je garde ça pour le chapitre sur MUTE ?}

\subsubsection{Définition du modèle de livraison}

Pour résumer, la couche de livraison des opérations associée à LogootSplit doit respecter le modèle de livraison suivant :

\begin{definition}[Exactly-once + Causal remove]
  Le modèle de livraison \emph{Exactly-once + Causal remove} définit les 3 règles suivantes sur la livraison des opérations :
  \begin{enumerate}
    \item Une opération doit être délivrée à l'ensemble des noeuds à terme,
    \item Une opération doit être délivrée qu'une seule et unique fois aux noeuds,
    \item Une opération \emph{remove} doit être délivrée à un noeud une fois que les opérations \emph{insert} des éléments concernés par la suppression ont été délivrées à ce dernier.
  \end{enumerate}
\end{definition}

Il est à noter que \textcite{2021-these-vic} a récemment proposé dans ses travaux de thèse Dotted LogootSplit, un nouveau Sequence \ac{CRDT} basée sur les différences.
Inspiré de Logoot et LogootSplit, ce nouveau \ac{CRDT} associe une séquence à identifiants densément ordonnés à un contexte causal.
Le contexte causal est une structure de données permettant à Dotted LogootSplit de représenter et de maintenir efficacement les informations des modifications déjà intégrées à l'état courant.
Cette association permet à Dotted LogootSplit de fonctionner de manière autonome, sans imposer de contraintes particulières à la couche livraison autres que la livraison à terme.

\subsection{Limites}

Comme indiqué précédemment, la taille des identifiants provenant d'un ordre total dense est variable.
Quand les noeuds insèrent de nouveaux éléments entre deux autres ayant la même valeur de \emph{position}, LogootSplit n'a pas d'autre choix que d'augmenter la taille de l'identifiant résultant.
La \autoref{fig:example-split} illustre de tels cas.
Dans cet exemple, puisque le noeud A insère un nouvel élément entre deux identifiants contigus \id{i}{B0}{0} et \id{i}{B0}{1}, LogootSplit ne peut pas générer un identifiant adapté de la même taille.
Pour respecter l'ordre souhaité, LogootSplit génère un identifiant en ajoutant un nouveau tuple à l'identifiant du prédecesseur : \id{i}{B0}{0}\id{f}{A0}{0}.

\begin{figure}[!ht]
  \centering
  \begin{tikzpicture}
      \path
          node {\textbf{A}}
          to ++(0:\widthletter) node[block, label=below:{\id{i}{B0}{0..2}}] (HLO) {HLO}
          to ++(0:5 * \widthletter) node[letter, label=below:{\id{i}{B0}{0}}] (H) {H}
          to ++(0:\widthletter) node[letter, fill=mydarkorange, label=above:{\color{mydarkorange}\id{i}{B0}{0}\id{f}{A0}{0}}] {E}
          to ++(0:\widthletter) node[block, label=below:{\id{i}{B0}{1..2}}] {LO};

      \draw[->, thick] (HLO) -- node[below, align=center]{\emph{insert "e"}\\\emph{between}\\\emph{"h" and "l"}} (H);
  \end{tikzpicture}
  \caption{Insertion leading to longer identifiers}
  \label{fig:example-split}
\end{figure}

Par conséquent, la taille des identifiants a tendance à croître alors que le système progresse.
Cette croissance impacte négativement les performances de la structure de données sur plusieurs aspects.
Puisque les identifiants attachés aux éléments deviennent plus long, le surcoût en métadonnées augmente.
Ceci augmente aussi la consommation en bande-passante puisque les noeuds doivent diffuser les identifiants aux autres.

\mnnote{TODO: Ajouter une phrase pour expliquer que la croissance des identifiants impacte aussi le temps d'intégration des modifications}

De plus, le nombre de blocs composant la séquence répliquée augmente au fil du temps.
En effet, plusieurs contraintes sur la génération d'identifiants empêchent les noeuds d'ajouter des nouveaux éléments aux blocs existants.
Par exemple, seul le noeud qui a généré un bloc peut ajouter un élément à ce dernier.
Ces limitations provoquent la génération de nouveau blocs.
La séquence se retrouve finalement fragmentée en de nombreux blocs de seulement quelques caractères chacun.
Cependant, aucun mécanisme pour fusionner les blocs à posteriori n'est fourni.
L'efficacité de la structure décroît donc puisque chaque bloc entraîne un surcoût.

Comme illustré plus loin, nous avons mesuré au cours de nos évaluations que le contenu représente à terme moins de 1\% de taille de la structure de données.
Les 99\% restants correspondent aux métadonnées utilisées par la séquence répliquée.
Il est donc nécessaire de proposer des mécanismes et techniques afin de mitiger les problèmes soulignés précédemments.

\section{Mitigation du surcoût des séquences répliquées sans conflits}

\begin{itemize}
  \item Plusieurs approches ont été proposées pour réduire croissance des métadonnées dans Sequence \acp{CRDT}
  \item RGA (et RGASplit) propose un mécanisme de GC des pierres tombales.
    Nécessite cependant stabilité causale des opérations de suppression.
    S'agit d'une contrainte forte, peu adaptée aux systèmes dynamiques à large échelle.
    \mnnote{TODO: Trouver référence sur la stabilité causale dans systèmes dynamiques}
  \item Core \& Nebula propose un mécanisme de ré-équilibrage de l'arbre pour Treedoc.
    Le ré-équilibrage a pour effet de supprimer des potentielles pierres tombales et de réduire la taille des identifiants.
    Repose sur un algorithme de consensus.
    S'agit de nouveau d'une contrainte forte pour systèmes dynamique à large échelle.
    Pour y pallier, propose de séparer les pairs entre deux ensembles : Core et Nebula.
    Permet de limiter le nombre participant au consensus.
    Un protocole de rattrapage permet aux noeuds de la Nebula de mettre à jour leurs modifications concurrentes à un ré-équilibrage.
  \item LSEQ adopte une autre approche.
    Part du constat que les identifiants dans Logoot croissent de manière linéaire.
    Vise une croissance logarithmique des identifiants.
    Pour cela, propose de nouvelles fonctions d'allocation des identifiants visant à maximiser le nombre d'identifiants insérés avant de devoir augmenter la taille de l'identifiant.
    Propose aussi d'utiliser une base exponentielle pour la valeur \emph{position} des identifiants.
    Atteint ainsi la croissance polylogarithmique des identifiants, sans coordination requise entre les noeuds et mécanisme supplémentaire.
    Solution adaptée aux systèmes distribués à large échelle.
    Conjecture cependant que cette approche se marie mal avec les Sequence \acp{CRDT} utilisant des blocs.
    En effet, ajoute une raison supplémentaire à la croissance des identifiants : l'insertion entre identifiants contigus.
    Force alors la croissance des identifiants.
\end{itemize}

\section{Synthèse}

% \include{assets/chapter_soa_ergm}

\NumberThisInToc
\chapter{Renommage dans une séquence répliquée}
\minitoc

\section{Présentation de l'approche}

Nous proposons un nouveau \ac{CRDT} pour la \emph{Sequence} appartenant à l'approche des identifiants densément ordonnées : RenamableLogootSplit \cite{nicolas:hal-01932552,nicolas:hal-02526724}.
Cette structure de données permet aux pairs d'insérer et de supprimer des éléments au sein d'une séquence répliquée.
Nous introduisons une opération \emph{rename} qui permet de
\begin{enumerate*}[label=(\roman*)]
  \item réassigner des identifiants plus courts aux différents éléments de la séquence
  \item fusionner les blocs composant la séquence.
\end{enumerate*}
Ces deux actions permettent à l'opération \emph{rename} de produire un nouvel état minimisant son surcoût en métadonnées.

\subsection{Modèle du système}

Le système est composé d'un ensemble dynamique de noeuds, les noeuds pouvant rejoindre puis quitter la collaboration tout au long de sa durée.
Les noeuds collaborent afin de construire et maintenir une séquence à l'aide de RenamableLogootSplit.
Chaque noeud possède une copie de la séquence et peut l'éditer sans se coordonner avec les autres.
Les modifications des noeuds prennent la forme d'opérations qui sont appliquées immédiatement à leur copie locale.
Les opérations sont ensuite transmises de manière asynchrone aux autres noeuds pour qu'ils puissent à leur tour appliquer les modifications à leur copie.

Les noeuds communiquement par l'intermédiaire d'un réseau \ac{P2P}.
Ce réseau est non-fiable : les messages peuvent être perdus, ré-ordonnés ou même livrés à plusieurs reprises.
Le réseau peut aussi être sujet à des partitions, qui séparent alors les noeuds en des sous-groupes disjoints.
Afin de compenser les limitations du réseau, les noeuds reposent sur une couche de livraison de messages.

Puisque RenamableLogootSplit est une extension de LogootSplit, il partage les mêmes contraintes sur la livraison de messages.
La couche de livraison de messages sert donc à livrer les messages à l'application exactement une fois.
La couche de livraison de messages a aussi pour tâche de garantir la livraison des opérations de suppression après les opérations d'insertion correspondantes.
Aucune autre contrainte n'existe sur l'ordre de livraison des opérations.
Finalement, la couche de livraison intègre aussi un mécanisme d'anti-entropie \cite{1983-anti-entropy-vv}.
Ce mécanisme permet aux noeuds de se synchroniser par paires, en détectant et ré-échangeant les messages perdus.

\subsection{Définition de l'opération de renommage}

L'objectif de l'opération \emph{rename} est de réassigner de nouveaux identifiants aux éléments de la séquence répliquée sans modifier son contenu.
Puisque les identifiants sont des métadonnées utilisées par la structure de données uniquement afin de résoudre les conflits, les utilisateurs ignorent leur existence.
Les opérations \emph{rename} sont donc des opérations systèmes : elles sont émises et appliquées par les noeuds en coulisses, sans aucune intervention des utilisateurs.

Afin de garantir le respect du modèle de cohérence \ac{SEC}, nous définissons plusieurs propriétés de sécurité que l'opération \emph{rename} doit respecter.
Ces propriétés sont inspirées principalement par celles proposées dans \cite{zawirski:hal-01248197}.

\begin{property}(Déterminisme)
  Les opérations \emph{rename} sont intégrées par les noeuds sans aucune coordination.
  Pour assurer que l'ensemble des noeuds atteigne un état équivalent à terme, une opération \emph{rename} donnée doit toujours générer le même nouvel identifiant à partir de l'identifiant courant.
\end{property}

\begin{property}(Préservation de l'intention de l'utilisateur)
  Bien que l'opération \emph{rename} n'est pas elle-même n'incarne pas une intention de l'utilisateur, elle ne doit pas entrer en conflit avec les actions des utilisateurs.
  Notamment, les opérations \emph{rename} ne doivent pas annuler ou altérer le résultat d'opérations \emph{insert} et \emph{remove} du point de vue des utilisateurs.
\end{property}

\begin{property}(Séquence bien formée)
  La séquence répliquée doit être bien formée.
  Appliquée une opération \emph{rename} sur une séquence bien formée doit produire une nouvelle séquence bien formée.
  Une séquence bien formée doit respecter les propriétés suivantes :
  \begin{itemize}[noitemsep]
    \item[~]
    \begin{subproperty}(Préservation de l'unicité)
      Chaque identifiant doit être unique.
      Donc, pour une opération \emph{rename} donnée, chaque identifiant doit être associé à un nouvel identifiant distinct.
    \end{subproperty}
    \item[~]
    \begin{subproperty}(Préservation de l'ordre)
      Les éléments de la séquence doivent être triés en fonction de leur identifiants.
      L'ordre existant entre les identifiants initiaux doit donc être préservé par l'opération \emph{rename}.
    \end{subproperty}
  \end{itemize}
\end{property}

\begin{property}(Commutativité avec les opérations concurrentes)
  \label{prop:commutativity}
  Les opérations concurrentes peuvent être délivrées dans des ordres différents à chaque noeud.
  Afin de garantir la convergence des réplicas, l'ordre d'application d'un ensemble d'opérations concurrentes ne doit pas avoir d'impact sur l'état obtenu.
  L'opération \emph{rename} doit donc être commutative avec n'importe quelle opération concurrente.
\end{property}

La \autoref{prop:commutativity} est particulièrement difficile à assurer.
Cette difficulté est dûe au fait que les opérations \emph{rename} modifient les identifiants assignés aux éléments.
Cependant, les autres opérations telles que les opérations \emph{insert} et \emph{remove} reposent sur ces identifiants pour spécifier où insérer les éléments ou lesquels supprimer.
Les opérations \emph{rename} sont donc intrinséquement incompatibles avec les opérations \emph{insert} et \emph{remove} concurrentes.
De la même manière, des opérations \emph{rename} concurrentes peuvent réassigner des identifiants différents à des mêmes éléments.
Les opérations \emph{rename} concurrentes ne sont donc pas commutatives.
Par conséquent, il est nécessaire de concevoir et d'utiliser des méthodes de résolution de conflits pour assurer la \autoref{prop:commutativity}.

Dans un souci de simplicité, la présentation de l'opération \emph{rename} est divisée en deux parties.
Dans le \autoref{sec:centralised-rls}, nous présentons l'opération \emph{rename} proposée avec l'hypothèse qu'aucune opération \emph{rename} concurrente ne peut être générée.
Cette hypothèse nous permet de nous concentrer sur le fonctionnement de l'opération \emph{rename} elle-même ainsi que sur comment gérer les opérations \emph{insert} et \emph{remove} concurrentes.
Ensuite, dans le \autoref{sec:distributed-rls}, nous supprimons cette hypothèse.
Nous présentons alors notre approche pour gérer les scénarios avec des opérations \emph{rename} concurrentes.

\section{RenamableLogootSplit}
\label{sec:centralised-rls}
\subsection{Opération de renommage proposée}

Notre opération de renommage permet à RenamableLogootSplit de réduire le surcoût en métadonnées des séquences répliquées.
Pour ce faire, elle réassigne des identifiants arbitraires aux éléments de la séquence.

\begin{figure}[t!]
  \centering
  \subfloat[Selecting the new identifier of the first element]{
      \begin{minipage}{\linewidth}
          \centering
          \begin{tikzpicture}
              \path
                  node {\textbf{A}}
                  to ++(0:\widthletter) node[letter, label=below:{\id{i}{B0}{0}}] {H}
                  to ++(0:\widthletter) node[letter, fill=mydarkorange, label=above:{\color{mydarkorange}\id{i}{B0}{0}\id{f}{A0}{0}}] {E}
                  to ++(0:\widthletter) node[block, label=below:{\id{i}{B0}{1..2}}] (LO) {LO}
                  to ++(0:4 * \widthletter) node[letter, fill=mydarkblue, label=below:{\color{mydarkblueid}\id{i}{A1}{0}}] (H) {H};

              \draw[->, thick] (LO) -- node[below, align=center]{\emph{rename}} (H);
          \end{tikzpicture}
          \label{fig:renaming-first-id}
      \end{minipage}}
  \hfil
  \subfloat[Selecting the new identifiers of the remaining ones]{
      \begin{minipage}{\linewidth}
          \centering
          \begin{tikzpicture}
              \path
                  node {\textbf{A}}
                  to ++(0:\widthletter) node[letter, label=below:{\id{i}{B0}{0}}] {H}
                  to ++(0:\widthletter) node[letter, fill=mydarkorange, label=above:{\color{mydarkorange}\id{i}{B0}{0}\id{f}{A0}{0}}] {E}
                  to ++(0:\widthletter) node[block, label=below:{\id{i}{B0}{1..2}}] (LO) {LO}
                  to ++(0:4 * \widthletter) node[letter, fill=mydarkblue, label=below:{\color{mydarkblueid}\id{i}{A1}{0}}] (H) {H}
                  to ++(0:\widthletter) node[letter, fill=mydarkblue, label=below:{\color{mydarkblueid}\id{i}{A1}{1}}] {E}
                  to ++(0:\widthletter) node[letter, fill=mydarkblue, label=below:{\color{mydarkblueid}\id{i}{A1}{2}}] {L}
                  to ++(0:\widthletter) node[letter, fill=mydarkblue, label=below:{\color{mydarkblueid}\id{i}{A1}{3}}] {O};

                  \draw[->, thick] (LO) -- node[below, align=center]{\emph{rename}} (H);
              \end{tikzpicture}
          \label{fig:renaming-second-id}
      \end{minipage}}
  \hfil
  \subfloat[Final state obtained]{
      \begin{minipage}{\linewidth}
          \centering
          \begin{tikzpicture}
              \path
                  node {\textbf{A}}
                  to ++(0:\widthletter) node[letter, label=below:{\id{i}{B0}{0}}] {H}
                  to ++(0:\widthletter) node[letter, fill=mydarkorange, label=above:{\color{mydarkorange}\id{i}{B0}{0}\id{f}{A0}{0}}] {E}
                  to ++(0:\widthletter) node[block, label=below:{\id{i}{B0}{1..2}}] (LO) {LO}
                  to ++(0:4 * \widthletter) node[block, fill=mydarkblue, label=below:{\color{mydarkblueid}\id{i}{A1}{0..3}}] (HELO) {HELO};

              \draw[->, thick] (LO) -- node[below, align=center]{\emph{rename}} (HELO);
          \end{tikzpicture}
          \label{fig:renaming-final-state}
      \end{minipage}}
  \caption{Renaming the sequence on node \emph{A}}
  \label{fig:renaming}
\end{figure}

Son comportement est illustré dans la \autoref{fig:renaming}.
Dans cet exemple, le noeud A initie une opération \emph{rename} sur son état local.
Tout d'abord, le noeud A génère un nouvel identifiant à partir du premier tuple de l'identifiant du premier élément de la séquence (\id{i}{B0}{0}).
Pour générer ce nouvel identifiant, le noeud A reprend la position de ce tuple (\emph{i}) mais utilise son propre identifiant de noeud (\textbf{A}) et numéro de séquence actuel (\emph{1}).
De plus, son offset est mis à 0.
Le noeud A réassigne l'identifiant résultant (\id{i}{A1}{0}) au premier élément de la séquence, comme décrit dans \autoref{fig:renaming-first-id}.
Ensuite, le noeud A dérive des identifiants contigus pour tous les éléments restants en incrémentant de manière successive l'offset (\id{i}{A1}{1}, \id{i}{A1}{2}, \id{i}{A1}{3}), comme présenté dans \autoref{fig:renaming-second-id}.
Comme nous assignons des identifiants consécutifs à tous les éléments de la séquence, nous pouvons au final aggréger ces éléments en un seul bloc, comme illustré en \autoref{fig:renaming-final-state}.
Ceci permet aux noeuds de bénéficier au mieux de la fonctionnalité des blocs et de minimiser le surcoût en métadonnés de l'état résultat.

Pour converger, les autres noeuds doivent renommer leur état de manière identique.
Cependant, ils ne peuvent pas simplement remplacer leur état courant par l'état généré par le renommage.
En effet, ils peuvent avoir modifié en concurrence leur état.
Afin de ne pas perdre ces modifications, les noeuds doivent traiter l'opération \emph{rename} eux-mêmes.
Pour ce faire, le noeud qui a généré l'opération \emph{rename} diffuse son \emph{ancien état} aux autres.

\begin{definition}[Ancien état]
  Un \emph{ancien état} est la liste des idInterval qui composent l'état courant de la séquence répliquée au moment du renommage.
\end{definition}

De ce fait, nous définissons l'opération \emph{rename} de la manière suivante :

\begin{definition}[rename]
  Une opération \emph{rename} est un triplet $\langle$nodeId, nodeSeq, formerState$\rangle$ où
  \begin{itemize}
    \item nodeId est l'identifiant du noeud qui a générée l'opération \emph{rename}.
    \item nodeSeq est le numéro de séquence du noeud au moment de la génération de l'opération \emph{rename}.
    \item formerState est l'ancien état du noeud au moment du renommage.
  \end{itemize}
\end{definition}

En utilisant ces données, les autres noeuds calculent le nouvel identifiant de chaque identifiant renommé.
Concernant les identifiants insérés de manière concurrente au renommage, nous expliquons dans \autoref{sec:ops-concurrent-to-rename} comment les noeuds peuvent les renommer de manière déterministe.

\subsection{Gestion des opérations concurrentes au renommage}

\label{sec:ops-concurrent-to-rename}

Après avoir appliqué des opérations \emph{rename} sur leur état local, les noeuds peuvent recevoir des opérations concurrentes.
La \autoref{fig:concurrent-insert-rename-inconsistent} illustre de tels cas.

\begin{figure}[!ht]
  \centering
  \resizebox{\columnwidth}{!}{
    \begin{tikzpicture}
        \path
            node {\textbf{A}}
            to ++(0:\widthletter) node[letter, label=below:{\id{i}{B0}{0}}] {H}
            to ++(0:\widthletter) node[letter, fill=mydarkorange, label=above:{\color{mydarkorange}\id{i}{B0}{0}\id{f}{\,A0}{0}}] {E}
            to ++(0:\widthletter) node[block, label=below:{\id{i}{B0}{1..2}}] (S0A-right) {LO}
            to ++(0:5 * \widthletter) node[block, fill=mydarkblue,
                    label={below:{\color{mydarkblueid}\id{i}{A1}{0..3}} }
                        ] (S1A) {HELO}
            to ++(0:8 * \widthletter) node[block, fill=mydarkblue,
                    label={below:{\color{mydarkblueid}\id{i}{A1}{0..3}} }
                        ] (S2A-left) {HELO}
            to ++(0:1.18 * \widthblock) node[letter, fill=mylightorange, cross,
                    label={above:{\color{mylightorange}\id{i}{B0}{0}\id{m}{B1}{0}} }
                        ] {L};

        \path
            to ++(270:2) node {\textbf{B}}
            to ++(0:\widthletter) node[letter, label=below:{\id{i}{B0}{0}}] {H}
            to ++(0:\widthletter) node[letter, fill=mydarkorange, label=above:{\color{mydarkorange}\id{i}{B0}{0}\id{f}{\,A0}{0}}] {E}
            to ++(0:\widthletter) node[block, label=below:{\id{i}{B0}{1..2}}] (S0B-right) {LO}
            to ++(0:5 * \widthletter) node[letter, label=below:{\id{i}{B0}{0}}] (S1B-left) {H}
            to ++(0:\widthletter) node[letter, fill=mydarkorange, label=above:{\color{mydarkorange}\id{i}{B0}{0}\id{f}{\,A0}{0}}] {E}
            to ++(0:\widthletter) node[letter, fill=mylightorange, label=below:{\color{mylightorange}\id{i}{B0}{0}\id{m}{B1}{0}}] {L}
            to ++(0:\widthletter) node[block, label=above:{\id{i}{B0}{1..2}}] (S1B-right) {LO};


        \draw[->, thick] (S0A-right) -- node[above, align=center]{\emph{rename}} (S1A);
        \draw[dotted] (S1A) -- (S2A-left);
        \draw[->, thick] (S0B-right) -- node[below, align=center]{\emph{insert "l"}\\\emph{between}\\\emph{"e" and "l"}} (S1B-left);
        \draw[dashed, ->, thick, shorten >= 3] (S1B-right.east) -- node[below right, align=center]{\emph{insert "l" at} {\color{mylightorange}\id{i}{B0}{0}\id{m}{B1}{0}}} (S2A-left.west);

    \end{tikzpicture}
  }
  \caption{Concurrent update leading to inconsistency}
  \label{fig:concurrent-insert-rename-inconsistent}
\end{figure}

Dans cet exemple, le noeud B insère un nouvel élément "L", lui assigne l'identifiant \id{i}{B0}{0}\id{m}{B1}{0} et diffuse cette modification, de manière concurrente à l'opération \emph{rename} décrite dans la \autoref{fig:concurrent-insert-rename-inconsistent}.
À la réception de l'opération \emph{insert}, le noeud A ajoute l'élément inséré au sein de sa séquence, en utilisant l'identifiant de l'élément pour déterminer sa position.
Cependant, puisque les identifiants ont été modifiés par l'opération \emph{rename} concurrente, le noeud A insère le nouvel élément à la fin de sa séquence (puisque \id{i}{A1}{3} \lid \id{i}{B0}{0}\id{m}{B1}{0}) au lieu d'à sa position prévue.
Comme décrit par cet exemple, appliquer naivement les modifications concurrentes provoquerait des anomalies.
Il est donc nécessaire de traiter les opérations concurrentes aux opérations \emph{rename} de manière particulière.

Tout d'abord, les noeuds doivent détecter les opérations concurrentes aux opérations \emph{rename}.
Pour cela, nous utilisons un système basé sur des \emph{époques}.
Initialement, la séquence répliquée débute à l'époque \emph{origine} notée \epoch{0}.
Chaque opération \emph{rename} introduit une nouvelle époque et permet aux noeuds d'y avancer depuis l'époque précédente.
Par exemple, l'opération \emph{rename} décrite dans \autoref{fig:concurrent-insert-rename-inconsistent} permet aux noeuds de faire progresser leur état de \epoch{0} à \epoch{A1}.
Nous définissons les époques de la manière suivante :

\begin{definition}[Époque]
  Une époque est un triplet $\langle$nodeId, nodeSeq, formerState$\rangle$ où
  \begin{itemize}
    \item nodeId est l'identifiant du noeud qui a générée cette époque.
    \item nodeSeq est le numéro de séquence du noeud au moment de la génération de cette époque.
    \item formerState est l'ancien état du noeud de la génération de cette époque.
  \end{itemize}
\end{definition}

Notons que l'époque générée est caractérisée et identifiée de manière unique par son couple $\langle$nodeId, nodeSeq$\rangle$.

Au fur et à mesure que les noeuds reçoivent des opérations \emph{rename}, ils construisent et maintiennent localement la \emph{chaîne des époques}, une structure de données ordonnant les époques en fonction de leur relation \emph{parent-enfant}.
De plus, les noeuds marquent chaque opération avec l'identifiant de leur époque courante au moment de génération de l'opération.
À la réception d'une opération, les noeuds comparent l'époque de l'opération à l'époque courante de leur séquence.

Si les époques diffèrent, les noeuds doivent transformer l'opération avant de pouvoir l'intégrer.
Les noeuds déterminent par rapport à quelles opérations \emph{rename} doit être transformée l'opération reçue en calculant le chemin entre l'époque de l'opération et leur époque courante en utilisant la \emph{chaîne des époques}.

Les noeuds utilisent la fonction \textsc{renameId}, décrite dans \autoref{alg:renameId}, pour transformer les opérations \emph{insert} et \emph{remove} par rapport aux opérations \emph{rename}.
Cet algorithme associe les identifiants d'une époque \emph{parente} aux identifiants correspondant dans l'époque \emph{enfant}.
L'idée principale de cet algorithme est de renommer les identifiants inconnus au moment de la génération de l'opération \emph{rename} en utilisant leur prédecesseur.
Un exemple est présenté dans la \autoref{fig:concurrent-insert-rename-fixed}.
Cette figure décrit le même scénario que la \autoref{fig:concurrent-insert-rename-inconsistent}, à l'exception que le noeud A utilise \textsc{renameId} pour renommer les identifiants générés de façon concurrente avant de les insérer dans son état.

\begin{figure}[!ht]
  \footnotesize
  \begin{algorithmic}
      \Function {renameId}{id, renamedIds, nId, nSeq}
          % \State \Comment{$id$ is the id to rename}
          % \State \Comment{$renamedIds$ is the list of ids of the \emph{former state}}
          % \State \Comment{$nId$ is $node~id$ of the node that issued the \emph{rename} op}
          % \State \Comment{$nSeq$ is $node~seq$ of the node that issued the \emph{rename} op}
          \State length $\gets$ renamedIds.length
          \State firstId $\gets$ renamedIds[0]
          \State lastId $\gets$ renamedIds[length - 1]
          \State pos $\gets$ position(firstId)
          \\
          \If{id < firstId}
              \State newFirstId $\gets$ \new~Id(pos, nId, nSeq, 0)
              \State \Return renIdLessThanFirstId(id, newFirstId)
          \ElsIf{id $\in$ renameIds}
              \State index $\gets$ findIndex(id, renamedIds)
              \State \Return \new~Id(pos, nId, nSeq, index)
          \ElsIf{lastId < id}
              \State newLastId $\gets$ \new~Id(pos, nId, nSeq, length - 1)
              \State \Return renIdGreaterThanLastId(id, newLastId)
          \Else
              \State \Return renIdFromPredId(id, renamedIds, pos, nId, nSeq)
          \EndIf
      \EndFunction
      \\
      \Function {renIdFromPredId}{id, renamedIds, pos, nId, nSeq}
          \State index $\gets$ findIndexOfPred(id, renamedIds)
          \State newPredId $\gets$ \new~Id(pos, nId, nSeq, index)
          \\
          \State \Return concat(newPredId, id)
      \EndFunction
  \end{algorithmic}
  \caption{Main functions to rename an identifier}
  \label{alg:renameId}
\end{figure}

\begin{figure}[!ht]
  \centering
  \resizebox{\columnwidth}{!}{
    \begin{tikzpicture}
        \path
            node {\textbf{A}}
            to ++(0:0.5 * \widthletter) node[epoch] {\epoch{0}}
            to ++(0:1.05 * \widthoriginepoch) node[letter, label=below:{\id{i}{B0}{0}}] {H}
            to ++(0:\widthletter) node[letter, fill=mydarkorange, label=above:{\color{mydarkorange}\id{i}{B0}{0}\id{f}{\,A0}{0}}] {E}
            to ++(0:\widthletter) node[block, label=below:{\id{i}{B0}{1..2}}] (S0A-right) {LO}
            to ++(0:5 * \widthletter) node[epoch] (S1A-left) {\epoch{A1}}
            to ++(0:1.3 * \widthepoch) node[block, fill=mydarkblue,
                    label={below:{\color{mydarkblueid}\id{i}{A1}{0..3}} }
                        ] (S1A-right) {HELO}
            to ++(0:8 * \widthletter) node[epoch] (S2A-left) {\epoch{A1}}
            to ++(0:1.3 * \widthepoch) node[block, fill=mydarkblue,
                    label={below:{\color{mydarkblueid}\id{i}{A1}{0..1}} }
                        ] {HE}
            to ++(0:\widthblock) node[letter, fill=mylightblue,
                    label={above:{\color{mylightblue!20!mydarkblueid}\id{i}{A1}{1}\id{i}{B0}{0}\id{m}{B1}{0}} }
                        ] {L}
            to ++(0:\widthletter) node[block, fill=mydarkblue,
                    label={below:{\color{mydarkblueid}\id{i}{A1}{2..3}} }
                        ] {LO};

        \path
            to ++(270:2) node {\textbf{B}}
            to ++(0:0.5 * \widthletter) node[epoch] {\epoch{0}}
            to ++(0:1.05 * \widthoriginepoch) node[letter, label=below:{\id{i}{B0}{0}}] {H}
            to ++(0:\widthletter) node[letter, fill=mydarkorange, label=above:{\color{mydarkorange}\id{i}{B0}{0}\id{f}{\,A0}{0}}] {E}
            to ++(0:\widthletter) node[block, label=below:{\id{i}{B0}{1..2}}] (S0B-right) {LO}
            to ++(0:5 * \widthletter) node[epoch] (S1B-left) {\epoch{0}}
            to ++(0:1.05 * \widthoriginepoch) node[letter, label=below:{\id{i}{B0}{0}}] {H}
            to ++(0:\widthletter) node[letter, fill=mydarkorange, label=above:{\color{mydarkorange}\id{i}{B0}{0}\id{f}{\,A0}{0}}] {E}
            to ++(0:\widthletter) node[letter, fill=mylightorange, label=below:{\color{mylightorange}\id{i}{B0}{0}\id{m}{B1}{0}}] {L}
            to ++(0:\widthletter) node[block, label=above:{\id{i}{B0}{1..2}}] (S1B-right) {LO};


        \draw[->, thick] (S0A-right) -- node[above, align=center]{\emph{rename to \epoch{A1}}} (S1A-left);
        \draw[dotted] (S1A-right) -- (S2A-left);
        \draw[->, thick] (S0B-right) -- node[below, align=center]{\emph{insert "l"}\\\emph{between}\\\emph{"e" and "l"}} (S1B-left);
        \draw[dashed, ->, thick, shorten >= 3] (S1B-right.east) -- node[below right, align=center]{\emph{insert "l" at} {\color{mylightorange}\id{i}{B0}{0}\id{m}{B1}{0}}} (S2A-left.west);

    \end{tikzpicture}
  }
  \caption{Renaming concurrent update using \textsc{renameId} before applying it to maintain intended order}
  \label{fig:concurrent-insert-rename-fixed}
\end{figure}

L'algorithme procède de la manière suivante.
Tout d'abord, le noeud récupère le prédecesseur de l'identifiant donné \id{i}{B0}{0}\id{m}{B1}{0} dans l'ancien état : \id{i}{B0}{0}\id{f}{A0}{0}.
Ensuite, il calcule l'équivalent de \id{i}{B0}{0}\id{f}{A0}{0} dans l'état renommé : \id{i}{A1}{1}.
Finalement, le noeud A concatène cet identifiant et l'identifiant donné pour générer l'identifiant correspondant l'époque \emph{enfant} : \id{i}{A1}{1}\id{i}{B0}{0}\id{m}{B1}{0}.
En réassignant cet identifiant à l'élément inséré de manière concurrente, le noeud A peut l'insérer à son état tout en préservant l'ordre souhaité.

\textsc{renameId} permet aussi aux noeuds de gérer le cas contraire : intégrer des opérations \emph{rename} distantes sur leur copie locale alors qu'ils ont précédemment intégré des modifications concurrentes.
Ce cas correspond à celui du noeud B dans la \autoref{fig:concurrent-insert-rename-fixed}.
À la réception de l'opération \emph{rename} du noeud A, le noeud B utilise \textsc{renameId} sur chacun des identifiants de son état pour le renommer et atteindre un état équivalent à celui du noeud A.

\autoref{alg:renameId} présente seulement le cas principal de \textsc{renameId}, \ie le cas où l'identifiant à renommer appartient à l'interval des identifiants formant l'ancien état ($\trm{firstId} \leq_{id} \trm{id} \leq_{id} \trm{lastId}$).
Les fonctions pour gérer les autres cas, \ie les cas où l'identifiant à renommer n'appartient pas à cet interval ($\trm{id} <_{id} \trm{firstId}$ ou $\trm{lastId} <_{id} \trm{id}$), sont présentées dans \autoref{app:rename-id}.

L'algorithme que nous présentons ici permet aux noeuds de renommer leur état identifiant par identifiant.
Une extension possible est de concevoir \textsc{renameBlock}, une version améliorée qui renomme l'état bloc par bloc.
\textsc{renameBlock} réduirait le temps d'intégration des opérations \emph{rename}, puisque sa complexité temporelle ne dépendrait plus du nombre d'identifiants (\ie du nombre d'éléments) mais du nombre de blocs.
De plus, son exécution réduirait le temps d'intégration des prochaines opérations \emph{rename} puisque le mécanisme de renommage regroupe les éléments en moins de blocs.

\subsection{Évolution du modèle de livraison des opérations}

\label{sec:renamablelogootsplit-delivery-model}

Finalement, considérons le cas où un noeud reçoit une opération marquée d'une époque encore inconnue.
Il serait incohérent que le noeud l'intègre à leur état.
En effet, comparer des identifiants d'une époque aux identifiants d'une autre époque n'a pas de sens et provoquerait des anomalies (cf. \autoref{fig:concurrent-insert-rename-inconsistent}).
\mnnote{TODO: Ajouter nouvel exemple où noeud reçoit déjà des opérations dépendant d'un \emph{rename} avant de recevoir le \emph{rename}}
Les noeuds devraient plutôt attendre l'opération \emph{rename} permettant de progresser jusqu'à cette époque pour intégrer cette opération.
Il est donc nécessaire de faire évoluer le modèle de livraison des opérations pour ajouter la contrainte suivante : toute opération doit désormais être livrée après l'opération \emph{rename} qui introduit son époque de génération.
\mnnote{QUESTION: Récapituler modèle de livraison ? (Epoch-based + causal remove)}
\mnnote{QUESTION: Indiquer que l'opération \emph{rename} ne nécessite pas de livraison particulière autre que epoch-based?}

\section{RenamableLogootSplit v2}
\label{sec:distributed-rls}
\subsection{Conflits en cas de renommages concurrents}

Nous considérons à présent les scénarios avec des opérations \emph{rename} concurrentes.
\autoref{fig:conflicting-rename-operations} développe le scénario décrit précédemment dans \autoref{fig:concurrent-insert-rename-fixed}.

\begin{figure}[!ht]
  \centering
  \resizebox{\columnwidth}{!}{
    \begin{tikzpicture}
      \path
          node {\textbf{A}}
          to ++(0:0.5 * \widthletter) node[epoch] {\epoch{0}}
          to ++(0:1.05 * \widthoriginepoch) node[letter, label=below:{\id{i}{B0}{0}}] {H}
          to ++(0:\widthletter) node[letter, fill=mydarkorange, label=above:{\color{mydarkorange}\id{i}{B0}{0}\id{f}{\,A0}{0}}] {E}
          to ++(0:\widthletter) node[block, label=below:{\id{i}{B0}{1..2}}] (S0A-right) {LO}
          to ++(0:5 * \widthletter) node[epoch] (S1A-left) {\epoch{A1}}
          to ++(0:1.3 * \widthepoch) node[block, fill=mydarkblue,
                  label={below:{\color{mydarkblueid}\id{i}{A1}{0..3}} }
                      ] (S1A-right) {HELO}
          to ++(0:8 * \widthletter) node[epoch] (S2A-left) {\epoch{A1}}
          to ++(0:1.3 * \widthepoch) node[block, fill=mydarkblue,
                  label={below:{\color{mydarkblueid}\id{i}{A1}{0..1}} }
                      ] {HE}
          to ++(0:\widthblock) node[letter, fill=mylightblue,
                  label={above:{\color{mylightblue!20!mydarkblueid}\id{i}{A1}{1}\id{i}{B0}{0}\id{m}{B1}{0}} }
                      ] {L}
          to ++(0:\widthletter) node[block, fill=mydarkblue,
                  label={below:{\color{mydarkblueid}\id{i}{A1}{2..3}} }
                      ] (S2A-right) {LO};

      \path
          to ++(270:3) node {\textbf{B}}
          to ++(0:0.5 * \widthletter) node[epoch] {\epoch{0}}
          to ++(0:1.05 * \widthoriginepoch) node[letter, label=below:{\id{i}{B0}{0}}] {H}
          to ++(0:\widthletter) node[letter, fill=mydarkorange, label=above:{\color{mydarkorange}\id{i}{B0}{0}\id{f}{\,A0}{0}}] {E}
          to ++(0:\widthletter) node[block, label=below:{\id{i}{B0}{1..2}}] (S0B-right) {LO}
          to ++(0:5 * \widthletter) node[epoch] (S1B-left) {\epoch{0}}
          to ++(0:1.05 * \widthoriginepoch) node[letter, label=below:{\id{i}{B0}{0}}] {H}
          to ++(0:\widthletter) node[letter, fill=mydarkorange, label=above:{\color{mydarkorange}\id{i}{B0}{0}\id{f}{\,A0}{0}}] {E}
          to ++(0:\widthletter) node[letter, fill=mylightorange, label=below:{\color{mylightorange}\id{i}{B0}{0}\id{m}{B1}{0}}] {L}
          to ++(0:\widthletter) node[block, label=above:{\id{i}{B0}{1..2}}] (S1B-right) {LO}
          to ++(0:5 * \widthletter) node[epoch] (S2B-left) {\epoch{B2}}
          to ++(0:1.3 * \widthepoch) node[block, fill=mydarkpurple,
                  label={ [] below:{\color{mydarkpurpleid}\id{i}{B2}{0..4}} }
              ] (S2B-right) {HELLO};

      \draw[->, thick] (S0A-right) -- node[above, align=center]{\emph{rename to \epoch{A1}}} (S1A-left);
      \draw[dotted] (S1A-right) -- (S2A-left); % (S2A-right) -- (A-sync) (S2B-right) -- (B-sync);
      \draw[->, thick] (S0B-right) -- node[below, align=center]{\emph{insert "l"}\\\emph{between}\\\emph{"e" and "l"}} (S1B-left);
      \draw[dashed, ->, thick, shorten >= 3] (S1B-right.east) -- node[right, xshift=5pt, align=center]{\emph{insert "l" at} {\color{mylightorange}\id{i}{B0}{0}\id{m}{B1}{0}}} (S2A-left.west);
      \draw[->, thick] (S1B-right) -- node[below, align=center]{\emph{rename to \epoch{B2}}} (S2B-left);

    \end{tikzpicture}
  }
  \caption{Concurrent \emph{rename} operations leading to divergent states}
  \label{fig:conflicting-rename-operations}
\end{figure}

Après avoir diffusé son opération \emph{insert}, le noeud B effectue une opération \emph{rename} sur son état.
Cette opération réassigne à chaque élément un nouvel identifiant à partir de l'identifiant du premier élément de la séquence (\id{i}{B0}{0}), de l'identifiant du noeud (\textbf{B}) et de son numéro de séquence courant (2).
Cette opération introduit aussi une nouvelle époque : \epoch{B2}.
Puisque l'opération \emph{rename} de A n'a pas encore été délivrée au noeud B à ce moment, les deux opérations \emph{rename} sont concurrentes.

Puisque des époques concurrentes sont générées, les époques forment désormais l'\emph{arbre des époques}.
Nous représentons dans la \autoref{fig:epoch-tree} l'\emph{arbre des époques} que les noeuds obtiennent une fois qu'ils se sont synchronisés à terme.
Les époques sont representées sous la forme de noeuds de l'arbre et la relation \emph{parent-enfant} entre elles est illustrée sous la forme de flèches noires.

\begin{figure}[!ht]
  \centering
  \begin{tikzpicture}[scale=0.8,every node/.style={scale=0.8}]
      \path
          node[op] (e0) {\epoch{0}}
          to ++(270:1.5)
          to ++(180:0.95) node[op] (eA1) {\epoch{A1}};
      \path
          to ++(270:1.5)
          to ++(0:0.95) node[op] (eB2) {\epoch{B2}};

      \draw[thick, ->] (e0) -- (eA1);
      \draw[thick, ->] (e0) -- (eB2);
  \end{tikzpicture}
  \caption{The \emph{epoch tree} corresponding to the scenario of \autoref{fig:conflicting-rename-operations}}
  \label{fig:epoch-tree}
\end{figure}

À l'issue du scénario décrit dans la \autoref{fig:conflicting-rename-operations}, les noeuds A et B sont respectivement aux époques \epoch{A1} et \epoch{B2}.
Pour converger, tous les noeuds devraient atteindre la même époque à terme.
Cependant, la fonction \textsc{renameId} décrite dans \autoref{alg:renameId} permet seulement au noeuds de progresser d'une époque \emph{parente} à une de ses époques \emph{enfants}.
Le noeud A (resp. B) est donc dans l'incapacité de progresser vers l'époque du noeud B (resp. A).
Il est donc nécessaire de faire évoluer notre mécanisme de renommage pour sortir de cette impasse.

Tout d'abord, les noeuds doivent se mettre d'accord sur une époque commune de l'\emph{arbre des époques} comme époque cible.
Afin d'éviter des problèmes de performances dûs à une coordination synchrone, les noeuds doivent sélectionner cette époque de manière non-coordonnée, \ie en utilisant seulement les données présentes dans l'\emph{arbre des époques}.
Nous présentons un tel mécanisme dans \autoref{sec:priority}.

Ensuite, les noeuds doivent se déplacer à travers l'\emph{arbre des époques} afin d'atteindre l'époque cible.
La fonction \textsc{renameId} permet déjà aux noeuds de descendre dans l'arbre.
Les cas restants à gérer sont ceux où les noeuds se trouvent actuellement à une époque \emph{soeur} ou \emph{cousine} de l'époque cible.
Dans ces cas, les noeuds doivent être capable de remonter dans l'\emph{arbre des époques} pour retourner au \ac{LCA} de l'époque courante et l'époque cible.
Ce déplacement est en fait similaire à annuler l'effet des opérations \emph{rename} précédemment appliquées.
Nous proposons un algorithme qui remplit cet objectif dans la \autoref{sec:reverting-rename-ops}.

\subsection{Relation de priorité entre renommages}

\label{sec:priority}

Pour que chaque noeud sélectionne la même époque cible de manière non-coordonnée, nous définissons la relation \emph{priority}.

\begin{definition}[Relation \emph{priority} \lepoch]
  La relation \emph{priority} \lepoch est un ordre total strict sur l'ensemble des époques.
  Elle permet aux noeuds de comparer n'importe quelle paire d'époques.
\end{definition}

En utilisant la relation \emph{priority}, nous définissons l'époque cible de la manière suivante :

\begin{definition}[Époque cible]
  L'époque de l'ensemble des époques vers laquelle les noeuds doivent progresser.
  Les noeuds sélectionnent comme époque cible l'époque maximale d'après l'ordre établit par \emph{priority}.
\end{definition}

Pour définir la relation \emph{priority}, nous pouvons choisir entre plusieurs stratégies.
Dans le cadre de ce travail, nous utilisons l'ordre lexicographique sur le chemin des époques dans l'\emph{arbre des époques}.
La \autoref{fig:priority-example} fournit un exemple.

\begin{figure}[!ht]
  \subfloat[Example of execution with concurrent \emph{rename} operations]{
      \begin{minipage}{\linewidth}
          \centering
          \begin{tikzpicture}[scale=0.8,every node/.style={scale=0.8}]
              \path
                  node {\textbf{A}}
                  to ++(0:1) node (a0) {}
                  to ++(0:1) node[point, label=above:{rename to \epoch{A1}}] (a1) {}
                  to ++(0:5) node (a-end) {};

              \draw[->, thick] (a0) --  (a1) -- (a-end);

              \path
                  to ++(270:1.5) node {\textbf{B}}
                  to ++(0:1) node (b0) {}
                  to ++(0:1) node[point, label=below:{rename to \epoch{B2}}] (b2) {}
                  to ++(0:4) node[point, label=above:{rename to \epoch{B7}}] (b7) {}
                  to ++(0:1) node (b-end) {};

              \draw[->, thick] (b0) -- (b2) -- (b7) -- (b-end);

              \path
                  to ++(270:3) node {\textbf{C}}
                  to ++(0:1) node (c0) {}
                  to ++(0:3) node (c-receives-a1) {}
                  to ++(0:1) node[point, label=below:{rename to \epoch{C6}}] (c6) {}
                  to ++(0:2) node (c-end) {};

              \draw[->, thick] (c0) -- (c6) -- (c-end);

              \draw[->, dashed, shorten >= 1] (a1) -- (c-receives-a1);
          \end{tikzpicture}
          \label{fig:priority-execution}
      \end{minipage}}
  \hfil
  \subfloat[Corresponding epoch tree with \emph{priority} relation shown]{
      \begin{minipage}{\linewidth}
          \centering
          \begin{tikzpicture}[scale=0.8,every node/.style={scale=0.8}]
              \path
                  node[op] (e0) {\epoch{0}}
                  to ++(225:1.5) node[op] (eA1) {\epoch{A1}}
                  to ++(270:1.5) node[op] (eC6) {\epoch{C6}};
              \path
                  to ++(315:1.5) node[op] (eB2) {\epoch{B2}}
                  to ++(270:1.5) node[op, red] (eB7) {\epoch{B7}};

              \draw[->, thick] (e0) edge (eA1) (eA1) edge (eC6) (e0) edge (eB2) (eB2) edge (eB7);
              \draw[->, dashed, thick, red] (eB7.135) -- (eB2.225) (eB2.180) -- (eC6.45) -- (eA1.315) (eA1.0) -- (e0.270);
          \end{tikzpicture}
          \label{fig:priority-epoch-tree}
      \end{minipage}}
  \caption{Selecting target epoch from execution with concurrent \emph{rename} operations}
  \label{fig:priority-example}
\end{figure}

La \autoref{fig:priority-execution} décrit une exécution dans laquelle trois noeuds A, B et C générent plusieurs opérations avant de se synchroniser à terme.
Comme seules les opérations \emph{rename} sont pertinentes pour le problème qui nous occupe, seules ces opérations sont représentées dans cette figure.
Initialement, le noeud A génère une opération \emph{rename} qui introduit l'époque \epoch{A1}.
Cette opération est délivrée au noeud C, qui génère ensuite sa propre opération \emph{rename} qui introduit l'époque \epoch{C6}.
De manière concurrente à ces opérations, le noeud B génère deux opérations \emph{rename}, introduisant \epoch{B2} et \epoch{B7}.

Une fois que les noeuds se sont synchronisés, ils obtiennent l'\emph{arbre des époques} représenté dans la \autoref{fig:priority-epoch-tree}.
Dans cette figure, la flèche pointillé rouge représente l'ordre entre les époques d'après la relation \emph{priority} tandis que l'époque cible choisie est représentée sous la forme d'un noeud rouge.

Pour déterminer l'époque cible, les noeuds reposent sur la relation \emph{priority}.
D'après l'ordre lexicographique sur le chemin des époques dans l'\emph{arbre des époques}, nous avons \epoch{0} < \epoch{0}\epoch{A1} < \epoch{0}\epoch{A1}\epoch{C6} < \epoch{0}\epoch{B2} < \epoch{0}\epoch{B2}\epoch{B7}.
Chaque noeud sélectionne donc \epoch{B7} comme époque cible de manière non-coordonnée.

D'autres stratégies pourraient être proposées pour définir la relation \emph{priority}.
Par exemple, \emph{priority} pourrait reposer sur des métriques intégrées au sein des opérations \emph{rename} pour représenter le travail accumulé sur le document.
Cela permettrait de favoriser la branche de l'\emph{arbre des époques} avec le plus de collaborateurs actifs pour minimiser la quantité globale de calculs effectués par les noeuds du système.

\subsection{Algorithme d'annulation de l'opération de renommage}

\label{sec:reverting-rename-ops}

Nous présentons maintenant la fonction \textsc{revertRenameId}.
Décrite dans \autoref{alg:revertRenameId}, cette fonction permet aux noeuds d'annuler une opération \emph{rename} appliquée précédemment.
Pour ce faire, \textsc{revertRenameId} associe les identifiants de l'époque \emph{enfant} aux identifiants correspondant dans l'époque \emph{parente}.

\begin{figure}[!ht]
  \footnotesize
  \begin{algorithmic}
      \Function{revertRenameId}{id, renamedIds, nId, nSeq}
          % \Statex \LeftComment{$id}$ is the identifier to reverse rename}
          % \Statex \LeftComment{$renamedIds}$ is the former state shared by the \emph{rename} op}
          % \Statex \LeftComment{$nId}$ is $node~id}$ of the node which issued the \emph{rename} op}
          % \Statex \LeftComment{$nSeq}$ is $node~seq}$ of the node which issued the \emph{rename} op}
          % \\
          \State length $\gets$ renamedIds.length
          \State firstId $\gets$ renamedIds[0]
          \State lastId $\gets$ renamedIds[length - 1]
          \State pos $\gets$ position(firstId)
          \\
          \State newFirstId $\gets$ \new~Id(pos, nId, nSeq, 0)
          \State newLastId $\gets$ \new~Id(pos, nId, nSeq, length - 1)
          \\
          \If{id < newFirstId}
              \State \Return revRenIdLessThanNewFirstId(id, firstId, newFirstId)
          \ElsIf{isRenamedId(id, pos, nId, nSeq, length)}
              \State index $\gets$ getFirstOffset(id)
              \State \Return renamedIds[index]
          \ElsIf{newLastId < id}
              \State \Return revRenIdGreaterThanNewLastId(id, lastId)
          \Else
              % Ajouter commentaire sur cas
              \State index $\gets$ getFirstOffset(id)
              \State \Return revRenIdfromPredId(id, renamedIds, index)
          \EndIf
      \EndFunction
      \\
      \Function{revRenIdfromPredId}{id, renamedIds, index}
          \State predId $\gets$ renamedIds[index]
          \State succId $\gets$ renamedIds[index + 1]
          \State tail $\gets$ getTail(id, 1)
          \\
          \If{tail < predId}
              \State \Comment{$id$ has been inserted causally after the \emph{rename} op}
              \State \Return concat(predId, MIN\_TUPLE, tail)
          \ElsIf{succId < tail}
              \State \Comment{$id$ has been inserted causally after the \emph{rename} op}
              \State offset $\gets$ getLastOffset(succId) - 1
              \State predOfSuccId $\gets$ createIdFromBase(succId, offset)
              \State \Return concat(predOfSuccId, MAX\_TUPLE, tail)
          \Else
              \State \Return tail
          \EndIf
      \EndFunction
  \end{algorithmic}
  \caption{Main functions to revert an identifier renaming}
  \label{alg:revertRenameId}
\end{figure}

Les objectifs de \textsc{revertRenameId} sont les suivants :
\begin{enumerate*}[label=(\roman*)]
  \item Restaurer à leur ancienne valeur les identifiants générés causalement avant ou de manière concurrente à l'opération \emph{rename} annulée
  \item Assigner de nouveaux identifiants respectant l'ordre souhaité aux éléments qui ont été insérés causalement après l'opération \emph{rename} annulée.
\end{enumerate*}
Nous illustrons son comportement à l'aide de la \autoref{fig:revertRenameId}.

\begin{figure}[!ht]
  \centering
  \resizebox{\columnwidth}{!}{
    \begin{tikzpicture}
        \path
            node {\textbf{A}}
            to ++(0:0.5 * \widthletter) node[epoch] (S1A-left) {\epoch{A1}}
            to ++(0:1.3 * \widthepoch) node[block, fill=mydarkblue,
                    label={below:{\color{mydarkblueid}\id{i}{A1}{0..3}} }
                        ] (S1A-right) {HELO}
            to ++(0:8 * \widthletter) node[epoch] (S2A-left) {\epoch{A1}}
            to ++(0:1.3 * \widthepoch) node[block, fill=mydarkblue,
                    label={below:{\color{mydarkblueid}\id{i}{A1}{0..1}} }
                        ] {HE}
            to ++(0:\widthblock) node[letter, fill=mylightblue,
                    label={above:{\color{mylightblue!20!mydarkblueid}\id{i}{A1}{1}\id{i}{B0}{0}\id{m}{B1}{0}} }
                        ] {L}
            to ++(0:\widthletter) node[block, fill=mydarkblue,
                    label={below:{\color{mydarkblueid}\id{i}{A1}{2..3}} }
                        ] (S2A-right) {LO}
            to ++(0:2 * \widthblock) node[point] (between-S2A-S3A) {}
            to ++(0:3 * \widthletter) node[epoch] (S3A-left) {\epoch{0}}
            to ++(0:1.05 * \widthoriginepoch) node[letter, label=below:{\id{i}{B0}{0}}] {H}
            to ++(0:\widthletter) node[letter, fill=mydarkorange, label=above:{\color{mydarkorange}\id{i}{B0}{0}\id{f}{\,A0}{0}}] {E}
            to ++(0:\widthletter) node[letter, fill=mylightorange, label=below:{\color{mylightorange}\id{i}{B0}{0}\id{m}{B1}{0}}] {L}
            to ++(0:\widthletter) node[block, label=above:{\id{i}{B0}{1..2}}] (S3A-right) {LO}
            to ++ (0:3 * \widthblock) node[epoch] (S4A-left) {\epoch{B2}}
            to ++(0:1.3 * \widthepoch) node[block, fill=mydarkpurple,
                    label={ [] below:{\color{mydarkpurpleid}\id{i}{B2}{0..4}} }
                ] {HELLO};

        \path
            to ++(270:3) node {\textbf{B}}
            to ++(0:0.5 * \widthletter) node[epoch] (S1B-left) {\epoch{0}}
            to ++(0:1.05 * \widthoriginepoch) node[letter, label=below:{\id{i}{B0}{0}}] {H}
            to ++(0:\widthletter) node[letter, fill=mydarkorange, label=above:{\color{mydarkorange}\id{i}{B0}{0}\id{f}{\,A0}{0}}] {E}
            to ++(0:\widthletter) node[letter, fill=mylightorange, label=below:{\color{mylightorange}\id{i}{B0}{0}\id{m}{B1}{0}}] {L}
            to ++(0:\widthletter) node[block, label=above:{\id{i}{B0}{1..2}}] (S1B-right) {LO}
            to ++(0:5 * \widthletter) node[epoch] (S2B-left) {\epoch{B2}}
            to ++(0:1.3 * \widthepoch) node[block, fill=mydarkpurple,
                    label={ [] below:{\color{mydarkpurpleid}\id{i}{B2}{0..4}} }
                ] (S2B-right) {HELLO};

        \draw[dotted] (S1A-right) -- (S2A-left);
        \draw[dashed, ->, thick, shorten >= 3] (S1B-right.east) -- node[right, xshift=5pt, align=center]{\emph{insert "l" at} {\color{mylightorange}\id{i}{B0}{0}\id{m}{B1}{0}}} (S2A-left.west);
        \draw[->, thick] (S1B-right) -- node[below, align=center]{\emph{rename to \epoch{B2}}} (S2B-left);
        \draw[dotted] (S2A-right) -- (between-S2A-S3A) (S3A-right) -- (S4A-left);
        \draw[->, dashed, thick] (S2B-right.east) -- node[right, xshift=5pt, align=center]{\emph{rename to \epoch{B2}}} (between-S2A-S3A);
        \draw[->, loosely dash dot, thick, shorten >= 1] (between-S2A-S3A) -- node[above, align=center]{\emph{revert to \epoch{0}}} (S3A-left);
        \draw[->, dashed, thick, shorten >= 3] (between-S2A-S3A) edge[bend right] (S4A-left.west);
    \end{tikzpicture}
  }
  \caption{Reverting a previously applied \emph{rename} operation}
  \label{fig:revertRenameId}
\end{figure}

Cette figure reprend le scénario de la \autoref{fig:concurrent-insert-rename-inconsistent}.
Le noeud A reçoit l'opération \emph{rename} du noeud B, qui est concurrente à l'opération \emph{rename} que le noeud A a appliqué précédemment.
Selon la relation \emph{priority} proposée, le noeud A sélectionne l'époque introduite \epoch{B2} comme l'époque cible (\epoch{A1} \lepoch \epoch{B2}).
Il procède donc à ramener son état à un état équivalent à l'époque \epoch{0}, le \ac{LCA} de son époque courante \epoch{A1} et de l'époque cible \epoch{B2}.
Pour ce faire, il applique \textsc{revertRenameId} à chaque identifiant de son état courant.

\textsc{revertRenameId} détermine quelle stratégie appliquer pour restaurer un identifiant donné en utilisant des motifs.
Par exemple, les identifiants de la forme \id{pos}{nId~nSeq}{offset} (\id{i}{A1}{offset} dans l'exemple courant) correspondent aux nouvelles valeurs des identifiants qui composent l'\emph{ancien état}.
Pour retrouver les identifiants d'origine, \textsc{revertRenameId} utilise simplement leur offset puisqu'il correspond à leur index dans l'\emph{ancien état}.
Par exemple, l'identifiant ayant pour offset 0 correspond au premier identifiant dans l'\emph{ancien état} (\id{i}{B0}{0}), celui ayant pour offset 1 au second identifiant (\id{i}{B0}{0}\id{f}{A0}{0}), et ainsi de suite.

Les identifiants de la forme \id{pos}{nId~nSeq}{offset}$\trm{tail}$ (\eg \id{i}{A1}{1}\id{i}{B0}{0}\id{m}{B1}{0}) correspondent à des identifiants qui ont été soit insérés de façon concurrente à l'opération \emph{rename}, soit causalement après.
Pour traiter ces identifiant, \textsc{revertRenameId} retire tout d'abord le premier tuple (\id{i}{A1}{1}) pour isoler la queue de l'identifiant (\id{i}{B0}{0}\id{m}{B1}{0}).
En faisant cela, \textsc{revertRenameId} annule la transformation appliquée à l'identifiant par \textsc{renIdFromPredId} si l'identifiant a été inséré de manière concurrente.
L'algorithme compare ensuite la queue de l'identifiant aux identifiants de l'élément précédant et de l'élément suivant dans l'\emph{ancien état}.
Dans cet exemple, nous avons \id{i}{B0}{0}\id{f}{A0}{0} \lid \id{i}{B0}{0}\id{m}{B1}{0} \lid \id{i}{B0}{1}.
L'algorithme peut alors retourner la queue comme identifiant résultant tout en préservant l'ordre souhaité, puisque sa valeur est comprise entre celles des identifiants du prédécesseur et du successeur.

Sinon, cela signifie que l'identifiant donné a été inséré de manière causale après l'opération \emph{rename}.
Puisqu'aucun identifiant correspondant n'existe encore à l'époque \emph{parente}, \textsc{revertRenameId} peut retourner n'importe quel identifiant tant qu'il préserve l'ordre souhaité.
Pour ce faire, \textsc{revertRenameId} génère l'identifiant à partir de l'identifiant du prédecesseur ou du successeur, et en utilisant des tuples exclusifs au mécanisme de renommage : $\trm{MIN\_TUPLE}$ et $\trm{MAX\_TUPLE}$.

\mnnote{TODO: Modifier exemple pour illustrer le cas de figure où on a besoin de MIN/MAX\_TUPLE}

Une fois que le noeud A a converti son état à un état équivalent à l'époque \epoch{0} en utilisant \textsc{revertRenameId}, il peut appliquer \textsc{renameId} pour calculer l'état correspondant à \epoch{B2}.

Comme pour \autoref{alg:renameId}, \autoref{alg:revertRenameId} ne présente seulement que le cas principal de \textsc{revertRenameId}.
Il s'agit du cas où l'identifiant à restaurer appartient à l'interval des identifiants renommés $\trm{newFirstId}$ \leqid $\trm{id}$ \leqid $\trm{newLastId}$).
Les fonctions pour gérer les cas restants sont présentées dans \autoref{app:revert-rename-id}.

Notons que \textsc{renameId} et \textsc{revertRenameId} ne sont pas des fonctions réciproques.
\textsc{revertRenameId} restaure à leur valeur initiale les identifiants insérés causalement avant ou de manière concurrente à l'opération \emph{rename}.
Par contre, \textsc{renameId} ne fait pas de même pour les identifiants insérés causalement après l'opération \emph{rename}.
Rejouer une opération \emph{rename} précédemment annulée altère donc ces identifiants.
Cette modification peut entraîner une divergence entre les noeuds, puis qu'un même élément sera désigné par des identifiants différents.

Ce problème est toutefois évité dans notre système grâce à la relation \emph{priority} utilisée.
Puisque la relation \emph{priority} est définie en utilisant l'ordre lexicographique sur le chemin des époques dans l'\emph{arbre des époques}, les noeuds se déplacent seulement vers l'époque la plus à droite de l'\emph{arbre des époques} lorsqu'ils changent d'époque.
Les noeuds évitent donc d'aller et revenir entre deux mêmes époques, et donc d'annuler et rejouer les opérations \emph{rename} correspondantes.

\subsection{Processus d'intégration d'une opération}

\begin{itemize}
  \item Distingue plusieurs cas.
  \item 1er cas de figure est si l'opération que l'on reçoit est une opération \emph{insert} ou \emph{remove}.
  \item Intégration de ces opérations se décompose en plusieurs étapes.
  \item Tout d'abord, compare l'époque de l'opération avec l'époque courante de la séquence.
  \item Si les époques correspondent, intègre l'opération de manière standard.
  \item Sinon, besoin de transformer l'opération avant de pouvoir l'intégrer.
  \item Besoin d'identifier les transformations à appliquer à l'opération.
  \item Pour cela, calcule le chemin entre l'époque de l'opération et l'époque courante.
    Utilise pour cela l'algorithme suivant :
    Calcule le chemin entre l'époque de l'opération et la racine.
    Calcule le chemin entre l'époque courante et la racine.
    Calcule la première intersection entre ces deux chemins.
    S'agit du \ac{LCA} entre ces deux époques.
  \item Le chemin entre l'époque de l'opération et l'époque \ac{LCA} correspond aux renommages dont on doit annuler les effets sur l'opération.
    Applique successivement \textsc{revertRenameId} sur chaque identifiant de l'opération.
  \item Le chemin entre l'époque \ac{LCA} et l'époque courante correspond aux renommages dont on doit appliquer les effets sur l'opération.
    Applique successivement \textsc{renameId} sur chaque identifiant de l'opération.
  \item Obtient alors la liste des identifiants concernés par l'opération
    Doit alors intégrer une opération pour chacun d'entre eux.
  \item Pour minimiser le nombre de parcours de la séquence, aggrège les identifiants en intervals d'identifiants au préalable.
    Pour cela, parcourt la liste des identifiants et vérifient s'ils sont contigus.
  \item Recréé à partir des intervals d'identifiants obtenus une liste d'opérations.
    Intègre successivement chacune d'entre elles à la séquence.
  \item 2nd cas de figure est si l'opération à intégrer est une opération \emph{rename}.
  \item Commence par ajouter l'époque introduite par l'opération \emph{rename} à l'\emph{arbre des époques}.
  \item Compare ensuite l'époque introduite avec l'époque courante de la séquence.
  \item Ici, de nouveaux 2 cas de figure différents.
  \item 1er cas de figure : si époque de l'opération plus grande que l'époque courante d'après \lepoch, doit renommer son état en accordance.
  \item Pour cela, calcule le chemin entre l'époque de l'opération et l'époque courante, comme indiqué précédemment.
  \item Procéde ensuite au renommage de chaque identifiant de la séquence.
  \item Obtient alors la liste des identifiants correspondant à ceux de l'état courant, mais à la nouvelle époque courante.
  \item Doit alors construire une nouvelle séquence avec ces nouveaux identifiants.
  \item Pour limiter le nombre de traitements, commence déjà par aggréger les identifiants en intervals d'identifiants.
  \item Génère ensuite les blocs de la future séquence à partir de la liste d'intervals d'identifiants obtenus et du contenu de la séquence courante.
  \item Puis génère une nouvelle séquence à partir de cette liste de blocs.
    Remplace la séquence actuelle par cette dernière.
    Met à jour l'époque courante de la séquence.
  \item 2nd cas de figure : si époque introduite par l'opération \emph{rename} plus petite que l'époque d'après \lepoch.
  \item Dans ce cas, aucun traitement supplémentaire n'est nécessaire.
    Intégration de cette opération \emph{rename} s'arrête là.
\end{itemize}

\subsection{Règles de récupération de la mémoire des états précédents}

Les noeuds stockent les époques et les \emph{anciens états} correspondant pour transformer les identifiants d'une époque à l'autre.
Au fur et à mesure que le système progresse, certaines époques et métadonnées associées deviennent obsolètes puisque plus aucune opération ne peut être émise depuis ces époques.
Les noeuds peuvent alors supprimer ces époques.
Dans cette section, nous présentons un mécanisme permettant aux noeuds de déterminer les époques obsolètes.

Pour proposer un tel mécanisme, nous nous reposons sur la notion de \emph{stabilité causale des opérations} \cite{10.1007/978-3-662-43352-2_11}.
Une opération est causalement stable une fois qu'elle a été délivrée à tous les noeuds.
Dans le contexte de l'opération \emph{rename}, cela implique que tous les noeuds ont progressé à l'époque introduite par cette opération ou à une époque plus grande d'après la relation \emph{priority}.
À partir de ce constat, nous définissons les \emph{potentielles époques courantes} :

\begin{definition}[Potentielles époques courantes]
  L'ensemble des époques auxquelles les noeuds peuvent se trouver actuellement et à partir desquelles ils peuvent émettre des opérations, du point de vue du noeud courant.
  Il s'agit d'un sous-ensemble de l'ensemble des époques, composé de l'époque maximale introduite par une opération \emph{rename} causalement stable et de toutes les époques plus grande que cette dernière d'après la relation \emph{priority}.
\end{definition}

Pour traiter les prochaines opérations, les noeuds doivent maintenir les chemins entre toutes les époques de l'ensemble des \emph{potentielles époques courantes}.
Nous appelons \emph{époques requises} l'ensemble des époques correspondant.

\begin{definition}[Époques requises]
  L'ensemble des époques qu'un noeud doit conserver pour traiter les potentielles prochaines opérations.
  Il s'agit de l'ensemble des époques qui forment les chemins entre chaque époque appartenant à l'ensemble des \emph{potentielles époques courantes} et leur \ac{LCA}.
\end{definition}

Il s'ensuit que toute époque qui n'appartient pas à l'ensemble des \emph{époques requises} peut être retirée par les noeuds.
La \autoref{fig:GC-epochs} illustre un cas d'utilisation du mécanisme de récupération de mémoire proposé.

\begin{figure}[!ht]
  \subfloat[Execution]{
      \begin{minipage}{\linewidth}
          \centering
          \begin{tikzpicture}[scale=0.8,every node/.style={scale=0.8}]
              \path
                  node {\textbf{A}}
                  to ++(0:0.5) node (a0) {}
                  to ++(0:1) node[point, label=above:{rename to \epoch{A1}}] (a1) {}
                  to ++(0:1) node (a2) {}
                  to ++(0:1) node (a7) {}
                  to ++(0:1) node[point, label=above:{rename to \epoch{A8}}] (a8) {}
                  to ++(0:1.5) node (a-receives-b2) {}
                  to ++(0:1.5) node[point, label=above:{rename to \epoch{A9}}] (a9) {}
                  to ++(0:1.5) node (a-end) {};

              \draw[->, thick] (a0) --  (a1) -- (a2);
              \draw[dotted] (a2) -- (a7);
              \draw[->, thick] (a7) -- (a8) -- (a9) -- (a-end);

              \path
                  to ++(270:2) node {\textbf{B}}
                  to ++(0:0.5) node (b0) {}
                  to ++(0:1) node[point, label=below:{rename to \epoch{B2}}] (b2) {}
                  to ++(0:1.5) node (b-receives-a1) {}
                  to ++(0:1.5) node (b3) {}
                  to ++(0:1) node (b6) {}
                  to ++(0:1) node[point, label=below:{rename to \epoch{B7}}] (b7) {}
                  to ++(0:2.5) node (b-end) {};

              \draw[->, thick] (b0) -- (b2) -- (b3);
              \draw[dotted] (b3) -- (b6);
              \draw[->, thick] (b6) -- (b7) -- (b-end);

              \draw[->, dashed, shorten >= 1] (a1) -- (b-receives-a1);
              \draw[->, dashed, shorten >= 1] (b2) -- (a-receives-b2);
          \end{tikzpicture}
          \label{fig:GC-execution}
      \end{minipage}}
  \hfil
  \subfloat[States of respective epoch trees with \emph{potential current epochs} and \emph{required epochs} displayed]{
      \begin{minipage}{\linewidth}
          \centering
          \begin{tikzpicture}[scale=0.8,every node/.style={scale=0.8}]
              \path
                  node {\textbf{A}}
                  to ++(270:1) node[causalop] (Ae0) {\epoch{0}}
                  to ++(225:1.5) node[op] (AeA1) {\epoch{A1}}
                  to ++(270:1.5) node[op] (AeA8) {\epoch{A8}};
              \path
                  to ++(270:1)
                  to ++(315:1.5) node[causalop] (AeB2) {\epoch{B2}}
                      node[outer sep=12pt] (A-tl-potential-epochs) {}
                      node[outer sep=15pt] (A-tl-required-epochs) {}
                  to ++(270:1.5) node[op, red] (AeA9) {\epoch{A9}}
                      node[outer sep=12pt] (A-br-potential-epochs) {}
                      node[outer sep=15pt] (A-br-required-epochs) {};

              \draw[dashed] (A-tl-potential-epochs.north west) rectangle (A-br-potential-epochs.south east);
              \draw[dotted, thick, darkgreen] (A-tl-required-epochs.north west) rectangle (A-br-required-epochs.south east);

              \path
                  to ++(0:4) node {\textbf{B}}
                  to ++(270:1) node[causalop] (Be0) {\epoch{0}}
                  to ++(315:1.5) node[op] (BeB2) {\epoch{B2}}
                  to ++(270:1.5) node[op, red] (BeB7) {\epoch{B7}}
                      node[outer sep=12pt] (B-br-potential-epochs) {}
                      node[outer sep=15pt] (B-br-required-epochs) {};
              \path
                  to ++(0:4)
                  to ++(270:1)
                  to ++(225:1.5) node[causalop] (BeA1) {\epoch{A1}}
                      node[outer sep=12pt] (B-tl-potential-epochs) {};

              \draw[dashed] (B-tl-potential-epochs.north west) rectangle (B-br-potential-epochs.south east);
              \draw[dotted, thick, darkgreen] (B-tl-potential-epochs.north west)[xshift=-3pt, yshift=33pt] rectangle (B-br-required-epochs.south east);

              \draw[thick, ->] (Ae0) edge (AeA1) (AeA1) edge (AeA8) (Ae0) edge (AeB2) (AeB2) edge (AeA9);
              \draw[thick, ->] (Be0) edge (BeB2) (BeB2) edge (BeB7) (Be0) edge (BeA1);
              \draw[->, dashed, thick, red] (AeA9.135) -- (AeB2.225) (AeB2.180) -- (AeA8.45) -- (AeA1.315) (AeA1.0) -- (Ae0.270);
              \draw[->, dashed, thick, red] (BeB7.135) -- (BeB2.225) (BeB2.180) -- (BeA1.0) -- (Be0.270);

          \end{tikzpicture}
          \label{fig:GC-epoch-trees}
      \end{minipage}}
  \caption{Garbage collecting epochs and corresponding \emph{former states}}
  \label{fig:GC-epochs}
\end{figure}

Dans la \autoref{fig:GC-execution}, nous représentons une exécution au cours de laquelle deux noeuds A et B génère respectivement plusieurs opérations \emph{rename}.
Dans la \autoref{fig:GC-epoch-trees}, nous représentons les \emph{arbre des époques} respectifs de chaque noeud.
Les époques introduites par des opérations \emph{rename} causalement stables sont representées en utilisant des doubles cercles.
L'ensemble des \emph{potentielles époques courantes} est montré sous la forme d'un rectangle noir pointillé, tandis que l'ensemble des \emph{époques requises} est représenté par un rectangle vert pointillé.

\mnnote{TODO: Trouver un autre terme que pointillé pour dotted}

Le noeud A génère tout d'abord une opération \emph{rename} vers \epoch{A1} et ensuite une opération \emph{rename} vers \epoch{A8}.
Il reçoit ensuite une opération \emph{rename} du noeud B qui introduit \epoch{B2}.
Puisque \epoch{B2} est plus grand que son époque courante actuelle (\epoch{e0}\epoch{A1}\epoch{A8} < \epoch{e0}\epoch{B2}), le noeud A la sélectionne comme sa nouvelle époque cible et procède au renommage de son état en conséquence.
Finalement, le noeud A génère une troisième opération \emph{rename} vers \epoch{A9}.

De manière concurrente, le noeud B génère l'opération \emph{rename} vers \epoch{B2}.
Il reçoit ensuite l'opération \emph{rename} vers \epoch{A1} du noeud A.
Cependant, le noeud B conserve \epoch{B2} comme époque courante (puisque \epoch{e0}\epoch{A1} < \epoch{e0}\epoch{B2}).
Après, le noeud B génère une autre opération \emph{rename} vers \epoch{B7}.

À la livraison de l'opération \emph{rename} introduisant l'époque \epoch{B2} au noeud A, cette opération devient causalement stable.
À partir de ce point, le noeud A sait que tous les noeuds ont progressé jusqu'à cette époque ou une plus grande d'après la relation \emph{priority}.
Les époques \epoch{B2} et \epoch{A9} forment donc l'ensemble des \emph{potentielles époques courantes} et les noeuds peuvent seulement émettre des opérations depuis ces époques ou une de leur descendante encore inconnue.
Le noeud A procède ensuite au calcul de l'ensemble des \emph{époques requises}.
Pour ce faire, il détermine le \ac{LCA} des \emph{potentielles époques courantes} : \epoch{B2}.
Il génère ensuite l'ensemble des \emph{époques requises} en ajoutant toutes les époques formant les chemins entre \epoch{B2} et les \emph{potentielles époques courantes}.
Les époques \epoch{B2} et \epoch{A9} forment donc l'ensemble des \emph{époques requises}.
Le noeud A déduit que les époques \epoch{0}, \epoch{A1} et \epoch{A8} peuvent être supprimées de manière sûre.

À l'inverse, la livraison de l'opération \emph{rename} vers \epoch{A1} au noeud B ne lui permet pas de supprimer la moindre métadonnée.
À partir de ses connaissances, le noeud B calcule que \epoch{A1}, \epoch{B2} et \epoch{B7} forment l'ensemble des \emph{potentielles époques courantes}.
De cette information, le noeud B détermine que ces époques et leur \ac{LCA} forment l'ensemble des \emph{époques requises}.
Toute époque connue appartient donc à l'ensemble des \emph{époques requises}, empêchant leur suppression.

À terme, une fois que le système devient inactif, les noeuds atteignent la même époque et l'opération \emph{rename} correspondante devient causalement stable.
Les noeuds peuvent alors supprimer toutes les autres époques et métadonnées associées, supprimant ainsi le surcoût mémoire introduit par le mécanisme de renommage.

Notons que le mécanisme de récupération de mémoire peut être simplifié dans les systèmes empêchant les opérations \emph{rename} concurrentes.
Puisque les époques forment une chaîne dans de tels systèmes, la dernière époque introduite par une opération \emph{rename} causalement stable devient le \ac{LCA} des \emph{potentielles époques courantes}.
Il s'ensuit que cette époque et ses descendantes forment l'ensemble des \emph{époques requises}.
Les noeuds n'ont donc besoin que de suivre les opérations \emph{rename} causalement stables pour déterminer quelles époques peuvent être supprimées dans les systèmes sans opérations \emph{rename} concurrentes.

Pour déterminer qu'une opération \emph{rename} donnée est causalement stable, les noeuds doivent être conscients des autres et de leur avancement.
Un protocole de gestion de groupe tel que \cite{swim2002,lifeguard2018} est donc requis.

La stabilité causale peut prendre un certain temps à atteinte.
En attendant, les noeuds peuvent en fait décharger les anciens états sur le disque dur puis qu'ils sont seulement nécessaires pour traiter les opérations concurrentes aux opérations \emph{rename}.
Nous approfondissons ce sujet dans la \autoref{sec:offloading-former-states}.

\section{Validation}

\subsection{Preuve de correction de \textsc{renameId}}

\subsection{Complexité temporelle}

\begin{table}[!ht]
  \centering
  \caption{Complexité temporelle des différentes opérations}
  \label{tab:time-complexity-operations-with-revertRenameId}
  %\resizebox{0.5\columnwidth}{!}{
  \begin{tabular}{lcc}
      \toprule
      Type d'opération & \multicolumn{2}{c}{Complexité temporelle} \\
      \cmidrule(lr){2-3}
        & Locale &   Distante \\
      \midrule
      \emph{insert} & log($b$) & $h$ + $k$ + $l$ $\cdot$ log($m$) + log($b$) \\
      \emph{remove} & log($b$) & $h$ + $k$ + $l$ $\cdot$ log($m$) + log($b$) \\
      \emph{naive rename} & $n$ & $h$ + $n$ ($k$ + $l$ $\cdot$ log($m$) + log($b$)) \\
      \emph{rename} & $n$ & $h$ + $k$ $\cdot$ $n$ + $l$ ($n$ + $m$) + $n$ + $b$ \\
      \bottomrule
  \end{tabular}
  \caption*{$b$ : nombre de blocs, $n$ : nombre d'éléments, $h$ : hauteur de l'\emph{arbre des époques}, $k$ : nombre de renommages à inverser, $l$ : nombre de renommages à appliquer, $m$ : nombre d'éléments de l'\emph{ancien état} des renommages à appliquer}
  %}
\end{table}

\begin{itemize}
  \item Nous supposons que le nombre d'identifiants dans la séquence $n$ croît au fur et à mesure
  \item La taille des identifiants est quant à elle réinitialisé périodiquement par l'opération \emph{rename}
  \item Nous considérons que celle-ci devient négligeable par rapport au nombre d'identifiants.
  \item Nous ne prenons donc pas en compte la taille des identifiants dans nos complexités.
    \mnnote{QUESTION: Ai-je le droit de faire ça ?}
  \item \textsc{renameId} a pour complexité O(log($m$)), où $m$ est le nombre d'éléments de $renamedIds$, l'\emph{ancien état}
  \begin{itemize}
    \item Puisque que pire cas consiste à parcourir $renamedIds$ à la recherche du prédecesseur de $id$
    \item $renamedIds$ étant une liste triée d'identifiants, une recherche binaire peut être effectuée
  \end{itemize}
  \item \textsc{revertRenameId} a pour complexité O(1)
  \begin{itemize}
    \item Ici, pas de recherche à effectuer
    \item L'offset du 1er tuple de $id$ permet de retrouver directement ses prédecesseur et successeur dans $renamedIds$
  \end{itemize}
  \item Considérons que nous utilisons un AVL comme structure de données pour représenter la séquence
  \item La complexité d'une opération \emph{insert} locale est en O(log($b$)) (cf. article LogootSplit), puisque cette action est inchangée par rapport à LogootSplit
  \item La complexité d'une opération \emph{insert} distante, elle, évolue par rapport à LogootSplit
  \item En effet, l'opération peut être issue d'une autre époque que l'époque courante du noeud
  \item Pour l'intégrer, doit au prélable déterminer l'époque \ac{LCA} entre l'époque de l'opération et l'époque courante
  \item Correspond à trouver l'intersection entre deux branches de l'arbre
  \item S'agit d'un algorithme avec une complexité en O($h$), où $h$ est la hauteur de l'\emph{arbre des époques}
  \item Ensuite, le noeud doit appliquer successivement $k$ fois \textsc{revertRenameId} sur l'identifiant à insérer
  \begin{itemize}
    \item Où $k$ est le nombre d'époques formant la branche entre l'époque de l'opération et l'époque \ac{LCA}
    \item Permet d'obtenir l'identifiant équivalent à l'époque \ac{LCA}
  \end{itemize}
  \item Le noeud applique ensuite $l$ fois \textsc{renameId} sur l'identifiant obtenu
  \begin{itemize}
    \item Où $l$ est le nombre d'époques formant la branche entre l'époque \ac{LCA} et l'époque courante
    \item Permet d'obtenir l'identifiant équivalent à l'époque courante
  \end{itemize}
  \item Il ne reste plus qu'au noeud à insérer l'élément dans la séquence
  \item Nous obtenons donc une complexité de O($h$ + $k$ + $l$ $\cdot$ log($m$) + log($b$)) pour l'opération \emph{insert} distante
  \item De la même manière, nous obtenons des complexités similaires pour l'opération \emph{remove}, en local et distant
  \item Concernant l'opération \emph{rename}
  \item L'opération \emph{rename} locale consiste à parcourir et à linéariser la structure actuelle pour ne garder que les idIntervals courant
  \item Et à créer une nouvelle séquence équivalente à l'ancienne, composée seulement du nouveau bloc
  \item Sa complexité est donc de O($n$)
  \item Pour l'opération \emph{rename} distante, une implémentation naïve consiste à générer le nouvel état en renommant et insérant chaque identifiant de l'état courant d'une manière similaire à l'opération \emph{insert} distante
  \item On obtient dans ce cas une complexité de O($h$ + $n$ ($k$ + $l$ $\cdot$ log($m$) + log($b$)))
  \item Mais peut améliorer
  \item Plutôt que d'effectuer une recherche binaire sur $renamedIds$ pour trouver le prédecesseur de $id$
  \item Peut tirer parti du fait qu'on va parcourir séquentiellement l'état, qui est une liste triée d'identifiants
  \item Et parcourir en parallèle, au fur et à mesure, $renamedIds$ pour trouver le prédecesseur du $id$ courant
  \item Permet de ramener la complexité du renommage des $n$ identifiants de l'époque \ac{LCA} à l'époque courante à O($l$ ($n$ + $m$))
  \item Plutôt que de générer une nouvelle séquence et d'y insérer un par un les $n$ éléments
  \item Peut parcourir les $n$ éléments pour reformer $b$ blocs
  \item Peut ensuite construire l'AVL de manière récursive en parcourant les $b$ blocs obtenus
  \item Peut donc obtenir une complexité de O($h$ + $k$ $\cdot$ $n$ + $l$ ($n$ + $m$) + $n$ + $b$) pour l'intégration de l'opération \emph{rename} distante
\end{itemize}

\subsection{Expérimentations}

Afin de valider l'approche que nous proposons, nous avons procédé à une évaluation expérimentale.
Les objectifs de cette évaluation étaient de mesurer
\begin{enumerate*}[label=(\roman*)]
  \item le surcoût mémoire de la séquence répliquée
  \item le surcoût en calculs ajouté aux opérations \emph{insert} et \emph{remove} par le mécanisme de renommage
  \item le coût d'intégration des opérations \emph{rename}.
\end{enumerate*}

Par le biais de simulations, nous avons généré le jeu de données utilisé par nos benchmarks.
Ces simulations reproduisent le scénario suivant.

\subsubsection{Scénario d'expérimentation}

Plusieurs auteurs rédigent de manière collaborative un article en temps réel.
Dans un premier temps, les auteurs spécifie principalement le contenu de l'article.
Quelques opérations \emph{remove} sont tout même générées pour simuler des fautes de frappes.
Une fois que le document atteint une taille arbitrairement définie comme critique, les collaborateurs passent à la seconde phase de la collaboration.
Au cours de cette phase, les auteurs arrêtent d'ajouter du nouveau contenu mais se concentre à la place sur le remaniement du contenu existant.
Ceci est simulé en équilibrant le ratio entre les opérations \emph{insert} et \emph{remove}.
Chaque auteur doit émettre un nombre donné d'opérations \emph{insert} et \emph{remove}.
La simulation prend fin une fois que tous les collaborateurs ont reçu toutes les opérations.
Au cours de la simulation, nous prenons des instantanés de l'état des pairs à des points donnés pour suivre leur évolution.

\subsubsection{Implémentation des simulations}

Nous avons effectué nos simulations avec les paramètres expérimentaux suivants : nous avons déployé 10 bots à l'aide de conteneurs Docker sur une même machine.
Chaque conteneur correspond à un processus Node.js mono-threadé et permet de simuler un auteur.
Les bots partagent et éditent de façon collaborative le document en utilisant soit LogootSplit soit RenamableLogootSplit en fonction de la session.
Dans chaque cas, chaque bot génère localement une opération \emph{insert} ou \emph{remove} toutes les 200 $\pm$ 50ms et la diffuse immédiatement aux autres noeuds via un réseau P2P maillé.
Au cours de la première phase, la probabilité d'émetter une opération \emph{insert} (resp. \emph{remove}) est de 80\% (resp. 20\%).
Une fois que le document atteint 60k caractères (environ 15 pages), les bots passent à la seconde phase et mettent chaque probabilité à 50\%.
Après chaque opération locale, le bot peut déplacer son curseur à une autre position aléatoire dans le document avec une probabilité de 5\%.
Chaque bot génère 15k opérations \emph{insert} ou \emph{remove} et s'arrête une fois qu'il a observé 150k opérations.
Des instantanés de l'état du bot sont pris de façon périodique, toutes les 10k opérations observées.

De plus, dans le cas de RenamableLogootSplit, 1 à 4 bots sont désignés de façon arbitraire comme des \emph{renaming bots} en fonction de la session.
Les \emph{renaming bots} génèrent des opérations \emph{rename} toutes les 30k opérations qu'ils observent.
Ces opérations \emph{rename} sont générées de façon à assurer qu'elles soient concurrentes.

Dans un but de reproductibilité, nous avons mis à disposition notre code, nos benchmarks et les résultats à l'adresse suivante : \url{https://github.com/coast-team/mute-bot-random/}.

\subsection{Résultats}

En utilisant les instantanés générés, nous avons effectué plusieurs benchmarks.
Ces benchmarks évaluent les performances de RenamableLogootSplit et les compare à celles de LogootSplit.
Les résultats sont présentés et analysés ci-dessous.

\subsubsection{Convergence}

Nous avons tout d'abord vérifié la convergence de l'état des noeuds à l'issue des simulations.
Pour chaque simulation, nous avons comparé l'état final de chaque noeud à l'aide de leur instantanés respectifs.
Nous avons pu confirmer que les noeuds convergaient sans aucune autre forme de communication que les opérations, satisfaisant donc le modèle de la \ac{SEC}.

Ce résultat établit un premier jalon dans la validation de la correction de RenamableLogootSplit.
Il n'est cependant qu'empirique.
Des travaux supplémentaires pour prouver formellement sa correction doivent être entrepris.

\subsubsection{Consommation mémoire}

Nous avons ensuite procédé à l'évaluation de l'évolution de la consommation mémoire du document au cours des simulations, en fonction du \ac{CRDT} utilisé et du nombre de \emph{renaming bots}.
Nous présentons les résultats obtenus dans la \autoref{fig:evolution-document-size}.

\begin{figure}[!ht]
  \centering
  \includegraphics[width=0.7\columnwidth]{img/snapshot-sizes-alt-legende-v2.png}
  \caption{Evolution of the size of the document}
  \label{fig:evolution-document-size}
\end{figure}

Pour chaque graphique dans la \autoref{fig:evolution-document-size}, nous représentons 4 données différentes.
La ligne pointillée bleue correspond à la taille du contenu du document, \ie du texte, tandis que la ligne continue rouge représente la taille complète du document LogootSplit.

La ligne verte en pointillés représente la taille du document RenamableLogootSplit dans son meilleur cas.
Dans ce scénario, les noeuds considèrent que les opérations \emph{rename} sont supprimables dès qu'ils les reçoivent.
Les noeuds peuvent alors bénéficier des effets du mécanisme de renommage tout en supprimant les métadonnées qu'il introduit : les \emph{anciens états} et époques.
Ce faisant, les noeuds peuvent minimiser de manière périodique le surcoût en métadonnées de la structure de données, indépendamment du nombre de \emph{renaming bots} et d'opérations \emph{rename} concurrentes générées.

La ligne pointillée orange représente la taille du document RenamableLogootSplit dans son pire cas.
Dans ce scénario, les noeuds considèrent que les opérations \emph{rename} ne deviennent jamais causalement stables et qu'elles ne peuvent donc jamais être supprimées.
Les noeuds doivent alors conserver de façon permanente les métadonnées introduites par le mécanisme de renommage.
Les performances de RenamableLogootSplit diminuent donc à mesure que le nombre de \emph{renaming bots} et d'opérations \emph{rename} générées augmente.
Néanmoins, nous observons que RenamableLogootSplit peut surpasser les performances de LogootSplit tant que le nombre de \emph{renaming bots} reste faible (1 ou 2).
Ce résultat s'explique par le fait que le mécanisme de renommage permet aux noeuds de supprimer les métadonnées de la structure de données utilisée en interne pour représenter la séquence.

Pour récapituler les résultats présentés, le mécanisme de renommage introduit un surcoût temporaire en métadonnées qui augmente avec chaque opération \emph{rename}.
Mais le surcoût se résorbe à terme une fois que le système devient quiescent et que les opérations \emph{rename} deviennent causalement stables.
Dans la section \autoref{sec:offloading-former-states}, nous détaillerons l'idée que les \emph{anciens états} peuvent être déchargés sur le disque en attendant que la stabilité causale soit atteinte pour atténuer le surcoût temporaire en métadonnées.

\subsubsection{Temps d'intégration des opérations standards}

Nous avons ensuite comparé l'évolution du temps d'intégration des opérations standards, \ie les opérations \emph{insert} et \emph{remove}, sur des documents LogootSplit et RenamableLogootSplit.
Puisque les deux types d'opérations partagent la même complexité temporelle, nous avons seulement utilisé des opérations \emph{insert} dans nos benchmarks.
Nous faisons par contre la différence entre les mises à jours \emph{locales} et \emph{distantes}.
Conceptuellement, les modifications locales peuvent être décomposées comme présenté dans \cite{baquero2017pure} en les deux étapes suivantes :
\begin{enumerate*}[label=(\roman*)]
  \item la génération de l'opération correspondante
  \item l'application de l'opération correspondante sur l'état local.
\end{enumerate*}
Cependant, pour des raisons de performances, nous avons fusionné ces deux étapes dans notre implémentation.
Nous distinguons donc les résultats des modifications \emph{locales} et des modifications \emph{distantes} dans nos benchmarks.
La \autoref{fig:evolution-integration-time-insert} présente les résultats obtenus.

\begin{figure}[!ht]
  \centering
  \subfloat[Local updates]{
      \includegraphics[width=0.45\columnwidth]{img/integration-time-boxplot-local-operations-without-outliers.pdf}
      \label{fig:evolution-integration-time-local-insert}}
  \hfil
  \subfloat[Remote updates]{
      \includegraphics[width=0.45\columnwidth]{img/integration-time-boxplot-remote-operations-without-outliers.pdf}
      \label{fig:evolution-integration-time-remote-insert}}
  \caption{Integration time of standard operations}
  \label{fig:evolution-integration-time-insert}
\end{figure}

Dans ces figures, les boxplots orange correspondent aux temps d'intégration sur des documents LogootSplit, les boxplots bleu sur des documents RenamableLogootSplit.
Bien que les deux mesures soient initialement équivalentes, les temps d'intégration sur des documents RenamableLogootSplit sont ensuite réduits par rapport à ceux de LogootSplit une fois que des opérations \emph{rename} ont été intégrées.
Cette amélioration s'explique par le fait que l'opération \emph{rename} optimise la représentation interne de la séquence.

Dans le cadre des opérations distantes, nous avons mesuré des temps d'intégration spécifiques à RenamableLogootSplit : le temps d'intégration d'opérations distantes provenant d'époques \emph{parentes} et d'époques \emph{soeurs}, respectivement affiché sous la forme de boxplots blanche et rouge dans la \autoref{fig:evolution-integration-time-remote-insert}.

Les opérations distantes provenant d'époques \emph{parentes} sont des opérations générées de manière concurrente à l'opération \emph{rename} mais appliquées après cette dernière.
Puisque l'opération doit être transformée au préalable en utilisant \textsc{renameId}, nous observons un surcoût computationnel par rapport aux autres opérations.
Mais ce surcoût est compensé par l'optimisation de la représentation interne de la séquence effectuée par l'opération \emph{rename}.

Concernant les opérations provenant d'époques \emph{soeurs}, nous observons un surcoût additionnel puisque les noeuds doivent tout d'abord annulé les effets de l'opération \emph{rename} concurrente en utilisant \textsc{revertRenameId}.
À cause de cette étape supplémentaire, les performances de RenamableLogootSplit pour ces opérations sont comparables à celles de LogootSplit.

Pour récapituler, les fonctions de transformation ajoutent un surcoût aux temps d'intégration des opérations concurrentes aux opérations \emph{rename}.
Malgré ce surcoût, RenamableLogootSplit obtient de meilleures performances que LogootSplit tant que la distance entre l'époque de génération de l'opération et l'époque courante du noeud reste limité.
Au fur et à mesure que la distance entre les deux époques augmente, les performances de RenamableLogootSplit diminuent, jusqu'à atteindre des performances moins bonnes que celles de LogootSplit, puisque le surcoût est multiplié.
Néanmoins, le mécanisme de renommage réduit le temps d'intégration de la majorité des opérations, \ie les opérations générées entre deux séries d'opérations \emph{rename}.

\subsubsection{Temps d'intégration de l'opération de renommage}

Finalement, nous avons mesuré l'évolution du temps d'intégration de l'opération \emph{rename} en fonction du nombre d'opérations depuis l'opération \emph{rename} précédente.
Comme précédemment, nous distinguons les performances des modifications \emph{locales} et \emph{distantes}.
Le cas des opérations \emph{rename distantes} se sous-divise en trois catégories.
Les opérations \emph{distantes directes} désignent les opérations \emph{rename distantes} qui introduisent une nouvelle époque \emph{enfant} de l'époque courante du noeud.
Les opérations \emph{concurrentes introduisant une plus grande} (resp. \emph{petite}) \emph{époque} désigne les opérations \emph{rename} qui introduisent une époque \emph{soeur} de l'époque courante du noeud.
D'après la relation \emph{priority}, l'époque introduite est plus grande (resp. petite) que l'époque courante du noeud.
Les résultats obtenus sont présentés dans le \autoref{tab:evolution-integration-time-rename}.

\begin{table}[!ht]
  \centering
  \caption{Integration time of rename operations}
  \label{tab:evolution-integration-time-rename}
  \resizebox{0.7\columnwidth}{!}{
      \begin{tabular}{lrrrrr}
          \toprule
          \multicolumn{2}{c}{Parameters} & \multicolumn{4}{c}{Integration Time (ms)} \\
          \cmidrule(lr){1-2} \cmidrule(lr){3-6}
          Type & Nb Ops (k) &   Mean &   Median & 99\textsuperscript{th} Quant. &    Std \\
          \midrule
          Local & 30  &    41.75 &    38.74 &      71.68 &   6.84 \\
                                  & 60  &    78.32 &    78.16 &      81.42 &   1.24 \\
                                  & 90  &   119.19 &   118.87 &     124.22 &   2.49 \\
                                  & 120 &   143.75 &   143.57 &     148.59 &   2.16 \\
                                  & 150 &   158.04 &   157.95 &     164.38 &   2.49 \\
          \cmidrule(lr){1-6}
          Direct remote & 30  &   481.32 &   477.13 &     537.30 &  17.11 \\
                                  & 60  &   981.62 &   978.24 &    1072.83 &  31.54 \\
                                  & 90  &  1491.28 &  1481.83 &    1657.58 &  51.10 \\
                                  & 120 &  1670.00 &  1663.85 &    1814.38 &  50.29 \\
                                  & 150 &  1694.17 &  1675.95 &    1852.55 &  59.94 \\
          \cmidrule(lr){1-6}
          Cc. int. greater epoch & 30  &   643.53 &   643.57 &     682.80 &  13.42 \\
                                  & 60  &  1317.66 &  1316.39 &    1399.55 &  28.67 \\
                                  & 90  &  1998.23 &  1994.08 &    2111.98 &  45.37 \\
                                  & 120 &  2239.71 &  2233.22 &    2368.45 &  50.06 \\
                                  & 150 &  2241.92 &  2233.61 &    2351.02 &  52.20 \\
          \cmidrule(lr){1-6}
          Cc. int. lesser epoch & 30  &     1.36 &     1.30 &       3.53 &   0.37 \\
                                  & 60  &     2.82 &     2.69 &       4.85 &   0.45 \\
                                  & 90  &     4.45 &     4.23 &       5.81 &   0.71 \\
                                  & 120 &     5.33 &     5.10 &       8.78 &   0.90 \\
                                  & 150 &     5.53 &     5.26 &       8.70 &   0.79 \\
          \bottomrule
      \end{tabular}
  }
\end{table}

Le principal résultat de ces mesures est que les opérations \emph{rename} sont généralement coûteuses quand comparées aux autres types d'opérations, puisque les noeuds doivent parcourir et renommer leur état courant complet.
Les opérations \emph{rename} locales s'intègrent en plusieurs centaines de millisecondes tandis que les opérations \emph{distantes directes} et \emph{concurrentes introduisant une plus grande époque} peuvent prendre des secondes si retardées trop longtemps.
Il est donc nécessaire de prendre en compte ce résultat pour concevoir des stratégies de génération des opérations \emph{rename} pour éviter d'impacter négativement l'expérience utilisateur.

Un autre résultat intéressant de ces benchmarks est que les opérations \emph{concurrentes introduisant une plus petite époque} sont rapides à intégrer.
Puisque ces opérations introduisent une époque qui n'est pas sélectionné comme nouvelle époque cible, les noeuds ne procède pas au renommage de leur état.
L'intégration des opérations \emph{concurrentes introduisant une plus petite époque} consiste simplement à ajouter l'époque introduite et l'\emph{ancien état} correspondant à l'\emph{arbre des époques}.
Les noeuds peuvent donc réduire de manière significative le coût d'intégration d'un ensemble d'opérations \emph{rename} concurrentes en les appliquant dans l'ordre le plus adapté en fonction du contexte.

\section{Discussion}

\subsection{Stockage des états précédents sur disque}

\label{sec:offloading-former-states}

Les noeuds doivent conserver les \emph{anciens états} associés aux opérations \emph{rename} pour transformer les opérations issues d'époques précédentes ou concurrentes.
Les noeuds peuvent recevoir de telles opérations dans deux cas précis :
\begin{enumerate*}[label=(\roman*)]
  \item des noeuds ont émis récemment des opérations \emph{rename}
  \item des noeuds se sont récemment reconnectés
\end{enumerate*}
Entre deux de ces évènements spécifiques, les \emph{anciens états} ne sont pas nécessaires pour traiter les opérations.

Nous pouvons donc proposer l'optimisation suivante : décharger les \emph{anciens états} sur le disque jusqu'à leur prochaine utilisation ou jusqu'à ce qu'ils puissent être supprimés de manière sûre.
Décharger les \emph{anciens états} sur le disque permet de mitiger le surcoût en mémoire introduit par le mécanisme de renommage.
En échange, cela augmente le temps d'intégration des opérations nécessitant un \emph{ancien état} qui a été déchargé précédemment.

Les noeuds peuvent adopter différentes stratégies, en fonction de leurs contraintes, pour déterminer les \emph{anciens états} comme déchargeables et pour les récupérer de manière préemptive.
La conception de ces stratégies peut reposer sur différentes heuristiques : les époques des noeuds actuellement connectés, le nombre de noeuds pouvant toujours émettre des opérations concurrentes, le temps écoulé depuis la dernière utilisation de l'\emph{ancien état}...

\subsection{Utilisation de l'opération de renommage comme snapshot}

\begin{itemize}
  \item L'opération de renommage embarque la somme de toutes les opérations passées sous la forme du \emph{former state}
  \item On peut à tout moment re-calculer l'état courant du document à partir de l'opération de renommage primaire, des opérations concurrentes à cette dernière et des opérations générées depuis
  \item Peut utiliser ce principe pour le mécanisme de synchronisation
  \item Lorsqu'un nouveau pair rejoint la collaboration, le noeud avec lequel il se synchronise peut lui fournir uniquement ce sous-ensemble des opérations
  \item De la même manière, on pourrait généraliser l'utilisation de cette méthode de synchronisation
  \item À la réception d'une demande de synchronisation d'un pair présentant un important retard, le noeud peut choisir d'employer cette méthode
  \begin{itemize}
    \item Plutôt que de lui envoyer et de lui faire rejouer l'ensemble des opérations
  \end{itemize}
  \item La question étant de comment procéder pour quantifier ce retard et pour définir le seuil à partir duquel ce retard est considéré comme important
  \item Cette méthode de synchronisation pose néanmoins le problème suivant
  \item Le pair synchronisé de cette manière ne possède qu'une partie du log des opérations
  \item S'il reçoit ensuite une demande de synchronisation d'un autre pair, il est possible qu'il ne puisse y répondre
  \begin{itemize}
    \item Cas où il manque à l'autre pair juste une opération d'avant le renommage (possible si les dépendances causales ne sont pas requises pour intégrer l'opération de renommage)
  \end{itemize}
  \item Dans ce cas, ne peut pas fournir la seule opération manquante au pair qui la demande
  \begin{itemize}
    \item \mnnote{NOTE: Mais dans ce cas, le pair peut tout à fait générer un état courant à jour à partir des infos qu'il possède puisque l'opération qui lui manque est intégrée dans l'opé de renommage}
  \end{itemize}
  \item \mnnote{TODO: Étudier si y a un intérêt à privilégier la synchronisation basée sur l'intégration successive de toutes les opérations quand on a cette méthode de synchronisation par snapshot/checkpoint de possible}
\end{itemize}

\subsection{Compression et limitation de la taille de l'opération \emph{rename}}

Pour limiter la consommation en bande passante des opérations \emph{rename}, nous proposons la technique de compression suivante.
Au lieu de diffuser les identifiants complets formant l'\emph{ancien état}, les noeuds peuvent diffuser seulement les éléments nécessaires pour identifier de manière unique les blocs.
En effet, un identifiant peut être caractérisé de manière unique par le triplet composé de l'\emph{identifiant de noeud}, du \emph{numéro séquentiel} et de l'\emph{offset} de son dernier tuple.
Par conséquent, un bloc peut être identifié de manière unique à partir du triplet signature de son identifiant de début et de sa longueur.
Cette méthode nous permet de réduire les données à diffuser dans le cadre de l'opération \emph{rename} à un montant fixe par bloc.

Pour décompresser l'opération reçue, les noeuds parcourent leur état courant ainsi que leur log des opérations \emph{remove} concurrentes.
De cette manière, ils peuvent retrouver les identifiants complets et reconstruire l'opération \emph{rename} originale.
\mnnote{TODO: Développer ce paragraphe : noeuds doivent retrouver les blocs renommés à partir des données reçues. Pour cela, parcourent leur état. Suffit de retrouver un identifiant avec le même couple $\langle$nodeId, nodeSeq$\rangle$ pour reformer un bloc. Certains couples $\langle$nodeId, nodeSeq$\rangle$ peuvent avoir été supprimés en concurrence et ne plus être présent dans la séquence. Donc besoin d'aussi parcourir le log des opérations \emph{remove} concurrentes.}

Grâce à cette méthode de compression, nous pouvons instaurer une taille maximale à l'opération \emph{rename}.
En effet, les noeuds peuvent émettre une opération \emph{rename} dès que leur état courant atteint un nombre donné de blocs, bornant ainsi la taille du message à diffuser.

\subsection{Définition de relations de priorité pour minimiser les traitements}

Bien que la relation \emph{priority} proposée dans la \autoref{sec:priority} est simple et garantit que tous les noeuds désignent la même époque comme époque cible, elle introduit un surcoût computationnel significatif dans certains cas.
Notamment, cette relation \emph{priority} autorise le cas où un simple noeud, déconnecté de la collaboration depuis longtemps, force l'ensemble des autres noeuds à annuler les opérations \emph{rename} qu'ils ont effectué pendant ce temps car sa propre opération \emph{rename} introduit la nouvelle époque cible.
\mnnote{TODO: ajouter figure d'un epoch tree où une longue branche se fait remplacer par une époque isolée}

La relation \emph{priority} devrait donc être conçue pour garantir la convergence des noeuds, mais aussi pour minimiser les calculs effectués globalement par les noeuds du système.
Pour concevoir une relation \emph{priority} efficace, nous pourrions incorporer dans les opérations \emph{rename} des métriques qui représentent l'état du système et le travail accumulé sur le document (nombre de noeuds actuellement à l'époque \emph{parente}, nombre d'opérations générées depuis l'époque parente, taille du document...).
De cette manière, nous pourrions favoriser la branche de l'\emph{arbre des époques} regroupant les collaborateurs les plus actifs et empêcher les noeuds isolés d'imposer leurs opérations \emph{rename}.

Afin d'offrir une plus grande flexibilité dans la conception de la relation \emph{priority}, il est nécessaire de retirer la contrainte interdisant aux noeuds de rejouer une opération \emph{rename}.
Pour cela, un couple de fonctions réciproques doit être proposée pour \textsc{renameId} et \textsc{revertRenameId}.
Une solution alternative est de proposer une implémentation du mécanisme de renommage qui repose sur les identifiants originaux plutôt que sur ceux transformés, par exemple en utilisant le log des opérations.

\subsection{Implémentation alternative à base d'operation-log}

\begin{itemize}
  \item Implémentation du mécanisme de renommage décrite et évaluée dans ce manuscrit consiste à appliquer les fonctions de transformations \textsc{renameId} et \textsc{revertRenameId} à l'ensemble des identifiants composant l'état lors de l'intégration d'une opération \emph{rename}
  \item Une implémentation alternative se basant sur log des opérations peut être proposée
  \item Consiste à partir de l'état produit par l'opération \emph{rename}
  \item Puis à transformer et à intégrer l'ensemble des opérations concurrentes connues par le noeud sur cet état
  \item Approche présente plusieurs avantages par rapport à l'implémentation actuelle
  \begin{itemize}
    \item Réduit le nombre de transformations calculées : ne transforme plus chaque élément (ou blocs si dispose de \textsc{renameBlock}) mais seulement chaque d'opération concurrente.
      En fonction de la fréquence de déclenchement de l'opération \emph{rename}, ce nombre peut être plus petit de plusieurs ordres de grandeur.
      Plus la distance entre l'époque courante et l'époque cible est grande, plus cette réduction est significative.
    \item Récupère et réutilise les identifiants originaux depuis les opérations.
      Permet d'éviter la situation où un noeud applique \textsc{revertRenameId} puis \textsc{renameId} à un identifiant donné.
      Retire donc la contrainte de proposer un couple de fonctions réciproques \textsc{renameId} et \textsc{revertRenameId} ou d'avoir une relation \emph{priority} qui prémunit des aller-retours dans l'\emph{arbre des époques} pour garantir la correction du système.
      Offre donc une plus grande flexibilité pour améliorer ces autres points du mécanisme de renommage.
  \end{itemize}
  \item Mais dispose aussi de quelques limites
  \begin{itemize}
    \item Besoin de maintenir le log des opérations.
      Ce défaut est néanmoins à relativiser.
      Les noeuds ont déjà besoin de maintenir le log des opérations de façon à se synchroniser avec les autres noeuds, que ça soit pour pallier à une perte d'un message ou pour permettre à un noeud qui rejoint la collaboration de se mettre à jour.
    \item Besoin de parcourir le log des opérations à la recherche d'opérations concurrentes.
      Là aussi, ce défaut est à relativiser.
      Afin de supprimer les métadonnées du mécanisme de renommage, les noeuds traquent déjà les informations de progression des autres noeuds.
      Ces informations permettent de déterminer la stabilité causale des opérations et peuvent être utilisées pour tronquer au fur et à mesure le log et ainsi limiter sa taille.
      Le parcours du log peut aussi être facilement parallélisé.
      \mnnote{
        Est-ce que log peut être plus petit qu'état courant ?
        Oui, si noeuds tous connectés et que système se stabilise.
        Est-ce probable ?
        Pas convaincu.
      }
    \item Besoin d'ajouter les informations de causalité pour permettre de détecter les opérations concurrentes.
      Pour cela, il suffit d'inclure un vecteur de version dans l'opération \emph{rename}.
      Ainsi, puisque le log ne peut contenir que des opérations précédant l'opération \emph{rename} ou concurrente à cette dernière (d'après la livraison basée sur les époques), il s'ensuit que toute opération présente dans le log et n'étant pas couverte par le vecteur de version est une opération concurrente à l'opération \emph{rename}.
      Cependant, la taille des vecteurs de version croît de façon monotone avec le nombre de noeuds qui participent à la collaboration.
      Diffuser cette structure de données à l'ensemble des noeuds peut donc avoir un coût significatif dans les collaborations à large échelle.
      Néanmoins, il faut rappeler que les noeuds échangent déjà régulièrement des vecteurs de version dans le cadre du fonctionnement du mécanisme d'anti-entropie.
      Les opérations \emph{rename} étant rares en comparaison, ce surcoût nous paraît acceptable.
  \end{itemize}
\end{itemize}

\subsection{Report de la transition vers la nouvelle epoch principale}

Comme illustré par \autoref{tab:evolution-integration-time-rename}, intégrer des opérations \emph{rename} distantes est généralement coûteux.
Ce traitement peut générer un surcoût computationnel significatif en cas de multiples opérations \emph{rename} concurrentes.
En particulier, un noeud peut recevoir et intégrer les opérations \emph{rename} concurrentes dans l'ordre inverse défini par la relation \emph{priority} sur leur époques.
Dans ce scénario, le noeud considérerait chaque nouvelle époque introduite comme la nouvelle époque cible et renommerait son état en conséquence à chaque fois.

\mnnote{TODO: Ajouter figure où noeud reçoit successivement plusieurs opérations \emph{rename} concurrentes et procéde au renommage de son état à chaque fois}

En cas d'un grand nombre d'opérations \emph{rename} concurrentes, nous proposons que les noeuds délaient le renommage de leur état vers l'époque cible jusqu'à ce qu'ils aient obtenu un niveau de confiance donné en l'époque cible.
Ce délai réduit la probabilité que les noeuds n'effectuent des traitements inutiles.
Plusieurs stratégies peuvent être proposées pour calculer le niveau de confiance en l'époque cible.
Ces stratégies peuvent reposer sur une variété de métriques pour produire le niveau de confiance, tel que le temps écoulé depuis que le noeud a reçu une opération \emph{rename} concurrente et le nombre de noeuds en ligne qui n'ont pas encore reçu l'opération \emph{rename}.

Durant cette période d'incertitude introduite par le report, les noeuds peuvent recevoir des opérations provenant d'époques différentes, notamment de l'époque cible.
Néanmoins, les noeuds peuvent toujours intégrer les opérations \emph{insert} et \emph{remove} en utilisant \textsc{renameId} et \textsc{revertRenameId} au prix d'un surcoût computationnel pour chaque identifiant.
Cependant, ce coût est négligeable (plusieurs centaines de microsecondes par identifiant d'après \autoref{fig:evolution-integration-time-remote-insert}) comparé au coût de renommer, de manière inutile, complètement l'état (plusieurs centaines de milliseconds à des secondes complètes d'après \autoref{tab:evolution-integration-time-rename}).

Notons que ce mécanisme nécessite que \textsc{renameId} et \textsc{revertRenameId} soient des fonctions réciproques.
En effet, au cours de la période d'incertitude, un noeud peut avoir à utiliser \textsc{revertRenameId} pour intégrer les identifiants d'opérations \emph{insert} distantes provenant de l'époque cible.
Ensuite, le noeud peut devoir renommer son état vers l'époque cible une fois que celle-ci a obtenu le niveau de confiance requis.
Il s'ensuit que \textsc{renameId} doit restaurer les identifiants précédemment transformés par \textsc{revertRenameId} à leur valeur initiale pour garantir la convergence.

\section{Comparaison avec les approches existantes}

\subsection{Core-Nebula}
\subsection{LSEQ}

\mnnote{Serait intéressant d'avoir une implémentation combinant LogootSplit et LSEQ pour vérifier si les contraintes sur la création de blocs dans LogootSplit ne "sabotent" pas la croissance polylogarithmique des identifiants de LSEQ}

\subsection{Eager stability determination}

\mnnote{Peut aussi aborder les travaux de Jim Bauwens et Elisa Gonzalez Boix \cite{10.1145/3358504.3361231, 10.1145/3380787.3393685, 10.1145/3426182.3426183} sur l'accélération de la stabilité causale : ne concerne pas seulement les séquences, mais les operation-based CRDTs. Permet de tronquer le log des opérations mais aussi d'accélerer le mécanisme de GC de RGA (et le mien aussi)}

\mnnote{Peut aborder les travaux de Weidner, Miller et Meiklejohn \cite{10.1145/3380787.3393687, 10.1145/3408976} qui combinent aussi CRDT et OT dans une certaine mesure. Pas vraiment dans le but de réduire les métadonnées du CRDT. Mais reste intéressant à présenter pour se différencier (eux proposent d'utiliser OT pour fusionner 2 CRDTs, moi pour ajouter une action qui est incompatible nativement avec les autres actions du CRDT)}

\section{Conclusion}

\subsection{Récapitulatif}

\begin{itemize}
  \item Propose un mécanisme de renommage optimiste adapté à LogootSplit
  \item Réduit ponctuellement taille des métadonnées grâce à une nouvelle opération, l'opération \emph{rename}
  \item Réduit métadonnées à un montant fixe à terme (1 seul bloc)
  \item Ajoute un surcoût computationnel pour les opérations concurrentes aux opérations \emph{rename}
  \item Mais surcoût compensé par l'optimisation de la structure de données utilisée en interne pour représenter la séquence (réduit taille de l'arbre)
  \item Et permet de réduire le temps d'intégration des opérations dans le cas général
\end{itemize}

\subsection{Limites}

\begin{itemize}
  \item Une limite est la nécessité de stabilité causale pour pouvoir supprimer de manière définitive les métadonnées liées au mécanisme de renommage
  \item Mais s'agit d'une limite commune avec les autres mécanismes de réduction des métadonnées, à l'exception de LSEQ (GC des tombstones dans RGA, catchup protocole de core et nebula (à re-vérifier, mais je vois pas comment ils feraient autrement))
  \item Cette limite pose problème dans les systèmes dynamiques puisque nous possédons aucune garantie qu'un noeud va se reconnecter un jour
  \item S'agit d'un problème encore non-résolu
  \item Peut mettre en place un mécanisme pour évincer les noeuds inactifs depuis trop longtemps
  \item Mais mène dans ce cas à des forks
  \item Une second limite est le surcoût produit par les opérations \emph{rename} concurrentes, aussi bien en termes de métadonnées que de calculs
  \item Peut être adressée en mettant en place une architecture à la core et nebula dans les systèmes qui le permettent
  \item Sinon peut mettre en place des stratégies pour limiter le risque de d'opérations \emph{rename} concurrentes (coordination simple au sein de sous-groupe), mais n'offrent aucune garantie forte sur le nombre d'opérations concurrentes possibles
\end{itemize}

\subsection{Perspectives}

\begin{itemize}
  \item Définir stratégie pour déclenchement de l'opération \emph{rename}
  \item Proposer nouvelles relations \emph{priority} \lepoch
  \item Utiliser combinaison de CRDT et OT pour concevoir structures de données répliquées complexes (move dans Sequence, DFS...)
\end{itemize}

% \include{assets/chapter_application_HAL}

\NumberThisInToc
\chapter{MUTE, un éditeur web collaboratif P2P temps réel}
\minitoc
\section{Présentation}

\begin{itemize}
  \item Plateforme d'expérimentation et de démonstration de l'équipe
\end{itemize}

\subsection{Objectifs}

\begin{itemize}
  \item Éditeur collaboratif
  \item Permettre collaboration synchrone (temps réel) et asynchrone (mode offline)
  \item À grande échelle
  \item Respecter privacy, limiter au maximum la confiance qu'on demande aux utilisateurs d'avoir dans l'outil
  \item Facile d'accès
  \item Facilement déployable par des tiers
  \item S'inscrit dans la mouvance Local-First Software \cite{localfirstsoftware2019,pushpin2020}
\end{itemize}

\subsection{Architecture}

\begin{itemize}
  \item Pour répondre à ces besoins, a effectué les choix suivants
  \item Application web
  \item Utilise CRDT pour représenter le document partagé
  \item Nous permet de supporter les différents modes de collaboration
  \item Nous permet aussi d'adopter une architecture P2P garantissant la privacy et le passage à l'échelle
  \item Mais présence de plusieurs serveurs, aux responsabilités limitées, pour simplifier la collaboration (signaling server, pulsar, bots(?))
  \item \mnnote{TODO: Insérer schéma de l'architecture d'une collaboration (noeuds, types de noeuds et lien)}
  \item L'architecture d'un pair se décompose en plusieurs couches
  \item \mnnote{TODO: Insérer schéma de l'architecture logicielle d'un pair}
\end{itemize}

\section{Couche interface}

\begin{itemize}
  \item Éditeur Markdown
  \item Permet d'incorporer le style des éléments directement dans la séquence représentant le document texte
  \item Mécanisme de conscience de groupe
  \begin{itemize}
    \item Liste des collaborateurs
    \item Curseurs et sélections des autres collaborateurs
    \item Indicateur de connexion
  \end{itemize}
  \item Stocke au sein du navigateur les données du document (état du document, log des opérations...)
  \item Glue le reste des couches ensemble
\end{itemize}

\section{Couche réplication}

\subsection{Modèle de données du document texte}

MUTE propose plusieurs alternatives pour représenter le document texte.
MUTE permet de soit utiliser une implémentation de LogootSplit\footnote{Les deux implémentations proviennent de la librairie \texttt{mute-structs} : \url{https://github.com/coast-team/mute-structs}}, soit de RenamableLogootSplit\footnotemark[\value{footnote}] ou soit de Dotted LogootSplit \footnote{Implémentation fournie par la librairie suivante : \url{https://github.com/coast-team/dotted-logootsplit}}.
Ce choix est effectué via une valeur de configuration de l'application choisie au moment de son déploiement.

Le modèle de données utilisé interagit avec l'éditeur de texte par l'intermédiaire de d'opérations textes.
Lorsque l'utilisateur effectue des modifications locales, celles-ci sont détectées et mises sous la forme d'opérations textes.
Elles sont transmises au modèle de données, qui les intègre alors à la structure de données répliquées.
Le \ac{CRDT} retourne en résultat l'opération distante à propager aux autres noeuds.

De manière complémentaire, lorsqu'une opération distante est délivrée au modèle de données, elle est intégrée par le \ac{CRDT} pour actualiser son état.
Le \ac{CRDT} génère les opérations textes correspondantes et les transmet à l'éditeur de texte pour mettre à jour la vue.

\subsection{Module de livraison des opérations}

\begin{itemize}
  \item Associée au composant gérant la livraison des opérations
  \item \mnnote{TODO: Expliquer comment assure la livraison Causal Remove}
  \item \mnnote{TODO: Expliquer comment assure la livraison Epoch-based}
  \item Mécanisme d'anti-entropie \cite{1983-anti-entropy-vv}
\end{itemize}

Dans le cadre de LogootSplit et de RenamableLogootSplit, le modèle de données utilisé pour représenter le document texte est couplé au composant \texttt{Sync}.
Le rôle de ce composant est d'assurer le respect du modèle de livraison des opérations au \ac{CRDT}.
Pour cela, le module \texttt{Sync} doit implémenter les contraintes présentées dans la \autoref{sec:logootsplit-delivery-model} et la \autoref{sec:renamablelogootsplit-delivery-model}.

\subsubsection{Livraison des opérations en exactement un exemplaire}

\begin{itemize}
  \item Afin de respecter l'exactly-once delivery, doit identifier de manière unique chaque opération
  \item Pour cela, ajoute un dot à chaque opération
  \item dot est formé de l'identifiant du pair et d'un numéro séquentiel
  \begin{itemize}
    \item numéro séquentiel différent de celui-ci utilisé par le CRDT, puisque celui doit augmenter avec chaque opération
  \end{itemize}
  \item doit alors maintenir une structure de données représentant l'ensemble des opérations reçues par le pair
  \item à la réception d'une opération, vérifie si son dot est présent dans cette structure
  \item si absent, peut délivrer l'opération
  \item sinon, opération déjà délivrée précédemment, peut l'ignorer sans risque
  \item pour maintenir l'ensemble des opérations reçues, plusieurs structures de données adaptées
  \item dans le cadre de MUTE, avons fait le choix d'utiliser un version vector
  \item nous permet de réduire à un dot par pair le surcoût en métadonnées du mécanisme
  \begin{itemize}
    \item maintient seulement le dot le plus récent
  \end{itemize}
  \item à la réception d'une opération, vérifie qu'il s'agit de la prochaine opération attendue, \ie que son le numéro séquentiel de son dot est le successeur du dernier dot enregistré pour ce pair
  \item si c'est le cas, délivre l'opération et met à jour l'entrée du pair dans le version vector avec ce nouveau numéro séquentiel
  \item si le dot est un dot futur, met en attente l'opération en attendant d'avoir reçu et délivré les opérations manquantes de ce pair
  \item sinon, si dot est déjà présent dans le version vector, ignore l'opération
  \item ce fonctionnement se traduit dans les faits par une livraison FIFO des opérations par noeud
  \item ajoute une contrainte non-nécessaire qui peut introduire des délais dans livraison des opérations si perd juste une opération d'un noeud
  \item nous paraît un compromis acceptable entre le surcoût du mécanisme de livraison et son impact sur l'expérience utilisateur
  \item une extension possible serait de remplacer cette structure par \mnnote{TODO: Retrouver nom de la structure couplant à un version vector un ensemble de dots}
  \item permettrait de supprimer la contrainte de livraison FIFO tout en permettant de compacter efficacement la représentation des opérations reçues par le noeud à terme
\end{itemize}

\subsection{Métadonnées}

\begin{itemize}
  \item Titre (Simple LWW-Register)
  \item Mode de chiffrement (fixe)
\end{itemize}

\subsection{Collaborateurs}

\begin{itemize}
  \item Implémente Swim \cite{swim2002}
  \item Découple protocole de détection des failures du protocole de diffusion de l'évolution du groupe
  \item Protocole de détection des failures
    \begin{itemize}
      \item Basé sur un système de rounds, basés sur un interval de temps
      \item À chaque round, chaque pair probe de manière aléatoire un autre pair
      \item Si pas de réponse, demande à un autre pair de le contacter
      \item Si pas de réponse par leur intermédiaire, pair devient suspect
      \item Si toujours pas de nouvelles du pair après un certain temps, le considère déconnecté
    \end{itemize}
  \item Protocole de diffusion de l'évolution du groupe
    \begin{itemize}
      \item Plutôt que de diffuser chaque évolution du groupe, adopte un modèle de diffusion épidémique
      \item Piggyback les évolutions aux messages du protocole de détection des failures
    \end{itemize}
  \item Modifie le fonctionnement du protocole pour en faire un CRDT
  \item Afin de permettre un nouveau pair de récupérer instantanément état courant du groupe
  \item Autorise aussi un pair à se déconnecter puis reconnecter en modifiant l'ordre de priorité entre les différents messages
  \begin{itemize}
    \item Dans protocole original, un pair déconnecté doit revenir sous une nouvelle identité
    \item Afin de maintenir l'identifiant du pair, notamment pour ses opérations sur le document
  \end{itemize}
\end{itemize}

\subsection{Curseurs}

\begin{itemize}
  \item Repose sur des identifiants pour indiquer la position des curseurs
  \item Vecteur de LWW-Registers, chaque LWW-Register étant associé à un pair actuellement connecté
    \mnnote{NOTE: C'est vrai ça ? J'ai un doute sur le fait qu'on avait mis en place un CRDT pour cette structure. À vérifier.}
\end{itemize}

\section{Couche sécurité}

\begin{itemize}
  \item Propose deux modes de fonctionnement
  \item Mode secret partagé
  \item Mode clé de chiffrement de groupe
\end{itemize}

\subsection{Mode secret partagé}

\subsubsection{Avantages et limites}

\begin{itemize}
  \item Peu coûteux
  \item Simple
  \item Ne permet pas de respecter la forward/backward secrecy
\end{itemize}

\subsection{Mode clé de chiffrement de groupe}

\subsubsection{Authenticité des clés publiques des participants}
\begin{itemize}
  \item Trusternity \cite{2018-trusternity-short, 2018-trusternity-long}
\end{itemize}

\subsubsection{Établissement de la clé de chiffrement de groupe}

\begin{itemize}
  \item n-party Diffie-Hellman \cite{1995-diffie-hellman}
\end{itemize}

\subsubsection{Avantages et limites}

\begin{itemize}
  \item Forward/backward secrecy
  \item Mais besoin de regénérer une nouvelle clé de groupe à chaque modification du groupe
  \item Nécessite pour cela la participation de l'ensemble des membres du groupe
  \item Peut être fait de manière asynchrone (\mnnote{TODO: Vérifier cette affirmation})
  \item Mais quid de la collaboration pendant ce temps ?
  \item Adapté aux groupes contrôlés
\end{itemize}

\section{Couche réseau}
\subsection{Netflux}
\begin{itemize}
  \item Réseau P2P
  \item Interface uniforme permettant d'interagir à la fois avec des navigateurs et des bots
  \item Connectent les noeuds en utilisant la technologie WebRTC
  \item Connectent les bots en utilisant la technologie WebSocket
  \item Repose sur l'utilisation d'un signaling server pour permettre aux pairs de rejoindre la collaboration
  \item Topologie maillée
\end{itemize}
\subsection{Pulsar}
\begin{itemize}
  \item Log-based message broker
  \item Propose plusieurs modes de fonctionnement
  \item En mode log-based message broker, maintient l'ensemble des messages reçus dans un log
  \item Permet, lorsque utilisé pour diffuser les opérations, de conserver le log complet des opérations
  \item Permet alors à un nouveau noeud de récupérer l'ensemble des opérations connues et de reconstruire l'état courant du document, même si actuellement aucun autre pair n'est connecté
  \item En mode message broker, diffuse seulement les messages aux noeuds actuellement connectés au topic
  \item Permet de communiquer les messages transients (protocole d'établissement de la clé de groupe, heartbeat de Swim, mécanisme d'anti-entropie du document) sans polluer le log
  \item Pose néanmoins des questions de sécurité et d'utilisabilité
  \item Besoin de chiffrer E2E les opérations
  \item Dans ce cas
    \begin{itemize}
      \item comment un nouveau pair peut obtenir la clé de chiffrement si les autres pairs ne sont pas connectés ? (à moins de revenir à un chiffrement à base de passphrase, avec les problèmes qui en découlent)
      \item comment un nouveau pair peut relire les opérations chiffrées avec l'ancienne clé ?
    \end{itemize}
\end{itemize}

\section{Pistes d'amélioration et de recherche}

\subsection{Composition de CRDTs}

\begin{itemize}
  \item
\end{itemize}

\subsection{CRDT pour les styles}

\begin{itemize}
  \item Permettrait de se découpler de Markdown pour gérer le style du document
  \item Permettrait de supporter un plus grand nombre d'options que Markdown ne le permet actuellement (couleurs, mise en page...)
\end{itemize}

\subsection{Réseaux}
\begin{itemize}
  \item Librairies alternatives (libP2P, hypercore)
  \item Topologies alternatives (SPRAY)
\end{itemize}
\subsection{Évolution de schéma}
\begin{itemize}
  \item Cambria \cite{2021-cambria-schema-evolution}
\end{itemize}
\subsection{Droits d'accès}
\begin{itemize}
  \item Access Control Conflict Resolution in Distributed File Systems \cite{2021-access-control-crdts}
  \item Travaux de PA
\end{itemize}
\subsection{Historique du document}
\subsection{Rôles et places des bots dans systèmes collaboratifs}
\begin{itemize}
  \item Stockage du document pour améliorer sa disponibilité
  \item Overleaf en P2P ?
  \item Comment réinsérer des bots dans la collaboration sans en faire des éléments centraux, sans créer des failles de confidentialité, et tout en rendant ces fonctionnalités accessibles ?
\end{itemize}

\NumberThisInToc
\chapter{Conclusions et perspectives}
\minitoc
\section{Résumé des contributions}
\section{Perspectives}
\subsection{Définition de relations de priorité pour minimiser les traitements}
\subsection{Redéfinition de la sémantique du renommage en déplacement d'éléments}
\subsection{Définition de types de données répliquées sans conflits plus complexes}
% Dans ce chapitre, nous avons présenté \acf{MUTE}, l'éditeur collaboratif temps réel \ac{P2P} chiffré de bout en bout développé par notre équipe de recherche.\\

MUTE permet d'éditer de manière collaborative des documents texte.
Pour représenter les documents, MUTE propose plusieurs \acp{CRDT} pour le type Séquence \cite{2013-logootsplit,2021-these-vic,2022-rls-tpds-nicolas} issus des travaux de l'équipe.
Ces \acp{CRDT} offrent de nouvelles méthodes de collaborer, notamment en permettant de collaborer de manière synchrone ou asynchrone de manière transparente.\\

Pour permettre aux noeuds de communiquer, MUTE repose sur la technologie WebRTC.
Cette technologie permet de construire un réseau \ac{P2P} directement entre plusieurs navigateurs.
Plusieurs serveurs sont néanmoins requis, notamment pour la découverte des pairs et pour la communication entre des noeuds lorsque leur pare-feux respectifs empêchent l'établissement d'une connexion directe.\\

Finalement, MUTE implémente un mécanisme de chiffrement de bout en bout garantissant l'authenticité et la confidentialité des échanges entre les noeuds.
Ce mécanisme repose sur une clé de groupe de chiffrement qui est établie à l'aide du protocole \cite{1995-burmester-desmedt}.

Ce protocole nécessite que chaque noeud possède une paire de clés de chiffrement et qu'ils partagent leur clé publique.
\ac{MUTE} repose sur des \acp{PKI} pour cela.
Avant de détecter tout éventuel comportement malicieux de la part de ces derniers, \ac{MUTE} intègre un mécanisme d'audit \cite{2018-trusternity-short,2018-trusternity-long}.\\


\Annex{Algorithmes \textsc{renameId}}
% \include{assets/annex_extension}

\label{app:rename-id}

\begin{figure}[!ht]
  \footnotesize
  \begin{algorithmic}
      \Function{renIdLessThanFirstId}{id, newFirstId}
      \If{id < newFirstId}
          \State \Return id
      \Else
          \State pos $\gets$ position(newFirstId)
          \State nId $\gets$ nodeId(newFirstId)
          \State nSeq $\gets$ nodeSeq(newFirstId)
          \State predNewFirstId $\gets$ \new~Id(pos, nId, nSeq, -1)
          \\
          \State \Return concat(predNewFirstId, id)
          \EndIf
      \EndFunction
      \\
      \Function{renIdGreaterThanLastId}{id, newLastId}
          \If{id < newLastId}
              \State \Return concat(newLastId, id)
          \Else
              \State \Return id
          \EndIf
      \EndFunction
  \end{algorithmic}
  \caption{Remaining functions to rename an identifier}
  \label{alg:appendix-rename-id}
\end{figure}

\Annex{Algorithmes \textsc{revertRenameId}}

\label{app:revert-rename-id}

\begin{figure}[!ht]
  \footnotesize
  \begin{algorithmic}
      \Function{revRenIdLessThanNewFirstId}{id, firstId, newFirstId}
          \State predNewFirstId $\gets$ createIdFromBase(newFirstId, -1)
          \If{isPrefix(predNewFirstId, id)}
              \State tail $\gets$ getTail(id, 1)
              \If{tail < firstId}
                  \State \Return tail
              \Else
                  \State \Comment{$id$ has been inserted causally after the \emph{rename} op}
                  \State offset $\gets$ getLastOffset(firstId)
                  \State predFirstId $\gets$ createIdFromBase(firstId, offset)
                  \State \Return concat(predFirstId, MAX\_TUPLE, tail)
              \EndIf
          \Else
              \State \Return id
          \EndIf
      \EndFunction
      \\
      \Function{revRenIdGreaterThanNewLastId}{id, lastId}
          \If{id < lastId}
              \State \Comment{$id$ has been inserted causally after the \emph{rename} op}
              \State \Return concat(lastId, MIN\_TUPLE, id)
          \ElsIf{isPrefix(newLastId, id)}
              \State tail $\gets$ getTail(id, 1)
              \If{tail < lastId}
                  \State \Comment{$id$ has been inserted causally after the \emph{rename} op}
                  \State \Return concat(lastId, MIN\_TUPLE, tail)
              \ElsIf{tail < newLastId}
                  \State \Return tail
              \Else
                  \State \Comment{$id$ has been inserted causally after the \emph{rename} op}
                  \State \Return id
              \EndIf
          \Else
              \State \Return id
          \EndIf
      \EndFunction
  \end{algorithmic}
  \caption{Remaining functions to revert an identifier renaming}
  \label{alg:appendix-revert-rename-id}
\end{figure}

%
%%-------------------------------------------------------------------
%%                         Le glossaire
%%-------------------------------------------------------------------
%\BeginGloWith{Voici un glossaire tout-à-fait fictif,
%              introduit par un texte sur toute la largeur
%              des deux colonnes.}
%\twocolumn
%\PrintGlossary

%-------------------------------------------------------------------
%              L'index (toujours sur deux colonnes)
%-------------------------------------------------------------------
\BeginIndWith{Voici un index}
\PrintIndex

\onecolumn

%-------------------------------------------------------------------
%                       La bibliographie
%-------------------------------------------------------------------

% La bibliographie (comme d'habitude)

%\nocite{*}
%\bibliographystyle{named}

\printbibliography

%-------------------------------------------------------------------
%                          Les résumés
%-------------------------------------------------------------------
% (si le résumé apparaît sur une colonne étroite, avec la
% bibliographie à gauche, c'est sans doute parce que vous avez
% oublié de générer les fichiers d'index et de glossaire...)

\NumberAbstractPages
\begin{ThesisAbstract}
  \begin{FrenchAbstract}
    Afin d'assurer leur haute disponibilité, les systèmes distribués à large échelle se doivent de répliquer leurs données tout en minimisant les coordinations nécessaires entre noeuds.
    Pour concevoir de tels systèmes, la littérature et l'industrie adoptent de plus en plus l'utilisation de types de données répliquées sans conflits (CRDTs).
    Les CRDTs sont des types de données qui offrent des comportements similaires aux types existants, tel l'Ensemble ou la Séquence.
    Ils se distinguent cependant des types traditionnels par leur spécification, qui supporte nativement les modifications concurrentes.
    À cette fin, les CRDTs incorporent un mécanisme de résolution de conflits au sein de leur spécification.

    Afin de résoudre les conflits de manière déterministe, les CRDTs associent généralement des identifiants aux éléments stockés au sein de la structure de données.
    Les identifiants doivent respecter un ensemble de contraintes en fonction du CRDT, telles que l'unicité ou l'appartenance à un ordre dense.
    Ces contraintes empêchent de borner la taille des identifiants.
    La taille des identifiants utilisés croît alors continuellement avec le nombre de modifications effectuées, aggravant le surcoût lié à l'utilisation des CRDTs par rapport aux structures de données traditionnelles.
    Le but de cette thèse est de proposer des solutions pour pallier ce problème.

    Nous présentons dans cette thèse deux contributions visant à répondre à ce problème :
    \begin{enumerate*}[label=(\roman*)]
      \item Un nouveau CRDT pour Séquence, RenamableLogootSplit, qui intègre un mécanisme de renommage à sa spécification.
      Ce mécanisme de renommage permet aux noeuds du système de réattribuer des identifiants de taille minimale aux éléments de la séquence.
      Cependant, cette première version requiert une coordination entre les noeuds pour effectuer un renommage.
      L'évaluation expérimentale montre que le mécanisme de renommage permet de réinitialiser à chaque renommage le surcoût lié à l'utilisation du CRDT.
      \item Une seconde version de RenamableLogootSplit conçue pour une utilisation dans un système distribué.
      Cette nouvelle version permet aux noeuds de déclencher un renommage sans coordination préalable.
      L'évaluation expérimentale montre que cette nouvelle version présente un surcoût temporaire en cas de renommages concurrents, mais que ce surcoût est à terme.
    \end{enumerate*}
    \KeyWords{CRDTs, édition collaborative en temps réel, cohérence à terme, optimisation mémoire, performance}
  \end{FrenchAbstract}
  \begin{EnglishAbstract}
    \KeyWords{CRDTs, real-time collaborative editing, eventual consistency, memory-wise optimisation, performance}
  \end{EnglishAbstract}
\end{ThesisAbstract}


\end{document}



