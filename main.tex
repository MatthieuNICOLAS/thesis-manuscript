\documentclass[12pt]{thesul}
%----------------------------------------------------------------------
%                               Packages
%----------------------------------------------------------------------
\usepackage[french]{babel}
\usepackage{acronym} % \ac[p], \acl[p], \acs[p], \acf[p]
\usepackage[maxbibnames=99,maxcitenames=1,sorting=none]{biblatex}
\bibliography{biblio.bib}
\usepackage{booktabs} % \toprule, \midrule, \cmidrule, \bottomrule
\usepackage{cancel} % \cancel
\usepackage{caption}
\usepackage{csquotes}
\usepackage{hyperref}
\hypersetup{hidelinks}
\usepackage[inline]{enumitem}
\setlist[enumerate]{label=(\roman*)} %% <- set the base level label separately
\usepackage{import} % \import
\usepackage[french]{minitoc}

\usepackage{xcolor}
\AtBeginDocument{
% white
\definecolor{uclwhite}{rgb}{1, 1, 1}

%   UCL style guide colours.
%   Number refer to level of tint.
%   100%    1
%    70%    2
%    50%    3
%    20%    4

 % UCL style guide dk purple
\definecolor{ucl1dkpurple}{RGB}{82,66,91}
\definecolor{ucl2dkpurple}{RGB}{134,122,140}
\definecolor{ucl3dkpurple}{RGB}{168,160,173}
\definecolor{ucl4dkpurple}{RGB}{220,217,222}

 % UCL style guide dk red
\definecolor{ucl1dkred}{RGB}{90,27,49}
\definecolor{ucl2dkred}{RGB}{139,95,110}
\definecolor{ucl3dkred}{RGB}{172,141,152}
\definecolor{ucl4dkred}{RGB}{222,209,214}

 % UCL style guide dk blue
\definecolor{ucl1dkblue}{RGB}{0,67,89}
\definecolor{ucl2dkblue}{RGB}{76,123,138}
\definecolor{ucl3dkblue}{RGB}{127,161,172}
\definecolor{ucl4dkblue}{RGB}{204,217,222}

 % UCL style guide dk green
\definecolor{ucl1dkgreen}{RGB}{75,70,32}
\definecolor{ucl2dkgreen}{RGB}{129,125,98}
\definecolor{ucl3dkgreen}{RGB}{165,162,143}
\definecolor{ucl4dkgreen}{RGB}{219,218,210}

 % UCL style guide black
\definecolor{ucl1black}{RGB}{0,0,0}
\definecolor{ucl2black}{RGB}{75,75,75}
\definecolor{ucl3black}{RGB}{128,128,128}
\definecolor{ucl4black}{RGB}{205,205,205}

 % UCL style guide pink
\definecolor{ucl1pink}{RGB}{145,24,83}
\definecolor{ucl2pink}{RGB}{178,93,134}
\definecolor{ucl3pink}{RGB}{200,139,169}
\definecolor{ucl4pink}{RGB}{233,209,221}

 % UCL style guide md red
\definecolor{ucl1mdred}{RGB}{195,58,45}
\definecolor{ucl2mdred}{RGB}{213,117,108}
\definecolor{ucl3mdred}{RGB}{225,156,150}
\definecolor{ucl4mdred}{RGB}{243,216,213}

 % UCL style guide md blue
\definecolor{ucl1mdblue}{RGB}{69,156,189}
\definecolor{ucl2mdblue}{RGB}{124,186,209}
\definecolor{ucl3mdblue}{RGB}{162,205,222}
\definecolor{ucl4mdblue}{RGB}{218,235,242}

 % UCL style guide md green
\definecolor{ucl1mdgreen}{RGB}{130,141,55}
\definecolor{ucl2mdgreen}{RGB}{167,175,115}
\definecolor{ucl3mdgreen}{RGB}{192,198,155}
\definecolor{ucl4mdgreen}{RGB}{230,232,215}

 % UCL style guide orange
\definecolor{ucl1orange}{RGB}{215,123,35}
\definecolor{ucl2orange}{RGB}{227,162,101}
\definecolor{ucl3orange}{RGB}{235,189,145}
\definecolor{ucl4orange}{RGB}{247,229,211}

  % UCL style guide lt purple
\definecolor{ucl1ltpurple}{RGB}{191,175,188}
\definecolor{ucl2ltpurple}{RGB}{210,199,208}
\definecolor{ucl3ltpurple}{RGB}{223,215,221}
\definecolor{ucl4ltpurple}{RGB}{242,239,242}

 % UCL style guide yellow
\definecolor{ucl1yellow}{RGB}{229,175,0}
\definecolor{ucl2yellow}{RGB}{237,199,76}
\definecolor{ucl3yellow}{RGB}{242,215,127}
\definecolor{ucl4yellow}{RGB}{250,239,204}

 % UCL style guide lt blue
\definecolor{ucl1ltblue}{RGB}{168,192,209}
\definecolor{ucl2ltblue}{RGB}{194,211,223}
\definecolor{ucl3ltblue}{RGB}{211,223,232}
\definecolor{ucl4ltblue}{RGB}{238,242,246}

% UCL style guide brt green
\definecolor{ucl1brtgreen}{RGB}{204,209,88}
\definecolor{ucl2brtgreen}{RGB}{219,223,138}
\definecolor{ucl3brtgreen}{RGB}{229,232,171}
\definecolor{ucl4brtgreen}{RGB}{245,246,222}

% UCL style guide stone
\definecolor{ucl1stone}{RGB}{217,214,204}
\definecolor{ucl2stone}{RGB}{228,226,219}
\definecolor{ucl3stone}{RGB}{236,234,229}
\definecolor{ucl4stone}{RGB}{255,255,255}

% UCL style guide lt green
\definecolor{ucl1ltgreen}{RGB}{185,193,147}
\definecolor{ucl2ltgreen}{RGB}{206,211,179}
\definecolor{ucl3ltgreen}{RGB}{220,224,201}
\definecolor{ucl4ltgreen}{RGB}{241,243,233}

\definecolor{darkgreen}{RGB}{75,70,32}
\definecolor{darkblue}{RGB}{0,67,89}
\definecolor{mydarkblue}{RGB}{76,123,138}
\definecolor{mydarkblueid}{RGB}{0,67,89}
\definecolor{mylightblue}{RGB}{168,192,209}
\definecolor{mydarkorange}{RGB}{215,123,35}
\definecolor{mylightorange}{RGB}{227,162,101}
\definecolor{mydarkred}{RGB}{90,27,49}
\definecolor{mydarkpurple}{RGB}{134,122,140}
\definecolor{mydarkpurpleid}{RGB}{82,66,91}
}

\usepackage{amssymb}
\usepackage{amsmath}
\usepackage{amsthm}
% \interdisplaylinepenalty=2500
\theoremstyle{definition}
\newtheorem{definition}{Définition}
\newtheorem{subdefinition}{Définition}[definition]
\newtheorem{myrule}{Règle}
\newtheorem{property}{Propriété}
\newtheorem{subproperty}{Propriété}[property]
\usepackage{MnSymbol} % \dashrightarrow

\usepackage{algorithm, algpseudocode}
\floatname{algorithm}{Algorithme} % Renomme caption de "Algorithm" -> "Algorithme"

\newcommand\CONDITION[2]%
  {\begin{tabular}[t]{@{}l@{}l@{}}
     #1&#2
   \end{tabular}%
  }
  \algdef{SE}[WHILE]{While}{EndWhile}[1]%
  {\algorithmicwhile\ \CONDITION{#1}{\ \algorithmicdo}}%
  {\algorithmicend\ \algorithmicwhile}
\algdef{SE}[FOR]{For}{EndFor}[1]%
  {\algorithmicfor\ \CONDITION{#1}{\ \algorithmicdo}}%
  {\algorithmicend\ \algorithmicfor}
\algdef{S}[FOR]{ForAll}[1]%
  {\algorithmicforall\ \CONDITION{#1}{\ \algorithmicdo}}
\algdef{SE}[REPEAT]{Repeat}{Until}{\algorithmicrepeat}[1]%
  {\algorithmicuntil\ \CONDITION{#1}{}}
\algdef{SE}[IF]{If}{EndIf}[1]%
  {\algorithmicif\ \CONDITION{#1}{\ \algorithmicthen}}%
  {\algorithmicend\ \algorithmicif}%
\algdef{C}[IF]{IF}{ElsIf}[1]%
  {\algorithmicelse\ \algorithmicif\ \CONDITION{#1}{\ \algorithmicthen}}

\usepackage[draft,inline,nomargin,index]{fixme}
\fxsetup{theme=color,mode=multiuser,inlineface=\itshape,envface=\itshape}
\FXRegisterAuthor{mn}{amn}{Matthieu}

\usepackage{tikz} % \begin{tikzpicture} \end{tikzpicture}
\usetikzlibrary{calc}
\usetikzlibrary{graphs}
\usetikzlibrary{quotes}
\usetikzlibrary{shapes.misc}

\usepackage[caption=false,font=footnotesize,labelfont=sf,textfont=sf]{subfig}

\usepackage{pifont} % \ding
\renewcommand{\checkmark}{\ding{51}}
\newcommand{\ballotx}{\ding{55}}

\usepackage{xspace} % \xspace

% Commands
%---------
\newcommand{\cf}[1]{(cf. \autoref{#1}, page \pageref{#1})}
\newcommand{\eg}{e.g.\xspace}
\newcommand{\ie}{c.-à-d.\xspace}

\newcommand{\hb}{\emph{happens-before}\xspace}

\newcommand{\inbb}[1]{\in \mathbb{#1}}
\newcommand{\new}{\textbf{new}}
\newcommand{\trm}[1]{\mathit{#1}}
\newcommand{\set}[1]{\left\{#1\right\}} % set brace notation

\newcommand{\betterid}[3]{\trm{#1}^{\trm{#2}}_{\trm{#3}}}
\newcommand{\id}[3]{$\trm{#1}^{\trm{#2}}_{\trm{#3}}$}
\newcommand{\epoch}[1]{$\varepsilon_{#1}$}
\newcommand{\lid}{<_{id}}
\newcommand{\leqid}{$\leq_{id}$~}
\newcommand{\lepoch}{$<_{\varepsilon}$~}
\newcommand{\leqepoch}{$\leq_{\varepsilon}$~}
\newcommand{\ltuple}{<_{t}}

\newcommand{\botn}{\bot_\mathbb{N}}
\newcommand{\topn}{\top_\mathbb{N}}
\newcommand{\logootuple}[1]{\langle \text{pos}_{#1},\text{nodeId}_{#1},\text{seq}_{#1} \rangle}

\newcommand{\bigO}[1]{$\mathcal{O}(#1)$}

\newcommand{\widthletter}{2em}
\newcommand{\widthblock}{3em}
\newcommand{\widthoriginepoch}{1.33em}
\newcommand{\widthepoch}{1.65em}

\newcommand{\withmathbreak}[1]{\parbox{\linewidth}{#1}}
\newcommand{\elt}{E}
\newcommand{\nat}{\mathbb{N}}


% Définit noms pour \autoref
%---------
\newcommand{\algorithmautorefname}{Algorithme}
\newcommand{\annexautorefname}{Annexe}
\newcommand{\definitionautorefname}{Définition}
\newcommand{\propertyautorefname}{Propriété}
\newcommand{\subfigureautorefname}{Figure}
\newcommand{\subpropertyautorefname}{Propriété}

% Tikz styles
\tikzset{
    common/.style={anchor=west, draw, rectangle, minimum height=6mm},
    letter/.style={common, minimum width=\widthletter},
    block/.style={common, minimum width=\widthblock},
    epoch/.style={letter, rounded rectangle, rounded rectangle east arc=0pt, minimum width=\widthepoch},
    point/.style={insert path={ node[scale=5*sqrt(\pgflinewidth)]{.} }},
    node/.style={draw, circle, minimum size=1em},
    op/.style={draw, circle, minimum size=2.7em},
    causalop/.style={op, double=white, inner sep=2pt},
    gc-rule-1/.style={dashed, thick, darkblue},
    gc-rule-2/.style={densely dotted, thick, darkgreen},
    cross/.style={
        path picture={
            \draw[ucl1dkred, very thick]
                (path picture bounding box.south east)--(path picture bounding box.north west)
                (path picture bounding box.south west)--(path picture bounding box.north east);
        }
    }
}


%-------------------------------------------------------------------
%                             Marges
%-------------------------------------------------------------------

% pour positionner les vraies marges:
%\SetRealMargins{1mm}{1mm}

%-------------------------------------------------------------------
%                             En-têtes
%-------------------------------------------------------------------

% Les en-têtes: quelques exemples
%\UppercaseHeadings
%\UnderlineHeadings
%\newcommand\bfheadings[1]{{\bf #1}}
%\FormatHeadingsWith{\bfheadings}
%\FormatHeadingsWith{\uppercase}
%\FormatHeadingsWith{\underline}
\newcommand\upun[1]{\uppercase{\underline{\underline{#1}}}}
\FormatHeadingsWith\upun

\newcommand\itheadings[1]{\textit{#1}}
\FormatHeadingsWith{\itheadings}

% pour avoir un trait sous l'en-tete:
\setlength{\HeadRuleWidth}{0.4pt}

%-------------------------------------------------------------------
%                         Les références
%-------------------------------------------------------------------

\NoChapterNumberInRef
\NoChapterPrefix

%-------------------------------------------------------------------
%                           Brouillons
%-------------------------------------------------------------------

% ceci ajoute une marque « brouillon » et la date
\ThesisDraft

%-------------------------------------------------------------------
%                   Pour collecter un glossaire et un index
%-------------------------------------------------------------------

\makeglossary
\makeindex

%-------------------------------------------------------------------
%                           Acronymes
%-------------------------------------------------------------------

% Acronyms
% --------
% \input{assets/acronyms.tex}
\acrodef{ADT}[ADT]{Abstract Data Type}
\acrodefplural{ADT}[ADTs]{Abstract Data Types}
\acrodef{API}[API]{Application Programming Interface}
\acrodef{AW}[AW]{\emph{Add-Wins}}
\acrodef{CCI}[CCI]{Convergence, Causality preservation, Intention preservation}
\acrodef{CL}[CL]{\emph{Causal-Length}}
\acrodef{CRDT}[CRDT]{Conflict-free Replicated Data Type}
\acrodefplural{CRDT}[CRDTs]{Conflict-free Replicated Data Types}
\acrodef{DAG}[DAG]{Directed Acyclic Graph}
\acrodef{FIFO}[FIFO]{First In, First Out}
\acrodef{GC}[GC]{Garbage Collection}
\acrodef{IPFS}[IPFS]{InterPlanetary File System}
\acrodef{JIT}[JIT]{Just-In-Time}
\acrodef{LCA}[PPAC]{Plus Petit Ancêtre Commun}
\acrodef{LFS}[LFS]{Local-First Software}
\acrodefplural{LFS}[LFS]{Local-First Softwares}
\acrodef{LUB}[LUB]{Least Upper Bound}
\acrodef{LWW}[LWW]{\emph{Last-Writer-Wins}}
\acrodef{MUTE}[MUTE]{Multi User Text Editor}
\acrodef{MV}[MV]{\emph{Multi-Value}}
\acrodef{OC}[OC]{Commutativité des Opérations}
\acrodef{OT}[OT]{Operational Transformation}
\acrodefplural{OT}[OT]{Operational Transformations}
\acrodef{P2P}[P2P]{Pair-à-Pair}
\acrodef{PKI}[PKI]{Public Key Infrastructure}
\acrodef{PoC}[PoC]{Proof of Concept}
\acrodef{PT}[PT]{Transitivité de la Précédence}
\acrodef{RADT}[RADT]{Replicated Abstract Data Type}
\acrodefplural{RADT}[RADTs]{Replicated Abstract Data Types}
\acrodef{RCB}[RCB]{Reliable Causal Broadcast}
\acrodef{RGA}[RGA]{Replicated Growable Array}
\acrodef{RW}[RW]{\emph{Remove-Wins}}
\acrodef{SEC}[SEC]{Cohérence forte à terme}
\acrodef{TTF}[TTF]{Tombstone Transformation Function}
\acrodefplural{TTF}[TTF]{Tombstone Transformation Functions}
\acrodef{WebRTC}[WebRTC]{Web Real-Time Communication}

%-------------------------------------------------------------------
%                           Couleurs
%-------------------------------------------------------------------

% \input{assets/colours.tex}

%-------------------------------------------------------------------
%                     Global custom tikz commands
%-------------------------------------------------------------------

% \input{assets/tikz_presets.tex}

\begin{document}


      \OddHead={{\leftmark\rightmark}{\hfil\slshape\rightmark}}
      \EvenHead={{\leftmark}{{\slshape\leftmark}\hfil}}
      \OddFoot={\hfil\thepage}
      \EvenFoot={\thepage\hfil}
      \pagestyle{ThesisHeadingsII}


%-------------------------------------------------------------------
%                          Encadrements
%-------------------------------------------------------------------

% encadre les chapitres dans la table des matières:
% (ces commandes doivent figurer apres \begin{document}

\FrameChaptersInToc
%\FramePartsInToc


%-------------------------------------------------------------------
%            Réinitialisation de la numérotation des chapitres
%-------------------------------------------------------------------

% Si la commande suivante est présente,
% elle doit figurer APRÈS \begin{document}
% et avant la première commande \part
\ResetChaptersAtParts

%-------------------------------------------------------------------
%               mini-tables des matières par chapitre
%-------------------------------------------------------------------

% préparer les mini-tables des matières par chapitre.
% (commande de minitoc.sty)
\dominitoc

%-------------------------------------------------------------------
%                         Page de titre:
%-------------------------------------------------------------------

\ThesisTitle{Ré-identification sans coordination dans les types de données répliquées sans conflits (CRDTs)}
\ThesisDate{TODO: Définir une date}
\ThesisAuthor{Matthieu Nicolas}

% Type de la these
\ThesisUL
% Jury:

% (ne pas mettre de \\ apres la dernière entree)

% Exemple de création d'une nouvelle catégorie dans le jury:

% \NewJuryCategory{Encadrants}{\it Encadrant :}
%                        {\it Encadrants :}

\def\blanc{\hspace*{1cm}}

\President    = {À déterminer}
\Rapporteurs  = {Hanifa Boucheneb & Professeure, Polytechnique Montréal\\
                 Davide Frey      & Chargé de recherche, HdR, Inria Rennes Bretagne-Atlantique}
\Examinateurs = {Hala Skaf-Molli  & Maîtresse de conférences, HdR, Nantes Université \\
                 Stephan Merz     & Directeur de Recherche, Inria Nancy - Grand Est}
\Encadrants= {Olivier Perrin      & Professeur des Universités, Université de Lorraine, LORIA \\
              Gérald Oster        & Maître de conférences, Université de Lorraine, LORIA}

%\Invites=       {}

% Création de la page de titre:
\MakeThesisTitlePage

%-------------------------------------------------------------------


%-------------------------------------------------------------------
%                          remerciements
%-------------------------------------------------------------------

%\DontFrameThisInToc
\begin{ThesisAcknowledgments}
WIP
\end{ThesisAcknowledgments}

%-------------------------------------------------------------------
%                            dédicace
%-------------------------------------------------------------------

\begin{ThesisDedication}
WIP
\end{ThesisDedication}


%-------------------------------------------------------------------
%                  écriture de `Chapitre' et `Partie'
%                      dans la table des matières
%-------------------------------------------------------------------

\WritePartLabelInToc
\WriteChapterLabelInToc

%-------------------------------------------------------------------
%                        table des matières
%-------------------------------------------------------------------

\tableofcontents

%-------------------------------------------------------------------
%              Exemple d'utilisation de \SpecialSection
%-------------------------------------------------------------------
%\SpecialSection{Introduction générale}

\DontWriteThisInToc
\listoffigures

\mainmatter
\NumberThisInToc
\chapter{Introduction}
\minitoc

\section{Contexte}
\label{sec:intro-contexte}

L'évolution des technologies du web a conduit à l'avènement de ce qui est communément appelé le Web 2.0.
La principale caractéristique de ce média est la possibilité aux utilisateur-rices non plus seulement de le consulter, mais aussi d'y contribuer.

Ces nouvelles fonctionnalités ont permis l'apparition d'applications incitant les utilisateur-rices à créer et partager leur propre contenu, ainsi que d'échanger avec d'autres utilisateur-rices à ce sujet.
Un cas particulier de ces applications proposent aux utilisateur-rices de travailler ensemble pour la création d'un même contenu, en d'autres termes de collaborer.
Nous appelons ces applications des \emph{systèmes collaboratifs} :
\begin{definition}[Système collaboratif]
  \label{def:collaborative-system}
  Un système collaboratif est un système supportant ses utilisateur-rices dans leurs processus de collaboration pour la réalisation de tâches.
\end{definition}

De nos jours, ces systèmes font parties des applications les plus populaires du paysage internet, \eg la suite logicielle dont fait partie GoogleDocs compte 2 milliards d'utilisateur-rices \cite{2020-google-g-suite-users}, Wikipedia 788 millions \cite{2022-09-monthly-active-users-wikipedia}, Quora 300 millions \cite{2022-01-monthly-active-users-social-networks} ou encore GitHub 60 millions \cite{2022-github-users}.
De leur côté, d'autres plateformes fédèrent leur communautés en organisant ponctuellement des collaborations éphémères et généralement massives, \eg r/Place \cite{2022-rplace} ou TwitchPlaysPokemon \cite{2014-twitch-plays-pokemon}.\\

En raison de leur popularité, les systèmes collaboratifs doivent assurer plusieurs propriétés pour garantir leur bon fonctionnement et qualité de service : une haute disponibilité, tolérance aux pannes et capacité de passage à l'échelle.
\begin{definition}[Disponibilité]
  \label{def:availability}
  La disponibilité d'un système indique sa capacité à répondre à tout moment à une requête d'un-e utilisateur-rice.
\end{definition}
\begin{definition}[Tolérance aux pannes]
  La tolérance aux pannes d'un système indique sa capacité à continuer à répondre aux requêtes malgré l'absence de réponse d'un ou plusieurs de ses composants.
\end{definition}
\begin{definition}[Capacité de passage à l'échelle]
  La capacité de passage à l'échelle d'un système indique sa capacité à traiter un volume toujours plus conséquent de requêtes.
\end{definition}

Pour cela, ces systèmes adoptent une architecture décentralisée, que nous illustrons par la \autoref{fig:decentralised-system}.
\begin{figure}[!ht]
  \centering
  \resizebox{0.3 \columnwidth}{!}{
    \includegraphics[trim=12cm 0cm 13cm 0cm, clip]{img/centralised-decentralised-distributed}
  }
  \caption[Caption for decentralised-system]{
    Représentation d'une architecture décentralisée \cite{1964-distributed-communications-networks-baran}.
    Les noeuds aux extrêmités du graphe correspondent à des clients, les noeuds internes à des serveurs et les arêtes du graphe représentent les connexions entre appareils.
  }
  \label{fig:decentralised-system}
\end{figure}

Dans ce type d'architecture, les responsabilités, tâches et la charge travail sont réparties entre un ensemble de serveurs.
Il convient toutefois de noter que les serveurs jouent de manière globale toujours un rôle central dans ces systèmes, malgré ce que le nom de cette architecture peut suggérer.
En effet, ces systèmes reposent toujours sur leurs serveurs pour authentifier les utilisateur-rices, stocker les données de leurs utilisateur-rices ou encore fusionner les modifications effectuées par ces dernier-es.\\
% La nuance porte seulement sur le fait que ce n'est plus un serveur unique qui est en charge de ces tâches, mais un ensemble de serveurs.

Bien que cette architecture système permette de répondre aux problèmes d'ordre technique que nous présentons précédemment, elle souffre néanmoins de limites.
Notamment, de part le rôle prédominant que jouent les serveurs dans les systèmes décentralisés, ces derniers échouent à assurer un second ensemble de propriétés que nous jugeons néanmoins fondamentales :
\begin{definition}[Confidentialité des données]
  \label{def:confidentialite}
  La confidentialité des données d'un système indique sa capacité à garantir à ses utilisateur-rices que leurs données ne seront pas accessibles par des tiers non autorisés ou par le système lui-même.
\end{definition}
\begin{definition}[Souveraineté des données]
  \label{def:souverainete}
  La souveraineté des données d'un système indique sa capacité à garantir à ses utilisateur-rices leur maîtrise de leurs données, \ie leur capacité à les consulter, modifier, partager, exporter; supprimer ou encore à décider de l'usage qui en est fait.
\end{definition}
\begin{definition}[Pérennité]
  \label{def:perennite}
  La pérennité d'un système indique sa capacité à garantir à ses utilisateur-rices son fonctionnement continu dans le temps.
\end{definition}
\begin{definition}[Résistance à la censure]
  \label{def:censorship}
  La résistance à la censure d'un système indique sa capacité à garantir à ses utilisateur-rices son fonctionnement malgré des actions de contrôle de l'information par des autorités.
\end{definition}

De plus, les serveurs ne sont pas une ressource libre.
En effet, ils sont déployés et maintenus par la ou les organisations qui proposent le système collaboratif.
Ces organisations font alors office d'\emph{autorités centrales} du système, \eg en se portant garantes de l'identité des utilisateur-rices, de l'authenticité d'un contenu ou encore de la disponibilité dudit contenu.

De part le fait que les autorités centrales possèdent les serveurs hébergeant le système, elles ont tout pouvoir sur ces derniers.
Ainsi, les utilisateur-rices de systèmes collaboratifs prennent, de manière consciente ou non, le risque que les propriétés présentées précédemment soient transgressées par les autorités auxquelles appartiennent ces applications ou par des tiers avec lesquelles ces autorités interagissent, \eg des gouvernements.
Plusieurs faits d'actualités nous ont malheureusement montré de tels faits, \eg la censure de Wikipedia par des gouvernements \cite{2022-wikipedia-censorship}, la fermeture de services par les entreprises les proposant \cite{2022-killed-by-google} ou encore la mise à disposition des données hébergées par des applications aux services de renseignement de différentes nations \cite{prism-guardian,prism-washington-post}.
Cependant, le coût conséquent de l'infrastructure nécessaire pour déployer des systèmes à large échelle équivalents entrave la mise en place d'alternatives, plus respectueuses de leurs utilisateur-rices.\\

\emph{Ainsi, il nous paraît fondamental de proposer des moyens technologiques rendant accessible la conception et le déploiement des systèmes collaboratifs alternatifs.
Ces derniers devraient minimiser le rôle des autorités centrales, voire l'éliminer, de façon à protéger et privilégier les intérêts de leurs utilisateur-rices.}\\

Dans cette optique, une piste de recherche que nous jugeons intéressante est celle des systèmes collaboratifs \acf{P2P}.
Cette architecture système, que nous illustrons par la \autoref{fig:distributed-system}, place les utilisateur-rices au centre du système et relègue les éventuels serveurs à un simple rôle de support de la collaboration, \eg la mise en relation des pairs.

\begin{figure}[!ht]
  \centering
  \resizebox{0.25 \columnwidth}{!}{
    \includegraphics[trim=26cm 0cm 1cm 0cm, clip]{img/centralised-decentralised-distributed}
  }
  \caption[Caption for distributed-system]{
    Représentation d'une architecture distribuée \cite{1964-distributed-communications-networks-baran}.
    Ici, tout noeud du graphe correspond à un pair du système \ac{P2P}.}
  \label{fig:distributed-system}
\end{figure}

Récemment, la conception de systèmes collaboratifs \ac{P2P} a gagné en traction suite à \cite{localfirstsoftware2019}.
Dans cet article, les auteurs définissent un ensemble de propriétés qui correspondent à celles que nous avons établies précédemment, de la \autoref{def:collaborative-system} à la \autoref{def:censorship}.
En utilisant ces propriétés comme critères, les auteurs comparent les fonctionnalités et garanties offertes par les différents types d'applications, notamment les applications lourdes et les applications basées sur le cloud.

Le résultat de cette comparaison est le suivant : alors que les applications basées sur le cloud permettent de nouveaux usages, notamment la collaboration entre utilisateur-rices ou la synchronisation automatique entre appareils, elles retirent à leurs utilisateur-rices toute garantie de pérennité, confidentialité des données et souveraineté des données.
Ces dernières propriétés sont pourtant communément offertes par les applications lourdes.
La \autoref{fig:lfs-comparison-apps} détaille ce résultat.

\begin{figure}[!ht]
  \centering
  \resizebox{\columnwidth}{!}{
    \includegraphics{img/lfs-comparison-apps}
  }
  \caption[Caption for lfs-comparison-apps]{
    Évaluation d'applications et de technologies vis-à-vis des 7 propriétés visées par les applications \aclp{LFS} \cite{localfirstsoftware2019}.
    {\color{lfsgreen} \checked}, {\color{lfsorange} \Flatsteel} et {\scriptsize \color{lfsred} \ding{108}} indiquent respectivement que l'application ou la technologie satisfait pleinement, partiellement ou aucunement le critère évalué.
  }
  \label{fig:lfs-comparison-apps}
\end{figure}

Malgré ce que ce résultat pourrait suggérer, les auteurs affirment que les nouveaux usages offerts par les applications basées sur le cloud ne sont pas antinomiques avec les propriétés de confidentialité, souveraineté, pérennité.

Ainsi, ils proposent un nouveau paradigme de conception d'applications collaboratives \ac{P2P}, nommées \acp{LFS}.
Ce paradigme vise à la conception d'applications offrant le meilleur des approches existantes, \ie des applications cochant l'intégralité des critères de la \autoref{fig:lfs-comparison-apps}.
Nous partageons cette vision.\\
% Ce type d'applications se démarque de ceux existants, \eg les applications basées sur le cloud, par la place centrale donnée aux utilisateur-rices et leurs propres appareils, les éventuels serveurs étant relegués qu'à de simples rôles de support.

Cependant, de nombreuses problématiques de recherche identifiées dans \cite{localfirstsoftware2019} sont encore non résolues et entravent la démocratisation des applications \acp{LFS}, notamment celles à large échelle.
Spécifiquement, les applications \acp{LFS} se doivent de répliquer les données entre les appareils pour permettre :
\begin{enumerate}
  \item Le fonctionnement en mode hors-ligne et le fonctionnement avec une faible latence.
  \item Le partage de contenu entre appareils d'un-e même utilisateur-rice.
  \item Le partage de contenu entre utilisateur-rices pour la collaboration.
\end{enumerate}

Toutefois, compte tenu des propriétés visées par les applications \acp{LFS}, plusieurs contraintes restreignent le choix des méthodes de réplication possibles.
Ainsi, pour permettre le fonctionnement en mode hors-ligne de l'application, \ie la consultation et la modification de contenu, les applications \acp{LFS} doivent relaxer la propriété de cohérence des données.
\begin{definition}[Cohérence]
  La cohérence d'un système indique sa capacité à présenter une vue uniforme de son état à chacun de ses utilisateur-rices à un moment donné.
\end{definition}

Les applications \acp{LFS} doivent donc adopter des méthodes de réplication dites optimistes \cite{2005-optimistic-replication-saito}.
Ces méthodes autorisent chaque noeud possédant une copie de la donnée à la consulter et à la modifier sans coordination au préalable avec les autres noeuds\footnote{Par opposition aux méthodes de réplication dites pessimistes, qui nécessitent une coordination préalable entre les noeuds avant toute modification de la donnée.}.
L'état des copies des noeuds peut donc diverger temporairement.
Un mécanisme de synchronisation permet ensuite aux noeuds de partager les modifications effectuées et de les intégrer de façon à converger à terme \cite{10.1145/224057.224070}, \ie obtenir à terme de nouveau des états équivalents.

Cependant, il convient de noter que les méthodes de réplication optimistes autorisent la génération en concurrence de modifications provoquant un conflit, \eg la modification et la suppression d'une même page dans un wiki.
Un mécanisme de résolution de conflits est alors nécessaire pour assurer la convergence à terme des noeuds.

De nouveau, le modèle du système des applications que nous visons, \ie des applications \acp{LFS} à large échelle, limitent les choix possibles concernant les mécanismes de résolution de conflits.
Notamment, ces applications ne disposent d'aucun contrôle sur le nombre de noeuds qui compose le système, \ie le nombre d'appareils utilisés par l'ensemble de leurs utilisateur-rices.
Ce nombre de noeuds peut donc croître de manière non-bornée.
Les mécanismes de résolution de conflits choisis devraient donc rester efficaces, de manière indépendante à l'évolution de ce paramètre.

De plus, les noeuds composant le système n'offrent aucune garantie sur leur stabilité.
Des noeuds peuvent donc rejoindre et participer au système, mais uniquement de manière éphèmère.
Ce phénonème est connu sous le nom de \emph{churn} \cite{understandingChurnP2PNetworks2006}.
Ainsi, de part l'absence de garantie sur le nombre de noeuds connectés de manière stable, les applications \acp{LFS} à large échelle ne peuvent pas utiliser des mécanismes de résolution de conflits reposant sur une coordination synchrone d'une proportion des noeuds du système, \ie sur des algorithmes de consensus \cite{1998-paxos-lamport, 2014-raft-ongaro}.

Ainsi, pour permettre la conception d'applications \acp{LFS} à large échelle, il convient de disposer de mécanismes de résolution de conflits pour l'ensemble des types de données avec une complexité algorithmique efficace peu importe le nombre de noeuds et ne nécessitant pas de coordination synchrone entre une proportion des noeuds du système.


\section{Questions de recherche et contributions}
\subsection{Ré-identification sans coordination synchrone pour les \acp{CRDT} pour le type Séquence}
\label{sec:research-questions-rls}
\begin{itemize}
    \item Les \acfp{CRDT} \cite{2007-crdt-shapiro,shapiro_2011_crdt} sont des types de données répliqués.
        Ils sont conçus pour permettre à un ensemble de noeuds d'un système de répliquer une donnée et pour leur permettre de la consulter et dee la modifier sans aucune coordination préalable.
        Dans ce but, les \acp{CRDT} incorporent des mécanismes de résolution de conflits automatiques directement au sein leur spécification.
    \item Cependant, ces mécanimes induisent un surcoût, aussi bien en termes de métadonnées et de calculs que de bande-passante.
        Ces surcoûts sont néanmoins jugés acceptables par la communauté pour une variété de types de données, \eg le Registre ou l'Ensemble.
        Cependant, le surcoût des \acp{CRDT} pour le type Séquence constitue toujours une problématique de recherche.
    \item En effet, la particuliarité des \acp{CRDT} pour le type Séquence est que leur surcoût croît de manière monotone au cours de la durée de vie de la donnée, \ie au fur et à mesure des modifications effectuées.
        Le surcoût introduit par les \acp{CRDT} pour ce type de données se révèle donc handicapant dans le contexte de collaborations sur de longues durées ou à large échelle.
    \item De manière plus précise, le surcoût des \acp{CRDT} pour le type Séquence provient de la croissance des métadonnées utilisées par leur mécanisme de résolution de conflits automatique.
        Ces métadonnées correspondent à des identifiants qui sont associés aux éléments de la Séquence.
        Ces identifiants permettent de résoudre les conflits, \eg en précisant quel est l'élement à supprimer ou en spécifiant la position d'un nouvel élément à insérer par rapport aux autres.
    \item Plusieurs approches ont été proposées pour réduire le coût induit par ces identifiants.
        Notamment, \cite{letia:hal-01248270,zawirski:hal-01248197} proposent un mécanisme de ré-assignation des identifiants pour réduire leur coût a posteriori.
        Ce mécanisme génère toutefois des conflits en cas de modifications concurrentes de la séquence, \ie l'insertion ou la suppression d'un élément.
        Les auteurs résolvent ce problème en proposant un mécanisme de transformation des modifications concurrentes par rapport à l'effet du mécanisme de ré-assignation des identifiants.
    \item Cependant, l'exécution en concurrence du mécanisme de ré-assignation des identifiants par plusieurs noeuds provoque elle-même un conflit.
        Pour éviter ce dernier type de conflit, les auteurs choisissent de subordonner à un algorithme de consensus l'exécution du mécanisme de ré-assignation des identifiants.
        Ainsi, le mécanisme de ré-assignation des identifiants ne peut être déclenché en concurrence par plusieurs noeuds du systèmes.
    \item Comme nous l'avons évoqué précédemment, reposer sur un algorithme de consensus qui requiert une coordination synchrone entre une proportion de noeuds du système est une contrainte incompatible avec les systèmes \ac{P2P} à large échelle sujets au churn.
        Notre problématique de recherche est donc la suivante : \emph{pouvons-nous proposer un mécanisme sans coordination synchrone de réduction du surcoût des \acp{CRDT} pour Séquence, \ie adapté aux applications \acp{LFS} ?}
    \item Pour répondre à cette problématiquee, nous proposons RenamableLogootSplit, un nouveau \ac{CRDT} pour le type Séquence.
        Ce \ac{CRDT} intègre un mécanisme de ré-assignation des identifiants, dit de renommage, directement au sein de sa spécification.
        Nous associons au mécanisme de renommage un mécanisme de résolution de conflits automatique additionnel pour gérer ses exécutions concurrentes.
        Ainsi, nous proposons un \ac{CRDT} pour le type Séquence dont le surcoût est périodiquement réduit par le biais d'un mécanisme n'introduisant aucune contrainte de coordination synchrone entre les noeuds du système.
\end{itemize}

\subsection{Éditeur de texte collaboratif \ac{P2P} temps réel chiffré de bout en bout}
\label{sec:research-questions-mute}
\begin{itemize}
    \item Les systèmes collaboratifs permettent à plusieurs utilisateur-rices de collaborer pour la réalisation d'une tâche.
        Les systèmes collaboratifs actuels adoptent principalement une architecture décentralisée, \ie un ensemble de serveurs avec lesquels les utilisateur-rices interagissent pour réaliser leur tâche, \eg Google Docs \cite{gdocs}.
        Par rapport à une architecture centralisée, cette architecture leur permet d'améliorer leur disponibilité et tolérance aux pannes, notamment grâce aux méthodes de réplication de données.
        Cette architecture à base de serveurs facilite aussi la collaboration, les serveurs permettant d'intégrer les modifications effectuées par les utilisateur-rices, de stocker les données, d'assurer la communication entre les utilisateur-rices ou encore de les authentifier.
    \item De part le rôle qui leur incombe, ces serveurs occupent une place primordiale dans ces systèmes.
        Il en découle plusieurs problématiques :
        \begin{enumerate}
            \item Ces serveurs manipulent et hébergent les données faisant l'objet de collaborations.
                Ces systèmes ont donc connaissance des données manipulées et de l'identité des auteur-rices des modifications.
                Les systèmes collaboratifs décentralisés demandent donc à leurs utilisateur-rices d'abandonner la souveraineté et la confidentialité de leur travail.
            \item Ces serveurs sont gérés par des autorités centrales, \eg Google.
                Les systèmes collaboratifs devenant non-fonctionnels en cas d'arrêt de leurs serveurs, les utilisateur-rices de ces systèmes dépendent de ces autorités centrales.
                Ainsi, de part leur pouvoir de vie et de mort sur les services qu'elles proposent, les autorités centrales représentent une menace pour la pérennité de ces systèmes, \eg \cite{2022-killed-by-google}.
        \end{enumerate}
    \item Pour répondre à ces problématiques, \ie confidentialité et souveraineté des données, dépendance envers des tiers, pérennité des systèmes, un nouveau paradigme de conception d'applications propose de concevoir des applications \acp{LFS}, \ie des applications mettant les utilisateur-rices et leurs appareils au coeur du système.
        Dans ce cadre d'applications \ac{P2P}, les serveurs sont relégués seulement à un rôle de support à la collaboration.
    \item Dans le cadre de ses travaux, notre équipe de recherche étudie notamment la conception d'applications respectant ce paradigme.
        Ce changement de modèle, d'une architecture décentralisée appartenant à des autorités centrales à une architecture \ac{P2P} sans autorités centrales, introduit un ensemble de problématiques de domaines variés, \eg
        \begin{enumerate}
            \item Comment permettre aux utilisateur-rices de collaborer en l'absence d'autorités centrales pour résoudre les conflits de modifications ?
            \item Comment authentifier les utilisateur-rices en l'absence d'autorités centrales ?
            \item Comment structurer le réseau de manière efficace, \ie en limitant le nombre de connexions par pair ?
        \end{enumerate}
    \item Cet ensemble de questions peut être résumé en la problématique suivante : \emph{pouvons-nous concevoir une application collaborative \ac{P2P} à large échelle, sûre et sans autorités centrales ?}
    \item Pour étudier cette problématique, l'équipe Coast développe l'application \acf{MUTE}\footnote{Disponible à l'adresse : \url{https://mutehost.loria.fr}} \cite{MUTE2017}.
        Il s'agit d'un éditeur de texte web collaboratif \ac{P2P} temps réel chiffré de bout en bout.
        Ce projet nous permet de présenter les travaux de recherche de l'équipe portant sur les mécanismes de résolutions de conflits automatiques pour le type Séquence \cite{2013-logootsplit,2021-these-vic,2022-rls-tpds-nicolas} et les mécanismes d'authentification des pairs dans les systèmes sans autorités centrales \cite{2018-trusternity-short,2018-trusternity-long}.
        Puis, ce projet nous donne l'opportunité d'étudier la littérature des nombreux domaines de recherche nécessaires à la conception d'un tel système, \ie le domaine des protocoles d'appartenance aux groupes \cite{swim2002, lifeguard2018}, des topologies réseaux \ac{P2P} \cite{2018-spray-nedelec} ou encore des protocoles d'établissement de clés de chiffrement de groupe \cite{1995-burmester-desmedt}.
        Ce projet nous permet ainsi de valoriser nos travaux et d'identifier de nouvelles perspectives de recherche.
        Finalement, il résulte de ce projet le \acf{PoC} le plus complet d'applications \acp{LFS}, à notre connaissance.
        \mnnote{TODO: Vérifier du côté des applis de IPFS}
\end{itemize}


\section{Plan du manuscrit}

Ce manuscrit de thèse est organisé de la manière suivante :

Dans le \autoref{chap:state-of-art}, nous introduisons le modèle du système que nous considérons, \ie les systèmes \ac{P2P} à large échelle sujets au churn et sans autorités centrales.
Puis nous présentons dans ce chapitre l'état de l'art des \acp{CRDT} et plus particulièrement celui des \acp{CRDT} pour le type Séquence.
À partir de cet état de l'art, nous identifions et motivons notre problématique de recherche, \ie l'absence de mécanisme adapté aux systèmes \ac{P2P} à large échelle sujets au churn permettant de réduire le surcoût induit par les mécanismes de résolution de conflits automatiques pour le type Séquence.

Dans le \autoref{chap:renamablelogootsplit}, nous présentons notre approche pour réaliser un tel mécanisme, \ie un mécanisme de résolution de conflits automatiques pour le type Séquence auquel nous associons un mécanisme de \acf{GC} de son surcoût ne nécessitant pas de coordination synchrone entre les noeuds du système.
Nous détaillons le fonctionnement de notre approche, sa validation par le biais d'une évaluation empirique puis comparons notre approche par rapport aux approches existantes
Finalement, nous concluons la présentation de notre approche en identifiant et en détaillant plusieurs de ses limites.

Dans le \autoref{chap:mute}, nous présentons \ac{MUTE}, l'éditeur de texte collaboratif temps réel \ac{P2P} chiffré de bout en bout que notre équipe de recherche développe dans le cadre de ses travaux de recherche.
Nous présentons les différentes couches logicielles formant un pair et les services tiers avec lesquels les pairs interagissent, et détaillons nos travaux dans le cadre de ce projet, \ie l'intégration de notre mécanisme de résolution de conflits automatiques pour le type Séquence et le développement de la couche de livraison des messages associée.
Pour chaque couche logicielle, nous identifions ses limites et présentons de potentielles pistes d'améliorations.

Finalement, nous récapitulons dans le \autoref{chap:conclusions-perspectives} les contributions réalisées dans le cadre de cette thèse.
Puis nous clotûrons ce manuscrit en introduisant plusieurs des pistes de recherches que nous souhaiterons explorer dans le cadre de nos travaux futurs.


\section{Publications}
Notre travail sur la problématique identifiée dans la \autoref{sec:research-questions-rls}, \ie la proposition d'un mécanisme ne nécessitant aucune coordination synchrone pour réduire le surcoût des \acp{CRDT} pour le type Séquence, a donné lieu à des publications à différents stades de son avancement :
\begin{enumerate}
    \item Dans \cite{2018-rls-middleware-nicolas}, nous motivons le problème identifié et présentons l'idée de notre approche pour y répondre.
    \item Dans \cite{2020-rls-papoc-nicolas}, nous détaillons une première partie de notre approche et présentons notre protocole d'évaluation expérimentale ainsi que ses premiers résultats.
    \item Dans \cite{2022-rls-tpds-nicolas}, nous détaillons notre proposition dans son entièreté.
        Nous accompagnons cette proposition d'une évaluation expérimentale poussée.
        Finalement, nous complétons notre travail d'une discussion identifiant plusieurs de ses limites et présentant des pistes de travail possibles pour y répondre.
\end{enumerate}
Nous précisons ci-dessous les informations relatives à chacun de ces articles.

\subsection*{Efficient renaming in CRDTs \cite{2018-rls-middleware-nicolas}}

\paragraph{Auteur} Matthieu Nicolas

\paragraph{Article de position} à Middleware 2018 - 19th ACM/IFIP International Middleware Conference (Doctoral Symposium), Dec 2018, Rennes, France.

\paragraph{Abstract}
\emph{Sequence Conflict-free Replicated Data Types (CRDTs)} allow to replicate and edit, without any kind of coordination, sequences in distributed systems.
To ensure convergence, existing works from the literature add metadata to each element but they do not bound its footprint, which impedes their adoption.
Several approaches were proposed to address this issue but they do not fit a fully distributed setting.
In this paper, we present our ongoing work on the design and validation of a fully distributed renaming mechanism, setting a bound to the metadata's footprint.
Addressing this issue opens new perspectives of adoption of these CRDTs in distributed applications.

\subsection*{Efficient Renaming in Sequence CRDTs \cite{2020-rls-papoc-nicolas}}

\paragraph{Auteurs} Matthieu Nicolas, Gérald Oster, Olivier Perrin

\paragraph{Article de workshop} à PaPoC 2020 - 7th Workshop on Principles and Practice of Consistency for Distributed Data, Apr 2020, Heraklion / Virtual, Greece.

\paragraph{Abstract}
To achieve high availability, large-scale distributed systems have to replicate data and to minimise coordination between nodes.
Literature and industry increasingly adopt \acfp{CRDT} to design such systems.
\acp{CRDT} are data types which behave as traditional ones, e.g. the Set or the Sequence.
However, unlike traditional data types, they are designed to natively support concurrent modifications.
To this end, they embed in their specification a conflict-resolution mechanism.

To resolve conflicts in a deterministic manner, \acp{CRDT} usually attach identifiers to elements stored in the data structure.
Identifiers have to comply with several constraints, such as uniqueness or belonging to a dense order.
These constraints may hinder the identifiers’ size from being bounded.
As the system progresses, identifiers tend to grow.
This inflation deepens the overhead of the \ac{CRDT} over time, leading to performance issues.

To address this issue, we propose a new CRDT for Sequence which embeds a renaming mechanism.
It enables nodes to reassign shorter identifiers to elements in an uncoordinated manner.
Experimental results demonstrate that this mechanism decreases the overhead of the replicated data structure and eventually limits it.

\subsection*{Efficient Renaming in Sequence CRDTs \cite{2022-rls-tpds-nicolas}}

\paragraph{Auteurs} Matthieu Nicolas, Gérald Oster, Olivier Perrin

\paragraph{Article de journal} dans IEEE Transactions on Parallel and Distributed Systems, Institute of Electrical and Electronics Engineers, 2022, 33 (12), pp.3870-3885.

\paragraph{Abstract}
To achieve high availability, large-scale distributed systems have to replicate data and to minimise coordination between nodes.
For these purposes, literature and industry increasingly adopt \acfp{CRDT} to design such systems.
\acp{CRDT} are new specifications of existing data types, e.g. Set or Sequence.
While \acp{CRDT} have the same behaviour as previous specifications in sequential executions, they actually shine in distributed settings as they natively support concurrent updates.
To this end, \acp{CRDT} embed in their specification conflict resolution mechanisms.
These mechanisms usually rely on identifiers attached to elements of the data structure to resolve conflicts in a deterministic and coordination-free manner.
Identifiers have to comply with several constraints, such as being unique or belonging to a dense total order.
These constraints may hinder the identifier size from being bounded.
Identifiers hence tend to grow as the system progresses, which increases the overhead of \acp{CRDT} over time and leads to performance issues.
To address this issue, we propose a novel Sequence \ac{CRDT} which embeds a renaming mechanism.
It enables nodes to reassign shorter identifiers to elements in an uncoordinated manner.
Experimental results demonstrate that this mechanism decreases the overhead of the replicated data structure and eventually minimises it.


\import{chapters/etat-art/}{etat-art}
% \import{chapters/rls/}{rls}
% \import{chapters/}{mute}

\NumberThisInToc
\chapter{Conclusions et perspectives}
\minitoc
\label{chap:conclusions-perspectives}


Dans ce chapitre, nous revenons sur les contributions présentés dans cette thèse.
Nous rappelons le contexte dans lequel elles s'inscrivent, récapitulons leurs spécificités et apports, et finalement présentons leurs limites que nous identifions.
Puis, nous concluons ce manuscrit en présentant plusieurs pistes de recherche qui nous restent à explorer à l'issue de cette thèse.
Les premières s'inscrivent dans la continuité directe de nos travaux sur un mécanisme de ré-identification pour \acp{CRDT} pour Séquence dans un système \ac{P2P} à large échelle sujet au churn.
Les dernières traduisent quant à elles notre volonté de recentrer nos travaux sur le domaine plus général des \acp{CRDT}.


\section{Résumés des contributions}

\subsection{Réflexions sur l'état de l'art des \acp{CRDT}}
Les \acfp{CRDT} \cite{shapiro_2011_crdt} sont de nouvelles spécifications des types de données.
Ils sont conçus pour permettre à un ensemble de noeuds d'un système de répliquer une même donnée et pour leur permettre de la consulter et de la modifier sans aucune coordination préalable.
Les noeuds se synchronisent

L'absence de coordination entre les noeuds avant modifications implique que des noeuds peuvent modifier la donnée en concurrence.
De telles modifications peuvent donner lieu à des conflits, \eg l'ajout et la suppression en concurrence d'un même élément dans un ensemble.
Pour pallier ce problème, les \acp{CRDT} incorporent un mécanisme de résolution de conflits automatiques directement au sein de leur spécification.

Il convient de noter qu'il existe plusieurs solutions possibles pour résoudre un conflit.
Pour reprendre l'exemple de l'élément ajouté et supprimé en concurrence d'un ensemble, nous pouvons par exemple soit le conserver l'élément, soit le supprimer.
Nous parlons alors de sémantique du mécanisme de résolution de conflits automatique.

De la même manière, il existe plusieurs approches possibles pour synchroniser les noeuds, \eg diffuser chaque modification de manière atomique ou diffuser l'entièreté de l'état périodiquement.
Ainsi, lors de la définition d'un \ac{CRDT}, il convient de préciser les sémantiques de résolution de conflits qu'il adopte et le modèle de synchronisation qu'il utilise \cite{2018-crdts-overview-preguica}.\\

Depuis leur formalisation, les travaux sur les \acp{CRDT} ont abouti à la conception de nouveaux, soit en spécifiant de nouvelles sémantiques de résolution de conflits pour un type de données \cite{2020-cl-set-weihai}, soit en spécifiant de nouveaux modèles de synchronisation \cite{Almeida_2018} ou en enrichissant les spécifications des modèles existants \cite{baquero2017pure,enes2019}.

Dans notre présentation des \acp{CRDT} \cf{sec:etat-art-crdts-intro}, nous présentons chacun de ces aspects.
Cependant, nous nous ne limitons pas à retranscrire l'état de l'art de la littérature.
Notamment au sujet du modèle de synchronisation par opérations, nous précisons que le modèle de livraison causal n'est pas nécessaire pour l'ensemble des \acp{CRDT} synchronisés par opérations, \ie que certains peuvent adopter des modèles de livraison moins contraints et donc moins coûteux.
Cette précision nous permet de proposer une étude comparative des différents modèles de synchronisation qui est, à notre connaissance, l'une des plus précises de la littérature \cf{def:synchro-synthese}.\\

Nous présentons ensuite les différents \acp{CRDT} pour le type Séquence de la littérature \cf{sec:seq-crdts}.
Nous mettons alors en exergue les deux approches proposées pour concevoir le mécanisme de résolution de conflits automatiques pour le type Séquence, \ie l'approche à pierres tombales et l'approche à identifiants densément ordonnés.
De nouveau, cette rétrospective nous permet d'expliciter des angles morts des articles d'origine, notamment vis-à-vis des modèle de livraison des opérations des \acp{CRDT} proposés.
Puis, nous mettons en lumière les limites des évaluations comparant les deux approches, \ie le couplage entretenu entre approche du mécanisme de résolution de conflits et choix d'implémentations.
Cette limite empêche d'établir la supériorité d'une des approches par rapport à l'autre.
Finalement, nous conjecturons que le surcoût de ces deux approches est le même, \ie le coût nécessaire à la représentation d'un espace dense.
Nous précisons dès lors par le biais de notre propre étude comparative comment ce surcoût s'exprime dans chacune des approches , \ie le compromis entre surcoût en métadonnées, calculs et bande-passante proposé par les deux approches \cf{sec:seq-crdts-synth}.\\

Ces réflexions que nous présentons sur l'état des \acp{CRDT} définissent plusieurs pistes de recherches.
Une première d'entre elles concerne notre étude comparative des modèles de synchronisation.
D'après les critères que nous utilisons, une conclusion possible de cette comparaison est que le modèle de synchronisation par différences d'états rend obsolètes les modèles de synchronisation par états et par opérations.
En effet, le modèle de synchronisation par différences d'états apparaît comme adapté à l'ensemble des contextes d'utilisation qui étaient jusqu'à lors exclusifs à ces derniers, de par les multiples stratégies qu'il permet, \eg synchronisation par états complets, synchronisation par états irréductibles...

Cette conclusion nous paraît cependant hâtive.
Il convient d'étendre notre étude comparative pour prendre en compte des critères de comparaison additionnels pour confirmer cette conjecture, ou l'invalider et définir plus précisément les spécificités de chacun des modèles de synchronisation.
Nous détaillons cette piste de recherche dans la \autoref{sec:perspective-comparison-sync-models}.\\

Une seconde piste de recherche possible concerne les deux approches utilisées pour concevoir le mécanisme de résolution de conflits des \acp{CRDT} pour le type Séquence.
Comme dit précédemment, nous conjecturons que ces deux approches ne sont finalement que deux manières différentes de représenter une même information : la position d'un élément dans un espace dense.
La différence entre ces approches résiderait uniquement dans la manière que chaque représentation influe sur les performances du \ac{CRDT}.
Une piste de travail serait donc de confirmer cette conjecture, en proposant une formalisation unique des \acp{CRDT} pour le type Séquence.


\subsection{Ré-identification sans coordination pour les \acp{CRDT} pour Séquence}
Pour privilégier leur disponibilité, latence et tolérance aux pannes, les systèmes distribués peuvent adopter le paradigme de la réplication optimiste \cite{2005-optimistic-replication-saito}.
Ce paradigme consiste à relaxer la cohérence de données entre les noeuds du système pour leur permettre de consulter et modifier leur copie locale sans se coordonner.
Leur copies peuvent alors temporairement diverger avant de converger de nouveau une fois les modifications de chacun propagées.
Cependant, cette approche nécessite l'emploi d'un mécanisme de résolution de conflits pour assurer la convergence même en cas de modifications concurrentes.
Pour cela, l'approche des \acp{CRDT} \cite{2007-crdt-shapiro,shapiro_2011_crdt} propose d'utiliser des types de données dont les modifications commutent nativement.

Depuis la spécification des \acp{CRDT}, la littérature a proposé plusieurs de ces mécanismes résolution de conflits automatiques pour le type de données Séquence \cite{2006-woot-oster,ROH2011354,2009-treedoc-preguica,2009-logoot-weiss}.
Cependant, ces approches souffrent toutes d'un surcoût croissant de manière monotone de leurs métadonnées et calculs.
Ce problème a été identifié par la communauté, et celle-ci a proposé pour y répondre des mécanismes permettant soit de réduire la croissance de leur surcoût \cite{lseq2013,lseq2017}, soit d'effectuer une \acf{GC} de leur surcoût \cite{ROH2011354,letia:hal-01248270,zawirski:hal-01248197}.
Nous avons cependant déterminé que ces mécanismes ne sont pas adaptés aux systèmes \ac{P2P} à large échelle souffrant de churn et utilisant des \acp{CRDT} pour le type Séquence à granularité variable.

Dans le cadre de cette thèse, nous avons donc souhaité proposer un nouveau mécanisme adapté à ce type de systèmes.
Pour cela, nous avons suivi l'approche proposée par \cite{letia:hal-01248270,zawirski:hal-01248197} : l'utilisation d'un mécanisme pour ré-assigner de nouveaux identifiants aux éléments stockées dans la séquence.
Nous avons donc proposé un nouveau mécanisme appartenant à cette approche pour le \ac{CRDT} LogootSplit \cite{2013-logootsplit}.\\

Notre proposition prend la forme d'un nouveau \ac{CRDT} pour le type Séquence à granularité variable : RenamableLogootSplit.
Ce nouveau \ac{CRDT} associe à LogootSplit un nouveau type de modification, $\trm{ren}$, permettant de produire un nouvel état de la séquence équivalent à son état précédent.
Cette nouvelle modification tire profit de la granularité variable de la séquence pour produire un état de taille minimale : elle assigne à tous les éléments des identifiants de position issus d'un même intervalle.
Ceci nous permet de minimiser les métadonnées que la séquence doit stocker de manière effective, \ie le premier et le dernier des identifiants composant l'intervalle.
% De plus, le passage à une représentation interne minimale de la séquence nous permet d'améliorer le coût des modifications suivantes en termes de calculs.

Afin de gérer les opérations concurrentes aux opérations $\trm{ren}$, nous définissons pour ces dernières un algorithme de transformation.
Pour cela, nous définissons un mécanisme d'époques nous permettant d'identifier la concurrence entre opérations.
Nous introduisons une relation d'ordre strict total, \emph{priority}, pour résoudre de manière déterministe le conflit provoqué par deux opérations $\trm{ren}$, \ie pour déterminer quelle opération $\trm{ren}$ privilégier.
Finalement, nous définissons deux algorithmes, \texttt{renameId} et \texttt{revertRenameId}, qui permettent de transformer les opérations concurrentes à une opération $\trm{ren}$ pour prendre en compte l'effet de cette dernière.
Ainsi, notre algorithme permet de détecter et de transformer les opérations concurrentes aux opérations $\trm{ren}$, sans nécessiter une coordination synchrone entre les noeuds.\\


Pour valider notre approche, nous proposons une évaluation expérimentale de cette dernière.
Cette évaluation se base sur des traces de sessions d'édition collaborative que nous avons générées par simulations.

Notre évaluation nous permet de valider de manière empirique les résultats attendus.
Le premier d'entre eux concerne la convergence des noeuds.
En effet, nos simulations nous ont permis de valider que l'ensemble des noeuds obtenaient des états finaux équivalents, même en cas d'opérations $\trm{ren}$ concurrentes.

Notre évaluation nous a aussi permis de valider que le mécanisme de renommage réduit à une taille minimale le surcoût du mécanisme de résolution de conflits incorporé dans le \ac{CRDT} pour le type Séquence.

L'évaluation expérimentale nous a aussi permis de prendre conscience d'effets additionnels du mécanisme de renommage que nous n'avions pas anticipé.
Notamment, elle montre que le surcoût éventuel du mécanisme de renommage, notamment en termes de calculs, est toutefois contrebalancé par l'amélioration précisée précédemment, \ie la réduction de la taille de la séquence.\\

Finalement, notons que le mécanisme que nous proposons est partiellement générique : il peut être adapté à d'autres \acp{CRDT} pour le type Séquence à granularité variable, \eg un \ac{CRDT} pour le type Séquence appartenant à l'approche à pierres tombales.
Dans le cadre d'une telle démarche, nous pourrions réutiliser le système d'époques, la relation \emph{priority} et l'algorithme de contrôle qui identifie les transformations à effectuer.
Pour compléter une telle adaptation, nous devrions cependant concevoir de nouveaux algorithmes \texttt{renameId} et \texttt{revertRenameId} spécifiques et adaptés au \ac{CRDT} choisi.\\

Le mécanisme de renommage que nous présentons souffre néanmoins de plusieurs limites.
Une d'entre elles concerne ses performances.
En effet, notre évaluation expérimentale a mis en lumière le coût important en l'état de la modification $\trm{ren}$ par rapport aux autres modifications en termes de calculs \cf{sec:integration-time-rename}.
De plus, chaque opération $\trm{ren}$ comporte une représentation de l'ancien état qui doit être maintenue par les noeuds jusqu'à leur stabilité causale.
Le surcoût en métadonnées introduit par un ensemble d'opérations $\trm{ren}$ concurrentes peut donc s'avérer important, voire pénalisant \cf{sec:evaluation-metadata}.
Pour répondre à ces problèmes, nous identifions trois axes d'amélioration :
\begin{enumerate}
    \item La définition de stratégies de déclenchement du renommage efficaces.
        Le but de ces stratégies serait de déclencher le mécanisme de renommage de manière fréquente, de façon à garder son temps d'exécution acceptable, mais tout visant à minimiser la probabilité que les noeuds produisent des opérations $\trm{ren}$ concurrentes, de façon à minimiser le surcoût en métadonnées.
    \item La définition de relations \emph{priority} efficaces.
        Nous développons ce point dans nos perspectives, \ie dans la \autoref{sec:alternative-priority}.
    \item La proposition d'algorithmes de renommage efficaces.
        Cette amélioration peut prendre la forme de nouveaux algorithmes pour \texttt{renameId} et \texttt{revertRenameId} offrant une meilleure complexité en temps.
        Il peut aussi s'agir de la conception d'une nouvelle approche pour renommer l'état et gérer les modifications concurrentes, \eg un mécanisme de renommage basé sur le journal des opérations \cf{sec:log-based-rename-mechanism}.
\end{enumerate}

Une seconde limite de RenamableLogootSplit que nous identifions concerne son mécanisme de \ac{GC} des métadonnées introduites par le mécanisme de renommage.
En effet, pour fonctionner, ce dernier repose sur la stabilité causale des opérations $\trm{ren}$.
Pour rappel, la stabilité causale représente le contexte causal commun à l'ensemble des noeuds du système.
Pour le déterminer, chaque noeud doit récupérer le contexte causal de l'ensemble des noeuds du système.
Ainsi, l'utilisation de la stabilité causale comme pré-requis pour la \ac{GC} de métadonnées constitue une contrainte forte, voire prohibitive, dans les systèmes \ac{P2P} à large échelle sujet au churn.
En effet, un noeud du système déconnecté de manière définitive suffit pour empêcher la stabilité causale de progresser, son contexte causal étant alors indéterminé du point de vue des autres noeuds.
Il s'agit toutefois d'une limite récurrente des mécanismes de \ac{GC} distribués et asynchrones \cite{ROH2011354,baquero2017pure,2018-prunable-authenticated-log-vic}.


\subsection{Éditeur de texte collaboratif \ac{P2P} chiffré de bout en bout}
Les applications collaboratives permettent à des utilisateur-rices de réaliser collaborativement une tâche.
Elles permettent à plusieurs utilisateur-rices de consulter la version actuelle du document, de la modifier et de partager leurs modifications avec les autres.
Ceci permet de mettre en place une réflexion de groupe, ce qui améliore la qualité du résultat produit \cite{2004-empirical-study-collaborative-writing,2005-internet-encyclopaedias-head-to-head}.

Cependant, les applications collaboratives sont historiquement des applications centralisées, \eg Google Docs \cite{gdocs}.
Ce type d'architecture induit des défauts d'un point de vue technique, \eg faible capacité de passage à l'échelle et faible tolérance aux pannes, mais aussi d'un point de vue utilisateur, \eg perte de la souveraineté des données et absence de garantie de pérennité.\\

Les travaux de l'équipe Coast s'inscrivent dans une mouvance souhaitant résoudre ces problèmes et qui a conduit à la définition d'un nouveau paradigme d'applications : les applications \acf{LFS} \cite{localfirstsoftware2019}.
Le but de ce paradigme est la conception d'applications collaboratives, \ac{P2P}, pérennes et rendant la souveraineté de leurs données aux utilisateur-rices.

Dans le cadre de cette démarche, l'équipe Coast développe depuis plusieurs années l'application \acf{MUTE}, un éditeur de texte web collaboratif \ac{P2P} temps réel chiffré de bout en bout.
Cette application sert à la fois de plateforme de démonstration et de recherche pour les travaux de l'équipe, mais aussi de \acf{PoC} pour les \ac{LFS}.
Ainsi, \ac{MUTE} propose au moment où nous écrivons ces lignes un aperçu des travaux de recherche existants concernant :
\begin{enumerate}
    \item Les mécanismes de résolution de résolutions de conflits automatiques pour l'édition collaborative de documents textes \cite{2013-logootsplit,2021-these-vic,2022-rls-tpds-nicolas}.
    \item Les protocoles distribués d'appartenance au groupe \cite{swim2002}.
    \item Les mécanismes d'anti-entropie \cite{1983-anti-entropy-vv}.
    \item Les protocoles distribués d'authentification d'utilisateur-rices \cite{2018-trusternity-short,2018-trusternity-long}.
    \item Les protocoles distribués d'établissement de clés de chiffrement de groupe \cite{1995-burmester-desmedt}.
    \item Les mécanismes de conscience de groupe.
\end{enumerate}
Dans cette liste, nous avons personnellement contribué à l'implémentation des \acp{CRDT} LogootSplit \cite{2013-logootsplit} et RenamableLogootSplit \cite{2022-rls-tpds-nicolas}, et du protocole d'appartenance au groupe SWIM \cite{swim2002}.\\

En son état actuel, \ac{MUTE} présente cependant plusieurs limites.
Tout d'abord, l'environnement web implique un certain nombre de contraintes, notamment au niveau des technologies et protocoles disponibles.
Notamment, le protocole \acf{WebRTC} repose sur l'utilisation de serveurs de signalisation, \ie de points de rendez-vous des pairs, et de serveurs de relais, \ie d'intermédiaires pour communiquer entre pairs lorsque les configurations de leur réseaux respectifs interdisent l'établissement d'une connexion directe.
Ainsi, les applications \ac{P2P} web doivent soit déployer et maintenir leur propre infrastructure de serveurs, soit reposer sur une infrastructure existante, \eg celle proposée par OpenRelay \cite{openrelay}.

Dans le cadre de \ac{MUTE}, nous avons opté pour cette seconde solution.
Cependant, ce choix introduit un \acf{SPOF}\footnote{\acf{SPOF} : Point de défaillance unique} dans \ac{MUTE} : OpenRelay elle-même.
Afin de garantir la pérennité de \ac{MUTE}, nous devrions reposer non pas sur une unique infrastructure de serveurs de signalisation et de relais mais sur une multitude.
Malheureusement, l'écosystème actuel brille par la rareté d'infrastructures publiques offrant ces services.
Nous devons donc encourager et supporter la mise en place de telles infrastructures par une pluralité d'organisations.\\

Une autre limite de ce système que nous identifions concerne l'utilisabilité des systèmes \ac{P2P} de manière générale.
L'expérience vécue suivante constitue à notre avis un exemple éloquent des limites actuelles de l'application \ac{MUTE} dans ce domaine.
Après avoir rédigé une version initiale d'un document, nous avons envoyé le lien du document à notre collaborateur pour relecture et validation.
Lorsque notre collaborateur a souhaité accéder au document, celui-ci s'est retrouvé devant une page blanche : comme nous nous étions déconnecté du système entretemps, plus aucun pair possédant une copie n'était disponible pour se synchroniser.

Notre collaborateur était donc dans l'incapacité de récupérer le document et d'effectuer sa tâche.
Afin de pallier ce problème, une solution possible est de faire reposer \ac{MUTE} sur un réseau \ac{P2P} global, \eg le réseau de \ac{IPFS} \cite{ipfs}, et d'utiliser les pairs de ce dernier, potentiellement des pairs étrangers à l'application, comme pairs de stockage pour permettre une synchronisation future.
Cette solution limiterait ainsi le risque qu'un pair ne puisse récupérer l'état du document faute de pairs disponibles.
Pour garantir l'utilisabilité du système \ac{P2P}, une telle solution devrait donc permettre à un pair de récupérer l'état du document à sa reconnexion, malgré la potentielle évolution du groupe des collaborateur-rices et des pairs de stockage, \eg l'ajout, l'éviction ou la déconnexion d'un des pairs.
Cependant, la solution devrait en parallèle garantir qu'elle n'introduit aucune vulnérabilité, \eg la possibilité pour les pairs de stockage selectionnés de reconstruire et consulter le document.
% \item Finalement, une dernière limite que nous identifions est la pérennité économique de ce type d'applications.
%     Selon nous, les systèmes \ac{P2P} chiffrés de bout en bout s'interdisent les modèles économiques dominants et acceptés par les organisations et utilisateur-rices, \ie la collecte et revente de données.
%     En effet,
%     , de par les propriétés qu'ils assurent, notamment la confidentialité des données .
%     \mnnote{TODO: Voir si on a des données sur les entreprises/organisations encourageant le chiffrement de bout-en-bout dans leurs outils internes, }


\section{Perspectives}

\subsection{Définition de relations de priorité pour minimiser les traitements}
\label{sec:alternative-priority}

Dans la \autoref{sec:priority}, nous avons spécifié la relation \emph{priority} \cf{def:priority-relation}.
Pour rappel, cette relation doit établir un ordre strict total sur les époques de notre mécanisme de renommage.

Cette relation nous permet ainsi de résoudre le conflit provoqué par la génération de modifications $\trm{ren}$ concurrentes en les ordonnant.
Grâce à cette relation relation d'ordre, les noeuds peuvent déterminer vers quelle époque de l'ensemble des époques connues progresser.
Cette relation permet ainsi aux noeuds de converger à une époque commune à terme.\\

La convergence à terme à une époque commune présente plusieurs avantages :
\begin{enumerate}
    \item Réduire la distance entre les époques courantes des noeuds, et ainsi minimiser le surcoût en calculs par opération du mécanisme de renommage.
        En effet, il n'est pas nécessaire de transformer une opérations livrée avant de l'intégrer si celle-ci provient de la même époque que le noeud courant.
    \item Définir un nouveau \ac{LCA} entre les époques courantes des noeuds.
        Cela permet aux noeuds d'appliquer le mécanisme de \ac{GC} pour supprimer les époques devenues obsolètes et leur anciens états associés, pour ainsi minimiser le surcoût en métadonnées du mécanisme de renommage.
\end{enumerate}

Il existe plusieurs manières pour définir la relation \emph{priority} tout en satisfaisant les propriétés indiquées.
Dans le cadre de ce manuscrit, nous avons utilisé l'ordre lexicographique sur les chemins des époques dans l'\emph{arbre des époques} pour définir \emph{priority}.
Cette approche se démarque par :
\begin{enumerate}
    \item Sa simplicité.
    \item Son surcoût limité, \ie cette approche n'introduit pas de métadonnées supplémentaires à stocker et diffuser, et l'algorithme de comparaison utilisé est simple.
    \item Sa propriété arrangeante sur les déplacements des noeuds dans l'arbre des époques.
        De manière plus précise, cette définition de \emph{priority} impose aux noeuds de se déplacer que vers l'enfant le plus à droite de l'arbre des époques.
        Ceci empêche les noeuds de faire un aller-retour entre deux époques données.
        Cette propriété permet de passer outre une contrainte concernant le couple de fonctions \texttt{renameId} et \texttt{revertRenameId} : leur reciprocité.
\end{enumerate}

Cette définition présente cependant plusieurs limites.
La limite que nous identifions est sa décorrélation avec le coût et le bénéfice de progresser vers l'époque cible désignée.
En effet, l'époque cible est désignée de manière arbitraire par rapport à sa position dans l'arbre des époques.
Il est ainsi possible que progresser vers cette époque détériore l'état de la séquence, \ie augmente la taille des identifiants et augmente le nombre de blocs.
De plus, la transition de l'ensemble des noeuds depuis leur époque courante respective vers cette nouvelle époque cible induit un coût en calculs, potentiellement important \cf{fig:worst-case-priority}.\\

Pour pallier ce problème, il est nécessaire de proposer une définition de \emph{priority} prenant l'aspect efficacité en compte.
L'approche considérée consisterait à inclure dans les opérations $\trm{ren}$ une ou plusieurs métriques qui représente le travail accumulé sur la branche courante de l'arbre des époques, \eg le nombre d'opérations intégrées, les noeuds actuellement sur cette branche...
L'ordre strict total entre les époques serait ainsi construit à partir de la comparaison entre les valeurs de ces métriques de leur opération $\trm{ren}$ respective.

Il conviendra d'adjoindre à cette nouvelle définition de \emph{priority} un nouveau couple de fonctions \texttt{renameId} et \texttt{revertRenameId} respectant la contrainte de réciprocité de ces fonctions, ou de mettre en place une autre implémentation du mécanisme de renommage ne nécessitant pas cette contrainte, telle qu'une implémentation basée sur le journal des opérations \cf{sec:log-based-rename-mechanism}.

Il conviendra aussi d'étudier la possibilité de combiner l'utilisation de plusieurs relations \emph{priority} pour minimiser le surcoût global du mécanisme de renommage, \eg en fonction de la distance entre deux époques.

Finalement, il sera nécessaire de valider l'approche proposée par une évaluation comparative par rapport à l'approche actuelle.
Cette évaluation pourrait consister à monitorer le coût du système pour observer si l'approche proposée permet de réduire les calculs de manière globale.
Plusieurs configurations de paramètres pourraient aussi être utilisées pour déterminer l'impact respectif de chaque paramètre sur les résultats.


\subsection{Détection et fusion manuelle de versions distantes}
\begin{itemize}
    \item À l'issue de cette thèse, nous constatons plusieurs limites des mécanismes de résolution de conflits automatiques dans les systèmes \ac{P2P} à large échelle.
        La première d'entre elles est l'utilisation d'un contexte causal.
        Le contexte causal est utilisé par les mécanismes de résolution de conflits pour :
        \begin{enumerate}
            \item Satisfaire le modèle de cohérence causale, \ie assurer que si nous avons deux modifications $m_1$ et $m_2$ telles que $m1 \to m_2$, alors l'effet de $m_2$ supplantera celui de $m_1$.
                Ceci permet d'éviter des anomalies de comportement de la part de la structure de données du point de vue des utilisateur-rices, par exemple la résurgence d'un élément supprimé au préalable.
            \item Permettre de préserver l'intention d'une modification malgré l'intégration préalable de modifications concurrentes.
        \end{enumerate}
    \item Le contexte causal est utilisé de manière différente en fonction du mécanisme de résolution de conflit.
        Dans l'approche \ac{OT}, le contexte causal est utilisé par l'algorithme de contrôle pour déterminer les modifications concurrentes à une modification lors de son intégration, afin de prendre en compte leurs effets.
        Dans l'approche \ac{CRDT}, le contexte causal est utilisé par la structure de données répliquée à la génération de la modification pour en faire une modification nativement commutative avec les modifications concurrentes, \ie pour en faire un élément du sup-demi-treillis représentant la structure de données répliquée.
    \item Le contexte causal peut être représenté de différentes manières.
        Par exemple, le contexte causal peut prendre la forme d'un vecteur de version \cite{1988-version-vector-mattern,1991-version-vector-fidge} ou d'un \ac{DAG} des modifications \cite{1997-causal-barrier}.
        Cependant, de manière intrinsèque, le contexte causal ne fait que de croître au fur et à mesure que des modifications sont effectuées ou que des noeuds rejoignent le système, incrémentant son surcoût en métadonnées, calculs et bande-passante.
    \item La stabilité causale permet cependant de réduire le surcoût lié au contexte causal.
        En effet, la stabilité causale permet d'établir le contexte commun à l'ensemble des noeuds, \ie l'ensemble des modifications que l'ensemble des noeuds ont intégré.
        Ces modifications font alors partie de l'histoire commune et n'ont plus besoin d'être considérées par les mécanismes de résolution de conflits.
        La stabilité causale permet donc de déterminer et de tronquer la partie commune du contexte causal pour éviter que ce dernier ne pénalise les performances du système à terme.
    \item La stabilité causale est cependant une contrainte forte dans les systèmes \ac{P2P} dynamiques à large échelle dans lesquels nous n'avons aucun contrôle sur les noeuds.
        Il ne suffit en effet que d'un noeud déconnecté pour empêcher la stabilité causale de progresser.
        Pour répondre à ce problème, nous avons dès lors tout un spectre d'approches possibles, proposant chacune un compromis entre le surcoût du contexte causal et la probabilité de rejeter des modifications.
        Les extremités de ce spectre d'approches sont les suivantes :
        \begin{enumerate}
            \item Considérer tout noeud déconnecté comme déconnecté de manière définitive, et donc les ignorer dans le calcul de la stabilité causale.
                Cette première approche permet à la stabilité causale de progresser, et ainsi aux noeuds connectés de travailler dans des conditions optimales.
                Mais elle implique cependant que les modifications potentielles des noeuds déconnectés soient perdues, \ie de ne plus pouvoir les intégrer en l'absence d'un lien entre leur contexte causal de génération et le contexte causal actuel de chaque autre noeud.
                Il s'agit là de la stratégie la plus aggressive en terme de \ac{GC} du contexte causal.
            \item Assurer en toutes circonstances la capacité d'intégration des modifications des noeuds, même ceux déconnectés.
                Cette seconde approche permet de garantir que les modifications potentielles des noeuds déconnectés pourront être intégrées automatiquement, dans l'éventualité où ces derniers se reconnectent à terme.
                Mais elle implique de bloquer potentiellement de manière définitive la stabilité causale et donc le mécanisme de \ac{GC} du contexte causal.
                Il s'agit là de la stratégie la plus timide en terme de \ac{GC} du contexte causal.
        \end{enumerate}
    \item La seconde limite que nous constatons est la limite des mécanismes actuels de résolution de conflits automatiques pour préserver l'intention des utilisateur-rices.
        Par exemple, les mécanismes de résolution de conflits automatiques pour le type Séquence présentés dans ce manuscrit \cf{sec:seq-crdts} définissent l'intention de la manière suivante : \emph{l'intégration de la modification par les noeuds distants doit reproduire l'effet de la modification sur la copie d'origine}.
        Cette définition assure que chaque modification est porteuse d'une intention, mais limite voire ignore toute la dimension sémantique de la dite intention.
        Nous conjecturons que l'absence de dimension sémantique réduit les cas d'utilisation de ces mécanismes.
    \item Considérons par exemple une édition collaborative d'un même texte par un ensemble de noeuds.
        Lors de la présence d'une faute de frappe dans le texte, \eg le mot "HLLO", plusieurs utilisateur-rices peuvent la corriger en concurrence, \ie insérer l'élément "E" entre "H" et "L".
        Les mécanismes de résolution de conflits automatiques permettent aux noeuds d'obtenir des résultats qui convergent mais à notre sens insatisfaisant, \eg "HEEEEEELLO".
        Nous considérons ce type de résultats comme des anomalies, au même titre que l'entrelacement \cite{2019-interleaving-anomalies-collaborative-editors-kleppmann}.
        Dans le cadre de collaborations temps réel à échelle limitée, nous conjecturons cependant qu'une granularité fine des modifications permet de pallier ce problème.
        En effet, les utilisateur-rices peuvent observer une anomalie produite par le mécanisme de résolution de conflits automatique, déduire l'intention initiale des modifications concernées et la restaurer par le biais d'actions supplémentaires de compensation.
    \item Cependant, dans le cadre de collaborations asynchrones ou à large échelle, nous conjecturons que ces anomalies de résolution de conflits s'accumulent.
        Cette accumulation peut atteindre un seuil rendant laborieuse la déduction et le rétablissement de l'intention initiale des modifications.
        Le travail imposé aux utilisateur-rices pour résoudre ces anomalies par le biais d'actions de compensation peut alors entraver la collaboration.
        Pour reprendre l'exemple de l'édition collaborative de texte, nous pouvons constater de tels cas suite à de la duplication de contenu et/ou l'entrelacement de mots, phrases voire paragraphes nuisant à la clarté et correction du texte.
        Il convient alors de s'interroger sur le bien-fondé de l'utilisation de mécanismes de résolutions de conflits automatiques pour intégrer un ensemble de modifications dans l'ensemble des situations.
    \item Ainsi, pour répondre aux limites des mécanismes de résolution conflits automatiques dans les systèmes \ac{P2P} à large échelle, \ie l'augmentation de leur surcoût et la pertinence de leur résultat, nous souhaitons proposer une approche combinant un ou des mécanismes de résolution de conflits automatiques avec un ou des mécanismes de résolution de conflits manuels.
        L'idée derrière cette approche est de faire varier le mécanisme de résolution de conflits utilisé pour intégrer des modifications.
        Le choix du mécanisme de résolution de conflits utilisé peut se faire à partir de la valeur d'une distance calculée entre la version courante de la donnée répliquée et celle de la génération de la modification à intégrer, ou d'une évaluation de la qualité du résultat de l'intégration de la modification.
        Par exemple :
        \begin{enumerate}
            \item Si la distance calculée se trouve dans un intervalle de valeurs pour lequel nous disposons d'un mécanisme de résolution de conflits automatique satisfaisant, utiliser ce dernier.
                Ainsi, nous pouvons envisager de reposer sur plusieurs mécanismes de résolution de conflits automatiques, de plus en plus complexes et pertinents mais coûteux, sans dégrader les performances du système dans le cas de base.
            \item Si la distance calculée dépasse la distance seuil, \ie que nous ne disposons plus à ce stade de mécanismes de résolution de conflits automatiques satisfaisants, faire intervenir les utilisateur-rices par le biais d'un mécanisme de résolution de conflits manuel.
                L'utilisation d'un mécanisme manuel n'exclut cependant pas tout pré-travail de notre part pour réduire la charge de travail des utilisateur-rices dans le processus de fusion.
        \end{enumerate}
        Dans un premier temps, cette approche pourra se focaliser sur un type d'application spécifique, \eg l'édition collaborative de texte.
    \item Cette approche nous permettrait de répondre aux limites soulevées précédemment.
        En effet, elle permettrait de limiter la génération d'anomalies par le mécanisme de résolution de conflits automatique en faisant intervenir les utilisateur-rices.
        Puis, puisque nous déléguons aux utilisateur-rices l'intégration des modifications à partir d'une distance seuil, nous pouvons dès lors reconsidérer les métadonnées conservées par les noeuds pour les mécanismes de résolution de conflits automatiques.
        Notamment, nous pouvons identifier les noeuds se trouvant au-delà de cette distance seuil d'après leur dernier contexte causal connu et ne plus les prendre en compte pour le calcul de la stabilité causal.
        Cette approche permettrait donc de réduire le surcoût lié au contexte causal et limiter la perte de modifications, tout en prenant en considération l'ajout de travail aux utilisateur-rices.
    \item Pour mener à bien ce travail, il conviendra tout d'abord de définir la notion de distance entre versions de la donnée répliquée.
        Nous envisageons de baser cette dernière sur les deux aspects temporel et spatial, \ie en utilisant la distance entre contextes causaux et la distance entre contenus.
        Dans le cadre de l'édition collaborative, nous pourrons pour cela nous baser sur les travaux existants pour évaluer la distance entre deux textes.
        \mnnote{TODO: Insérer refs distance de Hamming, Levenstein, String-to-string correction problem (Tichy et al)}
    \item Il conviendra ensuite de déterminer comment établir la valeur seuil à partir de laquelle la distance entre textes est jugée trop importante.
        Les approches d'évaluation de la qualité du résultat pourront être utilisées pour déterminer un couple $\langle \text{méthode de calcul de la distance}, \text{valeur de distance} \rangle$ spécifiant les cas pour lesquels les méthodes de résolution de conflits automatiques ne produisent plus un résultat satisfaisant.
        \mnnote{TODO: Insérer refs travaux Claudia et Vinh}
        Le couple obtenu pourra ensuite être confirmé par le biais d'expériences utilisateurs inspirées de \cite{2014-effect-delay-collaborative-editing-ignat,2015-cope-delay-collaborative-note-taking-ignat}.
    \item Finalement, il conviendra de proposer un mécanisme de résolution de conflits adapté pour gérer les éventuelles fusions d'une même modification de façon concurrente par un mécanisme automatique et par un mécanisme manuel, ou à défaut un mécanisme de conscience de groupe invitant les utilisateur-rices à effectuer des actions de compensation.
\end{itemize}


\subsection{Étude comparative des différents modèles de synchronisation pour \acp{CRDT}}
\begin{itemize}
    \item La spécification récente des Delta-based CRDTs .
      Ce nouveau type de CRDTs se base sur celui des State-based CRDTs.
      Partage donc les mêmes pré-requis :
      \begin{itemize}
        \item États du type de données répliqué forment un sup-demi-treillis
        \item Modifications locales entraînent une inflation de l'état
        \item Possède une fonction de \texttt{merge}, permettant de fusionner deux états S et S', et qui
        \begin{itemize}
          \item Est associative, commutative et idempotente
          \item Retourne S", la \ac{LUB} de S et S' (\ie $\nexists S''' \cdot merge(S, S') < S''' < S''$)
        \end{itemize}
      \end{itemize}
      Et bénéficie de son principal avantage : synchronisation possible entre deux pairs en fusionnant leur états, peu importe le nombre de modifications les séparant.
    \item Spécificité des Delta-based CRDTs est de proposer une synchronisation par différence d'états.
      Plutôt que de diffuser l'entièreté de l'état pour permettre aux autres pairs de se mettre à jour, idée est de seulement transmettre la partie de l'état ayant été mise à jour.
      Correspond à un élément irréductible du sup-demi-treillis.
      Permet ainsi de mettre en place une synchronisation en temps réel de manière efficace.
      Et d'utiliser la synchronisation par fusion d'états complets pour compenser les défaillances du réseau
    \item Ainsi, ce nouveau type de CRDTs semble allier le meilleur des deux mondes :
      \begin{itemize}
        \item Absence de contrainte sur le réseau autre que la livraison à terme
        \item Propagation possible en temps réel des modifications
      \end{itemize}
      Semble donc être une solution universelle :
      \begin{itemize}
        \item Utilisable peu importe la fiabilité réseau à disposition
        \item Empreinte réseau du même ordre de grandeur qu'un Op-based CRDT
        \item Utilisable peu importe la fréquence de synchronisation désirée
      \end{itemize}
      Pose la question de l'intérêt des autres types de CRDTs.
    \item Delta-based CRDT est un State-based CRDT dont on a identifié les éléments irréductibles et qui utilise ces derniers pour la propagation des modifications plutôt que l'état complet.
      Famille des State-based CRDTs semble donc rendue obsolète par celle des Delta-based CRDTs.
      À confirmer.
    \item Les Op-based CRDTs proposent une spécification différente du type répliqué de leur équivalent Delta-based, généralement plus simple.
      À première vue, famille des Op-based CRDTs semble donc avoir la simplicité comme avantage par rapport à celle des Delta-based CRDTs.
      S'agit d'un paramètre difficilement mesurable et auquel on peut objecter si on considère qu'un Op-based CRDT s'accompagne d'une couche livraison de messages, qui cache sa part de complexité.
      Intéressant d'étudier si la spécification différente des Op-based CRDTs présente d'autres avantages par rapport aux Delta-based CRDTs : performances (temps d'intégration des modifications, délai de convergence...), fonctionnalités spécifiques (composition, undo...)
    \item But serait de fournir des guidelines sur la famille de CRDT à adopter en fonction du cas d'utilisation.
  \end{itemize}


\subsection{Approfondissement du patron de conception de Pure Operation-Based \acp{CRDT}}
Plusieurs approches ont été proposées dans la littérature pour guider la conception de \acp{CRDT} :
\begin{enumerate}
    \item L'utilisation de la théorie des treillis pour la conception de \acp{CRDT} synchronisés par états et par différences d'états \cite{shapiro_2011_crdt,enes2019}.
    \item L'utilisation d'un journal partiellement ordonné des opérations, nommé PO-Log, pour la conception de \acp{CRDT} synchronisés par opérations \cite{baquero2017pure}.
\end{enumerate}
Cependant, ce framework proposé par \cite{baquero2017pure} souffre de plusieurs limitations.
Nous souhaitons donc proposer un nouveau framework pour la conception de \acp{CRDT} synchronisés par opérations, en nous basant sur ce dernier.\\

Le framework proposé dans \cite{baquero2017pure} possède plusieurs objectifs :
\begin{enumerate}
    \item Proposer une approche partiellement générique pour définir un \ac{CRDT} synchronisé par opérations.
    \item Factoriser les métadonnées utilisées par le \ac{CRDT} pour le mécanisme de résolution de conflits, notamment pour identifier les éléments, et celles utilisées par la couche livraison, notamment pour identifier les opérations.
    \item Inclure des mécanismes de \ac{GC} de ces métadonnées pour réduire la taille de l'état.
\end{enumerate}

Pour cela, les auteurs se limitent aux \acp{CRDT} purs synchronisés par opérations, \ie les \acp{CRDT} dont les modifications enrichies de leurs arguments et d'une estampille fournie par la couche de livraison des messages sont commutatives.
Pour ces \acp{CRDT}, les auteurs proposent un framework générique permettant leur spécification sous la forme d'un PO-Log.
Les auteurs associent le PO-Log à une couche de livraison \ac{RCB} des opérations.

Les auteurs définissent ensuite le concept de stabilité causale.
Ce concept leur permet de retirer les métadonnées de causalité des opérations du PO-Log lorsque celles-ci sont déterminées comme étant causalement stables.

Finalement, les auteurs définissent un ensemble de relations, spécifiques à chaque \ac{CRDT}, qui permettent d'exprimer la \emph{redondance causale}.
La redondance causale permet de spécifier quand retirer une opération du PO-Log, car rendue obsolète par une autre opération.\\

Comme évoqué précédemment, cette approche souffre toutefois de plusieurs limites.
Tout d'abord, elle repose sur l'utilisation d'une couche de livraison \ac{RCB}.
Cette couche satisfait le modèle de livraison causale.
Mais pour rappel, ce modèle induit l'ajout de données de causalité précises à chaque opération, sous la forme d'un vecteur de version ou d'une barrière causale.
Nous jugeons ce modèle trop coûteux pour les systèmes \ac{P2P} dynamiques à large échelle sujets au churn.

En plus du coût induit en termes de métadonnées et de bande-passante, le modèle de livraison causale peut aussi introduire un délai superflu dans la livraison des opérations.
En effet, ce modèle impose que tous les messages précédant un nouveau message d'après la relation \hb soient eux-mêmes livrés avant de livrer ce dernier.
Il en résulte que des opérations peuvent être mises en attente par la couche livraison, \eg suite à la perte d'une de leurs dépendances d'après la relation \hb, alors que leurs dépendances réelles ont déjà été livrées et que les opérations sont de fait intégrables en l'état.
Plusieurs travaux \cite{2020-flec-bauwens,2021-improving-reactivity-pure-op-based-crdts-bauwens} ont noté ce problème.
Pour y répondre et ainsi améliorer la réactivité du framework Pure Operation-Based, ils proposent d'exposer les opérations mises en attente par la couche livraison au \ac{CRDT}.
Bien que fonctionnelle, cette approche induit toujours le coût d'une couche de livraison respectant le modèle de livraison causale et nous fait considérer la raison de ce coût, le modèle de livraison n'étant dès lors plus respecté.

Ensuite, ce framework impose que la modification \textbf{prepare} ne puisse pas inspecter l'état courant du noeud.
Cette contrainte est compatible avec les \acp{CRDT} pour les types de données simples qui sont considérés dans \cite{baquero2017pure}, \eg le Compteur ou l'Ensemble.
Elle empêche cependant l'expression de \acp{CRDT} pour des types de données plus complexes, \eg la Séquence ou le Graphe.
\mnnote{TODO: À confirmer pour le graphe}
Nous jugeons dommageable qu'un framework pour la conception de \acp{CRDT} limite de la sorte son champ d'application.

Finalement, les auteurs ne considèrent que des types de données avec des modifications à granularité fixe.
Ainsi, ils définissent la notion de redondance causale en se limitant à ce type de modifications.
Par exemple, ils définissent que la suppression d'un élément d'un ensemble rend obsolète les ajouts précédents de cet élément.
Cependant, dans le cadre d'autres types de données, \eg la Séquence, une modification peut concerner un ensemble d'éléments de taille variable.
Une opération peut donc être rendue obsolète non pas par une opération, mais par un ensemble d'opérations.
Par exemple, les suppressions d'éléments formant une sous-chaîne rendent obsolète l'insertion de cette sous-chaîne.
Ainsi, la notion de redondance causale est incomplète et souffre de l'absence d'une notion d'obsolescence partielle d'une opération.\\

Pour répondre aux différents problèmes soulevés, nous souhaitons proposer un nouveau framework en nous basant sur \cite{baquero2017pure}.
Nos objectifs sont les suivants :
\begin{enumerate}
    \item Proposer un framework mettant en lumière la présence et le rôle de deux modèles de livraison :
        \begin{enumerate}
            \item Le modèle de livraison minimal requis par le \ac{CRDT} pour assurer la convergence forte à terme \cite{shapiro_2011_crdt}.
            \item Le modèle de livraison employé par le système qui utilise le \ac{CRDT}.
                Ce second modèle de livraison est une stratégie permettant au système de respecter un modèle de cohérence donné et régissant les règles de compaction de l'état.
                Il doit être égal ou plus contraint que modèle de livraison minimal du \ac{CRDT} et peut être amené à évoluer en fonction de l'état du système et de ses besoins.
                Par exemple, un système pourrait par défaut utiliser le modèle de livraison causale pour assurer le modèle de cohérence causal.
                Puis, lorsque le nombre de noeuds atteint un seuil donné et que le coût de la livraison causale devient trop élevé, le système pourrait passer au modèle de livraison FIFO pour assurer le modèle de cohérence PRAM afin de réduire les coûts en bande-passante.
        \end{enumerate}
    \item Étendre la notion de redondance causale pour prendre en compte la redondance partielle des opérations.
        De plus, nous souhaitons rendre cette notion accessible à la couche de livraison, pour détecter au plus tôt les opérations désormais obsolètes et prévenir leur diffusion.
    \item Identifier et classifier les mécanismes de résolution de conflits, pour déterminer lesquels sont indépendants de l'état courant pour la génération des opérations et lesquels nécessitent d'inspecter l'état courant dans \textbf{prepare}.
\end{enumerate}


% \subsection{Conduction d'expériences utilisateurs d'édition collaborative}
% \begin{itemize}
    \item Absence d'un dataset réel et réutilisable sur les sessions d'édition collaborative
    \item Généralement, expériences utilisent données d'articles de Wikipédia \mnnote{TODO: Revoir références, mais me semble que c'est celui utilisé pour Logoot, LogootSplit et RGASplit entre autres}.
      Mais ces données correspondent à une exécution séquentielle, \ie aucune édition concurrente ne peut être réalisée avec le système de résolution de conflits de Wikipédia.
      \mnnote{TODO:
        Me semble que Kleppmann a aussi utilisé et mis à disposition ses traces correspondant à la rédaction d'un de ses articles.
        Mais que cet article n'était rédigé que par lui.
        Peu de chances de présence d'éditions concurrentes.
        À retrouver et vérifier.
      }
    \item Inspiré par expériences de Claudia, pourrait mener des sessions d'édition collaborative sur des outils orchestrés pour produire ce dataset
    \item Devrait rendre ce dataset agnostique de l'approche choisie pour la résolution automatique de conflits
    \item Absence de retours sur les collaborations à grande échelle
    \item Comment on collabore lorsque plusieurs centaines d'utilisateur-rices ?
  \end{itemize}


% \subsection{Comparaison des mécanismes de synchronisation}
% Serait intéressant de comparer à d'autres méthodes de synchronisation : mécanisme d'anti-entropie basé sur un Merkle Tree\cite{2007-dynamo, 2015-approximate-hash-based-set-reconciliation, 2017-anti-entropy-without-merkle-trees}, synchronisation par états (state/delta-based \acp{CRDT}).
Dans le cadre des Delta-based \acp{CRDT}, pourrait évaluer un protocole de diffusion épidémique des deltas comme celui proposé par SWIM\cite{swim2002}.


% \subsection{Contrôle d'accès}
% \begin{itemize}
    \item Pour le moment, n'importe quel utilisateur ayant l'URL du document peut y accéder dans MUTE
    \item Pour des raisons de confidentialité, peut vouloir contrôler quels utilisateurs ont accès à un document
    \item Nécessite l'implémentation de liste de contrôle d'accès
    \item Mais s'agit d'une tâche complexe dans le cadre d'un système distribué
    \item Peut s'inspirer des travaux réalisés au sein de la communautée \acp{CRDT} \cite{2021-access-control-crdts, 2022-dist-access-control-pa} pour cela
  \end{itemize}


% \subsection{Détection et éviction de pairs malhonnêtes}
% \begin{itemize}
    \item À l'heure actuelle, MUTE suppose qu'ensemble des collaborateurs honnêtes
    \item Vulnérable à plusieurs types d'attaques par des adversaires byzantins, tel que l'équivoque
    \item Ce type d'attaque peut provoquer des divergences durables et faire échouer des collaborations
    \item \textcite{2021-these-vic} propose un mécanisme permettant de maintenir des logs authentifiés dans un système distribué
    \item Les logs authentifiés permettent de mettre en lumière les comportements malveillants des adversaires et de borner le nombre d'actions malveillantes qu'ils peuvent effectuer avant d'être évincé
    \item Implémenter ce mécanisme permettrait de rendre compatible MUTE avec des environnements avec adversaires byzantins
    \item Nécessiterait tout de même de faire évoluer le \ac{CRDT} pour résoudre les équivoques détectés
  \end{itemize}


% \subsection{Vecteur de version \emph{epoch-based}}
% \begin{itemize}
    \item S'agit d'une structure primordiale dans les systèmes distribués, dont pouvons difficilement nous passer.
      Utilisé notamment pour représenter le contexte causal de l'état d'un noeud, nécessaire pour :
      \begin{enumerate}
        \item Déterminer quelles opérations ont été observées (anti-entropie et couche de livraison)
        \item Déterminer quelles opérations ont observé les autres noeuds (stabilité causale)
        \item Préciser les dépendances causales d'un message
      \end{enumerate}
    \item Comme présenté précédemment, nous utilisons plusieurs vecteurs pour représenter des données dans l'application MUTE
    \item Notamment pour le vecteur de version, utilisé pour respecter le modèle de livraison requis par le \ac{CRDT}
    \item Et pour la liste des collaborateur-rices, utilisé pour offrir des informations nécessaires à la conscience de groupe aux utilisateurs
    \item Ces vecteurs sont maintenus localement par chacun des noeuds et sont échangés de manière périodique
    \item Cependant, la taille de ces vecteurs croit de manière linéaire au nombre de noeuds impliqués dans la collaboration
    \item Les systèmes \ac{P2P} à large échelle sont sujets au \emph{churn}
    \item Dans le cadre d'un tel système, ces structures croissent de manière non-bornée
    \item Ceci pose un problème de performances, notamment d'un point de vue consommation en bande-passante
    \item Cependant, même si on observe un grand nombre de pairs différents dans le cadre d'une collaboration à large échelle
    \item Intuition est qu'une collaboration repose en fait sur un petit noyau de collaborateur-rices principaux
    \item Et que majorité des collaborateur-rices se connectent de manière éphèmère
    \item Serait intéressant de pouvoir réduire la taille des vecteurs en oubliant les collaborateur-rices éphèmères
    \item Dynamo\cite{2007-dynamo} tronque le vecteur de version lorsqu'il dépasse une taille seuil
    \item Conduit alors à une perte d'informations
    \item Pour la liste des collaborateur-rices, approche peut être adoptée (pas forcément gênant de limiter à 100 la taille de la liste)
    \item Mais pour vecteur de version, conduirait à une relivraison d'opérations déjà observées
    \item Approche donc pas applicable pour cette partie
    \item Autre approche possible est de réutiliser le système d'époque
    \item Idée serait de ACK un vecteur avec un changement d'époque
    \item Et de ne diffuser à partir de là que les différences
    \item Un mécanisme de transformation (une simple soustraction) permettrait d'obtenir le dot dans la nouvelle époque d'une opération concurrente au renommage
    \item Peut facilement mettre en place un mécanisme d'inversion du renommage (une simple addition) pour revenir à une époque précédente
    \item Et ainsi pouvoir circuler librement dans l'arbre des époques et gérer les opérations \emph{rename} concurrentes
    \item Serait intéressant d'étudier si on peut aller plus loin dans le cadre de cette structure de données et notamment rendre commutatives les opérations de renommage concurrentes
  \end{itemize}


% \subsection{Rôles et places des bots dans systèmes collaboratifs}
% \begin{itemize}
    \item Stockage du document pour améliorer sa disponibilité
    \item Overleaf en P2P ?
    \item Comment réinsérer des bots dans la collaboration sans en faire des éléments centraux, sans créer des failles de confidentialité, et tout en rendant ces fonctionnalités accessibles ?
  \end{itemize}


\Annex{Entrelacement d'insertions concurrentes dans Treedoc}

\label{app:treedoc-interleaving}

\begin{figure}[!ht]

  \centering
  \resizebox{\columnwidth}{!}{
    \includegraphics{img/contre-exemple-treedoc}
  }
  \caption{Modifications concurrentes d'une séquence Treedoc résultant en un entrelacement}
\end{figure}

\mnnote{
  TODO: Réaliser au propre contre-exemple.
  Nécessite que $d_E < d_O$, inverser A et B histoire d'éviter toute confusion.
  En soi, C pas nécessaire, à voir si le conserve.
}

\Annex{Algorithmes \textsc{renameId}}

\label{app:rename-id}

\begin{algorithm}[!ht]
  \footnotesize
  \begin{algorithmic}
      \Function{renIdLessThanFirstId}{id, newFirstId}
      \If{id < newFirstId}
          \State \Return id
      \Else
          \State pos $\gets$ position(newFirstId)
          \State nId $\gets$ nodeId(newFirstId)
          \State nSeq $\gets$ nodeSeq(newFirstId)
          \State predNewFirstId $\gets$ \new~Id(pos, nId, nSeq, -1)
          \\
          \State \Return concat(predNewFirstId, id)
          \EndIf
      \EndFunction
      \\
      \Function{renIdGreaterThanLastId}{id, newLastId}
          \If{id < newLastId}
              \State \Return concat(newLastId, id)
          \Else
              \State \Return id
          \EndIf
      \EndFunction
  \end{algorithmic}
  \caption{Remaining functions to rename an identifier}
  \label{alg:appendix-rename-id}
\end{algorithm}

\Annex{Algorithmes \textsc{revertRenameId}}

\label{app:revert-rename-id}

\begin{algorithm}[!ht]
  \footnotesize
  \begin{algorithmic}
      \Function{revRenIdLessThanNewFirstId}{id, firstId, newFirstId}
          \State predNewFirstId $\gets$ createIdFromBase(newFirstId, -1)
          \If{isPrefix(predNewFirstId, id)}
              \State tail $\gets$ getTail(id, 1)
              \If{tail < firstId}
                  \State \Return tail
              \Else
                  \State \Comment{$id$ has been inserted causally after the \emph{rename} op}
                  \State offset $\gets$ getLastOffset(firstId)
                  \State predFirstId $\gets$ createIdFromBase(firstId, offset)
                  \State \Return concat(predFirstId, MAX\_TUPLE, tail)
              \EndIf
          \Else
              \State \Return id
          \EndIf
      \EndFunction
      \\
      \Function{revRenIdGreaterThanNewLastId}{id, lastId}
          \If{id < lastId}
              \State \Comment{$id$ has been inserted causally after the \emph{rename} op}
              \State \Return concat(lastId, MIN\_TUPLE, id)
          \ElsIf{isPrefix(newLastId, id)}
              \State tail $\gets$ getTail(id, 1)
              \If{tail < lastId}
                  \State \Comment{$id$ has been inserted causally after the \emph{rename} op}
                  \State \Return concat(lastId, MIN\_TUPLE, tail)
              \ElsIf{tail < newLastId}
                  \State \Return tail
              \Else
                  \State \Comment{$id$ has been inserted causally after the \emph{rename} op}
                  \State \Return id
              \EndIf
          \Else
              \State \Return id
          \EndIf
      \EndFunction
  \end{algorithmic}
  \caption{Remaining functions to revert an identifier renaming}
  \label{alg:appendix-revert-rename-id}
\end{algorithm}

%
%%-------------------------------------------------------------------
%%                         Le glossaire
%%-------------------------------------------------------------------
%\BeginGloWith{Voici un glossaire tout-à-fait fictif,
%              introduit par un texte sur toute la largeur
%              des deux colonnes.}
%\twocolumn
%\PrintGlossary

%-------------------------------------------------------------------
%              L'index (toujours sur deux colonnes)
%-------------------------------------------------------------------
\BeginIndWith{Voici un index}
\PrintIndex

\onecolumn

%-------------------------------------------------------------------
%                       La bibliographie
%-------------------------------------------------------------------

% La bibliographie (comme d'habitude)

%\nocite{*}
%\bibliographystyle{named}

\printbibliography

%-------------------------------------------------------------------
%                          Les résumés
%-------------------------------------------------------------------
% (si le résumé apparaît sur une colonne étroite, avec la
% bibliographie à gauche, c'est sans doute parce que vous avez
% oublié de générer les fichiers d'index et de glossaire...)

\NumberAbstractPages
\begin{ThesisAbstract}
  \begin{FrenchAbstract}
    Afin d'assurer leur haute disponibilité, les systèmes distribués à large échelle se doivent de répliquer leurs données tout en minimisant les coordinations nécessaires entre noeuds.
    Pour concevoir de tels systèmes, la littérature et l'industrie adoptent de plus en plus l'utilisation de types de données répliquées sans conflits (CRDTs).
    Les CRDTs sont des types de données qui offrent des comportements similaires aux types existants, tel l'Ensemble ou la Séquence.
    Ils se distinguent cependant des types traditionnels par leur spécification, qui supporte nativement les modifications concurrentes.
    À cette fin, les CRDTs incorporent un mécanisme de résolution de conflits au sein de leur spécification.

    Afin de résoudre les conflits de manière déterministe, les CRDTs associent généralement des identifiants aux éléments stockés au sein de la structure de données.
    Les identifiants doivent respecter un ensemble de contraintes en fonction du CRDT, telles que l'unicité ou l'appartenance à un ordre dense.
    Ces contraintes empêchent de borner la taille des identifiants.
    La taille des identifiants utilisés croît alors continuellement avec le nombre de modifications effectuées, aggravant le surcoût lié à l'utilisation des CRDTs par rapport aux structures de données traditionnelles.
    Le but de cette thèse est de proposer des solutions pour pallier ce problème.

    Nous présentons dans cette thèse deux contributions visant à répondre à ce problème :
    \begin{enumerate*}
      \item Un nouveau CRDT pour Séquence, RenamableLogootSplit, qui intègre un mécanisme de renommage à sa spécification.
      Ce mécanisme de renommage permet aux noeuds du système de réattribuer des identifiants de taille minimale aux éléments de la séquence.
      Cependant, cette première version requiert une coordination entre les noeuds pour effectuer un renommage.
      L'évaluation expérimentale montre que le mécanisme de renommage permet de réinitialiser à chaque renommage le surcoût lié à l'utilisation du CRDT.
      \item Une seconde version de RenamableLogootSplit conçue pour une utilisation dans un système distribué.
      Cette nouvelle version permet aux noeuds de déclencher un renommage sans coordination préalable.
      L'évaluation expérimentale montre que cette nouvelle version présente un surcoût temporaire en cas de renommages concurrents, mais que ce surcoût est à terme.
    \end{enumerate*}
    \KeyWords{CRDTs, édition collaborative en temps réel, cohérence à terme, optimisation mémoire, performance}
  \end{FrenchAbstract}
  \begin{EnglishAbstract}
    \KeyWords{CRDTs, real-time collaborative editing, eventual consistency, memory-wise optimisation, performance}
  \end{EnglishAbstract}
\end{ThesisAbstract}


\end{document}



