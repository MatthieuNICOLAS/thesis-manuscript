\documentclass[12pt]{thesul}
%----------------------------------------------------------------------
%                               Packages
%----------------------------------------------------------------------
\usepackage[french]{babel}
\usepackage{acronym} % \ac[p], \acl[p], \acs[p], \acf[p]
\usepackage[maxbibnames=99,maxcitenames=1,sorting=none]{biblatex}
\bibliography{biblio.bib}
\usepackage{booktabs} % \toprule, \midrule, \cmidrule, \bottomrule
\usepackage{cancel} % \cancel
\usepackage{caption}
\usepackage{csquotes}
\usepackage{hyperref}
\hypersetup{hidelinks}
\usepackage[inline]{enumitem}
\setlist[enumerate]{label=(\roman*)} %% <- set the base level label separately
\usepackage{import} % \import
\usepackage[french]{minitoc}

\usepackage{xcolor}
\AtBeginDocument{
\definecolor{lfsgreen}{RGB}{31,160,31}
\definecolor{lfsorange}{RGB}{255,178,2}
\definecolor{lfsred}{RGB}{160,30,31}

% white
\definecolor{uclwhite}{rgb}{1, 1, 1}

%   UCL style guide colours.
%   Number refer to level of tint.
%   100%    1
%    70%    2
%    50%    3
%    20%    4

 % UCL style guide dk purple
\definecolor{ucl1dkpurple}{RGB}{82,66,91}
\definecolor{ucl2dkpurple}{RGB}{134,122,140}
\definecolor{ucl3dkpurple}{RGB}{168,160,173}
\definecolor{ucl4dkpurple}{RGB}{220,217,222}

 % UCL style guide dk red
\definecolor{ucl1dkred}{RGB}{90,27,49}
\definecolor{ucl2dkred}{RGB}{139,95,110}
\definecolor{ucl3dkred}{RGB}{172,141,152}
\definecolor{ucl4dkred}{RGB}{222,209,214}

 % UCL style guide dk blue
\definecolor{ucl1dkblue}{RGB}{0,67,89}
\definecolor{ucl2dkblue}{RGB}{76,123,138}
\definecolor{ucl3dkblue}{RGB}{127,161,172}
\definecolor{ucl4dkblue}{RGB}{204,217,222}

 % UCL style guide dk green
\definecolor{ucl1dkgreen}{RGB}{75,70,32}
\definecolor{ucl2dkgreen}{RGB}{129,125,98}
\definecolor{ucl3dkgreen}{RGB}{165,162,143}
\definecolor{ucl4dkgreen}{RGB}{219,218,210}

 % UCL style guide black
\definecolor{ucl1black}{RGB}{0,0,0}
\definecolor{ucl2black}{RGB}{75,75,75}
\definecolor{ucl3black}{RGB}{128,128,128}
\definecolor{ucl4black}{RGB}{205,205,205}

 % UCL style guide pink
\definecolor{ucl1pink}{RGB}{145,24,83}
\definecolor{ucl2pink}{RGB}{178,93,134}
\definecolor{ucl3pink}{RGB}{200,139,169}
\definecolor{ucl4pink}{RGB}{233,209,221}

 % UCL style guide md red
\definecolor{ucl1mdred}{RGB}{195,58,45}
\definecolor{ucl2mdred}{RGB}{213,117,108}
\definecolor{ucl3mdred}{RGB}{225,156,150}
\definecolor{ucl4mdred}{RGB}{243,216,213}

 % UCL style guide md blue
\definecolor{ucl1mdblue}{RGB}{69,156,189}
\definecolor{ucl2mdblue}{RGB}{124,186,209}
\definecolor{ucl3mdblue}{RGB}{162,205,222}
\definecolor{ucl4mdblue}{RGB}{218,235,242}

 % UCL style guide md green
\definecolor{ucl1mdgreen}{RGB}{130,141,55}
\definecolor{ucl2mdgreen}{RGB}{167,175,115}
\definecolor{ucl3mdgreen}{RGB}{192,198,155}
\definecolor{ucl4mdgreen}{RGB}{230,232,215}

 % UCL style guide orange
\definecolor{ucl1orange}{RGB}{215,123,35}
\definecolor{ucl2orange}{RGB}{227,162,101}
\definecolor{ucl3orange}{RGB}{235,189,145}
\definecolor{ucl4orange}{RGB}{247,229,211}

  % UCL style guide lt purple
\definecolor{ucl1ltpurple}{RGB}{191,175,188}
\definecolor{ucl2ltpurple}{RGB}{210,199,208}
\definecolor{ucl3ltpurple}{RGB}{223,215,221}
\definecolor{ucl4ltpurple}{RGB}{242,239,242}

 % UCL style guide yellow
\definecolor{ucl1yellow}{RGB}{229,175,0}
\definecolor{ucl2yellow}{RGB}{237,199,76}
\definecolor{ucl3yellow}{RGB}{242,215,127}
\definecolor{ucl4yellow}{RGB}{250,239,204}

 % UCL style guide lt blue
\definecolor{ucl1ltblue}{RGB}{168,192,209}
\definecolor{ucl2ltblue}{RGB}{194,211,223}
\definecolor{ucl3ltblue}{RGB}{211,223,232}
\definecolor{ucl4ltblue}{RGB}{238,242,246}

% UCL style guide brt green
\definecolor{ucl1brtgreen}{RGB}{204,209,88}
\definecolor{ucl2brtgreen}{RGB}{219,223,138}
\definecolor{ucl3brtgreen}{RGB}{229,232,171}
\definecolor{ucl4brtgreen}{RGB}{245,246,222}

% UCL style guide stone
\definecolor{ucl1stone}{RGB}{217,214,204}
\definecolor{ucl2stone}{RGB}{228,226,219}
\definecolor{ucl3stone}{RGB}{236,234,229}
\definecolor{ucl4stone}{RGB}{255,255,255}

% UCL style guide lt green
\definecolor{ucl1ltgreen}{RGB}{185,193,147}
\definecolor{ucl2ltgreen}{RGB}{206,211,179}
\definecolor{ucl3ltgreen}{RGB}{220,224,201}
\definecolor{ucl4ltgreen}{RGB}{241,243,233}

\definecolor{darkgreen}{RGB}{75,70,32}
\definecolor{darkblue}{RGB}{0,67,89}
\definecolor{mydarkblue}{RGB}{76,123,138}
\definecolor{mydarkblueid}{RGB}{0,67,89}
\definecolor{mylightblue}{RGB}{168,192,209}
\definecolor{mydarkorange}{RGB}{215,123,35}
\definecolor{mylightorange}{RGB}{227,162,101}
\definecolor{mydarkred}{RGB}{90,27,49}
\definecolor{mydarkpurple}{RGB}{134,122,140}
\definecolor{mydarkpurpleid}{RGB}{82,66,91}
}

\usepackage{amssymb}
\usepackage{amsmath}
\usepackage{amsthm}
% \interdisplaylinepenalty=2500
\theoremstyle{definition}
\newtheorem{definition}{Définition}
\newtheorem{subdefinition}{Définition}[definition]
\newtheorem{myrule}{Règle}
\newtheorem{property}{Propriété}
\newtheorem{subproperty}{Propriété}[property]
\usepackage{MnSymbol} % \dashrightarrow

\usepackage{algorithm, algpseudocode}
\floatname{algorithm}{Algorithme} % Renomme caption de "Algorithm" -> "Algorithme"

\newcommand\CONDITION[2]%
  {\begin{tabular}[t]{@{}l@{}l@{}}
     #1&#2
   \end{tabular}%
  }
  \algdef{SE}[WHILE]{While}{EndWhile}[1]%
  {\algorithmicwhile\ \CONDITION{#1}{\ \algorithmicdo}}%
  {\algorithmicend\ \algorithmicwhile}
\algdef{SE}[FOR]{For}{EndFor}[1]%
  {\algorithmicfor\ \CONDITION{#1}{\ \algorithmicdo}}%
  {\algorithmicend\ \algorithmicfor}
\algdef{S}[FOR]{ForAll}[1]%
  {\algorithmicforall\ \CONDITION{#1}{\ \algorithmicdo}}
\algdef{SE}[REPEAT]{Repeat}{Until}{\algorithmicrepeat}[1]%
  {\algorithmicuntil\ \CONDITION{#1}{}}
\algdef{SE}[IF]{If}{EndIf}[1]%
  {\algorithmicif\ \CONDITION{#1}{\ \algorithmicthen}}%
  {\algorithmicend\ \algorithmicif}%
\algdef{C}[IF]{IF}{ElsIf}[1]%
  {\algorithmicelse\ \algorithmicif\ \CONDITION{#1}{\ \algorithmicthen}}

\usepackage[inline,nomargin,index]{fixme}
\fxsetup{theme=color,mode=multiuser,inlineface=\itshape,envface=\itshape}
\FXRegisterAuthor{mn}{amn}{Matthieu}

\usepackage{tikz} % \begin{tikzpicture} \end{tikzpicture}
\usetikzlibrary{calc}
\usetikzlibrary{graphs}
\usetikzlibrary{quotes}
\usetikzlibrary{shapes.misc}

\usepackage[caption=false,font=footnotesize,labelfont=sf,textfont=sf]{subfig}

\usepackage{marvosym} % \Flatsteel
\usepackage{wasysym} % \checked

\usepackage{pifont} % \ding
\renewcommand{\checkmark}{\ding{51}}
\newcommand{\ballotx}{\ding{55}}

\usepackage{xspace} % \xspace

% Commands
%---------
\newcommand{\cf}[1]{(cf. \autoref{#1}, page \pageref{#1})}
\newcommand{\eg}{e.g.\xspace}
\newcommand{\ie}{c.-à-d.\xspace}

\newcommand{\hb}{\emph{happens-before}\xspace}

\newcommand{\inbb}[1]{\in \mathbb{#1}}
\newcommand{\new}{\textbf{new}}
\newcommand{\trm}[1]{\mathit{#1}}
\newcommand{\set}[1]{\left\{#1\right\}} % set brace notation

\newcommand{\betterid}[3]{\trm{#1}^{\trm{#2}}_{\trm{#3}}}
\newcommand{\id}[3]{$\trm{#1}^{\trm{#2}}_{\trm{#3}}$}
\newcommand{\epoch}[1]{$\varepsilon_{#1}$}
\newcommand{\betterepoch}[1]{\varepsilon_{#1}}
\newcommand{\lid}{<_{id}}
\newcommand{\leqid}{$\leq_{id}$~}
\newcommand{\lepoch}{$<_{\varepsilon}$~}
\newcommand{\betterelpoch}{<_{\varepsilon}}
\newcommand{\leqepoch}{$\leq_{\varepsilon}$~}
\newcommand{\ltuple}{<_{t}}

\newcommand{\botn}{\bot_\mathbb{N}}
\newcommand{\topn}{\top_\mathbb{N}}
\newcommand{\bott}{\bot_\mathbb{T}}
\newcommand{\topt}{\top_\mathbb{T}}
\newcommand{\logootuple}[1]{\langle \text{pos}_{#1},\text{nodeId}_{#1},\text{nodeSeq}_{#1} \rangle}
\newcommand{\logootsplituple}[1]{\langle \text{pos}_{#1},\text{nodeId}_{#1},\text{nodeSeq}_{#1},\text{offset}_{#1} \rangle}
\newcommand{\predNewFirstId}{\langle \text{pos},\text{nodeId},\text{nodeSeq},-1 \rangle}
\newcommand{\newFirstId}{\langle \text{pos},\text{nodeId},\text{nodeSeq},0 \rangle}
\newcommand{\newLastId}{\langle \text{pos},\text{nodeId},\text{nodeSeq},n-1 \rangle}
\newcommand{\newlogootsplituple}[1]{\langle \text{pos},\text{nodeId},\text{nodeSeq},{#1} \rangle}
\newcommand{\commentbott}{with $\bott$ the minimal tuple}
\newcommand{\commenttopt}{with $\topt$ the maximal tuple}

\newcommand{\bigO}[1]{$\mathcal{O}(#1)$}

\newcommand{\widthletter}{2em}
\newcommand{\widthblock}{3em}
\newcommand{\widthoriginepoch}{1.33em}
\newcommand{\widthepoch}{1.65em}

\newcommand{\withmathbreak}[1]{\parbox{\linewidth}{#1}}
\newcommand{\elt}{E}
\newcommand{\nat}{\mathbb{N}}


% Définit noms pour \autoref
%---------
\newcommand{\algorithmautorefname}{Algorithme}
\newcommand{\annexautorefname}{Annexe}
\newcommand{\definitionautorefname}{Définition}
\newcommand{\propertyautorefname}{Propriété}
\newcommand{\subfigureautorefname}{Figure}
\newcommand{\subpropertyautorefname}{Propriété}

% Tikz styles
\tikzset{
    common/.style={anchor=west, draw, rectangle, minimum height=6mm},
    letter/.style={common, minimum width=\widthletter},
    block/.style={common, minimum width=\widthblock},
    epoch/.style={letter, rounded rectangle, rounded rectangle east arc=0pt, minimum width=\widthepoch},
    point/.style={insert path={ node[scale=5*sqrt(\pgflinewidth)]{.} }},
    node/.style={draw, circle, minimum size=1em},
    op/.style={draw, circle, minimum size=2.7em},
    causalop/.style={op, double=white, inner sep=2pt},
    gc-rule-1/.style={dashed, thick, darkblue},
    gc-rule-2/.style={densely dotted, thick, darkgreen},
    cross/.style={
        path picture={
            \draw[ucl1dkred, very thick]
                (path picture bounding box.south east)--(path picture bounding box.north west)
                (path picture bounding box.south west)--(path picture bounding box.north east);
        }
    }
}


%-------------------------------------------------------------------
%                             Marges
%-------------------------------------------------------------------

% pour positionner les vraies marges:
%\SetRealMargins{1mm}{1mm}

%-------------------------------------------------------------------
%                             En-têtes
%-------------------------------------------------------------------

% Les en-têtes: quelques exemples
%\UppercaseHeadings
%\UnderlineHeadings
%\newcommand\bfheadings[1]{{\bf #1}}
%\FormatHeadingsWith{\bfheadings}
%\FormatHeadingsWith{\uppercase}
%\FormatHeadingsWith{\underline}
\newcommand\upun[1]{\uppercase{\underline{\underline{#1}}}}
\FormatHeadingsWith\upun

\newcommand\itheadings[1]{\textit{#1}}
\FormatHeadingsWith{\itheadings}

% pour avoir un trait sous l'en-tete:
\setlength{\HeadRuleWidth}{0.4pt}

%-------------------------------------------------------------------
%                         Les références
%-------------------------------------------------------------------

\NoChapterNumberInRef
\NoChapterPrefix

%-------------------------------------------------------------------
%                           Brouillons
%-------------------------------------------------------------------

% ceci ajoute une marque « brouillon » et la date
% \ThesisDraft

%-------------------------------------------------------------------
%                   Pour collecter un glossaire et un index
%-------------------------------------------------------------------

\makeglossary
\makeindex

%-------------------------------------------------------------------
%                           Acronymes
%-------------------------------------------------------------------

% Acronyms
% --------
% \input{assets/acronyms.tex}
\acrodef{ADT}[ADT]{Abstract Data Type}
\acrodefplural{ADT}[ADTs]{Abstract Data Types}
\acrodef{API}[API]{Application Programming Interface}
\acrodef{AW}[AW]{\emph{Add-Wins}}
\acrodef{CCI}[CCI]{Convergence, Causality preservation, Intention preservation}
\acrodef{CL}[CL]{\emph{Causal-Length}}
\acrodef{CRDT}[CRDT]{Conflict-free Replicated Data Type}
\acrodefplural{CRDT}[CRDTs]{Conflict-free Replicated Data Types}
\acrodef{DAG}[DAG]{Directed Acyclic Graph}
\acrodef{FIFO}[FIFO]{First In, First Out}
\acrodef{GC}[GC]{Garbage Collection}
\acrodef{IPFS}[IPFS]{InterPlanetary File System}
\acrodef{JIT}[JIT]{Just-In-Time}
\acrodef{LCA}[PPAC]{Plus Petit Ancêtre Commun}
\acrodef{LFS}[LFS]{Local-First Software}
\acrodef{LUB}[LUB]{Least Upper Bound}
\acrodef{LWW}[LWW]{\emph{Last-Writer-Wins}}
\acrodef{MUTE}[MUTE]{Multi User Text Editor}
\acrodef{MV}[MV]{\emph{Multi-Value}}
\acrodef{OC}[OC]{Commutativité des Opérations}
\acrodef{OT}[OT]{Operational Transformation}
\acrodefplural{OT}[OT]{Operational Transformations}
\acrodef{P2P}[P2P]{pair-à-pair}
\acrodef{PKI}[PKI]{Public Key Infrastructure}
\acrodef{PoC}[PoC]{Proof of Concept}
\acrodef{PT}[PT]{Transitivité de la Précédence}
\acrodef{RADT}[RADT]{Replicated Abstract Data Type}
\acrodefplural{RADT}[RADTs]{Replicated Abstract Data Types}
\acrodef{RCB}[RCB]{Reliable Causal Broadcast}
\acrodef{RGA}[RGA]{Replicated Growable Array}
\acrodef{RW}[RW]{\emph{Remove-Wins}}
\acrodef{SEC}[SEC]{Cohérence forte à terme}
\acrodef{SPOF}[SPOF]{Single Point Of Failure}
\acrodef{TTF}[TTF]{Tombstone Transformation Function}
\acrodefplural{TTF}[TTF]{Tombstone Transformation Functions}
\acrodef{WebRTC}[WebRTC]{Web Real-Time Communication}

%-------------------------------------------------------------------
%                           Couleurs
%-------------------------------------------------------------------

% \input{assets/colours.tex}

%-------------------------------------------------------------------
%                     Global custom tikz commands
%-------------------------------------------------------------------

% \input{assets/tikz_presets.tex}

\begin{document}


      \OddHead={{\leftmark\rightmark}{\hfil\slshape\rightmark}}
      \EvenHead={{\leftmark}{{\slshape\leftmark}\hfil}}
      \OddFoot={\hfil\thepage}
      \EvenFoot={\thepage\hfil}
      \pagestyle{ThesisHeadingsII}


%-------------------------------------------------------------------
%                          Encadrements
%-------------------------------------------------------------------

% encadre les chapitres dans la table des matières:
% (ces commandes doivent figurer apres \begin{document}

\FrameChaptersInToc
%\FramePartsInToc


%-------------------------------------------------------------------
%            Réinitialisation de la numérotation des chapitres
%-------------------------------------------------------------------

% Si la commande suivante est présente,
% elle doit figurer APRÈS \begin{document}
% et avant la première commande \part
\ResetChaptersAtParts

%-------------------------------------------------------------------
%               mini-tables des matières par chapitre
%-------------------------------------------------------------------

% préparer les mini-tables des matières par chapitre.
% (commande de minitoc.sty)
\dominitoc

%-------------------------------------------------------------------
%                         Page de titre:
%-------------------------------------------------------------------

\ThesisTitle{Ré-identification sans coordination dans les types de données répliquées sans conflits (CRDTs)}
\ThesisDate{16 Décembre 2022}
\ThesisAuthor{Matthieu Nicolas}

% Type de la these
\ThesisUL
% Jury:

% (ne pas mettre de \\ apres la dernière entree)

% Exemple de création d'une nouvelle catégorie dans le jury:

% \NewJuryCategory{Encadrants}{\it Encadrant :}
%                        {\it Encadrants :}

\def\blanc{\hspace*{1cm}}

\President    = {À déterminer}
\Rapporteurs  = {Hanifa Boucheneb & Professeure, Polytechnique Montréal\\
                 Davide Frey      & Chargé de recherche, HdR, Inria Rennes Bretagne-Atlantique}
\Examinateurs = {Hala Skaf-Molli  & Maîtresse de conférences, HdR, Nantes Université, LS2N\\
                 Stephan Merz     & Directeur de Recherche, Inria Nancy - Grand Est}
\Encadrants= {Olivier Perrin      & Professeur des Universités, Université de Lorraine, LORIA \\
              Gérald Oster        & Maître de conférences, Université de Lorraine, LORIA}

%\Invites=       {}

% Création de la page de titre:
\MakeThesisTitlePage

%-------------------------------------------------------------------


%-------------------------------------------------------------------
%                          remerciements
%-------------------------------------------------------------------

%\DontFrameThisInToc
\begin{ThesisAcknowledgments}
WIP
\end{ThesisAcknowledgments}

%-------------------------------------------------------------------
%                            dédicace
%-------------------------------------------------------------------

\begin{ThesisDedication}
WIP
\end{ThesisDedication}


%-------------------------------------------------------------------
%                  écriture de `Chapitre' et `Partie'
%                      dans la table des matières
%-------------------------------------------------------------------

\WritePartLabelInToc
\WriteChapterLabelInToc

%-------------------------------------------------------------------
%                        table des matières
%-------------------------------------------------------------------

\tableofcontents

%-------------------------------------------------------------------
%              Exemple d'utilisation de \SpecialSection
%-------------------------------------------------------------------
%\SpecialSection{Introduction générale}

\DontWriteThisInToc
\listoffigures

\mainmatter

% \import{chapters/introduction/}{introduction}
\import{chapters/etat-art/}{etat-art}
\import{chapters/rls/}{rls}
% \import{chapters/mute/}{mute}
% \import{chapters/conclusion/}{conclusion}
% \import{chapters/annexes/}{annexes}

%
%%-------------------------------------------------------------------
%%                         Le glossaire
%%-------------------------------------------------------------------
%\BeginGloWith{Voici un glossaire tout-à-fait fictif,
%              introduit par un texte sur toute la largeur
%              des deux colonnes.}
%\twocolumn
%\PrintGlossary

%-------------------------------------------------------------------
%              L'index (toujours sur deux colonnes)
%-------------------------------------------------------------------
\BeginIndWith{Voici un index}
\PrintIndex

\onecolumn

%-------------------------------------------------------------------
%                       La bibliographie
%-------------------------------------------------------------------

% La bibliographie (comme d'habitude)

%\nocite{*}
%\bibliographystyle{named}

\printbibliography

%-------------------------------------------------------------------
%                          Les résumés
%-------------------------------------------------------------------
% (si le résumé apparaît sur une colonne étroite, avec la
% bibliographie à gauche, c'est sans doute parce que vous avez
% oublié de générer les fichiers d'index et de glossaire...)

\NumberAbstractPages
\begin{ThesisAbstract}
    \vspace{-3cm}
    \begin{FrenchAbstract}
        Un système collaboratif permet à plusieurs utilisateur-rices de créer ensemble un contenu.
        Afin de supporter des collaborations impliquant des millions d'utilisateurs, ces systèmes adoptent une architecture décentralisée pour garantir leur haute disponibilité, tolérance aux pannes et capacité de passage à l'échelle.
        % Cependant, de part le rôle prédominant des serveurs dans leur fonctionnement, ces sytèmes échouent à garantir un autre ensemble de propriétés : confidentialité des données, souveraineté des données, pérennité et résistance à la censure.
        Cependant, ces sytèmes échouent à garantir la confidentialité des données, souveraineté des données, pérennité et résistance à la censure.
        % Pour répondre à ce problème, la littérature propose la conception d'applications \acf{LFS} : des applications collaboratives \acf{P2P} réléguant les serveurs à un simple rôle de support de la collaboration.
        Pour répondre à ce problème, la littérature propose la conception d'applications \acf{LFS} : des applications collaboratives \acf{P2P}.

        Une pierre angulaire des applications \ac{LFS} sont les \acfp{CRDT}.
        Il s'agit de nouvelles spécifications des types de données, tels que l'Ensemble ou la Séquence, permettant à un ensemble de noeuds de répliquer une donnée.
        Les \acp{CRDT} permettent aux noeuds de consulter et de modifier la donnée sans coordination préalable, et incorporent un mécanisme de résolution de conflits pour intégrer les modifications concurrentes.
        Cependant, les \acp{CRDT} pour le type Séquence souffrent d'une croissance monotone du surcoût de leur mécanisme de résolution de conflits.
        Pouvons-nous proposer un mécanisme de réduction du surcoût des \acp{CRDT} pour le type Séquence qui soit compatible avec les applications \ac{LFS} ?
        Dans cette thèse, nous proposons un nouveau \ac{CRDT} pour le type Séquence, RenamableLogootSplit.
        Ce \ac{CRDT} intègre un mécanisme de renommage qui minimise périodiquement le surcoût de son mécanisme de résolution de conflits ainsi qu'un mécanisme de résolution de conflits pour intégrer les modifications concurrentes à un renommage.
        Finalement, nous proposons un mécanisme de \acf{GC} qui supprime à terme le propre surcoût du mécanisme de renommage.

      % \KeyWords{CRDTs, édition collaborative en temps réel, cohérence à terme, optimisation mémoire, performance}
    \end{FrenchAbstract}
    \begin{EnglishAbstract}
        A collaborative system enables multiple users to work together to create content.
        To support collaborations involving millions of users, these systems adopt a decentralised architecture to ensure high availability, fault tolerance and scalability.
        However, these systems fail to guarantee the data confidentiality, data sovereignty, longevity and resistance to censorship.
        To address this problem, the literature proposes the design of \acf{LFS} applications: collaborative peer-to-peer applications.

        A cornerstone of LFS applications are \acfp{CRDT}.
        \acp{CRDT} are new specifications of data types, \eg Set or Sequence, enabling a set of nodes to replicate a data.
        \acp{CRDT} enable nodes to access and modify the data without prior coordination, and incorporate a conflict resolution mechanism to integrate concurrent modifications.
        However, Sequence \acp{CRDT} suffer from a monotonous growth in the overhead of their conflict resolution mechanism.
        Can we propose a mechanism for reducing the overhead of Sequence-type CRDTs that is compatible with LFS applications?
        In this thesis, we propose a novel \ac{CRDT} for the Sequence type, RenamableLogootSplit.
        This \ac{CRDT} embeds a renaming mechanism that periodically minimizes the overhead of its conflict resolution mechanism as well as a conflict resolution mechanism to integrate concurrent modifications to a rename.
        Finally, we propose a mechanism of \acf{GC} that eventually removes the own overhead of the renaming mechanism.
      % \KeyWords{CRDTs, real-time collaborative editing, eventual consistency, memory-wise optimisation, performance}
    \end{EnglishAbstract}
\end{ThesisAbstract}


\end{document}



